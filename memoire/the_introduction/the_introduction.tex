\documentclass[12pt,]{article}
\usepackage{lmodern}
\usepackage{setspace}
\setstretch{1.15}
\usepackage{amssymb,amsmath}
\usepackage{ifxetex,ifluatex}
\usepackage{fixltx2e} % provides \textsubscript
\ifnum 0\ifxetex 1\fi\ifluatex 1\fi=0 % if pdftex
  \usepackage[T1]{fontenc}
  \usepackage[utf8]{inputenc}
\else % if luatex or xelatex
  \ifxetex
    \usepackage{mathspec}
  \else
    \usepackage{fontspec}
  \fi
  \defaultfontfeatures{Ligatures=TeX,Scale=MatchLowercase}
    \setmainfont[]{XITS}
\fi
% use upquote if available, for straight quotes in verbatim environments
\IfFileExists{upquote.sty}{\usepackage{upquote}}{}
% use microtype if available
\IfFileExists{microtype.sty}{%
\usepackage{microtype}
\UseMicrotypeSet[protrusion]{basicmath} % disable protrusion for tt fonts
}{}
\usepackage[margin=1in]{geometry}
\usepackage{hyperref}
\hypersetup{unicode=true,
            pdftitle={Performance comparison of the discrete choice models of consumer choice},
            pdfborder={0 0 0},
            breaklinks=true}
\urlstyle{same}  % don't use monospace font for urls
\usepackage{graphicx,grffile}
\makeatletter
\def\maxwidth{\ifdim\Gin@nat@width>\linewidth\linewidth\else\Gin@nat@width\fi}
\def\maxheight{\ifdim\Gin@nat@height>\textheight\textheight\else\Gin@nat@height\fi}
\makeatother
% Scale images if necessary, so that they will not overflow the page
% margins by default, and it is still possible to overwrite the defaults
% using explicit options in \includegraphics[width, height, ...]{}
\setkeys{Gin}{width=\maxwidth,height=\maxheight,keepaspectratio}
\IfFileExists{parskip.sty}{%
\usepackage{parskip}
}{% else
\setlength{\parindent}{0pt}
\setlength{\parskip}{6pt plus 2pt minus 1pt}
}
\setlength{\emergencystretch}{3em}  % prevent overfull lines
\providecommand{\tightlist}{%
  \setlength{\itemsep}{0pt}\setlength{\parskip}{0pt}}
\setcounter{secnumdepth}{5}
% Redefines (sub)paragraphs to behave more like sections
\ifx\paragraph\undefined\else
\let\oldparagraph\paragraph
\renewcommand{\paragraph}[1]{\oldparagraph{#1}\mbox{}}
\fi
\ifx\subparagraph\undefined\else
\let\oldsubparagraph\subparagraph
\renewcommand{\subparagraph}[1]{\oldsubparagraph{#1}\mbox{}}
\fi

%%% Use protect on footnotes to avoid problems with footnotes in titles
\let\rmarkdownfootnote\footnote%
\def\footnote{\protect\rmarkdownfootnote}

%%% Change title format to be more compact
\usepackage{titling}

% Create subtitle command for use in maketitle
\newcommand{\subtitle}[1]{
  \posttitle{
    \begin{center}\large#1\end{center}
    }
}

\setlength{\droptitle}{-2em}

  \title{Performance comparison of the discrete choice models of consumer choice}
    \pretitle{\vspace{\droptitle}\centering\huge}
  \posttitle{\par}
  \subtitle{Exploration of the Econometrics and Machine Learning model performances
in the presence of heterogeneous preferences and random effects
utilities}
  \author{\large Nikita Gusarov\\
\normalsize Master 2\\
MIASHS C2ES (UGA)\\
~\\
~\\
\hspace{50mm}\\
~\\
~\\
\raggedright Under supervision of:\\
\raggedright\hspace{10mm} Iragaël Joly, HDR (GAEL, UGA, Grenoble INP)\\
\raggedright\hspace{10mm} Beatrice Roussillon, MCF (GAEL, UGA)\\
\hspace{-40mm}}
    \preauthor{\centering\large\emph}
  \postauthor{\par}
    \date{}
    \predate{}\postdate{}
  
%%%%%%%%%%%%%%%%%
% Load Packages %
%%%%%%%%%%%%%%%%%

% Spacings
\usepackage{setspace}

% Tables
\usepackage{longtable}
\usepackage{tabu}

% Floats
\usepackage{morefloats}
\usepackage{float}
\usepackage{placeins}

% Highlighting
\usepackage{soul}

% Horizontal page position
\usepackage{pdflscape}

% Append pdfs
\usepackage{pdfpages}

% Add latex chunks
\usepackage{docmute}

% Short toc
\usepackage{shorttoc}
%\setcounter{tocdepth}{1}
%\usepackage{minitoc} - incompatible with document class

% Referencing mutliple things with a single command - \cref
\usepackage{cleveref}

% Array
\usepackage{array}

% Multiple columns
\usepackage{multicol}

% Image insertion and colors
\usepackage{graphicx}

% Latex comments
\newenvironment{dummy}{}{}

% Fonts
% \usepackage{fontspec}
% \setmainfont{Museo}

% drawing
\usepackage{tikz}
\usetikzlibrary{matrix,chains,positioning,decorations.pathreplacing,arrows}

\usepackage{dcolumn}
% \usepackage{subfig}
\usepackage[export]{adjustbox}
% \usepackage[demo]{graphicx}  
\usepackage{subcaption}
\usetikzlibrary{shapes,arrows}
\usetikzlibrary{arrows.meta}
\usepackage[edges]{forest}

\usepackage{calc}

%%%%%%%%%%%%%%%%%%%
% Make Title Page %
%%%%%%%%%%%%%%%%%%%

% Maketitle definition
\makeatletter
\def\@maketitle{
    \pagenumbering{gobble} % Ommit page numbering
    % Add logos
    \raggedright
    \includegraphics[height = 30mm]{../images/uga_logo.png}  
    \hspace{5mm}
    %\includegraphics[height = 30mm]{../images/gael_logo_1.jpg}
    \hspace{5mm}
    %\includegraphics[height = 30mm]{../images/lig_logo_1.png}
    % Add title
    \begin{center}
        \vspace*{\fill}
            {\Huge \@title}\\
            \par
            \rule{7cm}{0.4pt}
            \par
            \textbf{Exploration of the Econometrics and Machine Learning models' performances in the presence of heterogeneous preferences and random effects utilities}\\ %[10mm] % Subtitle
            \par
            \rule{3cm}{0.4pt}
            \par
            \textbf{Research master thesis}\\[10mm]
            {\Large \@author}\\
        \vspace*{\fill}
    \end{center}
    % Add supplementary information
    % \vspace{10mm}
    % {\large Sous diréction de : }\\
    % \hspace{10mm} {\large Directeur}\\ % A modifier ici
    % \vspace{20mm}
    % {\large Niveau d'études : }\\
    % \hspace{10mm} {\large Niveau}\\ % A modifier ici
    % {\large Parcours : }\\
    % \hspace{10mm} {\large Parcours}\\ % A modifier ici
    % \vspace{20mm}
    % Add bottom section
    \begin{center}
        {\large Université Grenoble Alpes}\\ % Université
        {\large Faculté d'Economie et Gestion - FEG}\\ % Faculty
        \vspace{5mm}
        2019 - 2020\\ % Year
    \end{center}
    \clearpage
}
\makeatother

\begin{document}
\maketitle

\begin{description}

\item[Abstract:] This works is a cross-disciplinary study of econometrics and machine learning (ML) models applied to consumer choice modelling. 
To breach the interdisciplinary gap an integrated simulation and theory-testing framework is proposed. 
It incorporates all essential steps from hypothetical setting generation to the comparison of various performance metrics. \vspace{0.1cm}\\ 
The flexibility of the framework in theory-testing and models comparison over economics and statistical indicators is illustrated based on the work of Michaud, Llerena and Joly (2012). 
Two datasets are generated using the predefined utility functions simulating the presence of homogeneous and heterogeneous individual preferences for alternatives' attributes. 
Then, three models issued from econometrics and ML disciplines are estimated and compared. \vspace{0.1cm}\\ 
This study shows the proposed methodological approach's efficiency, successfuly capturing the differences between the models issued from different fields given the homogeneous or heterogeneous consumer preferences. 

\item[Key words:] Consumer Choice, Preference Studies, Willingness to Pay, Econometrics, Data Science, Machine Learning, Classification Techniques, Synthetic Datasets

\item[Author:] Nikita Gusarov (UGA)

\item[Under supervision of:] Iragaël Joly, HDR (GAEL, UGA, Grenoble INP); Beatrice Roussillon, MCF (GAEL, UGA)

\vspace{0.5cm}
\hrule
\vspace{0.5cm}

\item[Abstrait:] Ce travail est une étude interdisciplinaire des modèles d'économétrie et d'apprentissage automatique (ML) appliqués à la modélisation des choix des consommateurs.
Pour briser la frontière interdisciplinaire, un cadre intégré pour tester des théorie est proposé.
Il intègre toutes les étapes essentielles de la génération de paramètres hypothétiques à la comparaison de diverses mesures de performance. \vspace{0.1cm}\\ 
La flexibilité du cadre dans les tests de théorie et la comparaison de modèles par rapport aux indicateurs économiques et statistiques est illustrée à partir des travaux de Michaud, Llerena et Joly (2012).
Deux ensembles de données sont générés à l'aide des fonctions d'utilité prédéfinies simulant la présence de préférences individuelles homogènes et hétérogènes pour les attributs des alternatives.
Trois modèles issus des disciplines économétrie et ML sont ensuite estimés et comparés. \vspace{0.1cm}\\ 
Cette étude montre l'efficacité de l'approche méthodologique proposée, en captant avec succès les différences entre les modèles issus de différents domaines compte tenu des préférences homogènes ou hétérogènes des consommateurs.

\item[Mots clés:] Choix du consommateur, \'Etudes de Préférences, Consentements à Payer, \'Econométrie, Science des Données, Apprentissage Automatique, Techniques de Classification, Données Synthétiques

\end{description}

\newpage

\begin{center}
\textbf{\Large Acknowledgements}
\end{center}
\vspace{2.3ex}

This work was accomplished with financial aid from Multidisciplinary
Institute in Artificial Intelligence (MIAI), supported by Sihem
Amer-Yahia, head of the SLIDE team at the LIG laboratory.

\vspace{10mm}

I would like to express my gratitude for the administrative and
technical support from the Grenoble Informatics Laboratory (LIG) and
Grenoble Applied Economics Laboratory (GAEL), which helped to fulfil
this work during COVID-19 crisis.

\vspace{10mm}

Credits for dataset generation algorithm go to Amirreza Talebijamalabad,
first year master student at Grenoble INP, who worked on the theory of
the artificial datasets generation.

\newpage

\pagenumbering{roman}

\setcounter{tocdepth}{1}

\shorttoc{Summary}{2}

\newpage

\pagenumbering{arabic}

\hypertarget{introduction}{%
\section*{Introduction}\label{introduction}}
\addcontentsline{toc}{section}{Introduction}

The advances in statistical learning, data analysis and data science of
the past decades have resulted in propagation of \emph{Machine Learning}
(ML) techniques to different applied fields, including social and human
sciences. Nowadays, it is impossible to imagine a field of science that
is not benefiting from the fruits of statistical learning. The works of
De Palma et al. (2011) and Cascetta (2009) on transportation modelling,
the publications of Molina and Garip (2019) dedicated to sociology
problematic, the articles of Coussement, Benoit, and Poel (2010)
concerning marketing decisions, actuary analysis studies (Denuit and
Trufin (2019), Denuit and Hainaut (2019)) or even psychology with an
example of Baayen et al. (2017) work reflect the literal omnipresence of
the newly developed techniques.

However, there exist two completely distinct approaches to applying
statistical learning, as described by Breiman and others (2001) and
latter by Athey and Imbens (2019): the \emph{Machine Learning} which
focuses on the predictive qualities (figure \ref{fig:parad3}) and
\emph{Econometrics} which attempts to decipher the underlying properties
of the data (figure \ref{fig:parad2}). In economics, where the research
is focused on hidden patterns exploration, the scientific community
prefers to implement the traditional econometrics techniques using the
more advanced statistical models only in some special cases or as some
assistance tools (Athey 2018). This discrepancy is explained by the fact
that econometrics, contrary to traditional ML paradigm focusses on the
accessibility of results. Consequently, many of the advanced ML
techniques rarely appear in economics publications because of their
believed lack of interpretability and excessive complexity in
application. Nevertheless, some multidisciplinary scientists make
attempts to breach this wall between \emph{ML} and \emph{Econometrics}:
Varian (2014), Mullainathan and Spiess (2017) or, among the most recent,
Athey and Imbens (2019). Their advances are mostly focused on resolving
the general interdisciplinary tool-set integration questions, without
considering the application specific details. Nevertheless, in the
attempt to breach the interdisciplinary barrier the details reveal
themselves to be of utmost importance in the solution of the problem.

\begin{figure}[hbtp]
\centering
\caption{The different paradigms}
\label{fig:parad}
\begin{subfigure}[c]{.4\linewidth}
    \centering
    \caption{Real world}
    \label{fig:parad1}
    \begin{tikzpicture}[box/.style = {draw, text width=2cm, align=center}]
        \node[box] (b) {Nature};
        \node[left=of b] (a) {$\mathcal{X}$};
        \node[right=of b] (c) {$\mathcal{Y}$};
        \draw[->] (a) -- (b);
        \draw[->] (b) -- (c);
    \end{tikzpicture}
\end{subfigure}\hspace{12pt}\vspace{12pt}

\begin{subfigure}[c]{.4\linewidth}
    \centering
    \caption{Econometrics}
    \label{fig:parad2}
    \begin{tikzpicture}[box/.style = {draw, text width=2cm, align=center}]
        \node[box] (b) {Theoretical\\model};
        \node[left=of b] (a) {$X$};
        \node[right=of b] (c) {$Y$};
        \draw[->] (a) -- (b);
        \draw[->] (b) -- (c);
    \end{tikzpicture}
\end{subfigure}\hspace{12pt}
\begin{subfigure}[c]{.4\linewidth}
    \centering
    \caption{Machine Learning}
    \label{fig:parad3}
    \begin{tikzpicture}[box/.style = {draw, text width=2cm, align=center}]
        \node[box] (b) {Nature};
        \node[box, below=of b] (d) {ML\\model};
        \node[left=of b] (a) {$X$};
        \node[right=of b] (c) {$Y$};
        \draw[->] (a) -- (b);
        \draw[->] (b) -- (c);
        \draw[->] (a) |- (d);
        \draw[->] (d) -| (c);
    \end{tikzpicture}
\end{subfigure}\hspace{12pt}
\end{figure}

There have already been a multitude of studies comparing the
performances of different econometric and ML models in various real
world scenarios: the study of machine learning methods to model the car
ownership demand estimation of Paredes et al. (2017), for example; or
the use of decision trees in microeconomics of Brathwaite, Vij, and
Walker (2017). However, there's no known to us work incorporating at
least all the baseline models, as it would require an unimaginable
amount of efforts to accomplish. For instance, in the literature the
performance of competing models are studied according to several
absolutely alien criteria: in terms of the quality of data adjustments,
in terms of predictive capacity, as well as in terms of the quality of
the economic and behavioural indicators derived from estimates and,
finally, according to their algorithmic efficiency and computational
costs. None of the known to us articles manages to incorporate all these
aspects into their benchmarks, limiting their studies only with several
performance criteria.

These various aspects, greatly impact the performance of particular
models or algorithms, although some of them are often ignored by the
researchers. Not only there exists inconsistency in the targeted
performance metrics in the contemporary models' comparisons, but there
is also omnipresent problems of theoretical background choice, dataset
selection or model's specifications. For example, speaking about the
datasets used to support their findings, many researchers explore the
impacts of different specifications on the same observed or simulated
choice situation (Munizaga and Alvarez-Daziano (2005), Fiebig et al.
(2010), McCausland and Marley (2013), Bouscasse, Joly, and Peyhardi
(2019)) as it appears to be the most theoretically reliable procedure.
However, there is still no established unified methodology documenting
this field.

From this unambiguity in the scientific community the main problematic
of this work arises. It is particularly important to establish a common
framework for performance comparison of the discrete choice models be
they from the econometrics or ML tool-set. However, this task cannot be
accomplished outside a precise context, which will potentially impose
some limitations over the models' structure, as well as influence the
choice of performance metrics. In economics the discrete choice models
are extensively used for consumer choice analysis (Anderson, De Palma,
and Thisse 1992), willingness to pay derivation (Michaud, Llerena, and
Joly 2012) and other preference studies. The field specific theories and
traditional research objectives frame and define this study's scope.

From the economics perspective there exist three major points of
interest to be taken into account. First, there is a strong interest in
economics to explore the different behavioural set-ups, under different
settings and assumptions. Secondly, given the different choice
situations there is a potential need to test how the available
mathematical models, potentially sensitive to the tested behavioural
hypotheses or dependent on these hypotheses, perform in a given context.
Last, but not least, a comprehensive implementation of a performance
evaluation methodology, combining reproducibility and control of
experimental conditions, should be introduced in the proposed framework.

\textbf{Consumer choice}

The economic decision theory derives mostly from the random utility
theory (RUM) of McFadden (1974) and more recently of McFadden (2001),
that were recently challenged by alternative visions such as random
regret minimisation theory (RRM) of Chorus (2010), with a related
relative advantage maximisation theory (RAM) of Leong and Hensher
(2015), or even quantum decision theory (QDT) of Yukalov and Sornette
(2017), which offers a wide range of tools for modelling under
uncertainty.

These different theories address various aspects of the decision making
process, under different suppositions and incorporating different
biases. For example, one of the basic assumptions of the traditional
choice theory is the transitivity of choice, meaning there exists a
strict hierarchy of individual preferences among alternatives. This
assumption may be unsuitable for real world choice situation and lead to
potential bias, which is addressed by quantum decision theory. QDT
manages to bypass this shortcoming and incorporate non-transitivity of
choices into the framework. There exist a multitude of other behavioural
elements unexplained by the most traditional models that may be
incorporated into the decision making framework, such as loss aversion
for example, that could be addressed with random regret minimisation
theory.

There is a particular interest in detecting the differences in the
models' performances depending on the choice context and the assumed
decision-making framework. It is important, because different consumer
behaviour in the individual choice context result in different choice
distributions, which may affect the models' performances. In economics
RUM theory is nowadays one of the most used choice settings in the
individual decision modelling. Nevertheless, there still exist some
unexplored limitations, that such theoretical framework may impose over
the estimation techniques, as well as to what potential biases a model's
misspecification may lead.

\textbf{Mathematical models}

In general any classification technique may be used to model individual
decisions, although nearly every model has some restrictions and
limitations, which may largely affect its performances in a given
context.

Usually the choice of model is rarely discussed in applied studies, as
the researchers tend to use either the simplest model possible or
attempt to implement one particular model of interest ignoring some
times the other possible choices. For example, many traditional
econometrics studies, given a multiple choice problem context, use a
multinomial logistic regression (MNL) or even simplify the problem to a
binary case, allowing to implement even more traditional models such as
binary logit or binary probit models. However, there exists a multitude
of particular cases in modelling individual choices, that require
specific techniques to be implemented. A family of duration models may
be used to model the individual decisions over time (Vitetta (2016));
network modelling that allows to incorporate spatial and social
dependencies for the explored data (Brock and Durlauf (2003));
preference learning techniques aiming to explore the positioning of
different alternatives by an individual (Tsoukiàs and Viappiani (2013),
Pigozzi, Tsoukiàs, and Viappiani (2016)) and many other advanced
techniques from \emph{machine learning} field such as neural networks or
support vector machines.

An incorrect choice of the modelling technique may have a strong impact
on the derived target values leading to some erroneous conclusions in
the end. For example, an incorrectly estimated willingness to pay for a
particular product may lead to significant losses. When conducting an
applied research study one should always be conscious of the eventual
biases introduced by the choice of the model and the eventual
consequences of these choices. Some models are not suitable to be
implemented on a particular set of data, while others are unable to
provide necessary information about the relationships within a
particular dataset or derive the particular target values of interest.

Taking into account the implications of RUM theory, there exists a
particular interest to make the focus on the state of art econometric
discrete choice models (Agresti (2013), Agresti (2007), Baltagi (2008),
Train (2009), McFadden (2001), McFadden (1974)) as well as their
counterparts used in ML (Hastie, Tibshirani, and Friedman (2009),
Kotsiantis, Zaharakis, and Pintelas (2006)). A comparison of some simple
models against more complex ones may reveal the trade-off between
precise estimates and the resources invested.

\textbf{Data}

Different sources of data are available for a researcher, that could be
divided into two groups (Japkowicz and Shah 2011): \emph{field
datasets}, which are gathered through an experiment or collected from
the real world observations or real world uncontrolled experiment; and
\emph{synthetic datasets}, which are artificially generated by the
researcher to suit his needs and respect some particular limitations.
Although this variability of dataset choices not that evident in the
context of applied studies, there is an ongoing debate concerning the
eventual impacts of data choice on the models' performances and
resulting metrics.

Given a task of performance evaluation and comparison for different
algorithms or mathematical models there is always a difficult choice of
the data type to be used in the study. Both of the mentioned above
dataset types have their advantages and disadvantages and require a
particular attention. However, having for objective the theory- and
model-testing framework construction there is a strong interest to use
the artificially generated data in order to have as much control as
possible over the situation.

\textbf{The framework and context}

Given these three key elements we propose an integrated simulation and
theory-testing framework which will encompass all the different aspects
of the model comparison task. The steps to be integrated into such
framework encompass many theoretical questions starting with the
underlying theoretical assumptions and ending with the choice of correct
performance metrics. Consequently, this work attempts to fill the gap
between two statistical paradigms: \emph{econometrics} and \emph{machine
learning}, taking into account the key elements, among which the
different combinations of decision theory assumptions, dataset
generation procedures, mathematical models and target performance
measures. The problematic arises from the insufficient points of contact
among researchers from different fields of applications, as well as
insufficiently unified methodology to put into relations the different
approaches. A work that uses unified knowledge from several disciplines
might be highly beneficial for the scientific community as it will lie a
foundation and provide support for future applied studies. Following the
logic of Athey (2018) and Mullainathan and Spiess (2017) the project
will attempt to merge the essentials of ML and econometrics paradigms,
retaining their key concepts in the context of consumer choice problem.

We propose to use an applied paper in econometrics of choice modelling
to facilitate understanding of the field of application and tools. This
means not that we will attempt to replicate the results, but rather to
use the context provided in the work for demonstration of the proposed
hypothesis-testing framework. We select the article of Michaud, Llerena,
and Joly (2012) as our reference paper, because of the advantages to
work directly with the authors of the paper. The work of Michaud,
Llerena, and Joly (2012) is focused on investigation of consumers'
willingness to pay (WTP) for environmental attributes of a non-food
agricultural products, taking roses as example. Authors constructed an
experimental framework to derive the premium the testing subjects were
ready to pay for such environmental attributes as lower carbon imprint
and ecological labelling, certifying the source of the environmentally
friendly practices. That study explored individual preferences for roses
with an eco-label and a carbon footprint using discrete choice modelling
techniques and real economic incentives resulting in real purchases of
roses. The gathered dataset was analysed with a mixed logit model
demonstrating notorious premiums for both attributes. We will benefit of
the obtained results to demonstrate all of the complexity of a proposed
theory-testing framework, its functionality and perspectives.

The present report is divided into two main parts. The first section
presents the chosen context for this work followed by short
presentations of all the theoretical aspects which play their major
roles in this study, tracing at the same time parallels with the
context. The second part presents the results of all the results
step-by-step, demonstrating the functionality of the designed framework.
Each of the sections has an identical logical structure of presentation
of the framework's components in the successive order: starting with the
behavioural modelling and data related questions, directly followed by
the models' presentation and the performance measures. The final section
concludes.

\newpage

\hypertarget{refs}{}
\leavevmode\hypertarget{ref-agresti2007cd}{}%
Agresti, Alan. 2007. \emph{An Introduction to Categorical Data Analysis,
Second Edition}.

\leavevmode\hypertarget{ref-agresti2013cd}{}%
---------. 2013. \emph{Categorical Data Analysis, Third Edition}.

\leavevmode\hypertarget{ref-R-keras}{}%
Allaire, JJ, and François Chollet. 2020. \emph{Keras: R Interface to
'Keras'}. \url{https://CRAN.R-project.org/package=keras}.

\leavevmode\hypertarget{ref-R-tensorflow}{}%
Allaire, JJ, and Yuan Tang. 2020. \emph{Tensorflow: R Interface to
'Tensorflow'}. \url{https://CRAN.R-project.org/package=tensorflow}.

\leavevmode\hypertarget{ref-R-rmarkdown}{}%
Allaire, JJ, Yihui Xie, Jonathan McPherson, Javier Luraschi, Kevin
Ushey, Aron Atkins, Hadley Wickham, Joe Cheng, Winston Chang, and
Richard Iannone. 2018. \emph{Rmarkdown: Dynamic Documents for R}.
\url{https://CRAN.R-project.org/package=rmarkdown}.

\leavevmode\hypertarget{ref-anderson1992discrete}{}%
Anderson, Simon P, Andre De Palma, and Jacques-Francois Thisse. 1992.
\emph{Discrete Choice Theory of Product Differentiation}. MIT press.

\leavevmode\hypertarget{ref-athey2018iml}{}%
Athey, Susan. 2018. ``The Impact of Machine Learning on Economics.''
Book. In \emph{The Economics of Artificial Intelligence: An Agenda}, by
Ajay Agrawal, Joshua Gans, and Avi Goldfarb, 507--47. National Bureau of
Economic Research; University of Chicago Press.
\url{https://doi.org/https://doi.org/10.7208/chicago/9780226613475.001.0001}.

\leavevmode\hypertarget{ref-athey2019ml}{}%
Athey, Susan, and Guido W. Imbens. 2019. ``Machine Learning Methods That
Economists Should Know About.'' \emph{Annual Review of Economics} 11
(1): 685--725.
\url{https://doi.org/10.1146/annurev-economics-080217-053433}.

\leavevmode\hypertarget{ref-baayen2017gam}{}%
Baayen, Harald, Shravan Vasishth, Reinhold Kliegl, and Douglas Bates.
2017. ``The Cave of Shadows: Addressing the Human Factor with
Generalized Additive Mixed Models.'' \emph{Journal of Memory and
Language} 94: 206--34.
\url{https://doi.org/https://doi.org/10.1016/j.jml.2016.11.006}.

\leavevmode\hypertarget{ref-baltagi2008econometric}{}%
Baltagi, Badi. 2008. \emph{Econometric Analysis of Panel Data}. John
Wiley \& Sons.

\leavevmode\hypertarget{ref-joly2019qcm}{}%
Bouscasse, Hélène, Iragaël Joly, and Jean Peyhardi. 2019. ``A new family
of qualitative choice models: An application of reference models to
travel mode choice.'' \emph{Transportation Research Part B:
Methodological} 121 (C): 74--91.
\url{https://doi.org/10.1016/j.trb.2018.12.010}.

\leavevmode\hypertarget{ref-brathwaite2017machine}{}%
Brathwaite, Timothy, Akshay Vij, and Joan L Walker. 2017. ``Machine
Learning Meets Microeconomics: The Case of Decision Trees and Discrete
Choice.'' \emph{arXiv Preprint arXiv:1711.04826}.

\leavevmode\hypertarget{ref-breiman2001stat}{}%
Breiman, Leo, and others. 2001. ``Statistical Modeling: The Two Cultures
(with Comments and a Rejoinder by the Author).'' \emph{Statistical
Science} 16 (3). Institute of Mathematical Statistics: 199--231.

\leavevmode\hypertarget{ref-brock2003mcsi}{}%
Brock, William, and Steven Durlauf. 2003. ``Multinomial Choice with
Social Interactions.'' NBER Technical Working Papers 0288. National
Bureau of Economic Research, Inc.
\url{https://EconPapers.repec.org/RePEc:nbr:nberte:0288}.

\leavevmode\hypertarget{ref-cascetta2009tr}{}%
Cascetta, Ennio. 2009. \emph{Transportation Systems Analysis: Models and
Applications}. Vol. 29. Springer Science \& Business Media.

\leavevmode\hypertarget{ref-chorus2010rrm}{}%
Chorus, Caspar G. 2010. ``A New Model of Random Regret Minimization.''
\emph{European Journal of Transport and Infrastructure Research} 10 (2).

\leavevmode\hypertarget{ref-coussement2010gam}{}%
Coussement, Kristof, Dries F. Benoit, and Dirk Van den Poel. 2010.
``Improved Marketing Decision Making in a Customer Churn Prediction
Context Using Generalized Additive Models.'' \emph{Expert Systems with
Applications} 37 (3): 2132--43.
\url{https://doi.org/https://doi.org/10.1016/j.eswa.2009.07.029}.

\leavevmode\hypertarget{ref-R-mlogit}{}%
Croissant, Yves. 2020. \emph{Mlogit: Multinomial Logit Models}.
\url{https://CRAN.R-project.org/package=mlogit}.

\leavevmode\hypertarget{ref-denuit2019as3}{}%
Denuit, Michel, and Donatien Hainaut. 2019. \emph{Effective Statistical
Learning Methods for Actuaries Iii: Neural Networks and Extentions}.
Springer.

\leavevmode\hypertarget{ref-denuit2019as1}{}%
Denuit, Michel, and Julien Trufin. 2019. \emph{Effective Statistical
Learning Methods for Actuaries I: GLMs and Extentions}. Springer.

\leavevmode\hypertarget{ref-depalma2011tr}{}%
De Palma, André, Robin Lindsey, Emile Quinet, and Roger Vickerman. 2011.
\emph{A Handbook of Transport Economics}. Edward Elgar Publishing.

\leavevmode\hypertarget{ref-fiebig2010gmlm}{}%
Fiebig, Denzil, Michael Keane, Jordan Louviere, and Nada Wasi. 2010.
``The Generalized Multinomial Logit Model: Accounting for Scale and
Coefficient Heterogeneity.'' \emph{Marketing Science} 29 (3): 393--421.
\url{https://EconPapers.repec.org/RePEc:inm:ormksc:v:29:y:2010:i:3:p:393-421}.

\leavevmode\hypertarget{ref-hastie2009sl}{}%
Hastie, Trevor, Robert Tibshirani, and Jerome Friedman. 2009. \emph{The
Elements of Statistical Learning: Data Mining, Inference, and
Prediction}. Springer Science \& Business Media.

\leavevmode\hypertarget{ref-R-arsenal}{}%
Heinzen, Ethan, Jason Sinnwell, Elizabeth Atkinson, Tina Gunderson, and
Gregory Dougherty. 2020. \emph{Arsenal: An Arsenal of 'R' Functions for
Large-Scale Statistical Summaries}.
\url{https://CRAN.R-project.org/package=arsenal}.

\leavevmode\hypertarget{ref-R-stargazer}{}%
Hlavac, Marek. 2018. \emph{Stargazer: Well-Formatted Regression and
Summary Statistics Tables}.
\url{https://CRAN.R-project.org/package=stargazer}.

\leavevmode\hypertarget{ref-japkowicz2011el}{}%
Japkowicz, Nathalie, and Mohak Shah. 2011. \emph{Evaluating Learning
Algorithms: A Classification Perspective}. Cambridge University Press.
\url{https://doi.org/10.1017/CBO9780511921803}.

\leavevmode\hypertarget{ref-kotsiantis2006tr}{}%
Kotsiantis, Sotiris, I. Zaharakis, and P. Pintelas. 2006. ``Machine
Learning: A Review of Classification and Combining Techniques.''
\emph{Artificial Intelligence Review} 26 (November): 159--90.
\url{https://doi.org/10.1007/s10462-007-9052-3}.

\leavevmode\hypertarget{ref-leong2015ram}{}%
Leong, Waiyan, and David A. Hensher. 2015. ``Contrasts of Relative
Advantage Maximisation with Random Utility Maximisation and Regret
Minimisation.'' \emph{Journal of Transport Economics and Policy (JTEP)}
49 (1): 167--86.
\url{https://www.ingentaconnect.com/content/lse/jtep/2015/00000049/00000001/art00010}.

\leavevmode\hypertarget{ref-mccausland2013pd}{}%
McCausland, William J., and A.A.J. Marley. 2013. ``Prior Distributions
for Random Choice Structures.'' \emph{Journal of Mathematical
Psychology} 57 (3): 78--93.
\url{https://doi.org/https://doi.org/10.1016/j.jmp.2013.05.001}.

\leavevmode\hypertarget{ref-mcfadden1974utd}{}%
McFadden, Daniel. 1974. ``The Measurement of Urban Travel Demand.''
\emph{Journal of Public Economics} 3 (4): 303--28.
\url{https://doi.org/https://doi.org/10.1016/0047-2727(74)90003-6}.

\leavevmode\hypertarget{ref-mcfadden2001ec}{}%
---------. 2001. ``Economic Choices.'' \emph{The American Economic
Review} 91 (3). American Economic Association: 351--78.
\url{http://www.jstor.org/stable/2677869}.

\leavevmode\hypertarget{ref-llerena2013rose}{}%
Michaud, Celine, Daniel Llerena, and Iragael Joly. 2012. ``Willingness
to pay for environmental attributes of non-food agricultural products: a
real choice experiment.'' \emph{European Review of Agricultural
Economics} 40 (2): 313--29. \url{https://doi.org/10.1093/erae/jbs025}.

\leavevmode\hypertarget{ref-R-foreach}{}%
Microsoft, and Steve Weston. 2020. \emph{Foreach: Provides Foreach
Looping Construct}. \url{https://CRAN.R-project.org/package=foreach}.

\leavevmode\hypertarget{ref-molina2019soc}{}%
Molina, Mario, and Filiz Garip. 2019. ``Machine Learning for
Sociology.'' \emph{Annual Review of Sociology} 45 (1): 27--45.
\url{https://doi.org/10.1146/annurev-soc-073117-041106}.

\leavevmode\hypertarget{ref-mullainathan2017ml}{}%
Mullainathan, Sendhil, and Jann Spiess. 2017. ``Machine Learning: An
Applied Econometric Approach.'' \emph{Journal of Economic Perspectives}
31 (2): 87--106. \url{https://doi.org/10.1257/jep.31.2.87}.

\leavevmode\hypertarget{ref-munizaga2005mlyp}{}%
Munizaga, Marcela A., and Ricardo Alvarez-Daziano. 2005. ``Testing Mixed
Logit and Probit Models by Simulation.'' \emph{Transportation Research
Record} 1921 (1): 53--62.
\url{https://doi.org/10.1177/0361198105192100107}.

\leavevmode\hypertarget{ref-paredes2017machine}{}%
Paredes, Miguel, Erik Hemberg, Una-May O'Reilly, and Chris Zegras. 2017.
``Machine Learning or Discrete Choice Models for Car Ownership Demand
Estimation and Prediction?'' In \emph{2017 5th Ieee International
Conference on Models and Technologies for Intelligent Transportation
Systems (Mt-Its)}, 780--85. IEEE.

\leavevmode\hypertarget{ref-pigozzi2016pai}{}%
Pigozzi, Gabriella, Alexis Tsoukiàs, and Paolo Viappiani. 2016.
``Preferences in Artificial Intelligence.'' \emph{Annals of Mathematics
and Artificial Intelligence} 77 (3-4). Springer Verlag: 361--401.
\url{https://doi.org/10.1007/s10472-015-9475-5}.

\leavevmode\hypertarget{ref-R-foreign}{}%
R Core Team. 2018a. \emph{Foreign: Read Data Stored by 'Minitab', 'S',
'Sas', 'Spss', 'Stata', 'Systat', 'Weka', 'dBase', ...}
\url{https://CRAN.R-project.org/package=foreign}.

\leavevmode\hypertarget{ref-R-base}{}%
---------. 2018b. \emph{R: A Language and Environment for Statistical
Computing}. Vienna, Austria: R Foundation for Statistical Computing.
\url{https://www.R-project.org/}.

\leavevmode\hypertarget{ref-R-sdcm}{}%
Talebijamalabad, Amirreza, Nikita Gusarov, and Iragael Joly. 2020.
\emph{Sdcm: Simulation of Discrete Choice Models}.

\leavevmode\hypertarget{ref-train2009dc}{}%
Train, Kenneth E. 2009. \emph{Discrete Choice Methods with Simulation}.
Cambridge university press.

\leavevmode\hypertarget{ref-tsoukias2013ph}{}%
Tsoukiàs, Alexis, and Paolo Viappiani. 2013. ``Tutorial on preference
handling.'' In \emph{ACM Conference on Recommender System (RecSys)},
497--98. Hong Kong, China.
\url{https://doi.org/10.1145/2507157.2508065}.

\leavevmode\hypertarget{ref-varian2014bd}{}%
Varian, Hal R. 2014. ``Big Data: New Tricks for Econometrics.''
\emph{Journal of Economic Perspectives} 28 (2): 3--28.
\url{https://doi.org/10.1257/jep.28.2.3}.

\leavevmode\hypertarget{ref-vitetta2016quantum}{}%
Vitetta, Antonino. 2016. ``A Quantum Utility Model for Route Choice in
Transport Systems.'' \emph{Travel Behaviour and Society} 3. Elsevier:
29--37.

\leavevmode\hypertarget{ref-R-tidyverse}{}%
Wickham, Hadley. 2019. \emph{Tidyverse: Easily Install and Load the
'Tidyverse'}. \url{https://CRAN.R-project.org/package=tidyverse}.

\leavevmode\hypertarget{ref-R-knitr}{}%
Xie, Yihui. 2020a. \emph{Knitr: A General-Purpose Package for Dynamic
Report Generation in R}. \url{https://CRAN.R-project.org/package=knitr}.

\leavevmode\hypertarget{ref-R-tinytex}{}%
---------. 2020b. \emph{Tinytex: Helper Functions to Install and
Maintain Tex Live, and Compile Latex Documents}.
\url{https://CRAN.R-project.org/package=tinytex}.

\leavevmode\hypertarget{ref-yukalov2017quantum}{}%
Yukalov, Vyacheslav I, and Didier Sornette. 2017. ``Quantum
Probabilities as Behavioral Probabilities.'' \emph{Entropy} 19 (3).
Multidisciplinary Digital Publishing Institute: 112.


\end{document}
