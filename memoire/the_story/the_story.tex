\documentclass[11pt,]{article}
\usepackage{lmodern}
\usepackage{amssymb,amsmath}
\usepackage{ifxetex,ifluatex}
\usepackage{fixltx2e} % provides \textsubscript
\ifnum 0\ifxetex 1\fi\ifluatex 1\fi=0 % if pdftex
  \usepackage[T1]{fontenc}
  \usepackage[utf8]{inputenc}
\else % if luatex or xelatex
  \ifxetex
    \usepackage{mathspec}
  \else
    \usepackage{fontspec}
  \fi
  \defaultfontfeatures{Ligatures=TeX,Scale=MatchLowercase}
\fi
% use upquote if available, for straight quotes in verbatim environments
\IfFileExists{upquote.sty}{\usepackage{upquote}}{}
% use microtype if available
\IfFileExists{microtype.sty}{%
\usepackage{microtype}
\UseMicrotypeSet[protrusion]{basicmath} % disable protrusion for tt fonts
}{}
\usepackage[margin=1in]{geometry}
\usepackage{hyperref}
\hypersetup{unicode=true,
            pdfborder={0 0 0},
            breaklinks=true}
\urlstyle{same}  % don't use monospace font for urls
\usepackage{graphicx,grffile}
\makeatletter
\def\maxwidth{\ifdim\Gin@nat@width>\linewidth\linewidth\else\Gin@nat@width\fi}
\def\maxheight{\ifdim\Gin@nat@height>\textheight\textheight\else\Gin@nat@height\fi}
\makeatother
% Scale images if necessary, so that they will not overflow the page
% margins by default, and it is still possible to overwrite the defaults
% using explicit options in \includegraphics[width, height, ...]{}
\setkeys{Gin}{width=\maxwidth,height=\maxheight,keepaspectratio}
\IfFileExists{parskip.sty}{%
\usepackage{parskip}
}{% else
\setlength{\parindent}{0pt}
\setlength{\parskip}{6pt plus 2pt minus 1pt}
}
\setlength{\emergencystretch}{3em}  % prevent overfull lines
\providecommand{\tightlist}{%
  \setlength{\itemsep}{0pt}\setlength{\parskip}{0pt}}
\setcounter{secnumdepth}{0}
% Redefines (sub)paragraphs to behave more like sections
\ifx\paragraph\undefined\else
\let\oldparagraph\paragraph
\renewcommand{\paragraph}[1]{\oldparagraph{#1}\mbox{}}
\fi
\ifx\subparagraph\undefined\else
\let\oldsubparagraph\subparagraph
\renewcommand{\subparagraph}[1]{\oldsubparagraph{#1}\mbox{}}
\fi

%%% Use protect on footnotes to avoid problems with footnotes in titles
\let\rmarkdownfootnote\footnote%
\def\footnote{\protect\rmarkdownfootnote}

%%% Change title format to be more compact
\usepackage{titling}

% Create subtitle command for use in maketitle
\newcommand{\subtitle}[1]{
  \posttitle{
    \begin{center}\large#1\end{center}
    }
}

\setlength{\droptitle}{-2em}

  \title{}
    \pretitle{\vspace{\droptitle}}
  \posttitle{}
    \author{}
    \preauthor{}\postauthor{}
    \date{}
    \predate{}\postdate{}
  
\usepackage{placeins}
\AtBeginDocument{\let\maketitle\relax}
\usepackage{tikz}
\usetikzlibrary{matrix,chains,positioning,decorations.pathreplacing,arrows}
\usetikzlibrary{shapes,arrows}
\usetikzlibrary{arrows.meta}
\usepackage{amsmath}
\usepackage[edges]{forest}
\usepackage{multicol}

\begin{document}

\hypertarget{the-framework-design}{%
\section{The framework design}\label{the-framework-design}}

This section introduces the design and provides an example of available
functionality of an integrated experimental framework for model
performance exploration. In doing so, we strive to reduce and simplify
the framework, illustrating the theoretical discussion of the eventual
questions that arise during the model evaluation.

There exist multiple ways to provide an illustration for the generalized
framework due to its extended flexibility on different levels of
scientific procedure. Nevertheless, in our work we are attempting to
extend this illustrative objective to all the possible levels available
by the devised tool-set. The idea is to demonstrate all the features of
different frameworks' layers in the context of a performance comparison.
First of all, there is a particular interest to demonstrate the
advantages of possibility to test different choice settings, providing
different artificial datasets for exploration. What is more, it would be
interesting to contrast different mathematical models and algorithms
used to study these datasets and evaluate their performance using
different criteria, which will allow for more flexibility.

The work of Michaud, Llerena, and Joly (2012) investigates the impacts
of the environmental characteristics in the context of a consumer choice
of non-alimentary agricultural goods taking roses as an example. We will
inspire ourselves with the context, assumptions and findings of this
study and build our work around these pre-sets. We may be interested to
observe how some minor changes in the model may affect the results,
which pushes us to consider some simple, yet educative changes in the
model.

The organisation of this section is as follows. First of all, we
introduce in detail the context and discuss which features and
characteristics to retain give the Michaud, Llerena, and Joly (2012)
work. We will provide a description of the procedure adopted for this
illustration procedure as well. After brief overview of the original
article and delimitation of the general assumptions we will provide a
detailed discussion over every single major part of testing framework
with extensive argumentation. Starting with the presentation of the
underlying concepts of the decision theories and dataset generation
procedure we will continue with a discussion of different modelling
techniques and a detailed description of the models to be tested over
the artificial dataset. Finally, we will provide a panorama of the
performance assessment metrics, before switching over to application.

\hypertarget{context-willingness-to-pay-for-environmental-attributes-of-non-food-agricultural-products}{%
\subsection{Context: Willingness to pay for environmental attributes of
non-food agricultural
products}\label{context-willingness-to-pay-for-environmental-attributes-of-non-food-agricultural-products}}

We choose to use the estimated results of Michaud, Llerena, and Joly
(2012) as a starting point for our work, copying the context of the
study with some minor adjustments. In this part we will provide only a
general overview of the assumptions made in the article ``as is''. This
description will serve us as a reference for future discussion, because
afterwards we will be presenting our changes, modifications and
additions to these given our needs.

In the article of Michaud, Llerena, and Joly (2012) the choice of roses
as the non-food agricultural product was determined by several criteria.
Initially, roses were supposed by authors to have characteristics that
respect the limitations imposed by the experimental economics. These are
popular widespread products known to all the test subjects, being not
easily available at the same time. What is more, the production of roses
have been the object of a growing attention because of potential
environmental damages inflicted in the process. This last feature made
them a perfect product to explore the impacts of the environmental
properties on the consumer choice.

Two environmental aspects of roses' production were explored by Michaud,
Llerena, and Joly (2012). The first one, eco-labelling, described the
cultivation environment and conditions, including the use of pesticides,
fertilizers, as well as reasonable consumption of water and energy. This
labelling was adopted shortly before 2010 by some of the producers, who
attempted to reduce the harm to environment, to signal their
eco-responsible position to consumer. Authors mention such dedicated
eco-labels as the American \emph{VeriFlora} ``Certified Sustainably
Grown'' label guaranteeing the low environmental impact of roses'
production, or the European equivalent: ``\emph{Fair Flowers Fair
Plants}'' (FFP) label certifying the environmental performance of
agricultures by several criteria such as the ``\emph{fertilizer use,
crop production, energy efficiency, waste management and a number of
social requirements}''. The second chosen environmental feature of roses
was their carbon footprint, measured by the greenhouse gases emissions
during the cultivation and transportation. This criteria being
particularly important because of an increase of roses production in
developing countries in Africa, South America and Asia, which are later
sold on the European market, resulting in immense amount of CO2
emissions during the transportation.

The authors assumed that the individuals had heterogeneous preferences
for the environmental attributes of roses. In other words, it was
assumed that each individual had his personal attitude to the eco-label
and carbon footprint of the roses, determined by their awareness of the
environmental issues. The experimental design took into account this
assumed dimension through observation of multiple simultaneous choices
for each of the subjects in order to capture individual specific
elements. To model such complex repeated choice framework, authors used
well developed RUM behavioural theory (McFadden 2001) paired with the
power of the mixed logit model, which is a generalisation of a simple
logit model, allowing for more flexibility, such as random effects
modelling.

The assumptions made by the researchers may be roughly divided into two
categories, which will define the structure of this section. First one
comprises the behavioural assumptions concerning the decision-making
procedure, which encompasses the different restrictions on the
experimental design, individual's behavioural strategy and choice
preferences, which aim at elimination of various behavioural biases and
simplification of future mathematical analysis and data treatment. The
second regroups the assumptions related to the modelling process. It
encompasses theoretical assumptions imposing restrictions on the
mathematical model, its choice and estimation techniques. Finally, we
present the target effects computed by the researchers in the context of
the study, as the main objective was not the general approximation and
modelling of a consumer choice, but rather extraction of particular
values of interest such as willingness to pay for the alternatives'
attributes.

\hypertarget{experimental-design}{%
\subsubsection{Experimental design}\label{experimental-design}}

First of all, we should start with a description of the experimental
design framework introduced by the authors in order to obtain valid
results. This would allow to correctly implement such complex
econometric model as mixed logit on the next stage. The experimental
design assumed that individuals make their decisions based on the
perceived utility of a particular alternative, following the traditional
restrictions described by McFadden (2001).

Because the study collected data through a controlled experiment
setting, some restrictions were imposed on the observed characteristics
in order to simplify the analysis. The roses, as available alternatives,
were defined by three attributes observed by subjects:

\begin{itemize}
\tightlist
\item
  the FFP EU eco-label (\emph{Label})
\item
  the carbon footprint (\emph{Carbon})
\item
  the price (\emph{Price})
\end{itemize}

These attributes varied across the available options of the alternatives
present in different choice sets. Precise written instructions were
transmitted to the subjects making available information about the
criteria certified by the FFP labelling as well as some briefing about
the organization issuing this labels (the Horticultural Commodity
Board). These data-sheets provided as well a summary of Cranfield
University's report about roses' carbon footprint. Both these attributes
(eco-label and carbon footprint) were understood as a binary variables
valuing 0 or 1 depending on the presence of a particular attribute for a
particular rose. Finally, in addition to the two environmental
attributes, a price was introduced into experimental design, which
varied by 0.50€ between 1.50€ and 4.50€, creating this way a seven level
factor. The table \ref{tab:attributes} regroups the main characteristics
for these variables.

\begin{table}[!htbp] \centering 
 \caption{Alternatives' attributes} 
 \label{tab:attributes} 
\begin{tabular}{@{\extracolsep{5pt}}lcccccc} 
\\[-1.8ex]\hline 
\hline \\[-1.8ex] 
Statistic & \multicolumn{1}{c}{Levels} & \multicolumn{1}{c}{Min} & \multicolumn{1}{c}{Max} & \multicolumn{1}{c}{Step} \\ 
\hline \\[-1.8ex] 
Price & 7 & 1.5€ & 4.5€ & 0.5€\\ 
Label & 2 & 0 & 1 & 1 \\ 
Carbon & 2 & 0 & 1 & 1 \\
\hline \\[-1.8ex] 
\end{tabular} 
\end{table}

In order to avoid the substitution bias\footnote{The substitution bias
  occurs when the individuals tend to switch to the less expensive
  alternative available, given the relative prices changes. In the
  experiment context, the customers might have preferred to buy
  identically priced roses in a better placed store, instead of waiting
  for bought in experiment roses to be delivered.} as the subjects might
have decided to purchase a rose for the experiment somewhere else for a
lower price rather than in the laboratory a special measure was
introduced into experimental design. In the experimental literature the
implemented method is known as the ``\emph{field price censoring}'',
which means that the values used in the laboratory are censored
according to the field market price ({\textbf{???}}).

The elicitation of the individual preferences for the different roses'
attributes was ensured through a combination of discrete choice
questions and real economic incentives. The stated choice surveys are a
popular choice for study of consumer preferences for public and private
goods. The discrete choice methodology and experimental design setting
provides the advantage to vary several attributes of a particular
product and to estimate the marginal rates of substitution between these
attributes. A particular accent was made on the derivation of the
willingness to pay (WTP) for specified features of interest. This tool
provides a great flexibility, allowing to test different scenarios all
of which could be presented in a single study, although, there is always
a danger that the choices made by consumers in experimental surveys
might not reflect their real preferences. The participants to
hypothetical surveys were generally stating higher WTP values for
private and public goods, leading to a potential bias in the estimates
when compared to the real world. Following this reasoning the authors
have introduced incentives into their choice experiment linking this way
the participants' decisions to real consequences by resulting in
acquisition of randomly chosen alternative from the pool of chosen
alternatives.

The choice set generation was devised with intention to resemble to
maximum the actual purchase decisions with the inclusion of a ``\emph{do
not buy}'' option, in order not to force the subjects to buy anything.
In other words, the presence of such alternative ensured that subjects
were never pushed to purchase a rose, imitating this way a real shopping
situation, when consumers always have the possibility of not purchasing
any roses if none of the alternatives suited them in a particular choice
set. Consumers were asked to make twelve different choices displayed.

In the case of prices allocation random design techniques were used to
configure the subsets of choice sets among subjects. The two level
factors standing for the roses' environmental attributes could be
regrouped into four different combinations defining four types of roses.
The experimental design introduced the roses in pairs to subjects,
creating this way several three alternatives choice sets. Even though
the different combinations of two roses potentially create sixteen
different alternative pairs, the authors limited their search to six
completely different pairs of roses. The resulting experimental sets of
six choice sets were repeated twice resulting in twelve cards, which
were then introduced to subjects. All the cards were distributed
simultaneously so that consumers could make their choices in any order.
Individuals were informed from the beginning that one of their decisions
would be randomly drawn at the end of the experiment. Finally, the
random draw resulted in the purchase of a real rose offered against
payment, this condition ensured that the subjects considered each choice
made during the experiment as a real purchase decision, weighting
carefully the available alternatives.

Generic titles were randomly allocated to the roses within choice sets:
rose A and rose B respectively. Such ``\emph{unbranded}'' alternatives'
titles allowed to ensure that they can only be differentiated according
to their attribute combinations. This way the choice between a
``\emph{Rose A}'' and a ``\emph{Rose B}'' can only be defined by their
attributes alone (Label, Carbon footprint and Price), but not by their
label. The same strategy applied to prices, which were randomly assigned
to the alternatives within the choice sets by a random number generator
setting prices within the defined limits.

Taking into account the experimental design we are going to follow the
authors' ideas in simulating an identical experimental design with
statistical methods available. The artificial choice situation will
assume three alternatives: two unlabelled ones doted with a common
utility function, while the third is the baseline alternative of ``no
choice'' option. In order to study the heterogeneity of the individual
preferences the subjects should be placed in a situation of repeated
choice, facing several choice situation. The alternatives will be
described by three attributes, while individuals will be distinguished
by four characteristics.

\hypertarget{econometric-model}{%
\subsubsection{Econometric model}\label{econometric-model}}

Consumers' decisions are analysed with the discrete choice framework
based on the utility maximisation assumption. This framework assumes
that consumers associate each alternative in a choice set with a utility
level and choose the option, which maximises this utility. The general
estimation framework of the Random Utility Model (RUM) proposed by
McFadden (1974) provides the opportunity to estimate the effects of
product attributes and individual characteristics and to compute
willingness to pay indicators.

Authors implemented the mixed logistic regression with random,
correlated attributes' effects to estimate the willingness to pay of the
individuals for each of the explored attributes of a rose. The mixed
logit model takes into account the repeated nature of the choices made
by the respondents. This model relaxes the Independence from Irrelevant
Alternatives (IIA) hypothesis of the more traditional multinomial logit,
allowing the random components of the alternatives to be correlated, at
the same time the error terms are still considered to be identically
distributed ({\textbf{???}}). The alternative specific parameters are
assumed to be randomly distributed across the population contrary to the
fixed parameters specification for a traditional multinomial logit
model. In other words, the mixed logit model provides the opportunity to
consider heterogeneous effects among individuals by allowing taste
parameters to vary in the population. The authors suppose that the
random taste heterogeneity should be evident in response to the
eco-label and the carbon footprint attributes of the roses, because of
different level of environmental awareness across population. Following
the ideas of ({\textbf{???}}) the authors introduce the cross-product
for eco-labelling and carbon footprint as a random parameter as well in
attempt to test the effect of the simultaneous presence of both of these
attributes on consumer choice. This addition results in a total of four
random parameters to be estimated: the two parameters describing roses
attributes, their cross-product and the ``\emph{Buy}'' option dummy
variable which captures heterogeneity in consumers' preferences for a
rose. All of the random parameters associated with the roses' attributes
are assumed to follow normal distribution, which is traditional for the
procedure of mixed logit modelling. Given that the normal distribution
is symmetric and unbounded, the resulting model allows for both positive
and negative effects to exist inside population. To simplify the
analysis and assuming the reasoning of ({\textbf{???}}), the authors
restrict the price coefficient to be fixed in the population. Such
choice of price's effects specification ensures that all respondents
have a negative price coefficient, leading to a normally distributed
estimate of willingness to pay.

The systematic part of the utility relatively to the ``No buy'' option
was expressed through a linear in parameters form:

\begin{multline}
V_{ij} = \alpha_{i,Buy} + \beta_{Buy, Sex} Sex_i + \beta_{Buy, Age} Age_i + \beta_{Buy, Income} Income_i + \beta_{Buy, Habit} Habit_i + \\
+ \gamma_{Price} Price_{ij} + \gamma_{i, Label} Label_{ij} + \gamma_{i, Carbon} Carbon_{ij} + \gamma_{i, Label \times Carbon} Label \times Carbon_{ij}
\end{multline}

Where \(j\) was an alternative among the available choice set of three
options: buy rose A, buy rose B or do not buy anything. The dummy
variable \(\alpha_{i,Buy}\) was introduced to capture the effect of a
decision to buy a rose, while the vectors of \(\beta\) and \(\gamma\)
regrouped the effect of individual characteristics and the attributes of
alternatives respectively.

In their article Michaud, Llerena, and Joly (2012) did not provide an
extensive demonstration or description of the model selection procedure.
What is more, we have little information as to what model comparison and
validation techniques were implemented. Only the final model, chosen by
authors, was presented to us, which brings some limitations for our
study.

In the end of this subsection, it is important to highlight that the
mixed logit models are usually specified with uncorrelated random
effects, although it's not the case in the context of this particular
study. The authors introduce correlation between the normally
distributed alternative specific coefficients: \(\alpha_{i,Buy}\),
\(\gamma_{i, Label}\), \(\gamma_{i, Carbon}\) and
\(\gamma_{i, Label \times Carbon}\).

\hypertarget{willingness-to-pay-and-premiums}{%
\subsubsection{Willingness to pay and
premiums}\label{willingness-to-pay-and-premiums}}

The only target metrics present in the article were the willingness to
pay (WTP) and premiums for particular attributes. The former could be
read as the value the consumers are willing to pay for a rose. The
latter may be translated as how much consumers are ready to pay for a
unit change of a given attribute of the product. Both the WTP for a
product and the premiums can be computed as the marginal rates of
substitution between the quantity expressed by the attributes and the
price ({\textbf{???}}). The WTP for a rose in this case could be
expressed as:

\begin{equation}
WTP = \frac{
 \frac{\Delta V}{\Delta BUY}
}{
 \frac{\Delta V}{\Delta Price}
} = \frac{
  - \alpha_{Buy}
}{
  \beta_{Price}
}
\end{equation}

Where \(\frac{\Delta V}{\Delta BUY}\) is the difference in the relative
utility \(V\) associated with the ``Buy'' and ``No buy'' choices. The
premiums for the particular attributes \(Z_k\) of a given product could
be identically expressed as:

\begin{equation}
WTP = \frac{
 \frac{\Delta V}{\Delta Z_k}
}{
 \frac{\Delta V}{\Delta Price}
}
\end{equation}

Since the random parameters of the utility function were assumed to be
correlated, authors used Krinsky and Robb parametric bootstrapping
method ({\textbf{???}}) with 1000 draws to estimate the standard
deviations and confidence intervals for these parameters.

\hypertarget{theories-of-consumer-choice}{%
\subsection{Theories of consumer
choice}\label{theories-of-consumer-choice}}

Once we have presented the assumed context for this study, we will dive
further into details and present all the key behavioural elements of
this work one by one. In this section we are going to present in detail
the questions and problematic associated with the behavioural modelling
of the consumer choice. Particularly, we are going to introduce the
terminology to be used in this work, some of which was already partially
presented in the previous section.

\hypertarget{general-terminology}{%
\subsubsection{General terminology}\label{general-terminology}}

For the presentation of general methodology we are going to adopt the
ideas of De Palma et al. (2011), introducing this way the principal
concepts and main components of the decision theory. Traditionally it
comprises several components: the decision makers or \emph{individuals},
described by their characteristics; a set (or sets) of available
\emph{alternatives}, defined by their attributes; and a decision rule or
set of rules, describing the procedure adopted by the individuals to
make actual decisions.

The individuals are supposed to have different tastes, and therefore we
must explicitly treat the differences in the decision-making processes
among individuals, doted with different characteristics. Therefore the
characteristics \(X_i\) of the decision maker \(i\) constitute an
important part of the problem.

The decision maker chooses from a finite and countable set of
alternatives \(\{\omega_i, \dots, \omega_j\}\), which consists of the
entire \emph{universal set of alternatives}
\(\{\omega_1, \dots, \omega_r \} \in \Omega\) as defined by the
particular choice environment. A decision maker \(i\) may only consider
a subset of this universal set \(\Omega\), and this consideration set is
conventionally named \emph{a choice set} \(\Omega_i\). In discrete
choice analysis, each alternative \(\omega_j\) is characterized by its
attributes \(Z_j\). For example, in the particular case study the
observed attributes of roses are their price, the eco-label and the
relative carbon footprint. Decision makers evaluate the attractiveness
of an alternative based on these attribute values before making their
choice.

Finally, the decision rule describes the process by which the decision
maker \(i\) evaluates the available information
\(Z_j \forall \omega_j \in \Omega_i\) and arrives at a unique choice.
There is a wide range of available decision rules, including dominance,
satisfaction, lexicographic, elimination by aspect, habitual, imitation,
and utility (De Palma et al. 2011). However, only the latter class is
most often associated with discrete choice analysis because to its
extensive use in the consumer choice behaviour modelling. The utility
theory takes its roots from the microeconomic consumer theory and is
adjusted according to the needs of the modeller. A utility \(U_{ij}\)
represents the attractiveness of a particular alternative \(\omega_j\)
for a particular individual \(j\) in a scalar form.

\hypertarget{random-utility-maximisation-models}{%
\subsubsection{Random utility maximisation
models}\label{random-utility-maximisation-models}}

The random utility maximisation models (RUM) were introduced and
developed by McFadden (1974). The theory of optimization implies that
this is a classical indirect utility function, with the following
properties: ``\emph{it has a closed graph and is quasi-convex and
homogeneous of degree zero in the economic variables}'' (McFadden 2001).
The last element in applying the standard model to discrete choice is to
require the consumer's choice among the feasible alternatives to
maximize conditional indirect utility based on some reference
alternative, rather than absolute utility.

In our work we use the notation introduced by Bhat (1995) and later
adopted by Cascetta (2009) when representing the utility functions as
they are more simple and easy to understand compared to initial McFadden
(1974) specification. The functional form of the canonical indirect
utility function depends on the structure of preferences, including the
trade-off between different available alternatives. The perceived
utility \(U_{ij}\) can be expressed as the sum of two terms: a
systematic utility and a random residual term:

\begin{equation}
U_{ij} = V_{ij} + \eta_{ij}
\end{equation}

Where \(U_{ij}\) stand for utility, \(V_{ij}\) at the same time
represent its deterministic part defined by some fixed deterministic
function and \(\eta_{ij}\) reflects some unobserved random effects. The
latter having being a random variable following Gumble distribution,
parametrized with \((\mu = 0, \theta = 1)\), which may be interpreted
as:

\begin{equation}
\eta_{ij} = - log( - log(\epsilon_{ij}))
\end{equation}

With \(\epsilon_{ij}\) a variable uniformly distributed and independent
across alternatives, the disturbances are independently identically
distributed Extreme Values (EV). This produces a MNL model in which the
systematic utility has a linear in parameters form for each alternative
\(\omega_j \in \Omega\). The systematic utility \(V_{ij}\) represents
the mean utility perceived by all decision-makers having the same choice
context decision-maker.

\begin{equation}
V_{ij} = f(X_i, X_j) + \eta_{ij}
\end{equation}

Traditionally in the most simple models this deterministic utility part
is represented by some linear in parameters function:

\begin{equation}
f(X_i, Z_j) = \alpha_j + \beta_j X_i + \gamma Z_j
\end{equation}

One family of RUM-consistent discrete choice models that is very
flexible is the random parameters or mixed multinomial logit (MMNL or
more often denoted as ML) model, which is used in the Michaud, Llerena,
and Joly (2012) work. The random parameters set-up assumes \(\gamma\)
effects to be randomly distributed across individuals, usually following
normal random distribution. In some cases, these parameters may be
assumed to be correlated, which potentially reflects better the real
world.

In our study we are going to explore two equally possible in real life
specification for data generation procedure: one assuming random effects
for alternative specific variables and another keeping these parameters
fixed. Speaking about the utility definition, we assume, that the work
of Michaud, Llerena, and Joly (2012) managed to obtain correct estimates
for a relative utility function of roses and we take this particular
function structure in order to generate utilities for a given dataset.
This assumption will offer us a baseline and target effects' values to
compare our estimation with.

\hypertarget{different-datasets-available-in-research}{%
\subsection{Different datasets available in
research}\label{different-datasets-available-in-research}}

There exist numerous difficult questions related to the models'
comparison task such as performance measures' choice or models'
specification, but beforehand there always stand the data related
questions. It is due to the fact that all the other questions and the
validity of the obtained responses rely entirely on the choice of the
inputs and the data available. Many of the existing applied econometrics
papers use the most simple specification of the Multinomial Logistic
Regression (MNL), that may lead to erroneous results and conclusions.

Many of the models' performances and performance measures depend on the
dataset properties and the particular application case. This means that
in comparison of different mathematical models, implementing some
complex tools such as a neural network models (NN), for example, we
should pay attention to use appropriate data-model to estimate such
model. This particular problem, as many others related to the models'
performance evaluation, was extensively described by Japkowicz and Shah
(2011).

When it comes to model comparison, the additional requirements arise to
the validation datasets and we should find answers to several questions:

\begin{itemize}
\tightlist
\item
  What datasets should be used?
\item
  Should the model be validated on one dataset or several several?
\item
  Should a synthetic or real-world data be used?
\item
  If several dataset are chosen, which ones should be used on different
  validation steps?
\item
  How the algorithms should be tuned face to the dataset selection?
\item
  What properties the studied data should have?
\end{itemize}

Moreover, different models may require different data adaptation methods
to be implemented. For example, the popular multinomial logistic
regression allows to take into account the individual characteristics as
well as the attributes of the various alternatives issued from some
limited set. The ML approaches, such as Support Vector Machines or
Linear Discriminant Analysis does not allow such flexibility. For these
models, even if we can represent each point in the modelled space as a
combination of individual characteristics and attributes of
alternatives, we can only classify the instances by iterative binary
separation (Tsoumakas and Katakis 2007). Consequently, the questions of
the dataset properties arises, which are tightly intertwined with the
available models choice and the implemented learning techniques.

In this section we will discuss the different existing approaches to
data management in theory testing and hypothesis verification. Firstly,
we will present the general questions and problematics. Then a solution
to be implemented in this particular study will be described and
discussed.

\hypertarget{theoretical-concerns-in-dataset-selection}{%
\subsubsection{Theoretical concerns in dataset
selection}\label{theoretical-concerns-in-dataset-selection}}

The data related problematic arise firstly during model generation step
of the standard statistical learning procedure and persists till the
stage of the model comparison. Speaking about the model validation, the
usual \emph{rule of thumb} approach is the cross-validation technique,
although some advanced users suggest that this method may not be always
appropriate (Japkowicz and Shah 2011). In econometrics, for example, as
well as in many other applied disciplines, researches tend to
oversimplify the validation step by completely avoiding this important
step, or by performing only \emph{single-fold} validation. On the other
hand, many advanced statistical model and ML methods require a separate
tuning step during model set-up, which alone requires verification and
validation on some dataset. It remains questionable whether the overall
model validation dataset and the dataset used for fine tuning should be
the same or not.

Of particular interest for our study is the ongoing discussion between
two sides of the statisticians' community, mentioned by Japkowicz and
Shah (2011), about whether the algorithms and statistical models should
be compared over the real world datasets or using some synthetically
generated data. On one hand, the datasets composed of the observations
or obtained through controlled experiments perfectly reflect the real
world situation, being at the same time too case specific. In other
words, it is always dubious that a model or a theory verified for one
particular real dataset has any external validity. The obtained insights
can rarely be extended over a larger population. Artificial data can be
designed in a controlled manner to study specific aspects of the
performance of algorithms and models. Moreover, the artificial data is
highly useful for testing particular theories, for example, the
behavioural theories or their impact on different models. Consequently
such data may allow for tighter control, which gives rise to more
carefully constructed and more enlightening experiments. Although, the
real data are hard to obtain and are difficult to analyse, the
artificial data introduces the danger of the problem's
oversimplification. In our case study these features are of utmost
importance, because, given the framework, the artificial data enables us
to test desired hypothesis in a controlled environment.

Generation of synthetic datasets is a common practice in many research
areas. Such data is often generated to meet specific needs or certain
conditions that may not be easily found in the original, real data. The
nature of the data varies according to the application filed and
includes text, graphs, social or weather data, among many others. In
this particular work we, for example, face the consumer choice data,
which describes individuals and their choice sets.\\
The common process to create such synthetic datasets is to implement
small scripts or programs, restricted to limited problems or to a
specific application.

As Garrow (2010) points it out, even observing the growing use of
artificial data in discrete choice and classification analysis, ``little
is known about how the methodology used to generate synthetic datasets
influences the properties of parameter estimates and the validity of
results based on these estimates''. That is, there are two potential
sources of biases when using synthetic discrete choice data:

\begin{itemize}
\tightlist
\item
  The unknown effect of the dataset generation method;
\item
  The parameter estimation bias.
\end{itemize}

The first one is rather complex and has many different element, that
could potentially affect the estimated results. There exist different
methods for artificial dataset generation, starting with use of
\emph{robots} (artificial observation instances) and ending with Markov
Chains Monte Carlo simulation and Neural Network use. One of the most
evident errors in this case could arise from the fact, that the closer
the estimated model is to the model implemented to generate the dataset,
the better would be the observed results, which may not be true in the
real world.

The second bias arises in the situation where the real world parameters
are used to generate artificial dataset, exactly as in this particular
work. The potential difference between the ideal simulated situation and
the real world situation lead to different choice structures. The
theoretical model supporting the data-generation process may be
potentially erroneous, leading to erroneous conclusions if only such
dataset was used for incorrect purpose.

\hypertarget{artificial-dataset-generation-procedure}{%
\subsubsection{Artificial dataset generation
procedure}\label{artificial-dataset-generation-procedure}}

For the objectives of this study we assume the best option is to
generate our own artificial dataset based on a predefined utility
function and given a predetermined statistical properties for individual
characteristics and alternatives' attributes. Such set-up ensures that
we know exactly the data generation process and have all the control
over the parameters and experimental design. As was mentioned above,
this choice may be dangerous in terms of justification of the resulting
external validity of obtained results in application to any other real
world dataset. However we ensure this way, that the obtained results
could be potentially compared with the baseline target parameters and
the initial effects are observed to us.

First step in the dataset generation is the generation of the
experimental design framework, imitating the original choice set-up, as
described in the Michaud, Llerena, and Joly (2012) article. Our first
steps are identical to the original work, as we start with the
generation of all possible combinations of binary factors for our
alternatives: roses described by two binary attributes and their price.
There exists only four different roses types, if described by their
binary attributes alone, as can be seen in the table \ref{tab:comb1}.

\begin{table}[!htbp] \centering 
 \caption{Possible attributes of roses} 
 \label{tab:comb1} 
\begin{tabular}{@{\extracolsep{5pt}}ccc} 
\\[-1.8ex]\hline 
\hline \\[-1.8ex] 
Type & \multicolumn{1}{c}{Eco-label} & \multicolumn{1}{c}{Carbon footprint} \\ 
\hline \\[-1.8ex] 
1 & 0 & 0 \\
2 & 0 & 1 \\
3 & 1 & 0 \\
4 & 1 & 1 \\
\hline \\[-1.8ex] 
\end{tabular} 
\end{table}

Given a multiple choice context when an individual is choosing among
three alternatives: two different roses, defined by labels A and B; and
a ``No buy'' option. Consequently there exist multiple possibilities to
regroup two roses into a choice set, for instance, in Michaud, Llerena,
and Joly (2012) are generating six choice sets ensuring that roses in a
given choice set always have different attributes, while in practice
there exist sixteen possible combination of two roses given they are
described by two binary factor variables. The choice of the choice set
delimitation in the article could be understood as the individuals
participating in the stated choice experiment are scarcely interested in
answering multiple questions, while six or twelve choices to consider
appear to be a reasonable number. On the contrary, our experimental
artificial set-up allows to ask as many questions to as many individuals
as we want. For example we can generate
\(7 \times 4 \times 7 \times 4 = 784\) choice sets for each individual,
containing all the possible combination of two different roses, each
described by two binary factor attributes as well as their price, which
has 7 different levels (varying by 0.50€ in a range from 1.50€ to
4.50€). However, such excessive set-up can have its toll on the
computation times, being in the same time absolutely unreasonable and
unrealistic, were we to replicate our results in a stated choice
experiment. Consequently, for price allocation we are going to implement
the same strategy as the authors of the article, meaning that the prices
will be randomly assigned inside the choice sets, while the choice sets
will follow a complete full-factorial design given two alternatives with
attributes. The following table \ref{tab:comb2} demonstrates this idea.

\begin{table}[!htbp] \centering 
 \caption{Choice sets attributes' combinations} 
 \label{tab:comb2} 
\begin{tabular}{@{\extracolsep{5pt}}ccccc} 
\\[-1.8ex]\hline 
\hline \\[-1.8ex] 
 & \multicolumn{2}{c}{Rose A} & \multicolumn{2}{c}{Rose B} \\ 
Choice set & \multicolumn{1}{c}{Eco-label} & \multicolumn{1}{c}{Carbon footprint} & \multicolumn{1}{c}{Eco-label} & \multicolumn{1}{c}{Carbon footprint} \\ 
\hline \\[-1.8ex] 
1 & 0 & 0 & 0 & 0 \\
2 & 0 & 0 & 0 & 1 \\
3 & 0 & 0 & 1 & 0 \\
\multicolumn{5}{c}{...} \\
15 & 1 & 1 & 1 & 0 \\
16 & 1 & 1 & 1 & 1 \\
\hline \\[-1.8ex]
\end{tabular} 
\end{table}

The prices are randomly allocated within given choice sets, although
there are some subtleties, which were discovered in attempt to replicate
the variability achieved in the original work. The main idea is to
ensure that both groups of roses (A and B) will have identical
characteristics, which is important for the later model estimation. At
the same time, we are interested in providing the test subjects with
identical choice sets to avoid eventual bias, which may be important if
we were facing a small number of observed individuals. Consequently, we
randomly allocate prices within a given choice set and distribute these
identical choice sets to all of the individuals. The variability in the
prices across alternatives is achieved through a replication of this
procedure \(n\) times. The resulting statistics and distribution will be
discussed in the second part of the work, where we will focus our
attention on the applied part.

On the next step we generate a population of ``\emph{robots}'', or
artificial individuals, who will be making their choices provided the
described above choice sets. It is important as well to mention, that
the distributions we use to generate the data are theoretical rather
than empirical ones. The individuals are generated based on the
descriptive statistics for population available in the reference paper.
This choice is done based on the final objective of the proposed testing
framework to allow the researchers to test and verify their hypothesis
related to the behavioural assumptions, modelling and performance
estimation in the consumer choice experimental context. We assume that
characteristics of the individuals are normally distributed, which is
rarely the case in practice, where skewed distributions are dominant.
Such choice imitates a replication attempt of a given empirical paper
given the information available in the article only, which are usually
the means and variances, rather than complete empirical distribution
descriptions.

Finally, having at our disposal a set of individuals as well as a number
of choice sets for the individuals to consider, we define the utility
function based on the estimates of the authors. Such choice implies,
that we assume all the hypothesis made when treating the original
dataset to be verified for the artificial model. The utility functions
are assumed as described in the preceding subsection to conform with the
standard random utility maximisation (RUM) definition as the individuals
are striving to maximise their perceived utility given their
characteristics and the observed attributes of the alternatives. The
utility is linear in parameters with additive error term.

Following this procedure we generate two synthetic datasets: one the
most basic one with only fixed effects present, while the other includes
random effects for the alternative specific attributes. These datasets
are then used to estimate, test and compare the models' performances.

To summarise, this section we will once again list the key hypothesis we
make in the artificial dataset creation:

\begin{itemize}
\tightlist
\item
  The dataset comprises:

  \begin{itemize}
  \tightlist
  \item
    4 individual characteristics (\(Sex\), \(Age\), \(Habit\) and
    \(Salary\))
  \item
    3 alternative's attributes (\(Price\), \(Label\) and \(Carbon\))
  \item
    2 product variables (\(Buy\) dummy variable and
    \(LC = Label \times Carbon\) cross-product)
  \end{itemize}
\item
  The individuals are assumed to maximise their utility, when making
  their choices, which corresponds to RUM behavioural framework;
\item
  The utility functions are linear, additive in parameters with an
  additive error term \(\epsilon\);
\item
  The error term is assumed to be iid. across population and follow a
  Gumble distribution: \(\epsilon \sim G(0, 1)\);
\item
  The individuals may (or may not) express heterogeneous preferences for
  the environmental attributes (eco-\(Label\) and \(Carbon\) footprint),
  which results in two different artificial datasets;
\item
  In the case of heterogeneous preferences a total of four random
  parameters are assumed to be correlated (\(Buy\) dummy, \(Label\),
  \(Carbon\) and their cross-product \(LC\)) and respect a multivariate
  normal distribution.
\end{itemize}

The detailed procedure of the choice modelling, as well as the exact
values of the parameters and some eventual difficulties in the dataset
generation are described in the applied section of this work.

\hypertarget{statistical-tools-for-choice-modelling}{%
\subsection{Statistical tools for choice
modelling}\label{statistical-tools-for-choice-modelling}}

As it was mentioned, there are different fields of application ranging
from \emph{econometrics} (Agresti 2013) to \emph{machine learning}
(Zielesny 2011), encompassing eventually such fields as transportation
systems analysis (Cascetta 2009) and logistics (De Palma et al. 2011),
actuarial science (Denuit and Trufin 2019), preference learning
(Fürnkranz and Hüllermeier 2010), psychology, sociology and more). The
more generalised models are regrouped under the \emph{statistical
models} label (Hastie, Tibshirani, and Friedman 2009), but nevertheless
they are mostly limited and are not taking into account many of the
field specific questions. Taking into account that our study is mostly
axed towards the study of the consumer choice data and related discrete
choice problems it is important to somehow limit the study's scope to a
number of selected models, without loosing the context.

Speaking about the econometrics models, this field of applied statistics
alone has a number of questions to answer before proceeding. For
example, we may question the particular task that we are performing
while applying the econometric models to some \emph{discrete choice}
problematic. Usually the economists are interested in deciphering and
understanding the underlying process (Athey and Imbens 2019), even
though there is a long lasting debate on the validity of obtained
measures as well as causality implications (Chen and Pearl 2013):
``\emph{The source of confusion surrounding econometric models stems
from the lack of a precise mathematical language to express causal
concepts.}'' This results in completely different cultures of the data
exploration and study objectives. This particular problem was largely
addressed by different researches, among which: Athey and Imbens (2019),
Mullainathan and Spiess (2017), Agrawal, Gans, and Goldfarb (2019),
Varian (2014) and Breiman and others (2001). Even as there are some
attempts to merge all the existing branches and approaches to
statistical modelling into some sort of a uniform culture (Donoho 2017),
the scientific community has a long route to make in order to achieve
this objective. There exist as well many other more subtle problems in
the econometric field. For example, different error term and different
link function specifications (Bouscasse, Joly, and Peyhardi 2019) in
econometrics models rise the question of what exactly we may consider as
single \emph{entry} to our list of models to evaluate.

On the other hand, speaking about the ML counterpart, the focus is
generally made on the predictive precision if we were to focus our
attention on the supervised ML sub-field (Mullainathan and Spiess 2017).
In their quest to achieve the best predictive precision with a
particular model, the \emph{machine learning} scientists study not only
the theoretical models themselves, but the algorithms used to estimate
these models (Zielesny 2011), that potentially augments the dimensions
to take into consideration in this particular work. Moreover, not only
there exist a confusion on what algorithms are to be associated with
each particular model (or potentially a number of models defined by
model/algorithm pairs), but many models are specified using a set of
hyper-parameters, which are to be chosen by the researcher. This aspect
immensely complexifies the task for us, as it is uncertain how exactly
should we define the values of these arbitrary chosen parameters. It's
worth mentioning that in many cases these parameters are case specific
and may vary from one application to another, resulting in different
performances over different datasets.

As it is mentioned by Kotsiantis, Zaharakis, and Pintelas (2006) the
choice of which specific learning algorithm to be implemented is a
critical step for any work, and a separate subset of training dataset is
usually used for this task. The classifier's evaluation is most often
based on prediction accuracy, which describes the percentage of correct
predictions among their total number, which requires some unrelated data
to be calculated as out of sample estimates provide more reliable
information about the performance of a particular algorithm.

This section will be opened by a brief introduction to the multitude of
the existing models, which is a particularly important point, given the
scope of the study. Each and every dataset, each and every relationship
between several variables may be modelled with different techniques and
different assumptions. There is a tremendous amount of work to be done
in order to systematise all the existing mathematical models, not
speaking about their extensions or their numerical implementations. The
first part of this section will demonstrate the complexity of the
models' choice given an application context. Only then, we are going to
present the selected models and their mathematical formulation: the MNL
model, the MMNL model and their artificial NN counterpart.

\hypertarget{taxonomy-of-statistical-models}{%
\subsubsection{Taxonomy of statistical
models}\label{taxonomy-of-statistical-models}}

Before proceeding with a discussion concerning eventual problems and
difficulties affecting the modelling part of every empirical study, we
will provide an overview of different families of models, encompassing
both the \emph{econometrics} and \emph{machine learning} fields. The
following presentation is a generalised vision of the existing discrete
modelling techniques, which can be used for classification tasks. As
general as it is, this part respect the setting of the discrete choice
behavioural modelling.

There exist several possibilities to divide ML algorithms into groups in
order to provide an exhaustive and complete taxonomy of this field and
the same reasoning may be applied to econometric models. However, the
existing taxonomies are rarely complete and focus mostly on one or
several grouping aspects. They define the general structure of a
particular taxonomy, but rarely take into account a sufficient number of
different descriptive features, which may vary across statistical
models. For example, we may take a look at Kotsiantis, Zaharakis, and
Pintelas (2006) work attempting to provide an overview of different
classification techniques on figure \ref{fig:kots}.

\begin{figure}
\centering
\caption{Taxonomy as proposed by Kotsiantis (2006)}
\label{fig:kots}
\begin{forest}
  for tree={
    align=center,
    edge+={ -{Stealth[]}},
    l sep'+=10pt,
    fork sep'=10pt,
  },
  forked edges,
  if level=0{
    inner xsep=0pt,
    tikz={\draw (.children first) -- (.children last);}
  }{},
  [Machine Learning 
    [Unsupervised]
    [Supervised
      [Logic-base\\techniques
        [Set of\\rules]
        [Inductive\\logic]
        [Decision\\trees]
      ]
      [Support Vector\\Machines]
      [Statistical\\techniques 
        [Neural\\Networks]
        [Instance-based\\learning]
        [Naive Bayesian\\networks]
      ]
    ]
  ]
\end{forest}
\end{figure}

This taxonomy is fairly simple and encompasses a large number of models'
families specifically designed for classification.\\
In the works of Hastie, Tibshirani, and Friedman (2009), Cascetta (2009)
and Ayodele (2010) we may see some more recent attempts to organise the
existing models into a single hierarchically related structure, although
neither of known to the author works offers sufficiently extended
reasoning over the relations between different classification techniques
(several of the resulting taxonomies could be seen in the Appendix A).
Moreover, not only the taxonomies may be based on the models'
themselves, but it can be constructed around their algorithmic
properties, as in Mullainathan and Spiess (2017). The resulting tree is
represented on the figure \ref{fig:mull}.

\begin{figure}
\centering
\caption{Taxonomy as proposed by Mullainathan (2017)}
\label{fig:mull}
\begin{forest}
  for tree={
    align=center,
    edge+={ -{Stealth[]}},
    l sep'+=10pt,
    fork sep'=10pt,
  },
  forked edges,
  if level=0{
    inner xsep=0pt,
    tikz={\draw (.children first) -- (.children last);}
  }{},
  [Machine Learning 
    [Combined\\predictors
      [Bagging]
      [Ensemble]
      [Boosting]
    ]
    [[Local/Nonparametric\\predictors
      [Kernel\\regression]
      [Decision\\trees]
      [Nearest\\neighbhors]
      [Random\\forest]
    ]]
    [Mixed\\predictors
      [Splines]
      [Neural\\Networks]
    ]
    [Global/Parametric\\predictors
      [Linear predictors\\(and generalizations)] 
    ]
  ]
\end{forest}
\end{figure}

In attempt to generalize the existing taxonomies and unite somehow the
different classification models and techniques, we may roughly divide
them in categories by different criteria. Usually there is no evident
hierarchical dependency between the different criteria, which immensely
complexifies the task of unified taxonomy construction.

First of all we may divide the models onto \emph{supervised} and
\emph{unsupervised} learning techniques (Hastie, Tibshirani, and
Friedman 2009), which is the most widely used model separation in ML
field. Sometimes this separation is complimented by various intermediate
combinations of these two. The supervised methods have the goal to
predict the value of an outcome measure based on a number of given input
measures, the outcome variable is available through the learning process
to guide the researcher and algorithm providing some baseline for
testing. In the statistical literature the inputs are often called the
predictors, the inputs, the features, or the independent variables. In
the econometrics the terms explicative or endogenous variables are more
popular. The outputs are denominated as responses, or, in econometrics,
the dependent or endogenous variables. The unsupervised learning is used
without any outcome measure available, with a main objective being to
describe the associations and patterns among a set of inputs. Such
formulation of a learning task is rather implemented to describe how the
data is organized or clustered, find the underlying patterns and
dependencies. As for the intermediate models' families, we may address
the article of Ayodele (2010), where authors present different mixed
types of learning tasks, although this particular classification is not
widely used. Among these models we find: \emph{semi-supervised}
learning, combining both labelled and unlabelled examples to generate an
appropriate function or classifier; \emph{reinforcement} learning, in
which algorithm learns to interact with the data generating source,
given an observation of the world, in this context every action of model
has some impact in the environment, and the environment provides
feedback that guides the learning algorithm; The \emph{transduction} is
nearly identical to supervised learning, although instead of an attempt
to construct a function it tries to predict new outputs based on
training inputs, training outputs, and new inputs; and finally
\emph{learning to learn}, when the algorithm learns its own inductive
bias based on previous experience, which is a more advanced
reinforcement learning problem.

Depending on the output variable structure we attempt to model we may
examine the taxonomy proposed by Agresti (2013). This taxonomy is based
on the output variable format: it may be either discrete or continuous.
The \emph{continuous} variables are the simplest case, where the output
is assumed to be continuous on a given interval and in the statistical
society is usually addressed as ``\emph{regression}'' task. It's
counterpart, the discrete dependent variable is sometimes addressed as
``\emph{classification}'' task and it is the focus of this particular
work. The \emph{categorical} variable has a measurement scale consisting
of a set of categories and these variables are of many types: binary
variables, nominal data, ordinal data or count variables. The
\emph{binary} data assumes that there exist only two categories, often
given the generic labels ``success'' and ``failure'' numerically
represented as 0 and 1. In the context of the undertaken study we may
imagine a binary variable representing the individual choice of ``Buy''
against ``No buy''. The \emph{nominal} variables represent categories
without a natural ordering and are measured on a nominal scale. The
perfect example for this data type is our choice set delimitation with
several unordered and independent options for individuals to consider:
buy rose A, buy rose B or do not buy anything. For nominal variables,
the order of listing the categories is irrelevant to the statistical
analysis, and the main importance is given by the choice of baseline
option, which is important for some of the statistical models.
\emph{Ordinal} data or ordered discrete data is an advanced
representation for nominal data, where many categorical variables do
have ordered categories, representing some given preferences order, for
example. For these variables, the distances between categories are
usually unknown and these intervals may be uneven between different
categories. An \emph{interval} variable is one that does have numerical
distances between any two values. For most variables of this type, it is
possible to compare two values by their ratio, in which case the
variable is also called a ratio variable. The final class if the
\emph{count} data, which is specific for special cases of
discrete-continuous data treatment.

By their structure the models may be separated into \emph{additive} and
\emph{non-additive} as described in Hastie, Tibshirani, and Friedman
(2009), both of which could be understood either as additive
(non-additive) in error term or having a full additive (non-additive)
structure. The first group encompasses different regression and
classification models where either the main function has additive
structure:

\begin{equation}
f(X) = E(Y \mid X)
\end{equation}

Or the error term is additive defining the following model:

\begin{equation}
Y = f(X) + \epsilon
\end{equation}

The \emph{non-additive} models, also denominated as
\emph{multiplicative} models, include all other eventual specifications
which could not be viewed or approximated by the additive relations.
This particular separation could be extended even further, as the models
could be viewed as \emph{linear} and \emph{non-linear} in their
parameters, or in their overall functional form. The former either
assume that the regression function \(E(Y \mid X)\) is linear, or that
the linear model is a reasonable approximation for the particular
situation. The non-linear models usually regroup the various extensions
and generalisations for the linear models integrating various non-linear
transformations.

One more possibility to separate different discrete choice models in
particular is by taking into account the probability structure they are
attempting to model as mentioned in Jebara (2004). The models are
separated into two major groups: generative and discriminative models,
to which sometimes a third ambiguous group of non-model techniques is
added. The \emph{generative} algorithms model the full structured joint
probability distribution over the examples and the labels given by
\(P(Y, X)\). The models in this context are typically cast in the
language of graphical models such as Bayesian networks. The joint
distribution modelling offers several attractive features such as the
ability to deal effectively with missing values, for example. On the
other hand, the \emph{discriminative} methods such as support vector
machines or boosting algorithms focus only on the conditional relation
of a label given the example, the probability being written as
\(P(Y \mid X)\). Their parametrized decision boundaries are optimized
directly according to the classification objective, encouraging a large
margin separation of the classes. They often lead to robust and highly
accurate classifiers.

The estimates structure differs across model families as well, as
described in Hastie, Tibshirani, and Friedman (2009). There are two
principal approaches to modelling given by \emph{parametric} estimators,
which are usually easy to read and interpret, and their
\emph{non-parametric} counterpart, offering the best results in terms of
precision in most cases. The multitude of non-parametric regression
techniques or learning methods can be separated into a number of classes
by the nature of the restrictions imposed, although we are not going to
provide an extensive description of all of them. What is more important,
that there exist different families of mixed models, profiting from both
the parametric and non-parametric feature. They are traditionally
regrouped into a single family of \emph{semi-parametric} models.

In this work we face a classification task which can be understood,
given the context, as consumer choice modelling. In order to correctly
model the consumer choice structure we will need to use the models
allowing to work with nominal discrete data, because the consumer
choices can not be positioned in some logical order defining a
continuous variable. The desire to obtain some explanatory results leads
us to restrict our choice to some additive and, moreover, linear models,
which would identify the parameters of a given relative utility
function. The latter argument implies that the models should be
parametric, producing some exact estimates for given set of parameters.

\hypertarget{description-of-models-to-be-compared}{%
\subsubsection{Description of models to be
compared}\label{description-of-models-to-be-compared}}

For our particular demonstrative task, which is restricted by the
context of the study of Michaud, Llerena, and Joly (2012), we have
already described the advantages and reasons behind the unrelenting
theoretical assumptions concerning the behaviour of individual, as well
as the dataset generation procedure. The two resulting datasets allow us
to explore the effects of the random effects of the alternatives'
attributes on the modelling. This possibility is particularly important,
as usually researchers ignore the possibility of random effects presence
in the population and use more simple and conventional multinomial
logistic models to model various discrete choice situations. However, we
are not going to test only one model over the obtained dataset, but
rather introduce several models with different specifications in order
to demonstrate a vast potential of our testing framework and its
advantages for research.

As we are exploring an over-simplified framework, we are going to study
first two different traditional models each perfectly adapted to model
one of the two generated datasets respectively. We are speaking about
the multinomial logistic regression, which should yield perfect fit
results on a fixed effects dataset and its counterpart - the mixed
multinomial logistic regression, which should be the most performant in
the presence of random effects in the utility functions. Many of the
existing applied econometrics papers use the most simple specification
of the Multinomial Logistic Regression (MNL), that may lead to erroneous
results and conclusions in the presence of random coefficients.
Eventually these models will allow us to verify, whether or not we are
able to obtain the same results as at the input.

What is more, as the main objective of this work is to demonstrate
proposed framework's flexibility, we are going to show how a completely
alien model to econometrics, such as neural networks model, may be
explored and compared with more traditional tools. More precisely, we
are going to use a neural networks imitating the procedure of the
multinomial logistic regression, while the other will be more
traditional multilayer neural network. It is because this model can be
viewed as an even wider generalisation of the generalised additive
models (GAM), that it is possible to simulate a model similar to MNL and
MMNL models. This choice was made because the seemingly identical model
by its structure may produce different results, depending on the
implemented estimation technique. The NN techniques offer us a great
number of different algorithms which are more advanced than the
algorithms traditionally implemented in econometrics, which make us
wonder, whether the changes in the estimation algorithm will allow us to
achieve better results.

In this part we will attempt as well to introduce some common notation
for the different models' families, issued from different disciplines.

\hypertarget{logistic-regressions}{%
\paragraph{Logistic regressions}\label{logistic-regressions}}

Multi-category logit models simultaneously use all pairs of categories
by specifying the odds of outcome in one category instead of another
(Agresti 2007). As described in Agresti (2013), many applications of
multinomial logit models relate to determining effects of explanatory
variables on a subject's choice from a discrete set of options.

\textbf{Multinomial Logit}

Even if in the original article of Michaud, Llerena, and Joly (2012) a
Mixed Logit model is used, here we start our study with an introduction
of the multinomial logistic regression (MNL) model, assuming the fixed
effects presence. This model will allow us to contrast the performances
in case of both fixed and random effect theoretical assumptions and
compare them with a more advanced version of mixed multinomial logistic
regression and NN model. This assumption is relaxed in the Mixed Logit
model (ML or MMNL), where coefficients (or some of them) vary by
individual (Agresti 2013). The logistic regression models are derived
from GLM specifications (Agresti 2007):

\begin{equation}
g(\mu_i) = \sum_r \beta_r x_{ir}
\end{equation}

Where \(g(.)\) is a link function, which is a logistic transformation
for binary logistic model. It is important to say that in this
theoretical introduction we ignore in some extent the previously
introduced terminology: \(i\) still denotes the individual observations,
laying in range of \(\{1, \dots, N \}\) in this case; the \(r\) index
here stands for different variables, because we do not use matrix
notation for the reasons of simplicity.

Here we propose the econometric specification of a \emph{multinomial
logit (MNL)} model as described by Cascetta (2009). The MNL model is one
of the simplest \emph{random utility model (RUM)} (McFadden 1974). This
class of models relies on the hypothesis, that an individual \(n\)
maximises his perceived utility over a set of alternatives \(\Omega\),
his utility determined by a fixed and a random parts, as described
earlier:

\begin{equation}
U_{ij} = V_{ij} + \eta_{ij} \text{ where } V_{ij} = \alpha_j + \beta_j X_i + \gamma Z_j
\end{equation}

Both \(\beta\), representing the alternative specific individual
coefficients, and \(\gamma\), standing for population-wide attributes
effects, are assumed to be fixed across population, meaning that all the
individuals have identical preferences and are subject to identical
effects. As precise in Agresti (2013) this approach enables
discrete-choice models to contain characteristics of the chooser and of
the choices. It offers the model an immense flexibility. The MNL is
based on the assumption that the residuals \(\eta_{ij}\) are identically
and independently distributed (iid.) as Gumbel random variables with
zero mean and scale parameter \(\theta\), which is usually equal to 1
(\(\theta = 1\)). This calibration is done due to computational reasons,
which will be explained later in this part.

One of the key concepts when it comes to modelling of the described
above process is the \emph{latent variable} notion. The latent variable
\(Y\) corresponds to its more meaningful counterpart \(V\) and is
sometimes understood as probability to choose a particular alternative.
Obviously, as in the experimental context we are unable to observe the
real choice probabilities, this variable takes values 0 or 1 depending
on whether or not a particular alternative was chosen:

\begin{equation}
Y_ij = I(V_{ij} > V_{il} | j \neq l, \forall l \in \Omega_i)
\end{equation}

Under the assumptions made here, the probability of choosing alternative
\(\omega_j\) from among those available
\(\{\omega_1, \dots, \omega_k\} \in \Omega\) by individual \(i\), can be
expressed in closed form as:

\begin{equation}
P_{ij} = \frac{
    e^{V_{ij} / \theta}
}{
    \sum_{l = 1}^{k} e^{V_{il} / \theta}
}
\end{equation}

The probability structure incorporates the theoretical assumptions of
the finite choice set, the uniqueness of the chosen alternative and the
idea of utility maximisation. In a more comprehensive form, we may say
that an individual chooses a particular alternative \(\omega_j\) or
simply \(j\) among all available for him alternatives \(\Omega_i\) only
if its utility is higher than any others' alternative utility:

\begin{equation}
P_{ij} = P(\eta_{il} - \eta_{ij} < V_{ij} - V_{il}) 
    \forall l : l \neq j, l \in \Omega_i
\end{equation}

Knowing the structure of \(V_{ij}\) and assuming the \(\theta\)
parameter for Gumble distribution of \(\eta\) is \(1\) we may rewrite
the probability as:

\begin{equation}
P_{ij} = \frac{
    e^{\alpha_j + \beta_j X_i + \gamma Z_j}
}{
    \sum_{l = 1}^{k} e^{\alpha_l + \beta_l X_i + \gamma Z_l}
}
\end{equation}

The alternative \(\omega_j\) in such case is denoted as reference
alternative or baseline alternative and is subject to several
restriction for the sake of identifiability. The most important one is
that we can not identify all the parameters in the probability function,
which require us to impose some restrictions over effects structure.
Traditionally (Agresti 2013) the reference level coefficients are
assumed to be 0, reducing this way the number of parameters to estimate.
This choice has some important consequences for the models'
interpretation, because the estimated effects for other alternatives in
this case should be treated as differences between the actual effects
for the baseline alternative and other alternative respectively. The
estimated parameters are in fact:

\begin{equation}
V_{ij} - V_{il} = (\alpha_j + \beta_j X_i + \gamma Z_j) - (\alpha_l + \beta_l X_i + \gamma Z_l)
\end{equation}

Where \(l \neq j\) and \(j, l \in \Omega_i\). Which could be transformed
into:

\begin{equation}
V_{ij} - V_{il} = (\alpha_j - \alpha_l) + (\beta_j - \beta_l) X_i + \gamma (Z_j - Z_l)
\end{equation}

At this stage an important remark should be made, which concerns the
understanding of individual characteristic effects and alternatives'
attributes effects. It is theoretically possible to estimate a common
individual effect for all the alternatives should we only wish to. The
main idea lies in the correct parametrisation of the initial framework.
To achieve identifiability for the individual characteristic specific
effects we should observe enough within choice set variance, as
otherwise the resulting singularity will incapacitate us to perform the
estimation. In other words, we can understand this procedure as manually
setting the individual effects to 0 for our baseline alternative and
estimating the resulting model. Speaking about the changes in the
dataset, the described above procedure is strictly equivalent to setting
the baseline alternative's individual characteristics vector to zeros
and estimating the resulting feature matrix as alternative specific
attributes.

The traditional vision of alternative specific individual
characteristics effects, assuming \(\beta_j = 0\), is:

\begin{equation}
(\beta_j - \beta_l) X_i = - \beta_l X_i \text{ if } \beta_j = 0
\end{equation}

The analogous vision for alternatives' attributes effects, when
reference attribute \(Z_j\) is set to 0 is:

\begin{equation}
\gamma (Z_j - Z_l) = - \gamma Z_l \text{ if } Z_j = 0
\end{equation}

As we can see \(\beta_l\) and \(\gamma\) parameters are roughly
equivalent in these two cases, assuming we are interested in means over
the set of individuals \(N\) and alternatives \(\Omega\).

\begin{equation}
E_{il} (- \beta_l X_i) = E_{il} (- \gamma Z_l) \forall i \in N, \forall l \in \Omega
\end{equation}

Which under transformation equals to:

\begin{equation}
- E_{l} (\beta_l) E_{i} (X_i) = - \gamma E_{l} (Z_l)
\end{equation}

Assuming \(X\) and \(Z\) here is the same variable, varying across
individuals and characteristics (\(Z_j = 0\)), we obtain that:

\begin{equation}
- E_{l} (\beta_l) X = - \gamma Z \Rightarrow E_{l} (\beta_l) = \gamma
\end{equation}

This could be empirically confirmed through estimation of two different
specifications and aggregation of obtained results.

However, were we in need to estimate an individual for all the
alternatives except the baseline one, we could benefit from this
transformation to do so. Such transformation allows us to take the
multiple choice context of the expiremental setup.

\textbf{Mixed Multinomial Logit}

Following Agresti (2007) presentation, generalized linear models (GLMs)
extend ordinary regression by allowing non-normal responses and a link
function of the mean. The generalized linear mixed model, denoted by
GLMM, is a further extension that permits random effects as well as
fixed effects in the linear predictor. We begin with the most common
case, in which is an intercept term in the model.

\begin{equation}
g(\mu_i) = \sum_r \beta_{ir} x_{ir}
\end{equation}

Where \(\beta_i\) is issued from some multivariate distribution.
Traditionally this distribution is assumed to be a multivariate normal
distribution (MNV) giving:

\begin{equation}
\beta_i \sim MNV(\beta, \Sigma)
\end{equation}

In more recent work of Agresti (2013) the more advanced models are
described. The multinomial logit and probability based discrete-choice
models can be further generalized by treating certain effects as random
rather than fixed.\\
A mixed logit model is the one in which choice probabilities are
obtained by integrating the logistic expression for choice probabilities
with respect to a distribution for certain model parameters. This allows
heterogeneity among subjects in the size of effects. It is useful as a
mechanism for inducing positive association among repeated responses
with panel data. Estimates of the parameters of the mixing distribution
provide information about the average effects and the extent of the
heterogeneity. Individual effects can also be predicted using this
technique.

The Mixed Logit is a further development and generalisation of a
traditional MNL and Conditional Logit models, because both of these
models may be constructed using Mixed Logit specification with a correct
parametrisation. The main difference from the more simple models is that
in this case it is assumed that effects vary across population and might
even be correlated. The utility specification in this case is
constructed identically to simple models, but the deterministic part
assumes that effects vary across population:

\begin{equation}
U_{ij} = V_{ij} + \eta_{ij} \text{ where } V_{ij} = \alpha_j + \beta_j X_i + \gamma_i Z_j
\end{equation}

Mathematically the random effects specification is achieved through the
parameter vector \(\gamma_i\), which is unobserved for each \(i\). The
\(\gamma\) in this case is assumed to vary in the population following
the continuous density \(f(\gamma_i \mid \theta)\), where \(\theta\) are
the parameters of this distribution. The simplest choice of the
distribution for the random effects is the normal distribution, which
was used by Michaud, Llerena, and Joly (2012), or more precisely a
multivariate normal distribution, because authors took into account the
correlation between coefficients:

\begin{equation}
\gamma_i \sim MVN(\gamma, \Sigma)
\end{equation}

In this case the vector of alternative specific effects can be
represented as:

\begin{equation}
\gamma_i = \gamma + L \sigma_i
\end{equation}

Where \(\sigma_i \sim N(0, I)\) , and \(L\) is the lower-triangular
Cholesky factor of \(\Sigma\) knowing which, the actual
variance-covariance matrix for random effects can be derived, as
presented in ({\textbf{???}}):

\begin{equation}
LL^T = V(\gamma_i) = \Sigma
\end{equation}

Here we do not present the eventual possibility to incorporate the
individual specific characteristics covariates into the given framework,
because we will not use it, but such possibility is definitely worth
mentioning.

Where \(\beta\) are some fixed mean effects across population and
\(\psi\) stand for the random part with \(0\) mean and some imposed
variance-covariance structure, as it is technically possible to assume
that only some of the effects are random.

A more advanced description of MMNL models is available in the work of
({\textbf{???}}), where some intuitions are given on the estimation
techniques necessary to evaluate such complex model. The authors
suggest, that numerical integration or approximation by simulation is
needed to evaluate MMNL probabilities. Maximum Simulated Likelihood
(MSLE) or Method of Simulated Moments (MSM) could be used to estimate
the MMNL model in practice, both of which are described in the reference
work ({\textbf{???}})

\hypertarget{neural-networks}{%
\paragraph{Neural Networks}\label{neural-networks}}

The second group of models focuses on more advanced and atypical
modelling techniques rarely implemented by the economists in their
studies, as usually this family is perceived as not offering enough
insight when it comes to the effects estimation. The ML techniques are
usually viewed by economists as some black boxes, which do not provide
any information about the underlying process. It is quite easy to comply
with their position, as even though the most advanced techniques perform
better in terms of predictive power, they rarely offer any insight into
the modelling process.

For this particular part we use the model's specifications described in
the handbook of Hastie, Tibshirani, and Friedman (2009) with some
additions and modifications, which aim at integration of this particular
specification in conformity with the specifications of the econometric
discrete model notation. \emph{Neural Networks (NN)} represent an
advanced class of models, being a further complexification of the
\emph{generalised additive models (GAM)}, which are a generalisation of
the \emph{generalised linear models (GLM)}, which was defined in
previous subsection. This GLM is generalised through assumption that
each explicative variable in \(X\) can undergo some transformation,
linear or not, resulting in a following GAM model:

\begin{equation}
g(\mu_i) = \sum_r s_r(x_{ir})
\end{equation}

Where \(s_r(.)\) is an unspecified smooth function of predictor
\(x_{ir}\). In order to obtain a NN model, this structure is further
developed as follows to obtain firstly a \emph{projection pursuit
regression (PPR)}:

\begin{equation}
f(X) = \sum_{r = m}^{M} g_m (\omega_{m}^{T} X)
\end{equation}

The \(X\) in this notation is a vector of inputs with \(p\) components,
and \(\omega_{m}\) with \(m \in \{1, 2, \dots, M \}\) are unit
\(p\)-vectors of unknown parameters. Before proceeding, we will
introduce some novelties to the notation used till this point by
introducing vectors \(X1\), \(X2\), \(\dots\), \(XS\), where \(X1\) is
the output of the first layer of neural network, each element of which
is some transformation (usually linear in parameters with some
``activation'' function) of the input vector \(X\). Then the simplest NN
for \(\Omega\) alternatives (classes) classification, with two layers,
may be represented as:

\begin{equation}
f_j (X) = g_j (X2) \text{ with } X2_j = \psi_{0j} + \psi_{k}^{T} X1
\end{equation}

Where \(f_j\) models the probability of a class \(j\), or in more
comprehensive language the probability that a given individual will
choose an alternative \(\omega_j\) from his choice set \(\Omega_i\):

\begin{equation}
X1_m = \sigma(\phi_{0m} + \phi_{m}^{T} X)
\end{equation}

While \(\sigma(.)\) is an activation function and \(g_k(.)\) a
probability transformation function, traditionally a \emph{softmax}
function. The latter is being used as well in \emph{multinomial logit
(MNL)} models:

\begin{equation}
g_j(T) = \frac{e^{T_j}}{\sum_{l = 1}^{\Omega} e^{T_l}} \text{ where } j,l \in \Omega
\end{equation}

This means, that single level NN with a softmax activation layer should
be identical to simple MNL model with all the coefficients varying by
alternatives. \(Z_m\) can be viewed as a basis expansion of the original
inputs \(X\) and the neural network is then a standard \emph{linear
multinomial logit (MNL)} model, using the transformations as inputs.

One of the supposed major problems for NN models in discrete choice
context is the inability to take into account all the influencing
factors across all the alternatives. Moreover, in this case study there
is major drawback in the ambiguity among choices A and B, as they are
interchangeable.

As we desire to obtain the effects assuming the alternatives A and B are
identical, this means that we should impose some additional restrictions
over the model. Traditional Multinomial Logistic regression (MNL) can be
potentially transcribed into a NN using convolution techniques. The
convolution layer operates iteratively on a given subset from the input
vector, calculating one single output per \(k\) inputs. In this case
\(k\) is denoted \emph{kernel size}. Another parameter, which defines a
convolutional layer is the \emph{stride} (\(s\)), which determines how
the ``window'' determined by kernel size should be moved over the input
layer. Consequently, the output layer consists of \(m\) values
determined as:

\begin{equation}
m = \frac{n - k}{s} + 1
\end{equation}

Where \(n\) is the length of the input vector to this layer. We may
attempt to define a convolution layer with linear activation function as
follows, assuming \(X = X_1, \dots, X_n\) is the input vector and
\(X1_1, \dots, X1_m\) is the output vector, while
\(\phi = \phi_1, \dots, \phi_k\) is the vector of weights:

\begin{align}
X1_1 = & \phi_1 X_1 + \phi_2 X_2 + \dots + \phi_k X_k \nonumber \\
& \vdots \\
X1_m = & \phi_1 X_{n-k} + \dots + \phi_k X_n \nonumber
\end{align}

The designed this way CNN consists of two transformation layers. The
first one is 1D convolutional layer with linear activation function,
which takes as input the dataset in ``wide'' format with 27 variables
overall (9 variables for each alternative), which produces a single
value as an output value for each individual for each choice set,
resulting in 3 output values in total. The second layer is a restricted
softmax transformation layer, which directly applies softmax
transformation over the inputs, without any supplementary permutations.

The vector of inputs issued from the dataset transformed into the
``wide'' format can be represented as:

\begin{multline}
X_i = Buy_{i,A}, Sex_{i,A}, Age_{i,A}, \dots, Habit_{i,C}, Price_{i,C}, Label_{i,C}, Carbon_{i,C}, LC_{i,C}
\end{multline}

Where all values with \(C\) index are set to zero in order to set the
baseline alternative. The first convolutional layer can be written as:

\begin{multline}
V_j = \alpha_{Buy} Buy_{ij} + \beta_{Sex} Sex_{ij} + \beta_{Age} Age_{ij} + \beta_{Income} Income_{ij} + \beta_{Habit} Habit_{ij} + \\
+ \gamma_{Price} Price_{ij} + \gamma_{Label} Label_{ij} + \gamma_{Carbon} Carbon_{ij} + \gamma_{Label \times Carbon} Label \times Carbon_{ij}
\end{multline}

Where \(j \in \{A, B, C\}\), with \(C\) denoting the ``No buy'' option.

We configure the convolution layer with linear activation function to
move across the input vector with strides 9, producing this way a vector
of length 3 as an output. This outputs of this layer may be interpreted
as utilities for each alternative respectively, identically to MNL
regression. The resulting design for a single convolution fold can be
schematically represented as in figure \ref{fig:convl}.

\begin{figure}[!htbp] \centering 
 \caption{Convolution layer} 
 \label{fig:convl} 
\begin{tikzpicture}[
    plain/.style={
        draw = none,
        fill = none,
    },
    net/.style={
        matrix of nodes,
        nodes={
            draw,
            circle,
            inner sep = 10pt
        },
        nodes in empty cells,
        column sep = 1.5cm,
        row sep = -9pt
    },
    >=latex
]

\matrix[net] (mat)
{
    |[plain]| \parbox{3cm}{\centering Convolution} & 
        |[plain]| \parbox{3cm}{\centering Deterministic Utility\\proxy} \\
    & |[plain]| \\
    |[plain]| & |[plain]| \\
    & |[plain]| \\
    |[plain]| \vdots & \\
    & |[plain]| \\
    |[plain]| & |[plain]| \\
    & |[plain]| \\
};


\draw[<-] (mat-2-1) -- node[above] {$Price_j$} +(-2cm,0);
\draw[<-] (mat-4-1) -- node[above] {$Label_j$} +(-2cm,0);
\draw[<-] (mat-6-1) -- node[above] {$Sex_j$} +(-2cm,0);
\draw[<-] (mat-8-1) -- node[above] {$Age_j$} +(-2cm,0);

\foreach \ai in {2,4,6,8}
    \draw[->] (mat-\ai-1) -- (mat-5-2);

\draw[->] (mat-5-2) -- node[above] {$V_j$} +(2cm,0);

\end{tikzpicture}
\end{figure}

The second transformation layer is a dense layer with a ``softmax''
activation function as described above, which has 3 coefficients for
each output, because it aggregates the inputs to an identical number of
outputs rescaling them in the process and producing choice
probabilities. Taking a set of \(V_A, V_B, V_C\) for inputs and
producing a vector of probabilities \(P(A), P(B), P(C)\) as outputs. The
second level may be synthetized as presented in figure \ref{fig:softl}.

\begin{figure}[!htbp] \centering 
 \caption{Softmax Layer} 
 \label{fig:softl} 
\begin{tikzpicture}[
    plain/.style={
        draw = none,
        fill = none,
    },
    net/.style={
        matrix of nodes,
        nodes={
            draw,
            circle,
            inner sep = 10pt
        },
        nodes in empty cells,
        column sep = 1.5cm,
        row sep = -9pt
    },
    >=latex
]

\matrix[net] (mat)
{
    |[plain]| \parbox{3cm}{\centering Deterministic Utility\\proxy} & 
        |[plain]| \parbox{3cm}{\centering Probability} \\
     & \\
    |[plain]| & |[plain]| \\
     & \\
    |[plain]| & |[plain]| \\
     & \\
};

\draw[<-] (mat-2-1) -- node[above] {$V_A$} +(-2cm,0);
\draw[<-] (mat-4-1) -- node[above] {$V_B$} +(-2cm,0);
\draw[<-] (mat-6-1) -- node[above] {$V_C$} +(-2cm,0);

\foreach \ai in {2,4,6}
    {\foreach \aii in {2,4,6}
        \draw[->] (mat-\ai-1) -- (mat-\aii-2);
    }

\draw[->] (mat-2-2) -- node[above] {$P_A$} +(2cm,0);
\draw[->] (mat-4-2) -- node[above] {$P_B$} +(2cm,0);
\draw[->] (mat-6-2) -- node[above] {$P_C$} +(2cm,0);

\end{tikzpicture}
\end{figure}

Finally, given the combination of these two layer we may construct the
whole CNN model. We may use the following graphical representation,
shown on figure \ref{fig:cnn} to visualise the resulting CNN
architecture:

\FloatBarrier

\begin{figure}[!htbp] \centering 
 \caption{Convolution Neural Network design} 
 \label{fig:cnn} 
\begin{tikzpicture}[
    plain/.style={
        draw = none,
        fill = none,
    },
    net/.style={
        matrix of nodes,
        nodes={
            draw,
            circle,
            inner sep = 10pt
        },
        nodes in empty cells,
        column sep = 1.5cm,
        row sep = -9pt
    },
    >=latex
]

\matrix[net] (mat)
{
    |[plain]| \parbox{1.3cm}{\centering Input\\layer} & 
        |[plain]| \parbox{1.3cm}{\centering Convolution\\layer} & 
        |[plain]| \parbox{1.3cm}{\centering Probability\\layer} & 
        |[plain]| \parbox{1.3cm}{\centering Output\\layer} \\
    & $V_A$ & $P(A)$ & |[plain]| \\
        |[plain]| & |[plain]| & |[plain]| \\
    & $V_B$ & $P(B)$ & \\
        |[plain]| & |[plain]| & |[plain]| \\
    & $V_C$ & $P(C)$ & |[plain]| \\
};


\draw[<-] (mat-2-1) -- node[above] {Rose A} +(-2.5cm,0);
\draw[<-] (mat-4-1) -- node[above] {Rose B} +(-2.5cm,0);
\draw[<-] (mat-6-1) -- node[above] {No buy (C)} +(-2.5cm,0);
\foreach \ai in {2,4,6}
    \draw[->] (mat-\ai-1) -- (mat-\ai-2);
\foreach \ai in {2,4,6}
    {\foreach \aii in {2,4,6}
        \draw[->] (mat-\ai-2) -- (mat-\aii-3);
    }
\foreach \ai in {2,4,6}
    \draw[->] (mat-\ai-3) -- (mat-4-4);
\draw[->] (mat-4-4) -- node[above] {Choice} +(2cm,0);

\end{tikzpicture}
\end{figure}

The figure \ref{fig:cnn} is no more than a simplified architecture
presentation for the chosen CNN design, imitating the MNL model in this
particular case. Each alternative input on this graph assumes entry of
the three attributes of a particular alternative, supported by five
individual characteristics each, the later being specific to a
particular alternative exactly as in the MNL model specification.

In this case the only difference between these two models is represented
by the algorithm used for estimation, which can yield absolutely
different results or even require some transformation of the input
dataset (ie. rescaling, which is used to prevent biases in weights
estimation). Consequently, the main interest of such implementation is
to observe, whether or not a ML algorithm will be able to bypass the MNL
model performances in the presence of heterogenous individual
preferences. Different convergence rates and different iterative
algorithms may result in absolutely distinct optimums for the parameters
vector. The particular algorithms implemented will be discussed later,
alongside the obtained results.

For NN modelling we use the advanced interface offered by Google's
\emph{Tensorflow} ({\textbf{???}}) with \emph{Keras} ({\textbf{???}})
back-end for \emph{R}-language. The flexibility offered by this
particular tool is astonishing compared to other neural networks
implementations in proposed in \emph{R}. This flexibility allows us to
simulate exactly the architecture of a MNL model and compare this way
how the different estimation techniques and algorithms perform in the
identical contexts.

\hypertarget{model-performance-evaluation-and-available-measures}{%
\subsection{Model performance evaluation and available
measures}\label{model-performance-evaluation-and-available-measures}}

In this subsection we are going to describe the different performance
measures, attempting at the same time to shun some light on the
complexity of this particular task and the multitude of different
questions that are usually aborded when a problem of performance
measures' choice arises.

The main problem in the case of classification context and particularly
in the multiple choice classification context relates to the fact that
rarely all of the models can use the same metrics for their comparison
(Baldi et al. 2000). The available metrics largely depend on the output
variable type, the models architecture and assumptions, the
specifications, the algorithms used and, finally and most importantly,
the context. As we have seen earlier, the work of Michaud, Llerena, and
Joly (2012) was focused on the identification of the willingness to pay
of consumers for particular environmental attributes of roses, rather
than general goodness of fit of particular model, which perfectly
illustrates the complexity of the posed question.

There exists a multitude of different target metrics to evaluate and
compare the performances of different models. For example, one may be
interested in exploration of a particular effects or the overall
goodness of fit, some predictive qualities or a possibility to derive
correct estimates for a particular socio-economic information. This
topic was already largely explored by some of the statisticians
(Japkowicz and Shah 2011) with some initial steps into producing an
integrated support containing all the necessary information for applied
studies. However, even given the amount of the work in reference, there
is still a strong need for contextualisation and constitution of
application specific methodological supports. The different possible
application scenarios require sometimes absolutely different metrics.
For example, econometricians rarely take into account the computational
efficiency of the models, while ML researchers are rarely considering
the possibility to derive the specific field specific metrics.

Nevertheless, this work aims at demonstrating the full potential of the
proposed experimental framework and we are bound to demonstrate at least
a fraction of its full potential, which inevitably addresses the
different performance metrics used to compare the models' performance in
terms of precision and predictive accuracy.

The measures available may roughly be divided into three parts following
the logic of Japkowicz and Shah (2011) (for an adaptation of the vision
of Japkowicz and Shah (2011) on the different measures' types see
Appendix B).

\begin{itemize}
\tightlist
\item
  The measures that take information solely from the \emph{confusion
  matrix}, which can be calculated using the estimated model over a know
  dataset (also denoted a test dataset). These measures are typically
  applied in the case of deterministic classification algorithms, but
  can be calculated for the probabilistic output algorithms as well.
\item
  The measures that not only use the confusion matrix, but integrate the
  information about the class distribution priors and classifier
  uncertainty. Logically, these metrics are useful for the
  \emph{scoring} classifiers" performance evaluation and could not be
  used with some more simple models.
\item
  Bayesian measures to account for probabilistic classifiers and
  measures for regression algorithms. Bayesian measures require a
  probabilistic structure of the models output.
\end{itemize}

The measures may be as well separated into two different groups by their
behaviour (Japkowicz and Shah 2011):

\begin{itemize}
\tightlist
\item
  A \emph{monotonic} performance measures \(pm(.)\), for which a strict
  increase (or decrease) in the value of \(pm(.)\) indicates a better
  (or worse) classifier throughout the range of the function \(pm(.)\)
  respectively.
\item
  \emph{Not strictly monotonic} can be thought of as the
  class-conditional probability estimate discussed in ({\textbf{???}})
  in the context of a multi-class problem.
\end{itemize}

As our framework can potentially treat multiple different aspects, we
will not only assess the general models' performances, but explore the
capacity to identify and estimate the target values of interest. Taking
into account the context of the target article we will be mostly
interested in exploring the willingness to pay (WTP) or the premium,
that the consumer is ready to add to the observed price for a particular
attribute.

\hypertarget{confusion-matrix}{%
\subsubsection{Confusion matrix}\label{confusion-matrix}}

Most of the performance measures for a classification task are derived
from the observed entries in the confusion matrix, denoted \(C\)
(Japkowicz and Shah 2011, @baldi2000ar). This matrix lies in the center
of most non-probabilistic performance measures for classification. A
confusion matrix \(C\) for a classifier defined by a function \(f(.)\)
over some dataset may be defined as:

\begin{equation}
C = {c_{ij}}, \text{  } i, j \in \{1, 2, \dots, k\}
\end{equation}

Where \(i\) is the row index and \(j\) is the column index, both
referring to some available alternatives for a given alternatives' set
\(\Omega\).

Generally, \(C\) is defined with respect to some fixed learning
algorithm. The confusion matrix can be extended to incorporate
information for the performance of more than one algorithm, resulting in
creation of a \emph{confusion tensor}, which can be imagined as a stack
of matrices. There exist specific metrics to be implemented on such
\emph{tensor}.

Given a training dataset and a test dataset, an algorithm learns on the
training set, outputting a fixed classifier \(f\). These datasets may be
identical, as it is frequently done in economics studies. The test-set
performance of \(f\) is then recorded in the confusion matrix. This
means that a confusion matrix, as well as its entries and the measures
derived from these are defined with respect to a fixed classifier \(f\)
over a given dataset. Consecutively, the matrix is sometimes denoted
with respect to \(f\) as \(C(f)\). It is a square \(k \times k\) matrix
for a dataset with \(k\) classes. Each element \(c_{ij}(f)\) of the
confusion matrix denotes the number of examples that actually have a
class \(i\) label and that the classifier \(f\) assigns to class \(j\).

In binary case these measures are simplified to four, that do not always
appear in matrix form for the sake of simplification. These measures, as
well as derived performance indicators are described in Baldi et al.
(2000). The binary classification case is the most common setting in
which the performance of the learning algorithm is measured. Also, this
setting serves well for illustration purposes with regard to the
strengths and limitations of the performance measures.

\hypertarget{general-performance-measures}{%
\subsubsection{General performance
measures}\label{general-performance-measures}}

The general measures (Baldi et al. 2000) describe the performance of a
given classifier \(f(.)\) (or shortly \(f\)) over a given set of
observation, taking into account all the possible classes, or choices in
the discrete choice context. In other words, these measures incorporate
all the information available for all the classes matches or mismatches,
which offers some good general overview of a given model performances,
but sometimes ignores some of the significant elements. For example,
given an unbalanced dataset, where one class dominates the other, the
general performance measures can have high positive values, signalling
the good overall performance, while all the observations will be
assigned to dominant class by the classifier.

The most known measures, which are usually implemented to assess the
general performance of the algorithms or even construct ``loss''
functions for some learning tasks include: the empirical risk, the
empirical error rate and the accuracy.

Accuracy and error rate effectively summarize the overall performance,
taking into account all data classes. This is the reason why these
measures are often implemented to assess general algorithms'
performances and are used in the learning tasks. Moreover, they offer an
insight into the generalization performance of the classifier by means
of studying their convergence behaviours, which may be important for
some algorithms.

Nevertheless, such general metrics have potential limitations (Japkowicz
and Shah 2011). Firstly, these measures suffer from the lack of
information on the varying degree of importance of different classes on
the performance. What is more, as we have already pointed out, the
metrics are incapacitated by the lack to produce any meaningful
information in the case of skewed class distribution. This results in
the situation, when as the distribution begins to skew in the direction
of a particular class, the more-prevalent class dominates the
measurement information in these metrics, making them biased.

\textbf{Empirical risk}

The \emph{empirical risk} \(R_{N} (f)\) of classifier \(f\) on test set
\(N\), defined as:

\begin{equation}
R_{N} (f) = \frac{1}{\mid N \mid} \sum_{i = 1}^{\mid N \mid} I (y_i \neq f (x_i))
\end{equation}

Where:

\begin{itemize}
\tightlist
\item
  \(I(a)\) is the indicator function if predicate \(a\) is true and zero
  otherwise;
\item
  \(f(x_i)\) is the label assigned to example \(x_i\) by classifier
  \(f\);
\item
  \(y_i\) is the true label of example \(x_i\), which indicates to ome
  of the alternatives \(\{\omega_1, \dots, \omega_k\} \in \Omega\);
\item
  \(\mid N \mid\) is the size of the test set.
\end{itemize}

This measure describes the average loss over the data points.

\textbf{Empirical error rate}

The \emph{empirical error rate} can be computed as follows:

\begin{equation}
R_N (f) = \frac
    {\sum_{i, j: i \neq j} c_{ij} (f)}
    {\sum_{i,j = 1}^{\Omega} c_{ij} (f)} =
  \frac
    {\sum_{i,j = 1}^{\Omega} c_{ij} (f) - \sum_{i = 1}^{\Omega} c_{ii} (f)}
    {\sum_{i,j = 1}^{\Omega} c_{ij} (f)}
\end{equation}

This rate measures the part of the instances from the given set that are
incorrectly classified by the learning algorithm \(f\).

\textbf{Accuracy}

The \emph{accuracy} describes the part of correctly classified instances
in a given set and is by its nature a complement to the empirical
error-rate measure. It can be computed as:

\begin{equation}
Acc_N (f) = \frac{1}{\mid N \mid} \sum_{i = 1}^{\mid N \mid} I (f (x_i) = y_i)
\end{equation}

Where \(y_i\) is the observed class for observation \(i\). Given a skew
ratio \(r_s\), it is possible to extend this measure and define the
\emph{skew-sensitive formulation of the accuracy}. Such modification
allows partially to solve the poor measures' utility problem on a skewed
class distribution dataset.

\hypertarget{single-class-performance-measures}{%
\subsubsection{Single-class performance
measures}\label{single-class-performance-measures}}

Apart from the general performance measures, there exist some more
specific performance measures, which instead of estimating the
performances of the overall classifier, target some specific aspects.
Usually in the modelling the consumer behaviour we may be interested in
his his choice ``Buy'' against ``No buy'' beforehand, and only
afterwards we are interested by his consumer habits and preferences.
Among these measure we may cite:

\begin{multicols}{2}
\begin{itemize}
\item True- and False-Positive/Negative Rates
\item Sensitivity
\item Specificity
\item Precision 
\item Recall 
\item Geometric means
\item Likelihood Ratio (LR) \footnote{This measure will be omitted in order to prevent the eventual confusion with Likelihood Ratio (LR) used in the MNL and MMNL models}
\item F-measured
\item Skew and Cost 
\end{itemize}
\end{multicols}

One of the important problems for discrete choice modelling and general
classification tasks resides in the form of the greater importance of
the algorithms' performance on a single class of interest. This
performance on a given class can be crucial with regard to the instances
of this class itself or with regard to the instances of other classes in
the training data. As it was mentioned earlier, in our particular study
case, we may be interested at how good the algorithm distinguishes the
``Buy'' and ``No buy'' choices.

A number of such measures can also allow us to measure the overall
performance of the classifier with an emphasis on the instances of each
individual class. Such precise metrics may be excessive, given a
particular case study, although they offer a good substitute for more
typical measures, such as the accuracy or error rate.

In this part we are going to introduce some new terminology, because
contrary to the precious parts, where we had to deal with classes, here
we are bound to simplify the problem to a binary case. This means that
one of the classes is considered as ``positive'', while the rest of the
alternatives is regrouped into a single ``negative'' class. Such
transformation allows us to define new variables, which will be used
later in the class-specific measures presentation. Among these values we
have:

\begin{itemize}
\tightlist
\item
  True Positive or \(TP\), which denotes the number of correctly
  classified observations which appertained to the ``positive'' class;
\item
  True Negative or \(TN\), where the number of correctly classified
  ``negative'' instances is regrouped;
\item
  False Positive or \(FP\) stands for the misclassified instances that
  in the dataset were encoded as ``positive'' class;
\item
  False Negative or \(FN\), which logically indicates the number of
  initially ``positive'' observations, which were identified as
  ``negative'' ones by the model.
\end{itemize}

All these values may be easily obtained from the confusion matrix \(C\).

\textbf{True- and False- positive/negative rates, specificity and
sensitivity}

The most natural metric aimed at measuring the performance of a learning
algorithm on instances of a single class is arguably its
\emph{true-positive rate}. The \emph{true-positive rate} of a classifier
is also referred to as the \emph{sensitivity} of the classifier. The
complement metric to this, in the case of the two-class scenario, would
focus on the proportion of negative instances is called the
\emph{specificity} of the learning algorithm. It is obtained as:

\begin{equation}
TPR_i (f) = \frac{c_{ii} (f)}{\sum_{j = 1}^{I} c_{ij} (f)} =
  \frac{c_{ii} (f)}{c_i (f)}
  \end{equation}

The \emph{false-positive rate} of a classifier:

\begin{equation}
FPR_i (f) = \frac{\sum_{j: j \neq i} c_{ji} (f)}
  {\sum_{j, k: k \neq i} c_{jk} (f)}
  \end{equation}

Some usefull derived formulas, which are easy to compute for a binary
case, are introduced hereafter. The True- and False- positive rates:

\begin{equation}
TPR (f) = \frac{TP}{TP + FN} = \text{Sensitivity} = 1 - FNR (f)
\end{equation}

\begin{equation}
FPR (f) = \frac{FP}{FP + TN}
\end{equation}

As well as their counterpart, the True- and False- negative rates, which
are focussed on the number of correctly classified instances from a
``negative class''.

\begin{equation}
TNR (f) = \frac{FN}{TN + FP} = \text{Specificity}
\end{equation}

\begin{equation}
FNR (f) = \frac{FN}{FN + TP}
\end{equation}

\textbf{Precision and recall}

The \emph{precision} or \emph{positive predictive value (PPV)} of a
classifier \(f\) on a given class of interest \(j\), denoted as well as
the ``positive'' class, in terms of the entries of \(C\), measures how
\emph{precise} the algorithm is when identifying the examples of a given
class and is defined as:

\begin{equation}
PPV_i (f) = Prec_i (f) \frac{c_{ii} (f)}{\sum_{j = 1}^{I} c_{ji} (f)} = \frac{c_{ii} (f)}{c_{.i} (f)}
\end{equation}

For binary case we can write the following simplified definition, which
should be more clear to the reader:

\begin{equation}
Prec (f) = PPV (f) = \frac{TP}{TP + FP}
\end{equation}

The PPV can be complimented with the sensitivity of the classifier over
this class. This measure is generally referred to as \emph{recall}:

\begin{equation}
Rec (f) = \frac{TP}{TP + FN}
\end{equation}

\textbf{Geometric means}

The \emph{geometric means} take into account the relative balance of
several performance measures for a given classifier. The most popular
option is to observe simultaneously the classifier's performance on both
the positive and the negative classes:

\begin{equation}
Gmean_1 (f) = \sqrt{TPR (f) \times TNR (f)}
\end{equation}

This implementation is of particular interest for our case study, as we
will be able to compare the performances of different models across
``Buy'' and ``No buy'' options. Another popular version of the measure,
which focusses on a single class of interest, can take the precision of
the classifier in combination with the classifiers performance on the
``positive'' class into account:

\begin{equation}
Gmean_2 (f) = \sqrt{TPR (f) \times Prec (f)}
\end{equation}

\textbf{F-measure}

The \emph{F-measure} as well attempts to address the issue of
convenience brought on by a single metric versus a pair of metrics. It
combines the information of precision and recall in a single value. More
precisely, the F-measure is a weighted harmonic mean of precision and
recall, with a weight \(\alpha\):

\begin{equation}
F_{\alpha} = \frac
  {(1 + \alpha)(Prec (f) \times Rec (f))}
  {\alpha Prec (f) + Rec (f)}
\end{equation}

For instance, the most comprehensive \emph{balanced F-measure} weights
the recall and precision of the classifier evenly:

\begin{equation}
F_{1} = \frac
  {2(Prec (f) \times Rec (f))}
  {Prec (f) + Rec (f)}
\end{equation}

In most practical cases, appropriate weights are generally not known,
which results in some complications in choice of the hyper-parameter
\(\alpha\) of such combinations of measures.

\textbf{Class ratio}

\emph{Class ratio} for a given class \(i\), which in the consumer choice
setting is usually denoted \(\omega_i\) refers to the number of
instances of class \(i\) as opposed to those of other classes in the
dataset:

\begin{equation}
ratio_i = r_i = \frac{\sum_j c_{ij}}
  {\sum_{j, j \neq i} c_{ji} + \sum_{j, j \neq i} c_{jj}}
\end{equation}

Or for a binary case:

\begin{equation}
ratio_{positive} = \frac{(TP + FN)}{(FP + TN)}
\end{equation}

Another issue worth considering when looking at misclassification is
that of classifier uncertainty. This lack of classifier uncertainty
information is also reflected in all the performance measures that rely
solely on the confusion matrix.

\hypertarget{information-theoretic-measures}{%
\subsubsection{Information-theoretic
measures}\label{information-theoretic-measures}}

These measures are probabilistic by their nature, as they explore the
performances of the classifier with respect to the (typically empirical)
prior distributions of the data. in contrast to the cost-sensitive
metrics that have been introduced earlier, the
\emph{information-theoretic measures}, because of accounting for the
data priors, are applicable only to probabilistic classifiers. What is
more, these metrics are independent of the cost considerations and can
be applied directly to the probabilistic output of a given model. These
measures are extensively implemented in Bayesian learning and take their
roots in physics. Among these metrics one may encounter:

\begin{itemize}
\tightlist
\item
  Kullback--Leibler Divergence, which estimates the difference between
  the entropies of the two distributions;
\item
  Kononenko and Bratko's Information Score, which explores the
  likelihood of correct classification.
\end{itemize}

In this work we will present only the first among these two.

\textbf{Kullback--Leibler Divergence}

Let the true probability distribution over the labels be denoted as
\(p(y)\). Let the posterior distribution generated by the learning
algorithm after seeing the data be denoted by \(P (y \mid f)\). Because
\(f\) is obtained after looking at the training samples \(x \in S\),
this empirically approximates \(P (y \mid x)\), the conditional
posterior distribution of the labels. Then the \emph{Kullback--Leibler
divergence} (KLD or KL) can be utilized to quantify the difference
between the estimated posterior distribution and the true underlying
distribution of the labels:

\begin{equation}
KLD [p(y) \mid \mid P (y \mid f) ] =
  \int p(y) ln p(y) dy - 
  \int p(y) ln P(y \mid f) dy
\end{equation}

\begin{equation}
KLD [p(y) \mid \mid P (y \mid f) ] =
  - \int p(y) ln \frac{P(y \mid f)}{p(y)} dy
\end{equation}

The KLD divergence basically just finds the difference between the
entropies of the two distributions \(P (y \mid f)\) and \(p(y)\). This
measure is also known as \emph{relative entropy} (see Baldi et al.
(2000)) for more information.

\begin{equation}
KLD [p(y) \mid \mid P (y \mid f) ] = 
  - \sum_{x \in S} p(y) ln \frac{P(y \mid f)}{p(y)} dy = \sum_{x \in S} p(y) ln \frac{p(y)}{P(y \mid f)} dy
\end{equation}

The KLD value is equal to zero if and only if the posterior distribution
is the same as the prior, when the perfect fit is achieved, meaning that
the classifier perfectly mimics the true underlying distribution of the
labels.

Even though the KLD measures the difference between the posterior
distribution obtained by the learner from the true distribution so there
is a significant drawback to it. The KLD needs the knowledge of the true
underlying prior distribution of the labels, which is rarely, if at all,
known in any practical application. In practice the estimated priors are
used, although in the experimental framework where a synthetic dataset
is used, we may theoretically impose some ``true'' structure over the
choice distribution.

\hypertarget{case-specific-metrics}{%
\subsubsection{Case specific metrics}\label{case-specific-metrics}}

The article of Michaud, Llerena, and Joly (2012) focuses on the WTP for
roses and derivation of the premiums for particular alternative
attributes of interest. This focus allows authors to explore the
consumer attitude towards the alternative specific environmental
attributes. Consequently, as we try to follow the logic introduced in
the article, we are going to attempt to derive the WTP and premiums for
attributes as well. However, before introducing the notion of the WTP
and premium, we should firstly describe the procedure of derivation of
the marginal effects, as the WTP and premiums are expressed using the
marginal effects.

In the conventional MNL models the coefficients \(\beta_{rj}\) can be
interpreted as the marginal effect of variable \(X_r\) on the log
odds-ratio of alternative \(j\) to the baseline alternative. The
marginal effect of \(X_r\) on the probability of choosing a specific
alternative \(j\) can be expressed as:

\begin{equation}
ME_{rj} = \frac{
    \Delta P(Y_i = \omega_j) 
}{
    \Delta X_r 
}
\end{equation}

Consequently, for the MNL model, the marginal effect of \(X_r\) on
alternative \(j\) not only takes into account the parameters specific to
\(j\) alternative, but the ones of all other alternatives as well. The
equation can be written in this case as:

\begin{equation}
\frac{
    \Delta P(Y_i = \omega_j) 
}{
    \Delta X_i 
} = P(Y_i = \omega_j) [
    \beta_{j1} - \sum_{l = 0}^k P(Y_i = \omega_l) \beta_{j1}
]
\end{equation}

The parameters such as WTP and premiums are more easy to interpret. They
can be estimated directly or can be obtained from the marginal utility
by dividing it by the effect estimate of a price, taken as a non random
parameter. The resulting ratio can afterwards be interpreted as a
monetary value. The WTP as it was described in the context presentation,
taking into account the case specific relative utility functions can be
represented as:

\begin{equation}
WTP = \frac{
  \frac{\Delta V}{\Delta BUY}
}{
  \frac{\Delta V}{\Delta Price}
} = \frac{
  - \alpha_{Buy}
}{
  \beta_{Price}
}
\end{equation}

The premiums for a given attribute \(X_r\) (\(Label\), \(Carbon\) or
their cross-product \(LC\)), can therefore be expressed as:

\begin{equation}
WTP = \frac{
  \frac{\Delta V}{\Delta X_r}
}{
   \frac{\Delta V}{\Delta Price}
}
\end{equation}

\hypertarget{selection-of-measures-to-implement}{%
\subsubsection{Selection of measures to
implement}\label{selection-of-measures-to-implement}}

In this work we are going to explore only a selection of the described
above most popular performance metrics, that are the most interesting
given the context of the study. Moreover, in our application we are
limited in the number of measures we can explore.

In the first place we are interested by the WTP for roses and the
premiums associated with particular alternative specific attributes.
These theoretical values could be easily derived for all the three
explored models and they will allow us to compare, how close are the
derived values from the theoretical input values, which were defined on
the dataset generation step.

Secondly, it is important to assess the overall goodness of fit over the
whole dataset for the selected models. For this particular task the most
suited measure is the \emph{accuracy}. This way we will be able to
observe the ratio of the overall correctly classified instances. We may
implement the KLD estimator for overall goodness of fit, based on the
probability distributions, because all the models predict the
probabilities for the available alternatives.

We may be interested as well in comparing the performances of the given
models in terms of distinguishing the ``Buy'' choice, irrelevant of the
alternative, and the ``No buy'' choice. This is a particularly
interesting question, because in the different choice settings and over
the datasets generated under different theoretical assumptions. For this
purpose the most interesting choice will be to select the F-measure or a
Geometric mean of the TPR and TNR.

Finally, we are going to observe the performance of these different
models in terms of computational efficiency in resources consumption.
For this task we will observe the computation times for given
models\footnote{This measure is one of the most complex, because it
  accounts at the same time for different models, different estimation
  algorithms, different numerical implementation in the statistical
  software and different PC configuration. It is valid in this
  particular case, because all models were estimated using the same
  hardware and software set-up.}. The obtained results will be discussed
at the end of this work.

\newpage

\hypertarget{refs}{}
\leavevmode\hypertarget{ref-agrawal2019nber}{}%
Agrawal, Ajay, Joshua Gans, and Avi Goldfarb. 2019. \emph{The Economics
of Artificial Intelligence: An Agenda}. Book. National Bureau of
Economic Research; University of Chicago Press.
\url{https://doi.org/https://doi.org/10.7208/chicago/9780226613475.001.0001}.

\leavevmode\hypertarget{ref-agresti2007cd}{}%
Agresti, Alan. 2007. \emph{An Introduction to Categorical Data Analysis,
Second Edition}.

\leavevmode\hypertarget{ref-agresti2013cd}{}%
---------. 2013. \emph{Categorical Data Analysis, Third Edition}.

\leavevmode\hypertarget{ref-athey2019ml}{}%
Athey, Susan, and Guido W. Imbens. 2019. ``Machine Learning Methods That
Economists Should Know About.'' \emph{Annual Review of Economics} 11
(1): 685--725.
\url{https://doi.org/10.1146/annurev-economics-080217-053433}.

\leavevmode\hypertarget{ref-ayodele2010tml}{}%
Ayodele, Taiwo Oladipupo. 2010. ``Types of Machine Learning
Algorithms.'' \emph{New Advances in Machine Learning}. InTech, 19--48.

\leavevmode\hypertarget{ref-baldi2000ar}{}%
Baldi, Pierre, Søren Brunak, Yves Chauvin, Claus A. F. Andersen, and
Henrik Nielsen. 2000. ``Assessing the accuracy of prediction algorithms
for classification: an overview.'' \emph{Bioinformatics} 16 (5):
412--24. \url{https://doi.org/10.1093/bioinformatics/16.5.412}.

\leavevmode\hypertarget{ref-bhat1995evm}{}%
Bhat, Chandra R. 1995. ``A Heteroscedastic Extreme Value Model of
Intercity Travel Mode Choice.'' Suggested. \emph{Transportation Research
Part B: Methodological} 29 (6): 471--83.
\url{https://EconPapers.repec.org/RePEc:eee:transb:v:29:y:1995:i:6:p:471-483}.

\leavevmode\hypertarget{ref-joly2019qcm}{}%
Bouscasse, Hélène, Iragaël Joly, and Jean Peyhardi. 2019. ``A new family
of qualitative choice models: An application of reference models to
travel mode choice.'' \emph{Transportation Research Part B:
Methodological} 121 (C): 74--91.
\url{https://doi.org/10.1016/j.trb.2018.12.010}.

\leavevmode\hypertarget{ref-breiman2001stat}{}%
Breiman, Leo, and others. 2001. ``Statistical Modeling: The Two Cultures
(with Comments and a Rejoinder by the Author).'' \emph{Statistical
Science} 16 (3). Institute of Mathematical Statistics: 199--231.

\leavevmode\hypertarget{ref-cascetta2009tr}{}%
Cascetta, Ennio. 2009. \emph{Transportation Systems Analysis: Models and
Applications}. Vol. 29. Springer Science \& Business Media.

\leavevmode\hypertarget{ref-chen2013rac}{}%
Chen, Bryant, and Judea Pearl. 2013. ``Regression and Causation: A
Critical Examination of Six Econometrics Textbooks.'' \emph{Real-World
Economics Review, Issue}, no. 65: 2--20.

\leavevmode\hypertarget{ref-denuit2019as1}{}%
Denuit, Michel, and Julien Trufin. 2019. \emph{Effective Statistical
Learning Methods for Actuaries I: GLMs and Extentions}. Springer.

\leavevmode\hypertarget{ref-depalma2011tr}{}%
De Palma, André, Robin Lindsey, Emile Quinet, and Roger Vickerman. 2011.
\emph{A Handbook of Transport Economics}. Edward Elgar Publishing.

\leavevmode\hypertarget{ref-donoho2017ds}{}%
Donoho, David. 2017. ``50 Years of Data Science.'' \emph{Journal of
Computational and Graphical Statistics} 26 (4). Taylor \& Francis:
745--66.

\leavevmode\hypertarget{ref-furnkranz2011p}{}%
Fürnkranz, J., and E. Hüllermeier. 2010. \emph{Preference Learning}.
Springer Verlag, Berlin.

\leavevmode\hypertarget{ref-garrow2010gs}{}%
Garrow, Tudor D.; Lee, Laurie A.; Bodea. 2010. ``Generation of Synthetic
Datasets for Discrete Choice Analysis.'' \emph{Transportation} 37 (2):
183--202. \url{https://doi.org/10.1007/s11116-009-9228-6}.

\leavevmode\hypertarget{ref-hastie2009sl}{}%
Hastie, Trevor, Robert Tibshirani, and Jerome Friedman. 2009. \emph{The
Elements of Statistical Learning: Data Mining, Inference, and
Prediction}. Springer Science \& Business Media.

\leavevmode\hypertarget{ref-japkowicz2011el}{}%
Japkowicz, Nathalie, and Mohak Shah. 2011. \emph{Evaluating Learning
Algorithms: A Classification Perspective}. Cambridge University Press.
\url{https://doi.org/10.1017/CBO9780511921803}.

\leavevmode\hypertarget{ref-jebara2004ml}{}%
Jebara, Tony. 2004. \emph{Machine Learning: Discriminative and
Generative}. Springer Science \& Business Media.

\leavevmode\hypertarget{ref-kotsiantis2006tr}{}%
Kotsiantis, Sotiris, I. Zaharakis, and P. Pintelas. 2006. ``Machine
Learning: A Review of Classification and Combining Techniques.''
\emph{Artificial Intelligence Review} 26 (November): 159--90.
\url{https://doi.org/10.1007/s10462-007-9052-3}.

\leavevmode\hypertarget{ref-mcfadden1974utd}{}%
McFadden, Daniel. 1974. ``The Measurement of Urban Travel Demand.''
\emph{Journal of Public Economics} 3 (4): 303--28.
\url{https://doi.org/https://doi.org/10.1016/0047-2727(74)90003-6}.

\leavevmode\hypertarget{ref-mcfadden2001ec}{}%
---------. 2001. ``Economic Choices.'' \emph{The American Economic
Review} 91 (3). American Economic Association: 351--78.
\url{http://www.jstor.org/stable/2677869}.

\leavevmode\hypertarget{ref-llerena2013rose}{}%
Michaud, Celine, Daniel Llerena, and Iragael Joly. 2012. ``Willingness
to pay for environmental attributes of non-food agricultural products: a
real choice experiment.'' \emph{European Review of Agricultural
Economics} 40 (2): 313--29. \url{https://doi.org/10.1093/erae/jbs025}.

\leavevmode\hypertarget{ref-mullainathan2017ml}{}%
Mullainathan, Sendhil, and Jann Spiess. 2017. ``Machine Learning: An
Applied Econometric Approach.'' \emph{Journal of Economic Perspectives}
31 (2): 87--106. \url{https://doi.org/10.1257/jep.31.2.87}.

\leavevmode\hypertarget{ref-tsoumakas2007cm}{}%
Tsoumakas, Grigorios, and Ioannis Katakis. 2007. ``Multi-Label
Classification: An Overview.'' \emph{International Journal of Data
Warehousing and Mining (IJDWM)} 3 (3): 1--13.
\url{https://EconPapers.repec.org/RePEc:igg:jdwm00:v:3:y:2007:i:3:p:1-13}.

\leavevmode\hypertarget{ref-varian2014bd}{}%
Varian, Hal R. 2014. ``Big Data: New Tricks for Econometrics.''
\emph{Journal of Economic Perspectives} 28 (2): 3--28.
\url{https://doi.org/10.1257/jep.28.2.3}.

\leavevmode\hypertarget{ref-zielesny2011cf}{}%
Zielesny, Achim. 2011. \emph{From Curve Fitting to Machine Learning}.
Vol. 18. Springer.


\end{document}
