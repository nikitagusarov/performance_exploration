\documentclass[11pt,]{article}
\usepackage{lmodern}
\usepackage{amssymb,amsmath}
\usepackage{ifxetex,ifluatex}
\usepackage{fixltx2e} % provides \textsubscript
\ifnum 0\ifxetex 1\fi\ifluatex 1\fi=0 % if pdftex
  \usepackage[T1]{fontenc}
  \usepackage[utf8]{inputenc}
\else % if luatex or xelatex
  \ifxetex
    \usepackage{mathspec}
  \else
    \usepackage{fontspec}
  \fi
  \defaultfontfeatures{Ligatures=TeX,Scale=MatchLowercase}
\fi
% use upquote if available, for straight quotes in verbatim environments
\IfFileExists{upquote.sty}{\usepackage{upquote}}{}
% use microtype if available
\IfFileExists{microtype.sty}{%
\usepackage{microtype}
\UseMicrotypeSet[protrusion]{basicmath} % disable protrusion for tt fonts
}{}
\usepackage[margin=1in]{geometry}
\usepackage{hyperref}
\hypersetup{unicode=true,
            pdfborder={0 0 0},
            breaklinks=true}
\urlstyle{same}  % don't use monospace font for urls
\usepackage{graphicx,grffile}
\makeatletter
\def\maxwidth{\ifdim\Gin@nat@width>\linewidth\linewidth\else\Gin@nat@width\fi}
\def\maxheight{\ifdim\Gin@nat@height>\textheight\textheight\else\Gin@nat@height\fi}
\makeatother
% Scale images if necessary, so that they will not overflow the page
% margins by default, and it is still possible to overwrite the defaults
% using explicit options in \includegraphics[width, height, ...]{}
\setkeys{Gin}{width=\maxwidth,height=\maxheight,keepaspectratio}
\IfFileExists{parskip.sty}{%
\usepackage{parskip}
}{% else
\setlength{\parindent}{0pt}
\setlength{\parskip}{6pt plus 2pt minus 1pt}
}
\setlength{\emergencystretch}{3em}  % prevent overfull lines
\providecommand{\tightlist}{%
  \setlength{\itemsep}{0pt}\setlength{\parskip}{0pt}}
\setcounter{secnumdepth}{0}
% Redefines (sub)paragraphs to behave more like sections
\ifx\paragraph\undefined\else
\let\oldparagraph\paragraph
\renewcommand{\paragraph}[1]{\oldparagraph{#1}\mbox{}}
\fi
\ifx\subparagraph\undefined\else
\let\oldsubparagraph\subparagraph
\renewcommand{\subparagraph}[1]{\oldsubparagraph{#1}\mbox{}}
\fi

%%% Use protect on footnotes to avoid problems with footnotes in titles
\let\rmarkdownfootnote\footnote%
\def\footnote{\protect\rmarkdownfootnote}

%%% Change title format to be more compact
\usepackage{titling}

% Create subtitle command for use in maketitle
\newcommand{\subtitle}[1]{
  \posttitle{
    \begin{center}\large#1\end{center}
    }
}

\setlength{\droptitle}{-2em}

  \title{}
    \pretitle{\vspace{\droptitle}}
  \posttitle{}
    \author{}
    \preauthor{}\postauthor{}
    \date{}
    \predate{}\postdate{}
  
\usepackage{placeins}
\AtBeginDocument{\let\maketitle\relax}
\usepackage{subcaption}

\begin{document}

\hypertarget{simulating-individual-choices}{%
\subsection{Simulating individual
choices}\label{simulating-individual-choices}}

Based on the article of Michaud, Llerena, and Joly (2012) we generate a
synthetic dataset assuming the utility function is as described in the
paper with some minor changes and adjustments. We have already delimited
the scope of study and delimited our area of interest to the exploration
of different models performance given the theoretical structure of
consumer preferences for the alternative specific attributes. For
simplicity we relax some of the assumptions made in the paper in order
to generate two different datasets. For the first dataset we assume that
estimations made in the paper and the derived utility functions are
correct and reflect the real world situation. For the second one, we
relax some of the advanced assumptions and regenerate a simplified
version, which will allow us to contrast the performances of different
models in different choice context assuming different nature of choice
functions.

In both situations the utility functions are determined as in paper: we
use the exact means for the coefficients estimates, assuming they are
correct. The relative utility's deterministic part for each individual
is defined by the following function, which was presented in a more
detailed way in previous section:

\begin{multline}
V_{ij} = \alpha_{i,Buy} + \beta_{Buy, Sex} Sex_i + \beta_{Buy, Age} Age_i + \beta_{Buy, Salary} Salary_i + \beta_{Buy, Habit} Habit_i + \\
+ \gamma_{Price} Price_{ij} + \gamma_{i, Label} Label_{ij} + \gamma_{i, Carbon} Carbon_{ij} + \gamma_{i, LC} LC_{ij}
\end{multline}

Where \(LC = Label \times Carbon\). The random component of the relative
utility \(U_{ij}\) is defined as identically and independently
distributed random variable \(\epsilon_{ij}\) issued from the Gumble
distribution parametrised with \((0, 1)\). The mean effects for the
components of the deterministic part are given as presented in the table
\ref{tab:params1}

\begin{table}[!htbp]\centering
    \caption{The assumed relative utility function parameters}
    \label{tab:params}
\begin{subfigure}[c]{.4\linewidth}
    \centering
    \caption{Mean effects}
    \label{tab:params1}
\begin{tabular}[b]{@{\extracolsep{5pt}}lc} 
\\[-1.8ex]\hline 
\hline \\[-1.8ex] 
& \multicolumn{1}{c}{\textit{Effects}} \\ 
\cline{2-2} 
\\[-1.8ex] & \multicolumn{1}{c}{\textit{Means}} \\
\hline \\[-1.8ex] 
\textbf{Individual characteristics ($\beta$)} & \\
 ~~~Sex & 1.420 \\ 
 ~~~Age & 0.009 \\ 
 ~~~Salary & 0.057 \\ 
 ~~~Habit & 1.027 \\ 
\textbf{Alternatives' attributes ($\gamma$)} & \\
 ~~~Price & $-$1.631 \\ 
 ~~~Buy & 2.285 \\ 
 ~~~Label & 2.824 \\ 
 ~~~Carbon & 6.665 \\ 
 ~~~LC & $-$2.785 \\
 ~~~ & \\
\hline \\[-1.8ex] 
\end{tabular} 
\end{subfigure}
\hspace{1.5cm}
\begin{subfigure}[c]{.4\linewidth}
    \centering
    \caption{Variance-covariance structure}
    \label{tab:params2}
\begin{tabular}{@{\extracolsep{5pt}}lcc} 
\\[-1.8ex]\hline 
\hline \\[-1.8ex] 
 & \multicolumn{2}{c}{\textit{Effects}} \\ 
\cline{2-3} 
\\[-1.8ex] & Fixed & Random \\ 
\hline \\[-1.8ex] 
\textbf{Variance} & & \\
 ~~~Buy & 0 & 3.202 \\  
 ~~~Label & 0 & 2.654 \\  
 ~~~Carbon & 0 & 3.535 \\  
 ~~~LC & 0 & 2.711 \\ 
\textbf{Covariance} & & \\ 
 ~~~Buy:Label & 0 & -0.54 \\  
 ~~~Buy:Carbon & 0 & -4.39 \\  
 ~~~Buy:LC & 0 & 6.17 \\  
 ~~~Label:Carbon & 0 & 8.77 \\  
 ~~~Label:LC & 0 & -2.33 \\  
 ~~~Carbon:LC & 0 & -4.82 \\ 
\hline \\[-1.8ex] 
\end{tabular} 
\end{subfigure}
\end{table}

The only difference between the two generated datasets is in the
specification of the randomness of these coefficients as they may vary
or not across population. It means, that the first dataset is generated
assuming the variance-covariance matrix for correlated random
coefficients is composed with 0's only and the resulting multivariate
normal distribution produces exact means for the coefficients. The
second dataset is generated using the exact estimates of the
variance-covariance matrix as provided in the article. The assumed
parameters for effects distributions are represented in the table
\ref{tab:params2}.

Additionally we impose some supplementary constraints to our data due to
the limitations of the simulation tool. Particularly, the individual
characteristics are supposed to be not correlated, which can be
explained by the fact that the context of a controlled experiment offers
a possibility to control this particular feature. Obviously, this is not
optimal decision, as naturally the age, sex, income and environmental
habits of individuals should be correlated. Unfortunately, the original
article does not provide information about the characteristics'
variance-covariance matrix.

The targeted features and requirements to the resulting dataset are
numerous and they make a contrast compared to the initial empirical
dataset.

The simulated dataset allows us to explore significant number of choice
sets for numerous artificial individuals, which ensures statistical
validity for obtained results and permits us to use advanced estimation
algorithms (such as neural networks, for example). It means that we
generate a large sample with exhaustive number of choice sets, in which
all the possible combinations of alternative attributes are represented.
Here by \emph{attributes} we understand the binary factors describing
rose's labelling and carbon footprint and ignore the price, the latter
being added afterwards using randomisation techniques. This choice is
similar to the experimental design described in the Michaud, Llerena,
and Joly (2012) work and is easily explained when we take a closer look
at the number of choice sets for different specifications. In simulated
datasets it is traditional to use Full-Factorial (FF) experimental
design as it uncovers completely the full potential of simulation tools:
it allows to observe all the possible combinations of factors affecting
some process and fully explore their implications. In our case, a simple
full factorial design for a binary choice context has 28 combinations of
factors (seven levels of prices, two levels for eco-label and two levels
for Carbon imprint), but a complete full factorial design for a choice
context with two alternatives implies 784 different combinations (as we
have two alternatives each having 28 possible variants), which is
unrealistic in a standard experimental study context and risks to be too
demanding in terms of calculation times.

The dataset should be equilibrated with relatively identical number of
choices for all three alternatives. In the field experiment the authors
managed to achieve satisfying result with 67.5\% of ``Buy'' choices and
32.5\% for ``Not to buy'' choices, although the ``A'' and ``B''
alternatives showed different properties. The resulting observed
descriptive statistics derived from the data proposed by Michaud,
Llerena, and Joly (2012) are presented in table \ref{tab:altdata}. The
table focusses on the choice ``Buy'' descriptive statistics, ignoring
the ``No buy'' option, for which all the attributes are considered to be
equal to 0. The \(p\)-values are the results of the two subsets (``A''
and ``B'') comparison\footnote{\(\chi^2\) test is used for discrete
  variables, while \emph{Anova} is implemented for continuous ones.}.

\begin{table}[!htbp] \centering 
  \caption{Alternatives' descriptive statistics by group, correlated random effects} 
  \label{tab:altdata} 
\begin{tabular}{@{\extracolsep{5pt}}lcccr}
\\[-1.8ex]\hline 
\hline \\[-1.8ex] 
 & A  & B  & Total  & p value\\
 & (N=1186) & (N=1186) & (N=2372) &  \\
\hline \\[-1.8ex] 
\textbf{Choice} &  &  &  & < 0.001\\
~~~Mean (SD) & 0.517 (0.500) & 0.159 (0.366) & 0.338 (0.473) & \\
~~~Range & 0.000 - 1.000 & 0.000 - 1.000 & 0.000 - 1.000 & \\
\textbf{Price} &  &  &  & 0.418\\
~~~Mean (SD) & 2.990 (0.881) & 3.020 (0.893) & 3.005 (0.887) & \\
~~~Range & 1.500 - 4.500 & 1.500 - 4.500 & 1.500 - 4.500 & \\
\textbf{Carbon} &  &  &  & < 0.001\\
~~~Mean (SD) & 0.167 (0.373) & 0.832 (0.374) & 0.500 (0.500) & \\
~~~Range & 0.000 - 1.000 & 0.000 - 1.000 & 0.000 - 1.000 & \\
\textbf{Label} &  &  &  & 0.837\\
~~~Mean (SD) & 0.502 (0.500) & 0.497 (0.500) & 0.500 (0.500) & \\
~~~Range & 0.000 - 1.000 & 0.000 - 1.000 & 0.000 - 1.000 & \\
\hline
\end{tabular}
\end{table}

Of particular interest in the table \ref{tab:altdata} to us is the
unbalanced structure of the resulting dataset. The \(Carbon\) imprint of
the different alternatives has not identical properties, which leads to
different \(Choice\) statistics, where the alternative with higher
carbon imprint is chosen less frequently. In the original study such
difference was not dangerous, because only the ``Buy'' option was
compared against ``No Buy'' one. However, in case of the NN modelling
such unbalanced dataset may lead to erroneous results, where the more
popular alternative will always have a higher choice probability. The
distribution inside the ``Buy'' group for different alternatives (``A''
and ``B'') should be quasi-identical, producing equally distributed
three groups of choices each nearing 33.3\%. Even if this property is
not as important for a traditional MNL model, we are interested to
observe the same choice structure in our artificial dataset, because it
may highly affect the performance of more advanced models, such as NN
for example.

\FloatBarrier

\hypertarget{generated-dataset-presentation}{%
\subsubsection{Generated dataset
presentation}\label{generated-dataset-presentation}}

\FloatBarrier

In this section we will discuss the resulting datasets simulated under
the listed above assumptions.

For our dataset we choose to generate 160000 observations, for 1000
individuals, each facing 16 different choice sets 10 times. The 16
choice sets include all the possible combinations of two roses (``A''
and ``B'') described by two environmental attributes, while prices are
randomly assigned within the choice sets. The prices are assumed to be
uniformly distributed over the choice sets, following a discrete uniform
distribution. The prices vary among the different replications. This
procedure resulted in sufficiently large dataset, which in the same time
was not difficult to treat without implementation of Big Data specific
techniques.

The original experimental design used to generate the choice sets
assumed no branding for the alternatives to avoid any undesired bias in
the results. Theoretically this design should have provided an
equilibrated dataset with no correlation between different attributes,
although the size of the final dataset might have affected the results.
In our case we assume that individuals have no additional information
about the roses in choice sets except the three observed attributes. As
in the original work we assign insignificant labels ``A'' and ``B'' to
the roses within choice sets, which is done mostly for convenience and
has no impact on the individuals' decisions.

It is interesting to explore the statistical properties of the resulting
datasets: the original one (Original), gathered by Michaud, Llerena, and
Joly (2012) and made available in anonymised format by Iragaël Joly; and
the two generated artificial datasets, assuming homogeneous (Generated
FE) and heterogeneous (Generated RE) preferences respectively of the
individuals for the environmental attributes. First of all, we may
observe the individuals descriptive statistics for three datasets in the
table \ref{tab:indivdata}.

\begin{table}[!htbp] \centering 
  \caption{Individuals' characteristics descriptive statistics by dataset} 
  \label{tab:indivdata} 
\begin{tabular}{@{\extracolsep{5pt}}lcccr}
\\[-1.8ex]\hline 
\hline \\[-1.8ex] 
 & Fixed Effects  & Random Effects  & Target  & p value\\
 & (N=1000) & (N=1000) & (N=102) &  \\
\hline \\[-1.8ex] 
\textbf{Sex} &  &  &  & 0.851\\
~~~Mean (SD) & 0.506 (0.500) & 0.515 (0.500) & 0.490 (0.502) & \\
~~~Range & 0.000 - 1.000 & 0.000 - 1.000 & 0.000 - 1.000 & \\
\textbf{Habit} &  &  &  & 0.182\\
~~~N-Miss & 0 & 0 & 1 & \\
~~~Mean (SD) & 0.683 (0.466) & 0.657 (0.475) & 0.604 (0.492) & \\
~~~Range & 0.000 - 1.000 & 0.000 - 1.000 & 0.000 - 1.000 & \\
\textbf{Salary} &  &  &  & < 0.001\\
~~~Mean (SD) & 2.750 (1.476) & 2.671 (1.438) & 2.147 (1.222) & \\
~~~Range & 1.000 - 6.000 & 1.000 - 6.000 & 1.000 - 6.000 & \\
\textbf{Age} &  &  &  & 0.255\\
~~~Mean (SD) & 41.862 (13.685) & 42.161 (13.820) & 39.755 (18.895) & \\
~~~Range & 18.000 - 84.000 & 18.000 - 84.000 & 18.000 - 85.000 & \\
\hline
\end{tabular}
\end{table}

Even though the \(p\)-values show no evident differences between the
simulated datasets and the original one, except for the \(Age\)
variable, we observe the differences in the means. This is explained by
the implemented dataset generation procedure. The variables in the
original dataset are integers, assuming continuous nature of the real
world variables. When synthesizing the dataset, we assume the quasi
continuous variables, such as \(Age\) and \(Salary\) (denoted as
\(Income\) in original work) to be issued from normal distribution with
parameters as figuring in the descriptive statistics for the original
dataset, and only afterwards we convert the resulting values to
integers. The binary variables \(Sex\) and \(Habit\) are generated with
random draws from Bernoully distribution and consequently produce more
realistic results. This procedure leads to potential biases in the
resulting datasets, which is true not only for the individual variables,
but for the alternatives' attributes as well.

\begin{table}[!htbp] \centering 
  \caption{Alternatives' descriptive statistics by dataset} 
  \label{tab:alt1} 
\begin{tabular}{@{\extracolsep{5pt}}lcccr}
\\[-1.8ex]\hline 
\hline \\[-1.8ex] 
 & Fixed Effects  & Random Effects  & Target  & p value\\
 & (N=320000) & (N=320000) & (N=2372) &  \\
\hline \\[-1.8ex] 
\textbf{Price} &  &  &  & 0.002\\
~~~Mean (SD) & 2.936 (0.958) & 2.936 (0.958) & 3.005 (0.887) & \\
~~~Range & 1.500 - 4.500 & 1.500 - 4.500 & 1.500 - 4.500 & \\
\textbf{Carbon} &  &  &  & 0.999\\
~~~Mean (SD) & 0.500 (0.500) & 0.500 (0.500) & 0.500 (0.500) & \\
~~~Range & 0.000 - 1.000 & 0.000 - 1.000 & 0.000 - 1.000 & \\
\textbf{Label} &  &  &  & 0.999\\
~~~Mean (SD) & 0.500 (0.500) & 0.500 (0.500) & 0.500 (0.500) & \\
~~~Range & 0.000 - 1.000 & 0.000 - 1.000 & 0.000 - 1.000 & \\
\hline
\end{tabular}
\end{table}

Secondly, we may as well observe the alternative specific descriptive
statistics. They are presented in table \ref{tab:alt1}. In this table we
present the cumulative statistics for the ``Buy'' option, including both
rose ``A'' and rose ``B'' properties, while 160000 entries (1186 entries
for the original dataset) describing the ``No buy'' alternative are
omitted, because their attributes are reduced to zeros in order to
achieve identifiability of the models (a complete presentation of
descriptive statistics par dataset and stratified by alternative may be
found in Appendix C). The distributions of \(Carbon\) footprint and
Eco-\(Label\) attributes follows perfectly the ones inside the original
dataset, although the prices differ. This particular divergence, may be
explained by the procedure implemented to assign prices to the
alternatives inside choice sets, because the random generator algorithms
different across statistical programs and potentially the procedures
implemented in \(R\) and \(SAS\) are not identical.

What is more interesting, is the difference in the \(Choice\)
statistics. We may be interested in comparing the statistics for
different classes in our sample to ensure that they are not biased in
favour of label ``A'' or label ``B'', as in this case it risks to bias
the estimates. For the artificial dataset the ratio of choices per
``Buy'' alternative is higher than 40\% and reaches 47.3\% for the fixed
effect utility (table \ref{tab:alt2}), while for the random effects
specification the numbers are lower, reaching only 42\% in mean for two
classes (table \ref{tab:alt3}). This particular observation is rather
interesting as it demonstrates how the heterogeneous effects for
alternatives' features the consumer decisions.

We will start with a close examination of the fixed effects dataset,
where we can see, that prices are not equally distributed among the
different choices.

\begin{table}[!htbp] \centering 
  \caption{Alternatives' descriptive statistics by group, fixed coefficients} 
  \label{tab:alt2} 
\begin{tabular}{@{\extracolsep{5pt}}lcccr}
\\[-1.8ex]\hline 
\hline \\[-1.8ex] 
 & A  & B  & Total  & p value\\
 & (N=160000) & (N=160000) & (N=320000) &  \\
\hline \\[-1.8ex] 
\textbf{Choice} &  &  &  & < 0.001\\
~~~Mean (SD) & 0.427 (0.495) & 0.518 (0.500) & 0.473 (0.499) & \\
~~~Range & 0.000 - 1.000 & 0.000 - 1.000 & 0.000 - 1.000 & \\
\textbf{Price} &  &  &  & < 0.001\\
~~~Mean (SD) & 3.069 (0.979) & 2.803 (0.917) & 2.936 (0.958) & \\
~~~Range & 1.500 - 4.500 & 1.500 - 4.500 & 1.500 - 4.500 & \\
\hline
\end{tabular}
\end{table}

The unbalanced prices potentially bias our dataset and we can see how
the option with inferior mean prices is chosen less frequently. Even
thought this differences do not affect the MNL and MMNL models, which
calculate average effects for all the alternatives, there may be an
impact over the performances of the NN models performances.

For the dataset with correlated random effects of the alternative
specific variables, we observe an identical situation in table
\ref{tab:alt3}. The class with lower average prices is chosen more
rarely by the consumers, while the overall choices are less frequent due
to the presence of stochastic individual preferences for particular
alternatives' attributes.

\begin{table}[!htbp] \centering 
  \caption{Alternatives' descriptive statistics by group, correlated random effects} 
  \label{tab:alt3} 
\begin{tabular}{@{\extracolsep{5pt}}lcccr}
\\[-1.8ex]\hline 
\hline \\[-1.8ex] 
 & A  & B  & Total  & p value\\
 & (N=160000) & (N=160000) & (N=320000) &  \\
\hline \\[-1.8ex] 
\textbf{Choice} &  &  &  & < 0.001\\
~~~Mean (SD) & 0.382 (0.486) & 0.462 (0.499) & 0.422 (0.494) & \\
~~~Range & 0.000 - 1.000 & 0.000 - 1.000 & 0.000 - 1.000 & \\
\textbf{Price} &  &  &  & < 0.001\\
~~~Mean (SD) & 3.069 (0.979) & 2.803 (0.917) & 2.936 (0.958) & \\
~~~Range & 1.500 - 4.500 & 1.500 - 4.500 & 1.500 - 4.500 & \\
\hline
\end{tabular}
\end{table}

We may conclude the preliminary datasets study and comparison with the
main impression that two artificial datasets may be assumed to be
quasi-identical. The slight differences in prices, captured by
statistical tests may be considered insignificant in comparison with the
biases present in the original dataset. What is more, even if the biases
were more significant, the models' specification, which assumes no
variable specific coefficients for choice A and B would have lead to the
correct estimates, exactly as it was done by Michaud, Llerena, and Joly
(2012). The heterogeneous preferences result in less probable decisions
to buy a rose in the population, which should definitely impact the
performances of our models. Now it rests to verify how well the number
of selected models will be able to derive the target values for the
relative utility function.

\FloatBarrier

\hypertarget{refs}{}
\leavevmode\hypertarget{ref-llerena2013rose}{}%
Michaud, Celine, Daniel Llerena, and Iragael Joly. 2012. ``Willingness
to pay for environmental attributes of non-food agricultural products: a
real choice experiment.'' \emph{European Review of Agricultural
Economics} 40 (2): 313--29. \url{https://doi.org/10.1093/erae/jbs025}.


\end{document}
