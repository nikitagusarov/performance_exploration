\documentclass[11pt,]{article}
\usepackage{lmodern}
\usepackage{amssymb,amsmath}
\usepackage{ifxetex,ifluatex}
\usepackage{fixltx2e} % provides \textsubscript
\ifnum 0\ifxetex 1\fi\ifluatex 1\fi=0 % if pdftex
  \usepackage[T1]{fontenc}
  \usepackage[utf8]{inputenc}
\else % if luatex or xelatex
  \ifxetex
    \usepackage{mathspec}
  \else
    \usepackage{fontspec}
  \fi
  \defaultfontfeatures{Ligatures=TeX,Scale=MatchLowercase}
\fi
% use upquote if available, for straight quotes in verbatim environments
\IfFileExists{upquote.sty}{\usepackage{upquote}}{}
% use microtype if available
\IfFileExists{microtype.sty}{%
\usepackage{microtype}
\UseMicrotypeSet[protrusion]{basicmath} % disable protrusion for tt fonts
}{}
\usepackage[margin=1in]{geometry}
\usepackage{hyperref}
\hypersetup{unicode=true,
            pdfborder={0 0 0},
            breaklinks=true}
\urlstyle{same}  % don't use monospace font for urls
\usepackage{graphicx,grffile}
\makeatletter
\def\maxwidth{\ifdim\Gin@nat@width>\linewidth\linewidth\else\Gin@nat@width\fi}
\def\maxheight{\ifdim\Gin@nat@height>\textheight\textheight\else\Gin@nat@height\fi}
\makeatother
% Scale images if necessary, so that they will not overflow the page
% margins by default, and it is still possible to overwrite the defaults
% using explicit options in \includegraphics[width, height, ...]{}
\setkeys{Gin}{width=\maxwidth,height=\maxheight,keepaspectratio}
\IfFileExists{parskip.sty}{%
\usepackage{parskip}
}{% else
\setlength{\parindent}{0pt}
\setlength{\parskip}{6pt plus 2pt minus 1pt}
}
\setlength{\emergencystretch}{3em}  % prevent overfull lines
\providecommand{\tightlist}{%
  \setlength{\itemsep}{0pt}\setlength{\parskip}{0pt}}
\setcounter{secnumdepth}{0}
% Redefines (sub)paragraphs to behave more like sections
\ifx\paragraph\undefined\else
\let\oldparagraph\paragraph
\renewcommand{\paragraph}[1]{\oldparagraph{#1}\mbox{}}
\fi
\ifx\subparagraph\undefined\else
\let\oldsubparagraph\subparagraph
\renewcommand{\subparagraph}[1]{\oldsubparagraph{#1}\mbox{}}
\fi

%%% Use protect on footnotes to avoid problems with footnotes in titles
\let\rmarkdownfootnote\footnote%
\def\footnote{\protect\rmarkdownfootnote}

%%% Change title format to be more compact
\usepackage{titling}

% Create subtitle command for use in maketitle
\newcommand{\subtitle}[1]{
  \posttitle{
    \begin{center}\large#1\end{center}
    }
}

\setlength{\droptitle}{-2em}

  \title{}
    \pretitle{\vspace{\droptitle}}
  \posttitle{}
    \author{}
    \preauthor{}\postauthor{}
    \date{}
    \predate{}\postdate{}
  
\usepackage{placeins}
\AtBeginDocument{\let\maketitle\relax}

\begin{document}

\hypertarget{performance-evaluation-and-comparison}{%
\subsection{Performance evaluation and
comparison}\label{performance-evaluation-and-comparison}}

This section comprises the results we managed to achieve in the
exploration of different performance metrics and provides insights on
the functioning of the discussed mathematical models in a given context.
As we have seen in the previous part, where the effects' estimates were
provided, all of the models are able to provide some estimates for the
retaliate utility function parameters in different discrete choice
set-ups. The most simple models performed well on the dataset defined by
the homogeneous preferences in the population for environmental
attributes, underestimating the effects in the presence of preference
heterogeneity. In the same time the more complex MMNL model performed
sufficiently well in both behavioural set-ups, although it demonstrated
some potential problems with the algorithmic implementation.

\hypertarget{overall-precision}{%
\subsubsection{Overall precision}\label{overall-precision}}

\FloatBarrier

First of all we focus our attention on the general performance metrics,
describing how well the estimated models fit the predicted outcomes over
an original dataset. As we have discussed earlier we use only some of
the available measures in an attempt not to make this work too
cumbersome. The retained performance metrics are: accuracy, describing
the overall goodness of fit over observed choices of the subjects; and
more complex KDL measure, which compares the distributions instead of
more simple metrics, which use only the information available in the
confusion matrix.

We can observe the values of these general performance measures,
describing overall performance of a given classifier in the table
\ref{tab:gpm}. The table regroups the metrics' values for all the
estimated models.

\begin{table}[!htbp] \centering 
  \caption{General performance measures} 
  \label{tab:gpm} 
\begin{tabular}{@{\extracolsep{5pt}} lcccccc} 
\\[-1.8ex]\hline 
\hline \\[-1.8ex] 
& \multicolumn{3}{c}{\textit{Fixed effects}} & \multicolumn{3}{c}{\textit{Random effects}} \\ 
\cline{2-4}\cline{5-7} 
\\[-1.8ex] & MNL & MMNL & CNN & MNL & MMNL & CNN \\ 
\hline \\[-1.8ex] 
\textbf{Overall measures} $ $ $ $ $ $ \\
~~~Accuracy & $0.863$ & $0.863$ & $0.723$ & $0.725$ & $0.863$ & $0.721$ \\ 
\textbf{Probabilistic measures} $ $ $ $ $ $ \\
~~~KLD & $0.623$ & $0.623$ & $0.328$ & $0.349$ & $0.625$ & $0.317$ \\ 
\hline \\[-1.8ex] 
\end{tabular} 
\end{table}

As we have underlined earlier we observe quite natural situation when
the best model in terms of overall performance is the model, which was
used in the data generation step. This situation perfectly demonstrates
the potential bias, which is explained by our choice of the artificial
data-generation algorithm. Nevertheless, it should be noted, that the
MNL and MMNL models perform equally well on the fixed effects dataset,
where the preferences for the environmental attributes are homogeneous.
This fact supports our initial hypothesis that an implementation of a
more complex model is preferred when the real effects are unknown to the
researcher.

Focusing our attention on the CNN model observe that the \emph{Adam}
algorithm did not outperform the \emph{BFGS} procedure. This observation
may be explained by the data-generation set-up, where the generative
algorithm favoured the MNL model, rather than \emph{Adam}. The latter
not supporting the fine tuning over the error distribution.

We can observe the results for the resources efficiency we managed to
obtain, which are regrouped in the table \ref{tab:time}. Even though we
present all the time values, we are mostly interested with the ``user''
and ``system'' time values. The first one indicates the CPU time charged
for the execution of user instructions of the calling process, while the
second one stand for the CPU time spent for execution by the system on
behalf of the calling process.

\begin{table}[!htbp] \centering 
  \caption{Ressources efficiency} 
  \label{tab:time} 
\begin{tabular}{@{\extracolsep{5pt}} lcccccc} 
\\[-1.8ex]\hline 
\hline \\[-1.8ex] 
& \multicolumn{3}{c}{\textit{Fixed effects}} & \multicolumn{3}{c}{\textit{Random effects}} \\ 
\cline{2-4}\cline{5-7} 
\\[-1.8ex] & MNL & MMNL & CNN & MNL & MMNL & CNN \\ 
\hline \\[-1.8ex] 
User & 20.910 & 452.414 & 17.433 & 18.722 & 2066.934 & 16.806 \\
System & 0.153 & 1.712 & 0.714 & 0.004 & 16.112 & 0.415 \\
Total & 21.068 & 454.192 & 8.412 & 18.726 & 2083.221 & 7.604 \\
\hline \\[-1.8ex] 
\end{tabular} 
\end{table}

The more advanced \emph{Adam} algorithm easily bypasses the algorithms
available in the \emph{mlogit} package, although this boost in
efficiency goes at the cost of lower overall performance and goodness of
fit. At the same time, the MMNL implementation is far less efficient and
takes 128 times more time, than CNN model. This situation clearly
illustrates us how the precision and flexibility come at higher costs.

\FloatBarrier

\hypertarget{alternative-specific-metrics}{%
\subsubsection{Alternative specific
metrics}\label{alternative-specific-metrics}}

We proceed with a look at some more specific measures. The table
\ref{tab:vspm} regroups response specific metrics, that describe the
precision of model in predicting only one target class of the dataset.
These metrics are mostly used when we are interested in some in-depth
insight into the model performance and allow to identify the models
which perform the best over a single class of interest. Given the
context of Michaud, Llerena, and Joly (2012) study we are interested in
identifying the algorithm which predicts the best ``buy'' (A and B
alternatives) against ``not buy'' (C) alternative, providing at the same
time some information about the alternative chosen. In order to evaluate
the performance at this dimension we use Geometric mean and the
F-measure performance estimators.

\begin{table}[!htbp] \centering 
  \caption{Variable specific performance measures, fixed effects data} 
  \label{tab:vspm} 
\begin{tabular}{@{\extracolsep{5pt}} lcccccc} 
\\[-1.8ex]\hline 
\hline \\[-1.8ex] 
& \multicolumn{3}{c}{\textit{Fixed effects}} & \multicolumn{3}{c}{\textit{Random effects}} \\
\cline{2-4}\cline{5-7} 
\\[-1.8ex] & C & A & B & C & A & B \\ 
\hline \\[-1.8ex] 
\textbf{Geometric mean} & & & & & & \\
  ~~~MNL & $0.454$ & $0.848$ & $0.868$ & $0.432$ & $0.696$ & $0.693$ \\
  ~~~MMNL & $0.454$ & $0.849$ & $0.867$ & $0.452$ & $0.848$ & $0.867$ \\ 
  ~~~CNN & $0.443$ & $0.697$ & $0.698$ & $0.447$ & $0.697$ & $0.700$ \\
\textbf{F-measure} & & & & & & \\
  ~~~MNL & $0.318$ & $0.834$ & $0.873$ & $0.282$ & $0.666$ & $0.704$ \\
  ~~~MMNL & $0.318$ & $0.834$ & $0.873$ & $0.316$ & $0.833$ & $0.873$ \\ 
  ~~~CNN & $0.291$ & $0.665$ & $0.706$ & $0.294$ & $0.665$ & $0.707$ \\
\hline \\[-1.8ex] 
\end{tabular} 
\end{table}

In the table \ref{tab:vspm} we are interested with the entries in the
columns corresponding to the ``No buy'' alternative (C). For the dataset
with fixed effects across the population, the MNL and MMNL models
perform identically according to both of the selected measures. The CNN
model falls behind the econometrics models on the fixed effects dataset,
although situation changes in the presence of heterogeneous effects. In
the more complex case scenario, when the individuals have varying across
population preferences towards one or another attribute, the CNN model
outperforms the simple MNL model in detecting ``No buy'' decisions for
given choice sets, which is rather interesting, because the overall
model's performance is still inferior to the MNL, as it was shown in
table \ref{tab:vspm}.

\FloatBarrier

\hypertarget{willingness-to-pay-and-premiums}{%
\subsubsection{Willingness to pay and
premiums}\label{willingness-to-pay-and-premiums}}

Here we should present the most important results comparing the
estimates for the WTP, as well as the premiums for particular attributes
derived for different models. The Premium to pay for a rose's particular
attribute as it was described previously can be represented as:

\begin{equation}
Premium = \frac{
  \frac{\delta V}{\delta X_k}
}{
   \frac{\delta V}{\delta Price}
}
\end{equation}

At the same time, the WTP for a rose may be seen as the ratio of two
corresponding coefficients of dummy variable and price. The table
\ref{tab:wtp} presents the estimated WTP and premiums for the models,
which output fixed coefficient estimates, without taking into account
the randomness of the individual effects. In other words, this table
regroups the results, which do not require bootstrapping for confidence
interval estimation.

\begin{table}[!htbp] \centering  
   \caption{WTP and Premiums obtained with MNL and CNN}  
   \label{tab:wtp}  
 \begin{tabular}{@{\extracolsep{5pt}} lccccc}  
 \\[-1.8ex]\hline  
 \hline \\[-1.8ex]  
& \multicolumn{2}{c}{\textit{Fixed effects}} & \multicolumn{2}{c}{\textit{Random effects}} & \multicolumn{1}{c}{\textit{Target}} \\
\cline{2-3}\cline{4-5} 
\\[-1.8ex] & MNL & CNN & MNL & CNN & \\  
 \hline \\[-1.8ex]  
 WTP & $1.421$ & $1.377$ & $0.747$ & $0.751$ & $1.401$ \\  
 Label & $1.731$ & $1.737$ & $1.445$ & $1.442$ & $1.731$ \\  
 Carbon & $4.091$ & $4.101$ & $3.679$ & $3.669$ & $4.086$ \\  
 LC & $4.112$ & $4.129$ & $3.378$ & $3.352$ & $4.110$ \\  
 \hline \\[-1.8ex]  
 \end{tabular}  
 \end{table}

For the estimation of the WTP and the premiums for more complex models
(the MMNL in our case) we use the same procedure, as was implemented by
Michaud, Llerena, and Joly (2012). Because the random parameters are
assumed to be correlated in the MMNL model's specification, the
estimated standard deviations and confidence intervals are obtained
using the Krinsky and Robb parametric bootstrapping method
({\textbf{???}}). This procedure consists of generating of multiple
random draws from a multivariate normal distribution and using the
obtained results to obtain the confidence interval estimates. Exactly as
in the original study we generate 1000 draws from a multivariate normal
distribution (\(MNV(\mu, \Sigma)\)), with the coefficient estimates as
means \(\mu\) and the estimated variance-covariance matrix of the random
parameters as \(\Sigma\).

The obtained results are then summarised as follows in the table
\ref{tab:wtpr}

\begin{table}[!htbp] \centering 
  \caption{WTP and Premiums obtained with MMNL} 
  \label{tab:wtpr} 
\begin{tabular}{@{\extracolsep{5pt}}lccccccc} 
\\[-1.8ex]\hline 
\hline \\[-1.8ex]  
& \multicolumn{6}{c}{\textit{Statistics}} \\
\cline{2-7} 
\\[-1.8ex] & \multicolumn{1}{c}{Mean} & \multicolumn{1}{c}{St. Dev.} & \multicolumn{1}{c}{Min} & \multicolumn{1}{c}{Pctl(25)} & \multicolumn{1}{c}{Pctl(75)} & \multicolumn{1}{c}{Max} \\ 
\hline \\[-1.8ex] 
\textbf{Fixed effects} & & & & & & \\
  ~~~WTP & 1.416 & 0.058 & 1.233 & 1.377 & 1.455 & 1.613 \\ 
  ~~~Label & 1.732 & 0.019 & 1.672 & 1.720 & 1.745 & 1.791 \\ 
  ~~~Carbon & 4.097 & 0.103 & 3.730 & 4.026 & 4.166 & 4.434 \\ 
  ~~~LC & 4.116 & 0.098 & 3.741 & 4.051 & 4.182 & 4.421 \\ 
\textbf{Random effects} & & & & & & \\ 
  ~~~WTP & 1.360 & 1.887 & $-$4.239 & 0.073 & 2.662 & 7.893 \\ 
  ~~~Label & 1.243 & 1.667 & $-$3.867 & 0.104 & 2.330 & 6.638 \\ 
  ~~~Carbon & 3.467 & 2.323 & $-$4.026 & 1.880 & 5.043 & 11.671 \\ 
  ~~~LC & 3.036 & 3.240 & $-$7.430 & 0.908 & 5.160 & 14.259 \\ 
\textbf{Target} & & & & & & \\
  ~~~WTP & 1.418 & 1.973 & $-$4.474 & 0.058 & 2.798 & 6.706 \\
  ~~~Label & 1.735 & 1.611 & $-$2.652 & 0.653 & 2.849 & 6.709 \\ 
  ~~~Carbon & 4.076 & 2.134 & $-$1.774 & 2.608 & 5.543 & 11.217 \\ 
  ~~~LC & 4.106 & 3.379 & $-$6.304 & 1.913 & 6.439 & 14.612 \\ 
\hline \\[-1.8ex] 
\textit{Note:}  & \multicolumn{6}{r}{The estimates are obtained with 1000 draws from MNV distribution} \\ 
\end{tabular} 
\end{table}

Comparing the estimates to the input values we observe that the variance
of the WTP and Premiums estimates, estimated over a fixed effects
dataset, do not potentially affect the conclusion one can derive from
the results. The values stay positive with the 75\% interval within 0.2€
of the mean estimate. Assuming the model is not re-estimated and
adjusted after the insignificant estimators are obtained for Choleski
matrix elements, the results remain valid.

We may conclude, that given sufficiently large dataset the
implementation of more complex model is preferable, because it will
allow to control for unknown parameters without adding a risk of
obtaining biased results. The more simple models, should be preferred in
a more restricted context. They allow to obtain the valid results only
in the case of correct theoretical assumptions, biasing the estimates in
other conditions. Consequently, in the presence of uncertainty about the
presence of heterogeneity in the customer choice modelling questions
there is a strong interest to implement a more complex model,
readjusting it afterwards if needed.

\FloatBarrier

\hypertarget{refs}{}
\leavevmode\hypertarget{ref-llerena2013rose}{}%
Michaud, Celine, Daniel Llerena, and Iragael Joly. 2012. ``Willingness
to pay for environmental attributes of non-food agricultural products: a
real choice experiment.'' \emph{European Review of Agricultural
Economics} 40 (2): 313--29. \url{https://doi.org/10.1093/erae/jbs025}.


\end{document}
