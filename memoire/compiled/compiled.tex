\documentclass[12pt,]{article}
\usepackage{lmodern}
\usepackage{setspace}
\setstretch{1.15}
\usepackage{amssymb,amsmath}
\usepackage{ifxetex,ifluatex}
\usepackage{fixltx2e} % provides \textsubscript
\ifnum 0\ifxetex 1\fi\ifluatex 1\fi=0 % if pdftex
  \usepackage[T1]{fontenc}
  \usepackage[utf8]{inputenc}
\else % if luatex or xelatex
  \ifxetex
    \usepackage{mathspec}
  \else
    \usepackage{fontspec}
  \fi
  \defaultfontfeatures{Ligatures=TeX,Scale=MatchLowercase}
    \setmainfont[]{XITS}
\fi
% use upquote if available, for straight quotes in verbatim environments
\IfFileExists{upquote.sty}{\usepackage{upquote}}{}
% use microtype if available
\IfFileExists{microtype.sty}{%
\usepackage{microtype}
\UseMicrotypeSet[protrusion]{basicmath} % disable protrusion for tt fonts
}{}
\usepackage[margin=1in]{geometry}
\usepackage{hyperref}
\hypersetup{unicode=true,
            pdftitle={Performance comparison of the discrete choice models of consumer choice},
            pdfborder={0 0 0},
            breaklinks=true}
\urlstyle{same}  % don't use monospace font for urls
\usepackage{color}
\usepackage{fancyvrb}
\newcommand{\VerbBar}{|}
\newcommand{\VERB}{\Verb[commandchars=\\\{\}]}
\DefineVerbatimEnvironment{Highlighting}{Verbatim}{commandchars=\\\{\}}
% Add ',fontsize=\small' for more characters per line
\usepackage{framed}
\definecolor{shadecolor}{RGB}{248,248,248}
\newenvironment{Shaded}{\begin{snugshade}}{\end{snugshade}}
\newcommand{\AlertTok}[1]{\textcolor[rgb]{0.94,0.16,0.16}{#1}}
\newcommand{\AnnotationTok}[1]{\textcolor[rgb]{0.56,0.35,0.01}{\textbf{\textit{#1}}}}
\newcommand{\AttributeTok}[1]{\textcolor[rgb]{0.77,0.63,0.00}{#1}}
\newcommand{\BaseNTok}[1]{\textcolor[rgb]{0.00,0.00,0.81}{#1}}
\newcommand{\BuiltInTok}[1]{#1}
\newcommand{\CharTok}[1]{\textcolor[rgb]{0.31,0.60,0.02}{#1}}
\newcommand{\CommentTok}[1]{\textcolor[rgb]{0.56,0.35,0.01}{\textit{#1}}}
\newcommand{\CommentVarTok}[1]{\textcolor[rgb]{0.56,0.35,0.01}{\textbf{\textit{#1}}}}
\newcommand{\ConstantTok}[1]{\textcolor[rgb]{0.00,0.00,0.00}{#1}}
\newcommand{\ControlFlowTok}[1]{\textcolor[rgb]{0.13,0.29,0.53}{\textbf{#1}}}
\newcommand{\DataTypeTok}[1]{\textcolor[rgb]{0.13,0.29,0.53}{#1}}
\newcommand{\DecValTok}[1]{\textcolor[rgb]{0.00,0.00,0.81}{#1}}
\newcommand{\DocumentationTok}[1]{\textcolor[rgb]{0.56,0.35,0.01}{\textbf{\textit{#1}}}}
\newcommand{\ErrorTok}[1]{\textcolor[rgb]{0.64,0.00,0.00}{\textbf{#1}}}
\newcommand{\ExtensionTok}[1]{#1}
\newcommand{\FloatTok}[1]{\textcolor[rgb]{0.00,0.00,0.81}{#1}}
\newcommand{\FunctionTok}[1]{\textcolor[rgb]{0.00,0.00,0.00}{#1}}
\newcommand{\ImportTok}[1]{#1}
\newcommand{\InformationTok}[1]{\textcolor[rgb]{0.56,0.35,0.01}{\textbf{\textit{#1}}}}
\newcommand{\KeywordTok}[1]{\textcolor[rgb]{0.13,0.29,0.53}{\textbf{#1}}}
\newcommand{\NormalTok}[1]{#1}
\newcommand{\OperatorTok}[1]{\textcolor[rgb]{0.81,0.36,0.00}{\textbf{#1}}}
\newcommand{\OtherTok}[1]{\textcolor[rgb]{0.56,0.35,0.01}{#1}}
\newcommand{\PreprocessorTok}[1]{\textcolor[rgb]{0.56,0.35,0.01}{\textit{#1}}}
\newcommand{\RegionMarkerTok}[1]{#1}
\newcommand{\SpecialCharTok}[1]{\textcolor[rgb]{0.00,0.00,0.00}{#1}}
\newcommand{\SpecialStringTok}[1]{\textcolor[rgb]{0.31,0.60,0.02}{#1}}
\newcommand{\StringTok}[1]{\textcolor[rgb]{0.31,0.60,0.02}{#1}}
\newcommand{\VariableTok}[1]{\textcolor[rgb]{0.00,0.00,0.00}{#1}}
\newcommand{\VerbatimStringTok}[1]{\textcolor[rgb]{0.31,0.60,0.02}{#1}}
\newcommand{\WarningTok}[1]{\textcolor[rgb]{0.56,0.35,0.01}{\textbf{\textit{#1}}}}
\usepackage{graphicx,grffile}
\makeatletter
\def\maxwidth{\ifdim\Gin@nat@width>\linewidth\linewidth\else\Gin@nat@width\fi}
\def\maxheight{\ifdim\Gin@nat@height>\textheight\textheight\else\Gin@nat@height\fi}
\makeatother
% Scale images if necessary, so that they will not overflow the page
% margins by default, and it is still possible to overwrite the defaults
% using explicit options in \includegraphics[width, height, ...]{}
\setkeys{Gin}{width=\maxwidth,height=\maxheight,keepaspectratio}
\IfFileExists{parskip.sty}{%
\usepackage{parskip}
}{% else
\setlength{\parindent}{0pt}
\setlength{\parskip}{6pt plus 2pt minus 1pt}
}
\setlength{\emergencystretch}{3em}  % prevent overfull lines
\providecommand{\tightlist}{%
  \setlength{\itemsep}{0pt}\setlength{\parskip}{0pt}}
\setcounter{secnumdepth}{5}
% Redefines (sub)paragraphs to behave more like sections
\ifx\paragraph\undefined\else
\let\oldparagraph\paragraph
\renewcommand{\paragraph}[1]{\oldparagraph{#1}\mbox{}}
\fi
\ifx\subparagraph\undefined\else
\let\oldsubparagraph\subparagraph
\renewcommand{\subparagraph}[1]{\oldsubparagraph{#1}\mbox{}}
\fi

%%% Use protect on footnotes to avoid problems with footnotes in titles
\let\rmarkdownfootnote\footnote%
\def\footnote{\protect\rmarkdownfootnote}

%%% Change title format to be more compact
\usepackage{titling}

% Create subtitle command for use in maketitle
\newcommand{\subtitle}[1]{
  \posttitle{
    \begin{center}\large#1\end{center}
    }
}

\setlength{\droptitle}{-2em}

  \title{Performance comparison of the discrete choice models of consumer choice}
    \pretitle{\vspace{\droptitle}\centering\huge}
  \posttitle{\par}
  \subtitle{Exploration of the Econometrics and Machine Learning model performances
in the presence of heterogeneous preferences and random effects
utilities}
  \author{\large Nikita Gusarov\\
\normalsize Master 2\\
MIASHS C2ES (UGA)\\
~\\
~\\
\hspace{50mm}\\
~\\
~\\
\raggedright Under supervision of:\\
\raggedright\hspace{10mm} Iragaël Joly, HDR (GAEL, UGA, Grenoble INP)\\
\raggedright\hspace{10mm} Beatrice Roussillon, MCF (GAEL, UGA)\\
\hspace{-40mm}}
    \preauthor{\centering\large\emph}
  \postauthor{\par}
    \date{}
    \predate{}\postdate{}
  
%%%%%%%%%%%%%%%%%
% Load Packages %
%%%%%%%%%%%%%%%%%

% Spacings
\usepackage{setspace}

% Tables
\usepackage{longtable}
\usepackage{tabu}

% Floats
\usepackage{morefloats}
\usepackage{float}
\usepackage{placeins}

% Highlighting
\usepackage{soul}

% Horizontal page position
\usepackage{pdflscape}

% Append pdfs
\usepackage{pdfpages}

% Add latex chunks
\usepackage{docmute}

% Short toc
\usepackage{shorttoc}
%\setcounter{tocdepth}{1}
%\usepackage{minitoc} - incompatible with document class

% Referencing mutliple things with a single command - \cref
\usepackage{cleveref}

% Array
\usepackage{array}

% Multiple columns
\usepackage{multicol}

% Image insertion and colors
\usepackage{graphicx}

% Latex comments
\newenvironment{dummy}{}{}

% Fonts
% \usepackage{fontspec}
% \setmainfont{Museo}

% drawing
\usepackage{tikz}
\usetikzlibrary{matrix,chains,positioning,decorations.pathreplacing,arrows}

\usepackage{dcolumn}
% \usepackage{subfig}
\usepackage[export]{adjustbox}
% \usepackage[demo]{graphicx}  
\usepackage{subcaption}
\usetikzlibrary{shapes,arrows}
\usetikzlibrary{arrows.meta}
\usepackage[edges]{forest}

\usepackage{calc}

%%%%%%%%%%%%%%%%%%%
% Make Title Page %
%%%%%%%%%%%%%%%%%%%

% Maketitle definition
\makeatletter
\def\@maketitle{
    \pagenumbering{gobble} % Ommit page numbering
    % Add logos
    \raggedright
    \includegraphics[height = 30mm]{../images/uga_logo.png}  
    \hspace{5mm}
    %\includegraphics[height = 30mm]{../images/gael_logo_1.jpg}
    \hspace{5mm}
    %\includegraphics[height = 30mm]{../images/lig_logo_1.png}
    % Add title
    \begin{center}
        \vspace*{\fill}
            {\Huge \@title}\\
            \par
            \rule{7cm}{0.4pt}
            \par
            \textbf{Exploration of the Econometrics and Machine Learning models' performances in the presence of heterogeneous preferences and random effects utilities}\\ %[10mm] % Subtitle
            \par
            \rule{3cm}{0.4pt}
            \par
            \textbf{Research master thesis}\\[10mm]
            {\Large \@author}\\
        \vspace*{\fill}
    \end{center}
    % Add supplementary information
    % \vspace{10mm}
    % {\large Sous diréction de : }\\
    % \hspace{10mm} {\large Directeur}\\ % A modifier ici
    % \vspace{20mm}
    % {\large Niveau d'études : }\\
    % \hspace{10mm} {\large Niveau}\\ % A modifier ici
    % {\large Parcours : }\\
    % \hspace{10mm} {\large Parcours}\\ % A modifier ici
    % \vspace{20mm}
    % Add bottom section
    \begin{center}
        {\large Université Grenoble Alpes}\\ % Université
        {\large Faculté d'Economie et Gestion - FEG}\\ % Faculty
        \vspace{5mm}
        2019 - 2020\\ % Year
    \end{center}
    \clearpage
}
\makeatother

\begin{document}
\maketitle

\begin{description}

\item[Abstract:] This works is a cross-disciplinary study of econometrics and machine learning (ML) models applied to consumer choice modelling. 
To breach the interdisciplinary gap an integrated simulation and theory-testing framework is proposed. 
It incorporates all essential steps from hypothetical setting generation to the comparison of various performance metrics. \vspace{0.1cm}\\ 
The flexibility of the framework in theory-testing and models comparison over economics and statistical indicators is illustrated based on the work of Michaud, Llerena and Joly (2012). 
Two datasets are generated using the predefined utility functions simulating the presence of homogeneous and heterogeneous individual preferences for alternatives' attributes. 
Then, three models issued from econometrics and ML disciplines are estimated and compared. \vspace{0.1cm}\\ 
This study shows the proposed methodological approach's efficiency, successfuly capturing the differences between the models issued from different fields given the homogeneous or heterogeneous consumer preferences. 

\item[Key words:] Consumer Choice, Preference Studies, Willingness to Pay, Econometrics, Data Science, Machine Learning, Classification Techniques, Synthetic Datasets

\item[Author:] Nikita Gusarov (UGA)

\item[Under supervision of:] Iragaël Joly, HDR (GAEL, UGA, Grenoble INP); Beatrice Roussillon, MCF (GAEL, UGA)

\vspace{0.5cm}
\hrule
\vspace{0.5cm}

\item[Abstrait:] Ce travail est une étude interdisciplinaire des modèles d'économétrie et d'apprentissage automatique (ML) appliqués à la modélisation des choix des consommateurs.
Pour briser la frontière interdisciplinaire, un cadre intégré pour tester des théorie est proposé.
Il intègre toutes les étapes essentielles de la génération de paramètres hypothétiques à la comparaison de diverses mesures de performance. \vspace{0.1cm}\\ 
La flexibilité du cadre dans les tests de théorie et la comparaison de modèles par rapport aux indicateurs économiques et statistiques est illustrée à partir des travaux de Michaud, Llerena et Joly (2012).
Deux ensembles de données sont générés à l'aide des fonctions d'utilité prédéfinies simulant la présence de préférences individuelles homogènes et hétérogènes pour les attributs des alternatives.
Trois modèles issus des disciplines économétrie et ML sont ensuite estimés et comparés. \vspace{0.1cm}\\ 
Cette étude montre l'efficacité de l'approche méthodologique proposée, en captant avec succès les différences entre les modèles issus de différents domaines compte tenu des préférences homogènes ou hétérogènes des consommateurs.

\item[Mots clés:] Choix du consommateur, \'Etudes de Préférences, Consentements à Payer, \'Econométrie, Science des Données, Apprentissage Automatique, Techniques de Classification, Données Synthétiques

\end{description}

\newpage

\begin{center}
\textbf{\Large Acknowledgements}
\end{center}
\vspace{2.3ex}

This work was accomplished with financial aid from Multidisciplinary
Institute in Artificial Intelligence (MIAI), supported by Sihem
Amer-Yahia, head of the SLIDE team at the LIG laboratory.

\vspace{10mm}

I would like to express my gratitude for the administrative and
technical support from the Grenoble Informatics Laboratory (LIG) and
Grenoble Applied Economics Laboratory (GAEL), which helped to fulfil
this work during COVID-19 crisis.

\vspace{10mm}

Credits for dataset generation algorithm go to Amirreza Talebijamalabad,
first year master student at Grenoble INP, who worked on the theory of
the artificial datasets generation.

\newpage

\pagenumbering{roman}

\setcounter{tocdepth}{1}

\shorttoc{Summary}{2}

\newpage

\pagenumbering{arabic}

\hypertarget{introduction}{%
\section*{Introduction}\label{introduction}}
\addcontentsline{toc}{section}{Introduction}

The advances in statistical learning, data analysis and data science of
the past decades have resulted in propagation of \emph{Machine Learning}
(ML) techniques to different applied fields, including social and human
sciences. Nowadays, it is impossible to imagine a field of science that
is not benefiting from the fruits of statistical learning. The works of
De Palma et al. (2011) and Cascetta (2009) on transportation modelling,
the publications of Molina and Garip (2019) dedicated to sociology
problematic, the articles of Coussement, Benoit, and Poel (2010)
concerning marketing decisions, actuary analysis studies (Denuit and
Trufin (2019), Denuit and Hainaut (2019)) or even psychology with an
example of Baayen et al. (2017) work reflect the literal omnipresence of
the newly developed techniques.

However, there exist two completely distinct approaches to applying
statistical learning, as described by Breiman and others (2001) and
latter by Athey and Imbens (2019): the \emph{Machine Learning} which
focuses on the predictive qualities (figure \ref{fig:parad3}) and
\emph{Econometrics} which attempts to decipher the underlying properties
of the data (figure \ref{fig:parad2}). In economics, where the research
is focused on hidden patterns exploration, the scientific community
prefers to implement the traditional econometrics techniques using the
more advanced statistical models only in some special cases or as some
assistance tools (Athey 2018). This discrepancy is explained by the fact
that econometrics, contrary to traditional ML paradigm focusses on the
accessibility of results. Consequently, many of the advanced ML
techniques rarely appear in economics publications because of their
believed lack of interpretability and excessive complexity in
application. Nevertheless, some multidisciplinary scientists make
attempts to breach this wall between \emph{ML} and \emph{Econometrics}:
Varian (2014), Mullainathan and Spiess (2017) or, among the most recent,
Athey and Imbens (2019). Their advances are mostly focused on resolving
the general interdisciplinary tool-set integration questions, without
considering the application specific details. Nevertheless, in the
attempt to breach the interdisciplinary barrier the details reveal
themselves to be of utmost importance in the solution of the problem.

\begin{figure}[hbtp]
\centering
\caption{The different paradigms}
\label{fig:parad}
\begin{subfigure}[c]{.4\linewidth}
    \centering
    \caption{Real world}
    \label{fig:parad1}
    \begin{tikzpicture}[box/.style = {draw, text width=2cm, align=center}]
        \node[box] (b) {Nature};
        \node[left=of b] (a) {$\mathcal{X}$};
        \node[right=of b] (c) {$\mathcal{Y}$};
        \draw[->] (a) -- (b);
        \draw[->] (b) -- (c);
    \end{tikzpicture}
\end{subfigure}\hspace{12pt}\vspace{12pt}

\begin{subfigure}[c]{.4\linewidth}
    \centering
    \caption{Econometrics}
    \label{fig:parad2}
    \begin{tikzpicture}[box/.style = {draw, text width=2cm, align=center}]
        \node[box] (b) {Theoretical\\model};
        \node[left=of b] (a) {$X$};
        \node[right=of b] (c) {$Y$};
        \draw[->] (a) -- (b);
        \draw[->] (b) -- (c);
    \end{tikzpicture}
\end{subfigure}\hspace{12pt}
\begin{subfigure}[c]{.4\linewidth}
    \centering
    \caption{Machine Learning}
    \label{fig:parad3}
    \begin{tikzpicture}[box/.style = {draw, text width=2cm, align=center}]
        \node[box] (b) {Nature};
        \node[box, below=of b] (d) {ML\\model};
        \node[left=of b] (a) {$X$};
        \node[right=of b] (c) {$Y$};
        \draw[->] (a) -- (b);
        \draw[->] (b) -- (c);
        \draw[->] (a) |- (d);
        \draw[->] (d) -| (c);
    \end{tikzpicture}
\end{subfigure}\hspace{12pt}
\end{figure}

There have already been a multitude of studies comparing the
performances of different econometric and ML models in various real
world scenarios: the study of machine learning methods to model the car
ownership demand estimation of Paredes et al. (2017), for example; or
the use of decision trees in microeconomics of Brathwaite, Vij, and
Walker (2017). However, there's no known to us work incorporating at
least all the baseline models, as it would require an unimaginable
amount of efforts to accomplish. For instance, in the literature the
performance of competing models are studied according to several
absolutely alien criteria: in terms of the quality of data adjustments,
in terms of predictive capacity, as well as in terms of the quality of
the economic and behavioural indicators derived from estimates and,
finally, according to their algorithmic efficiency and computational
costs. None of the known to us articles manages to incorporate all these
aspects into their benchmarks, limiting their studies only with several
performance criteria.

These various aspects, greatly impact the performance of particular
models or algorithms, although some of them are often ignored by the
researchers. Not only there exists inconsistency in the targeted
performance metrics in the contemporary models' comparisons, but there
is also omnipresent problems of theoretical background choice, dataset
selection or model's specifications. For example, speaking about the
datasets used to support their findings, many researchers explore the
impacts of different specifications on the same observed or simulated
choice situation (Munizaga and Alvarez-Daziano (2005), Fiebig et al.
(2010), McCausland and Marley (2013), Bouscasse, Joly, and Peyhardi
(2019)) as it appears to be the most theoretically reliable procedure.
However, there is still no established unified methodology documenting
this field.

From this unambiguity in the scientific community the main problematic
of this work arises. It is particularly important to establish a common
framework for performance comparison of the discrete choice models be
they from the econometrics or ML tool-set. However, this task cannot be
accomplished outside a precise context, which will potentially impose
some limitations over the models' structure, as well as influence the
choice of performance metrics. In economics the discrete choice models
are extensively used for consumer choice analysis (Anderson, De Palma,
and Thisse 1992), willingness to pay derivation (Michaud, Llerena, and
Joly 2012) and other preference studies. The field specific theories and
traditional research objectives frame and define this study's scope.

From the economics perspective there exist three major points of
interest to be taken into account. First, there is a strong interest in
economics to explore the different behavioural set-ups, under different
settings and assumptions. Secondly, given the different choice
situations there is a potential need to test how the available
mathematical models, potentially sensitive to the tested behavioural
hypotheses or dependent on these hypotheses, perform in a given context.
Last, but not least, a comprehensive implementation of a performance
evaluation methodology, combining reproducibility and control of
experimental conditions, should be introduced in the proposed framework.

\textbf{Consumer choice}

The economic decision theory derives mostly from the random utility
theory (RUM) of McFadden (1974) and more recently of McFadden (2001),
that were recently challenged by alternative visions such as random
regret minimisation theory (RRM) of Chorus (2010), with a related
relative advantage maximisation theory (RAM) of Leong and Hensher
(2015), or even quantum decision theory (QDT) of Yukalov and Sornette
(2017), which offers a wide range of tools for modelling under
uncertainty.

These different theories address various aspects of the decision making
process, under different suppositions and incorporating different
biases. For example, one of the basic assumptions of the traditional
choice theory is the transitivity of choice, meaning there exists a
strict hierarchy of individual preferences among alternatives. This
assumption may be unsuitable for real world choice situation and lead to
potential bias, which is addressed by quantum decision theory. QDT
manages to bypass this shortcoming and incorporate non-transitivity of
choices into the framework. There exist a multitude of other behavioural
elements unexplained by the most traditional models that may be
incorporated into the decision making framework, such as loss aversion
for example, that could be addressed with random regret minimisation
theory.

There is a particular interest in detecting the differences in the
models' performances depending on the choice context and the assumed
decision-making framework. It is important, because different consumer
behaviour in the individual choice context result in different choice
distributions, which may affect the models' performances. In economics
RUM theory is nowadays one of the most used choice settings in the
individual decision modelling. Nevertheless, there still exist some
unexplored limitations, that such theoretical framework may impose over
the estimation techniques, as well as to what potential biases a model's
misspecification may lead.

\textbf{Mathematical models}

In general any classification technique may be used to model individual
decisions, although nearly every model has some restrictions and
limitations, which may largely affect its performances in a given
context.

Usually the choice of model is rarely discussed in applied studies, as
the researchers tend to use either the simplest model possible or
attempt to implement one particular model of interest ignoring some
times the other possible choices. For example, many traditional
econometrics studies, given a multiple choice problem context, use a
multinomial logistic regression (MNL) or even simplify the problem to a
binary case, allowing to implement even more traditional models such as
binary logit or binary probit models. However, there exists a multitude
of particular cases in modelling individual choices, that require
specific techniques to be implemented. A family of duration models may
be used to model the individual decisions over time (Vitetta (2016));
network modelling that allows to incorporate spatial and social
dependencies for the explored data (Brock and Durlauf (2003));
preference learning techniques aiming to explore the positioning of
different alternatives by an individual (Tsoukiàs and Viappiani (2013),
Pigozzi, Tsoukiàs, and Viappiani (2016)) and many other advanced
techniques from \emph{machine learning} field such as neural networks or
support vector machines.

An incorrect choice of the modelling technique may have a strong impact
on the derived target values leading to some erroneous conclusions in
the end. For example, an incorrectly estimated willingness to pay for a
particular product may lead to significant losses. When conducting an
applied research study one should always be conscious of the eventual
biases introduced by the choice of the model and the eventual
consequences of these choices. Some models are not suitable to be
implemented on a particular set of data, while others are unable to
provide necessary information about the relationships within a
particular dataset or derive the particular target values of interest.

Taking into account the implications of RUM theory, there exists a
particular interest to make the focus on the state of art econometric
discrete choice models (Agresti (2013), Agresti (2007), Baltagi (2008),
Train (2009), McFadden (2001), McFadden (1974)) as well as their
counterparts used in ML (Hastie, Tibshirani, and Friedman (2009),
Kotsiantis, Zaharakis, and Pintelas (2006)). A comparison of some simple
models against more complex ones may reveal the trade-off between
precise estimates and the resources invested.

\textbf{Data}

Different sources of data are available for a researcher, that could be
divided into two groups (Japkowicz and Shah 2011): \emph{field
datasets}, which are gathered through an experiment or collected from
the real world observations or real world uncontrolled experiment; and
\emph{synthetic datasets}, which are artificially generated by the
researcher to suit his needs and respect some particular limitations.
Although this variability of dataset choices not that evident in the
context of applied studies, there is an ongoing debate concerning the
eventual impacts of data choice on the models' performances and
resulting metrics.

Given a task of performance evaluation and comparison for different
algorithms or mathematical models there is always a difficult choice of
the data type to be used in the study. Both of the mentioned above
dataset types have their advantages and disadvantages and require a
particular attention. However, having for objective the theory- and
model-testing framework construction there is a strong interest to use
the artificially generated data in order to have as much control as
possible over the situation.

\textbf{The framework and context}

Given these three key elements we propose an integrated simulation and
theory-testing framework which will encompass all the different aspects
of the model comparison task. The steps to be integrated into such
framework encompass many theoretical questions starting with the
underlying theoretical assumptions and ending with the choice of correct
performance metrics. Consequently, this work attempts to fill the gap
between two statistical paradigms: \emph{econometrics} and \emph{machine
learning}, taking into account the key elements, among which the
different combinations of decision theory assumptions, dataset
generation procedures, mathematical models and target performance
measures. The problematic arises from the insufficient points of contact
among researchers from different fields of applications, as well as
insufficiently unified methodology to put into relations the different
approaches. A work that uses unified knowledge from several disciplines
might be highly beneficial for the scientific community as it will lie a
foundation and provide support for future applied studies. Following the
logic of Athey (2018) and Mullainathan and Spiess (2017) the project
will attempt to merge the essentials of ML and econometrics paradigms,
retaining their key concepts in the context of consumer choice problem.

We propose to use an applied paper in econometrics of choice modelling
to facilitate understanding of the field of application and tools. This
means not that we will attempt to replicate the results, but rather to
use the context provided in the work for demonstration of the proposed
hypothesis-testing framework. We select the article of Michaud, Llerena,
and Joly (2012) as our reference paper, because of the advantages to
work directly with the authors of the paper. The work of Michaud,
Llerena, and Joly (2012) is focused on investigation of consumers'
willingness to pay (WTP) for environmental attributes of a non-food
agricultural products, taking roses as example. Authors constructed an
experimental framework to derive the premium the testing subjects were
ready to pay for such environmental attributes as lower carbon imprint
and ecological labelling, certifying the source of the environmentally
friendly practices. That study explored individual preferences for roses
with an eco-label and a carbon footprint using discrete choice modelling
techniques and real economic incentives resulting in real purchases of
roses. The gathered dataset was analysed with a mixed logit model
demonstrating notorious premiums for both attributes. We will benefit of
the obtained results to demonstrate all of the complexity of a proposed
theory-testing framework, its functionality and perspectives.

The present report is divided into two main parts. The first section
presents the chosen context for this work followed by short
presentations of all the theoretical aspects which play their major
roles in this study, tracing at the same time parallels with the
context. The second part presents the results of all the results
step-by-step, demonstrating the functionality of the designed framework.
Each of the sections has an identical logical structure of presentation
of the framework's components in the successive order: starting with the
behavioural modelling and data related questions, directly followed by
the models' presentation and the performance measures. The final section
concludes.

\newpage

\hypertarget{the-framework-design}{%
\section{The framework design}\label{the-framework-design}}

This section introduces the design and provides an example of available
functionality of an integrated experimental framework for model
performance exploration. In doing so, we strive to reduce and simplify
the framework, illustrating the theoretical discussion of the eventual
questions that arise during the model evaluation.

There exist multiple ways to provide an illustration for the generalized
framework due to its extended flexibility on different levels of
scientific procedure. Nevertheless, in our work we are attempting to
extend this illustrative objective to all the possible levels available
by the devised tool-set. The idea is to demonstrate all the features of
different frameworks' layers in the context of a performance comparison.
First of all, there is a particular interest to demonstrate the
advantages of possibility to test different choice settings, providing
different artificial datasets for exploration. What is more, it would be
interesting to contrast different mathematical models and algorithms
used to study these datasets and evaluate their performance using
different criteria, which will allow for more flexibility.

The work of Michaud, Llerena, and Joly (2012) investigates the impacts
of the environmental characteristics in the context of a consumer choice
of non-alimentary agricultural goods taking roses as an example. We will
inspire ourselves with the context, assumptions and findings of this
study and build our work around these pre-sets. We may be interested to
observe how some minor changes in the model may affect the results,
which pushes us to consider some simple, yet educative changes in the
model.

The organisation of this section is as follows. First of all, we
introduce in detail the context and discuss which features and
characteristics to retain give the Michaud, Llerena, and Joly (2012)
work. We will provide a description of the procedure adopted for this
illustration procedure as well. After brief overview of the original
article and delimitation of the general assumptions we will provide a
detailed discussion over every single major part of testing framework
with extensive argumentation. Starting with the presentation of the
underlying concepts of the decision theories and dataset generation
procedure we will continue with a discussion of different modelling
techniques and a detailed description of the models to be tested over
the artificial dataset. Finally, we will provide a panorama of the
performance assessment metrics, before switching over to application.

\hypertarget{context-willingness-to-pay-for-environmental-attributes-of-non-food-agricultural-products}{%
\subsection{Context: Willingness to pay for environmental attributes of
non-food agricultural
products}\label{context-willingness-to-pay-for-environmental-attributes-of-non-food-agricultural-products}}

We choose to use the estimated results of Michaud, Llerena, and Joly
(2012) as a starting point for our work, copying the context of the
study with some minor adjustments. In this part we will provide only a
general overview of the assumptions made in the article ``as is''. This
description will serve us as a reference for future discussion, because
afterwards we will be presenting our changes, modifications and
additions to these given our needs.

In the article of Michaud, Llerena, and Joly (2012) the choice of roses
as the non-food agricultural product was determined by several criteria.
Initially, roses were supposed by authors to have characteristics that
respect the limitations imposed by the experimental economics. These are
popular widespread products known to all the test subjects, being not
easily available at the same time. What is more, the production of roses
have been the object of a growing attention because of potential
environmental damages inflicted in the process. This last feature made
them a perfect product to explore the impacts of the environmental
properties on the consumer choice.

Two environmental aspects of roses' production were explored by Michaud,
Llerena, and Joly (2012). The first one, eco-labelling, described the
cultivation environment and conditions, including the use of pesticides,
fertilizers, as well as reasonable consumption of water and energy. This
labelling was adopted shortly before 2010 by some of the producers, who
attempted to reduce the harm to environment, to signal their
eco-responsible position to consumer. Authors mention such dedicated
eco-labels as the American \emph{VeriFlora} ``Certified Sustainably
Grown'' label guaranteeing the low environmental impact of roses'
production, or the European equivalent: ``\emph{Fair Flowers Fair
Plants}'' (FFP) label certifying the environmental performance of
agricultures by several criteria such as the ``\emph{fertilizer use,
crop production, energy efficiency, waste management and a number of
social requirements}''. The second chosen environmental feature of roses
was their carbon footprint, measured by the greenhouse gases emissions
during the cultivation and transportation. This criteria being
particularly important because of an increase of roses production in
developing countries in Africa, South America and Asia, which are later
sold on the European market, resulting in immense amount of CO2
emissions during the transportation.

The authors assumed that the individuals had heterogeneous preferences
for the environmental attributes of roses. In other words, it was
assumed that each individual had his personal attitude to the eco-label
and carbon footprint of the roses, determined by their awareness of the
environmental issues. The experimental design took into account this
assumed dimension through observation of multiple simultaneous choices
for each of the subjects in order to capture individual specific
elements. To model such complex repeated choice framework, authors used
well developed RUM behavioural theory (McFadden 2001) paired with the
power of the mixed logit model, which is a generalisation of a simple
logit model, allowing for more flexibility, such as random effects
modelling.

The assumptions made by the researchers may be roughly divided into two
categories, which will define the structure of this section. First one
comprises the behavioural assumptions concerning the decision-making
procedure, which encompasses the different restrictions on the
experimental design, individual's behavioural strategy and choice
preferences, which aim at elimination of various behavioural biases and
simplification of future mathematical analysis and data treatment. The
second regroups the assumptions related to the modelling process. It
encompasses theoretical assumptions imposing restrictions on the
mathematical model, its choice and estimation techniques. Finally, we
present the target effects computed by the researchers in the context of
the study, as the main objective was not the general approximation and
modelling of a consumer choice, but rather extraction of particular
values of interest such as willingness to pay for the alternatives'
attributes.

\hypertarget{experimental-design}{%
\subsubsection{Experimental design}\label{experimental-design}}

First of all, we should start with a description of the experimental
design framework introduced by the authors in order to obtain valid
results. This would allow to correctly implement such complex
econometric model as mixed logit on the next stage. The experimental
design assumed that individuals make their decisions based on the
perceived utility of a particular alternative, following the traditional
restrictions described by McFadden (2001).

Because the study collected data through a controlled experiment
setting, some restrictions were imposed on the observed characteristics
in order to simplify the analysis. The roses, as available alternatives,
were defined by three attributes observed by subjects:

\begin{itemize}
\tightlist
\item
  the FFP EU eco-label (\emph{Label})
\item
  the carbon footprint (\emph{Carbon})
\item
  the price (\emph{Price})
\end{itemize}

These attributes varied across the available options of the alternatives
present in different choice sets. Precise written instructions were
transmitted to the subjects making available information about the
criteria certified by the FFP labelling as well as some briefing about
the organization issuing this labels (the Horticultural Commodity
Board). These data-sheets provided as well a summary of Cranfield
University's report about roses' carbon footprint. Both these attributes
(eco-label and carbon footprint) were understood as a binary variables
valuing 0 or 1 depending on the presence of a particular attribute for a
particular rose. Finally, in addition to the two environmental
attributes, a price was introduced into experimental design, which
varied by 0.50€ between 1.50€ and 4.50€, creating this way a seven level
factor. The table \ref{tab:attributes} regroups the main characteristics
for these variables.

\begin{table}[!htbp] \centering 
 \caption{Alternatives' attributes} 
 \label{tab:attributes} 
\begin{tabular}{@{\extracolsep{5pt}}lcccccc} 
\\[-1.8ex]\hline 
\hline \\[-1.8ex] 
Statistic & \multicolumn{1}{c}{Levels} & \multicolumn{1}{c}{Min} & \multicolumn{1}{c}{Max} & \multicolumn{1}{c}{Step} \\ 
\hline \\[-1.8ex] 
Price & 7 & 1.5€ & 4.5€ & 0.5€\\ 
Label & 2 & 0 & 1 & 1 \\ 
Carbon & 2 & 0 & 1 & 1 \\
\hline \\[-1.8ex] 
\end{tabular} 
\end{table}

In order to avoid the substitution bias\footnote{The substitution bias
  occurs when the individuals tend to switch to the less expensive
  alternative available, given the relative prices changes. In the
  experiment context, the customers might have preferred to buy
  identically priced roses in a better placed store, instead of waiting
  for bought in experiment roses to be delivered.} as the subjects might
have decided to purchase a rose for the experiment somewhere else for a
lower price rather than in the laboratory a special measure was
introduced into experimental design. In the experimental literature the
implemented method is known as the ``\emph{field price censoring}'',
which means that the values used in the laboratory are censored
according to the field market price (Harrison, Harstad, and Rutström
2004).

The elicitation of the individual preferences for the different roses'
attributes was ensured through a combination of discrete choice
questions and real economic incentives. The stated choice surveys are a
popular choice for study of consumer preferences for public and private
goods. The discrete choice methodology and experimental design setting
provides the advantage to vary several attributes of a particular
product and to estimate the marginal rates of substitution between these
attributes. A particular accent was made on the derivation of the
willingness to pay (WTP) for specified features of interest. This tool
provides a great flexibility, allowing to test different scenarios all
of which could be presented in a single study, although, there is always
a danger that the choices made by consumers in experimental surveys
might not reflect their real preferences. The participants to
hypothetical surveys were generally stating higher WTP values for
private and public goods, leading to a potential bias in the estimates
when compared to the real world. Following this reasoning the authors
have introduced incentives into their choice experiment linking this way
the participants' decisions to real consequences by resulting in
acquisition of randomly chosen alternative from the pool of chosen
alternatives.

The choice set generation was devised with intention to resemble to
maximum the actual purchase decisions with the inclusion of a ``\emph{do
not buy}'' option, in order not to force the subjects to buy anything.
In other words, the presence of such alternative ensured that subjects
were never pushed to purchase a rose, imitating this way a real shopping
situation, when consumers always have the possibility of not purchasing
any roses if none of the alternatives suited them in a particular choice
set. Consumers were asked to make twelve different choices displayed.

In the case of prices allocation random design techniques were used to
configure the subsets of choice sets among subjects. The two level
factors standing for the roses' environmental attributes could be
regrouped into four different combinations defining four types of roses.
The experimental design introduced the roses in pairs to subjects,
creating this way several three alternatives choice sets. Even though
the different combinations of two roses potentially create sixteen
different alternative pairs, the authors limited their search to six
completely different pairs of roses. The resulting experimental sets of
six choice sets were repeated twice resulting in twelve cards, which
were then introduced to subjects. All the cards were distributed
simultaneously so that consumers could make their choices in any order.
Individuals were informed from the beginning that one of their decisions
would be randomly drawn at the end of the experiment. Finally, the
random draw resulted in the purchase of a real rose offered against
payment, this condition ensured that the subjects considered each choice
made during the experiment as a real purchase decision, weighting
carefully the available alternatives.

Generic titles were randomly allocated to the roses within choice sets:
rose A and rose B respectively. Such ``\emph{unbranded}'' alternatives'
titles allowed to ensure that they can only be differentiated according
to their attribute combinations. This way the choice between a
``\emph{Rose A}'' and a ``\emph{Rose B}'' can only be defined by their
attributes alone (Label, Carbon footprint and Price), but not by their
label. The same strategy applied to prices, which were randomly assigned
to the alternatives within the choice sets by a random number generator
setting prices within the defined limits.

Taking into account the experimental design we are going to follow the
authors' ideas in simulating an identical experimental design with
statistical methods available. The artificial choice situation will
assume three alternatives: two unlabelled ones doted with a common
utility function, while the third is the baseline alternative of ``no
choice'' option. In order to study the heterogeneity of the individual
preferences the subjects should be placed in a situation of repeated
choice, facing several choice situation. The alternatives will be
described by three attributes, while individuals will be distinguished
by four characteristics.

\hypertarget{econometric-model}{%
\subsubsection{Econometric model}\label{econometric-model}}

Consumers' decisions are analysed with the discrete choice framework
based on the utility maximisation assumption. This framework assumes
that consumers associate each alternative in a choice set with a utility
level and choose the option, which maximises this utility. The general
estimation framework of the Random Utility Model (RUM) proposed by
McFadden (1974) provides the opportunity to estimate the effects of
product attributes and individual characteristics and to compute
willingness to pay indicators.

Authors implemented the mixed logistic regression with random,
correlated attributes' effects to estimate the willingness to pay of the
individuals for each of the explored attributes of a rose. The mixed
logit model takes into account the repeated nature of the choices made
by the respondents. This model relaxes the Independence from Irrelevant
Alternatives (IIA) hypothesis of the more traditional multinomial logit,
allowing the random components of the alternatives to be correlated, at
the same time the error terms are still considered to be identically
distributed (Greene 2008). The alternative specific parameters are
assumed to be randomly distributed across the population contrary to the
fixed parameters specification for a traditional multinomial logit
model. In other words, the mixed logit model provides the opportunity to
consider heterogeneous effects among individuals by allowing taste
parameters to vary in the population. The authors suppose that the
random taste heterogeneity should be evident in response to the
eco-label and the carbon footprint attributes of the roses, because of
different level of environmental awareness across population. Following
the ideas of Bernard and Bernard (2009) the authors introduce the
cross-product for eco-labelling and carbon footprint as a random
parameter as well in attempt to test the effect of the simultaneous
presence of both of these attributes on consumer choice. This addition
results in a total of four random parameters to be estimated: the two
parameters describing roses attributes, their cross-product and the
``\emph{Buy}'' option dummy variable which captures heterogeneity in
consumers' preferences for a rose. All of the random parameters
associated with the roses' attributes are assumed to follow normal
distribution, which is traditional for the procedure of mixed logit
modelling. Given that the normal distribution is symmetric and
unbounded, the resulting model allows for both positive and negative
effects to exist inside population. To simplify the analysis and
assuming the reasoning of Revelt and Train (1998), the authors restrict
the price coefficient to be fixed in the population. Such choice of
price's effects specification ensures that all respondents have a
negative price coefficient, leading to a normally distributed estimate
of willingness to pay.

The systematic part of the utility relatively to the ``No buy'' option
was expressed through a linear in parameters form:

\begin{multline}
V_{ij} = \alpha_{i,Buy} + \beta_{Buy, Sex} Sex_i + \beta_{Buy, Age} Age_i + \beta_{Buy, Income} Income_i + \beta_{Buy, Habit} Habit_i + \\
+ \gamma_{Price} Price_{ij} + \gamma_{i, Label} Label_{ij} + \gamma_{i, Carbon} Carbon_{ij} + \gamma_{i, Label \times Carbon} Label \times Carbon_{ij}
\end{multline}

Where \(j\) was an alternative among the available choice set of three
options: buy rose A, buy rose B or do not buy anything. The dummy
variable \(\alpha_{i,Buy}\) was introduced to capture the effect of a
decision to buy a rose, while the vectors of \(\beta\) and \(\gamma\)
regrouped the effect of individual characteristics and the attributes of
alternatives respectively.

In their article Michaud, Llerena, and Joly (2012) did not provide an
extensive demonstration or description of the model selection procedure.
What is more, we have little information as to what model comparison and
validation techniques were implemented. Only the final model, chosen by
authors, was presented to us, which brings some limitations for our
study.

In the end of this subsection, it is important to highlight that the
mixed logit models are usually specified with uncorrelated random
effects, although it's not the case in the context of this particular
study. The authors introduce correlation between the normally
distributed alternative specific coefficients: \(\alpha_{i,Buy}\),
\(\gamma_{i, Label}\), \(\gamma_{i, Carbon}\) and
\(\gamma_{i, Label \times Carbon}\).

\hypertarget{willingness-to-pay-and-premiums}{%
\subsubsection{Willingness to pay and
premiums}\label{willingness-to-pay-and-premiums}}

The only target metrics present in the article were the willingness to
pay (WTP) and premiums for particular attributes. The former could be
read as the value the consumers are willing to pay for a rose. The
latter may be translated as how much consumers are ready to pay for a
unit change of a given attribute of the product. Both the WTP for a
product and the premiums can be computed as the marginal rates of
substitution between the quantity expressed by the attributes and the
price (Louviere, Hensher, and Swait 2000). The WTP for a rose in this
case could be expressed as:

\begin{equation}
WTP = \frac{
 \frac{\Delta V}{\Delta BUY}
}{
 \frac{\Delta V}{\Delta Price}
} = \frac{
  - \alpha_{Buy}
}{
  \beta_{Price}
}
\end{equation}

Where \(\frac{\Delta V}{\Delta BUY}\) is the difference in the relative
utility \(V\) associated with the ``Buy'' and ``No buy'' choices. The
premiums for the particular attributes \(Z_k\) of a given product could
be identically expressed as:

\begin{equation}
WTP = \frac{
 \frac{\Delta V}{\Delta Z_k}
}{
 \frac{\Delta V}{\Delta Price}
}
\end{equation}

Since the random parameters of the utility function were assumed to be
correlated, authors used Krinsky and Robb parametric bootstrapping
method (Krinsky and Robb 1986) with 1000 draws to estimate the standard
deviations and confidence intervals for these parameters.

\hypertarget{theories-of-consumer-choice}{%
\subsection{Theories of consumer
choice}\label{theories-of-consumer-choice}}

Once we have presented the assumed context for this study, we will dive
further into details and present all the key behavioural elements of
this work one by one. In this section we are going to present in detail
the questions and problematic associated with the behavioural modelling
of the consumer choice. Particularly, we are going to introduce the
terminology to be used in this work, some of which was already partially
presented in the previous section.

\hypertarget{general-terminology}{%
\subsubsection{General terminology}\label{general-terminology}}

For the presentation of general methodology we are going to adopt the
ideas of De Palma et al. (2011), introducing this way the principal
concepts and main components of the decision theory. Traditionally it
comprises several components: the decision makers or \emph{individuals},
described by their characteristics; a set (or sets) of available
\emph{alternatives}, defined by their attributes; and a decision rule or
set of rules, describing the procedure adopted by the individuals to
make actual decisions.

The individuals are supposed to have different tastes, and therefore we
must explicitly treat the differences in the decision-making processes
among individuals, doted with different characteristics. Therefore the
characteristics \(X_i\) of the decision maker \(i\) constitute an
important part of the problem.

The decision maker chooses from a finite and countable set of
alternatives \(\{\omega_i, \dots, \omega_j\}\), which consists of the
entire \emph{universal set of alternatives}
\(\{\omega_1, \dots, \omega_r \} \in \Omega\) as defined by the
particular choice environment. A decision maker \(i\) may only consider
a subset of this universal set \(\Omega\), and this consideration set is
conventionally named \emph{a choice set} \(\Omega_i\). In discrete
choice analysis, each alternative \(\omega_j\) is characterized by its
attributes \(Z_j\). For example, in the particular case study the
observed attributes of roses are their price, the eco-label and the
relative carbon footprint. Decision makers evaluate the attractiveness
of an alternative based on these attribute values before making their
choice.

Finally, the decision rule describes the process by which the decision
maker \(i\) evaluates the available information
\(Z_j \forall \omega_j \in \Omega_i\) and arrives at a unique choice.
There is a wide range of available decision rules, including dominance,
satisfaction, lexicographic, elimination by aspect, habitual, imitation,
and utility (De Palma et al. 2011). However, only the latter class is
most often associated with discrete choice analysis because to its
extensive use in the consumer choice behaviour modelling. The utility
theory takes its roots from the microeconomic consumer theory and is
adjusted according to the needs of the modeller. A utility \(U_{ij}\)
represents the attractiveness of a particular alternative \(\omega_j\)
for a particular individual \(j\) in a scalar form.

\hypertarget{random-utility-maximisation-models}{%
\subsubsection{Random utility maximisation
models}\label{random-utility-maximisation-models}}

The random utility maximisation models (RUM) were introduced and
developed by McFadden (1974). The theory of optimization implies that
this is a classical indirect utility function, with the following
properties: ``\emph{it has a closed graph and is quasi-convex and
homogeneous of degree zero in the economic variables}'' (McFadden 2001).
The last element in applying the standard model to discrete choice is to
require the consumer's choice among the feasible alternatives to
maximize conditional indirect utility based on some reference
alternative, rather than absolute utility.

In our work we use the notation introduced by Bhat (1995) and later
adopted by Cascetta (2009) when representing the utility functions as
they are more simple and easy to understand compared to initial McFadden
(1974) specification. The functional form of the canonical indirect
utility function depends on the structure of preferences, including the
trade-off between different available alternatives. The perceived
utility \(U_{ij}\) can be expressed as the sum of two terms: a
systematic utility and a random residual term:

\begin{equation}
U_{ij} = V_{ij} + \eta_{ij}
\end{equation}

Where \(U_{ij}\) stand for utility, \(V_{ij}\) at the same time
represent its deterministic part defined by some fixed deterministic
function and \(\eta_{ij}\) reflects some unobserved random effects. The
latter having being a random variable following Gumble distribution,
parametrized with \((\mu = 0, \theta = 1)\), which may be interpreted
as:

\begin{equation}
\eta_{ij} = - log( - log(\epsilon_{ij}))
\end{equation}

With \(\epsilon_{ij}\) a variable uniformly distributed and independent
across alternatives, the disturbances are independently identically
distributed Extreme Values (EV). This produces a MNL model in which the
systematic utility has a linear in parameters form for each alternative
\(\omega_j \in \Omega\). The systematic utility \(V_{ij}\) represents
the mean utility perceived by all decision-makers having the same choice
context decision-maker.

\begin{equation}
V_{ij} = f(X_i, X_j) + \eta_{ij}
\end{equation}

Traditionally in the most simple models this deterministic utility part
is represented by some linear in parameters function:

\begin{equation}
f(X_i, Z_j) = \alpha_j + \beta_j X_i + \gamma Z_j
\end{equation}

One family of RUM-consistent discrete choice models that is very
flexible is the random parameters or mixed multinomial logit (MMNL or
more often denoted as ML) model, which is used in the Michaud, Llerena,
and Joly (2012) work. The random parameters set-up assumes \(\gamma\)
effects to be randomly distributed across individuals, usually following
normal random distribution. In some cases, these parameters may be
assumed to be correlated, which potentially reflects better the real
world.

In our study we are going to explore two equally possible in real life
specification for data generation procedure: one assuming random effects
for alternative specific variables and another keeping these parameters
fixed. Speaking about the utility definition, we assume, that the work
of Michaud, Llerena, and Joly (2012) managed to obtain correct estimates
for a relative utility function of roses and we take this particular
function structure in order to generate utilities for a given dataset.
This assumption will offer us a baseline and target effects' values to
compare our estimation with.

\hypertarget{different-datasets-available-in-research}{%
\subsection{Different datasets available in
research}\label{different-datasets-available-in-research}}

There exist numerous difficult questions related to the models'
comparison task such as performance measures' choice or models'
specification, but beforehand there always stand the data related
questions. It is due to the fact that all the other questions and the
validity of the obtained responses rely entirely on the choice of the
inputs and the data available. Many of the existing applied econometrics
papers use the most simple specification of the Multinomial Logistic
Regression (MNL), that may lead to erroneous results and conclusions.

Many of the models' performances and performance measures depend on the
dataset properties and the particular application case. This means that
in comparison of different mathematical models, implementing some
complex tools such as a neural network models (NN), for example, we
should pay attention to use appropriate data-model to estimate such
model. This particular problem, as many others related to the models'
performance evaluation, was extensively described by Japkowicz and Shah
(2011).

When it comes to model comparison, the additional requirements arise to
the validation datasets and we should find answers to several questions:

\begin{itemize}
\tightlist
\item
  What datasets should be used?
\item
  Should the model be validated on one dataset or several several?
\item
  Should a synthetic or real-world data be used?
\item
  If several dataset are chosen, which ones should be used on different
  validation steps?
\item
  How the algorithms should be tuned face to the dataset selection?
\item
  What properties the studied data should have?
\end{itemize}

Moreover, different models may require different data adaptation methods
to be implemented. For example, the popular multinomial logistic
regression allows to take into account the individual characteristics as
well as the attributes of the various alternatives issued from some
limited set. The ML approaches, such as Support Vector Machines or
Linear Discriminant Analysis does not allow such flexibility. For these
models, even if we can represent each point in the modelled space as a
combination of individual characteristics and attributes of
alternatives, we can only classify the instances by iterative binary
separation (Tsoumakas and Katakis 2007). Consequently, the questions of
the dataset properties arises, which are tightly intertwined with the
available models choice and the implemented learning techniques.

In this section we will discuss the different existing approaches to
data management in theory testing and hypothesis verification. Firstly,
we will present the general questions and problematics. Then a solution
to be implemented in this particular study will be described and
discussed.

\hypertarget{theoretical-concerns-in-dataset-selection}{%
\subsubsection{Theoretical concerns in dataset
selection}\label{theoretical-concerns-in-dataset-selection}}

The data related problematic arise firstly during model generation step
of the standard statistical learning procedure and persists till the
stage of the model comparison. Speaking about the model validation, the
usual \emph{rule of thumb} approach is the cross-validation technique,
although some advanced users suggest that this method may not be always
appropriate (Japkowicz and Shah 2011). In econometrics, for example, as
well as in many other applied disciplines, researches tend to
oversimplify the validation step by completely avoiding this important
step, or by performing only \emph{single-fold} validation. On the other
hand, many advanced statistical model and ML methods require a separate
tuning step during model set-up, which alone requires verification and
validation on some dataset. It remains questionable whether the overall
model validation dataset and the dataset used for fine tuning should be
the same or not.

Of particular interest for our study is the ongoing discussion between
two sides of the statisticians' community, mentioned by Japkowicz and
Shah (2011), about whether the algorithms and statistical models should
be compared over the real world datasets or using some synthetically
generated data. On one hand, the datasets composed of the observations
or obtained through controlled experiments perfectly reflect the real
world situation, being at the same time too case specific. In other
words, it is always dubious that a model or a theory verified for one
particular real dataset has any external validity. The obtained insights
can rarely be extended over a larger population. Artificial data can be
designed in a controlled manner to study specific aspects of the
performance of algorithms and models. Moreover, the artificial data is
highly useful for testing particular theories, for example, the
behavioural theories or their impact on different models. Consequently
such data may allow for tighter control, which gives rise to more
carefully constructed and more enlightening experiments. Although, the
real data are hard to obtain and are difficult to analyse, the
artificial data introduces the danger of the problem's
oversimplification. In our case study these features are of utmost
importance, because, given the framework, the artificial data enables us
to test desired hypothesis in a controlled environment.

Generation of synthetic datasets is a common practice in many research
areas. Such data is often generated to meet specific needs or certain
conditions that may not be easily found in the original, real data. The
nature of the data varies according to the application filed and
includes text, graphs, social or weather data, among many others. In
this particular work we, for example, face the consumer choice data,
which describes individuals and their choice sets.\\
The common process to create such synthetic datasets is to implement
small scripts or programs, restricted to limited problems or to a
specific application.

As Garrow (2010) points it out, even observing the growing use of
artificial data in discrete choice and classification analysis, ``little
is known about how the methodology used to generate synthetic datasets
influences the properties of parameter estimates and the validity of
results based on these estimates''. That is, there are two potential
sources of biases when using synthetic discrete choice data:

\begin{itemize}
\tightlist
\item
  The unknown effect of the dataset generation method;
\item
  The parameter estimation bias.
\end{itemize}

The first one is rather complex and has many different element, that
could potentially affect the estimated results. There exist different
methods for artificial dataset generation, starting with use of
\emph{robots} (artificial observation instances) and ending with Markov
Chains Monte Carlo simulation and Neural Network use. One of the most
evident errors in this case could arise from the fact, that the closer
the estimated model is to the model implemented to generate the dataset,
the better would be the observed results, which may not be true in the
real world.

The second bias arises in the situation where the real world parameters
are used to generate artificial dataset, exactly as in this particular
work. The potential difference between the ideal simulated situation and
the real world situation lead to different choice structures. The
theoretical model supporting the data-generation process may be
potentially erroneous, leading to erroneous conclusions if only such
dataset was used for incorrect purpose.

\hypertarget{artificial-dataset-generation-procedure}{%
\subsubsection{Artificial dataset generation
procedure}\label{artificial-dataset-generation-procedure}}

For the objectives of this study we assume the best option is to
generate our own artificial dataset based on a predefined utility
function and given a predetermined statistical properties for individual
characteristics and alternatives' attributes. Such set-up ensures that
we know exactly the data generation process and have all the control
over the parameters and experimental design. As was mentioned above,
this choice may be dangerous in terms of justification of the resulting
external validity of obtained results in application to any other real
world dataset. However we ensure this way, that the obtained results
could be potentially compared with the baseline target parameters and
the initial effects are observed to us.

First step in the dataset generation is the generation of the
experimental design framework, imitating the original choice set-up, as
described in the Michaud, Llerena, and Joly (2012) article. Our first
steps are identical to the original work, as we start with the
generation of all possible combinations of binary factors for our
alternatives: roses described by two binary attributes and their price.
There exists only four different roses types, if described by their
binary attributes alone, as can be seen in the table \ref{tab:comb1}.

\begin{table}[!htbp] \centering 
 \caption{Possible attributes of roses} 
 \label{tab:comb1} 
\begin{tabular}{@{\extracolsep{5pt}}ccc} 
\\[-1.8ex]\hline 
\hline \\[-1.8ex] 
Type & \multicolumn{1}{c}{Eco-label} & \multicolumn{1}{c}{Carbon footprint} \\ 
\hline \\[-1.8ex] 
1 & 0 & 0 \\
2 & 0 & 1 \\
3 & 1 & 0 \\
4 & 1 & 1 \\
\hline \\[-1.8ex] 
\end{tabular} 
\end{table}

Given a multiple choice context when an individual is choosing among
three alternatives: two different roses, defined by labels A and B; and
a ``No buy'' option. Consequently there exist multiple possibilities to
regroup two roses into a choice set, for instance, in Michaud, Llerena,
and Joly (2012) are generating six choice sets ensuring that roses in a
given choice set always have different attributes, while in practice
there exist sixteen possible combination of two roses given they are
described by two binary factor variables. The choice of the choice set
delimitation in the article could be understood as the individuals
participating in the stated choice experiment are scarcely interested in
answering multiple questions, while six or twelve choices to consider
appear to be a reasonable number. On the contrary, our experimental
artificial set-up allows to ask as many questions to as many individuals
as we want. For example we can generate
\(7 \times 4 \times 7 \times 4 = 784\) choice sets for each individual,
containing all the possible combination of two different roses, each
described by two binary factor attributes as well as their price, which
has 7 different levels (varying by 0.50€ in a range from 1.50€ to
4.50€). However, such excessive set-up can have its toll on the
computation times, being in the same time absolutely unreasonable and
unrealistic, were we to replicate our results in a stated choice
experiment. Consequently, for price allocation we are going to implement
the same strategy as the authors of the article, meaning that the prices
will be randomly assigned inside the choice sets, while the choice sets
will follow a complete full-factorial design given two alternatives with
attributes. The following table \ref{tab:comb2} demonstrates this idea.

\begin{table}[!htbp] \centering 
 \caption{Choice sets attributes' combinations} 
 \label{tab:comb2} 
\begin{tabular}{@{\extracolsep{5pt}}ccccc} 
\\[-1.8ex]\hline 
\hline \\[-1.8ex] 
 & \multicolumn{2}{c}{Rose A} & \multicolumn{2}{c}{Rose B} \\ 
Choice set & \multicolumn{1}{c}{Eco-label} & \multicolumn{1}{c}{Carbon footprint} & \multicolumn{1}{c}{Eco-label} & \multicolumn{1}{c}{Carbon footprint} \\ 
\hline \\[-1.8ex] 
1 & 0 & 0 & 0 & 0 \\
2 & 0 & 0 & 0 & 1 \\
3 & 0 & 0 & 1 & 0 \\
\multicolumn{5}{c}{...} \\
15 & 1 & 1 & 1 & 0 \\
16 & 1 & 1 & 1 & 1 \\
\hline \\[-1.8ex]
\end{tabular} 
\end{table}

The prices are randomly allocated within given choice sets, although
there are some subtleties, which were discovered in attempt to replicate
the variability achieved in the original work. The main idea is to
ensure that both groups of roses (A and B) will have identical
characteristics, which is important for the later model estimation. At
the same time, we are interested in providing the test subjects with
identical choice sets to avoid eventual bias, which may be important if
we were facing a small number of observed individuals. Consequently, we
randomly allocate prices within a given choice set and distribute these
identical choice sets to all of the individuals. The variability in the
prices across alternatives is achieved through a replication of this
procedure \(n\) times. The resulting statistics and distribution will be
discussed in the second part of the work, where we will focus our
attention on the applied part.

On the next step we generate a population of ``\emph{robots}'', or
artificial individuals, who will be making their choices provided the
described above choice sets. It is important as well to mention, that
the distributions we use to generate the data are theoretical rather
than empirical ones. The individuals are generated based on the
descriptive statistics for population available in the reference paper.
This choice is done based on the final objective of the proposed testing
framework to allow the researchers to test and verify their hypothesis
related to the behavioural assumptions, modelling and performance
estimation in the consumer choice experimental context. We assume that
characteristics of the individuals are normally distributed, which is
rarely the case in practice, where skewed distributions are dominant.
Such choice imitates a replication attempt of a given empirical paper
given the information available in the article only, which are usually
the means and variances, rather than complete empirical distribution
descriptions.

Finally, having at our disposal a set of individuals as well as a number
of choice sets for the individuals to consider, we define the utility
function based on the estimates of the authors. Such choice implies,
that we assume all the hypothesis made when treating the original
dataset to be verified for the artificial model. The utility functions
are assumed as described in the preceding subsection to conform with the
standard random utility maximisation (RUM) definition as the individuals
are striving to maximise their perceived utility given their
characteristics and the observed attributes of the alternatives. The
utility is linear in parameters with additive error term.

Following this procedure we generate two synthetic datasets: one the
most basic one with only fixed effects present, while the other includes
random effects for the alternative specific attributes. These datasets
are then used to estimate, test and compare the models' performances.

To summarise, this section we will once again list the key hypothesis we
make in the artificial dataset creation:

\begin{itemize}
\tightlist
\item
  The dataset comprises:

  \begin{itemize}
  \tightlist
  \item
    4 individual characteristics (\(Sex\), \(Age\), \(Habit\) and
    \(Salary\))
  \item
    3 alternative's attributes (\(Price\), \(Label\) and \(Carbon\))
  \item
    2 product variables (\(Buy\) dummy variable and
    \(LC = Label \times Carbon\) cross-product)
  \end{itemize}
\item
  The individuals are assumed to maximise their utility, when making
  their choices, which corresponds to RUM behavioural framework;
\item
  The utility functions are linear, additive in parameters with an
  additive error term \(\epsilon\);
\item
  The error term is assumed to be iid. across population and follow a
  Gumble distribution: \(\epsilon \sim G(0, 1)\);
\item
  The individuals may (or may not) express heterogeneous preferences for
  the environmental attributes (eco-\(Label\) and \(Carbon\) footprint),
  which results in two different artificial datasets;
\item
  In the case of heterogeneous preferences a total of four random
  parameters are assumed to be correlated (\(Buy\) dummy, \(Label\),
  \(Carbon\) and their cross-product \(LC\)) and respect a multivariate
  normal distribution.
\end{itemize}

The detailed procedure of the choice modelling, as well as the exact
values of the parameters and some eventual difficulties in the dataset
generation are described in the applied section of this work.

\hypertarget{statistical-tools-for-choice-modelling}{%
\subsection{Statistical tools for choice
modelling}\label{statistical-tools-for-choice-modelling}}

As it was mentioned, there are different fields of application ranging
from \emph{econometrics} (Agresti 2013) to \emph{machine learning}
(Zielesny 2011), encompassing eventually such fields as transportation
systems analysis (Cascetta 2009) and logistics (De Palma et al. 2011),
actuarial science (Denuit and Trufin 2019), preference learning
(Fürnkranz and Hüllermeier 2010), psychology, sociology and more). The
more generalised models are regrouped under the \emph{statistical
models} label (Hastie, Tibshirani, and Friedman 2009), but nevertheless
they are mostly limited and are not taking into account many of the
field specific questions. Taking into account that our study is mostly
axed towards the study of the consumer choice data and related discrete
choice problems it is important to somehow limit the study's scope to a
number of selected models, without loosing the context.

Speaking about the econometrics models, this field of applied statistics
alone has a number of questions to answer before proceeding. For
example, we may question the particular task that we are performing
while applying the econometric models to some \emph{discrete choice}
problematic. Usually the economists are interested in deciphering and
understanding the underlying process (Athey and Imbens 2019), even
though there is a long lasting debate on the validity of obtained
measures as well as causality implications (Chen and Pearl 2013):
``\emph{The source of confusion surrounding econometric models stems
from the lack of a precise mathematical language to express causal
concepts.}'' This results in completely different cultures of the data
exploration and study objectives. This particular problem was largely
addressed by different researches, among which: Athey and Imbens (2019),
Mullainathan and Spiess (2017), Agrawal, Gans, and Goldfarb (2019),
Varian (2014) and Breiman and others (2001). Even as there are some
attempts to merge all the existing branches and approaches to
statistical modelling into some sort of a uniform culture (Donoho 2017),
the scientific community has a long route to make in order to achieve
this objective. There exist as well many other more subtle problems in
the econometric field. For example, different error term and different
link function specifications (Bouscasse, Joly, and Peyhardi 2019) in
econometrics models rise the question of what exactly we may consider as
single \emph{entry} to our list of models to evaluate.

On the other hand, speaking about the ML counterpart, the focus is
generally made on the predictive precision if we were to focus our
attention on the supervised ML sub-field (Mullainathan and Spiess 2017).
In their quest to achieve the best predictive precision with a
particular model, the \emph{machine learning} scientists study not only
the theoretical models themselves, but the algorithms used to estimate
these models (Zielesny 2011), that potentially augments the dimensions
to take into consideration in this particular work. Moreover, not only
there exist a confusion on what algorithms are to be associated with
each particular model (or potentially a number of models defined by
model/algorithm pairs), but many models are specified using a set of
hyper-parameters, which are to be chosen by the researcher. This aspect
immensely complexifies the task for us, as it is uncertain how exactly
should we define the values of these arbitrary chosen parameters. It's
worth mentioning that in many cases these parameters are case specific
and may vary from one application to another, resulting in different
performances over different datasets.

As it is mentioned by Kotsiantis, Zaharakis, and Pintelas (2006) the
choice of which specific learning algorithm to be implemented is a
critical step for any work, and a separate subset of training dataset is
usually used for this task. The classifier's evaluation is most often
based on prediction accuracy, which describes the percentage of correct
predictions among their total number, which requires some unrelated data
to be calculated as out of sample estimates provide more reliable
information about the performance of a particular algorithm.

This section will be opened by a brief introduction to the multitude of
the existing models, which is a particularly important point, given the
scope of the study. Each and every dataset, each and every relationship
between several variables may be modelled with different techniques and
different assumptions. There is a tremendous amount of work to be done
in order to systematise all the existing mathematical models, not
speaking about their extensions or their numerical implementations. The
first part of this section will demonstrate the complexity of the
models' choice given an application context. Only then, we are going to
present the selected models and their mathematical formulation: the MNL
model, the MMNL model and their artificial NN counterpart.

\hypertarget{taxonomy-of-statistical-models}{%
\subsubsection{Taxonomy of statistical
models}\label{taxonomy-of-statistical-models}}

Before proceeding with a discussion concerning eventual problems and
difficulties affecting the modelling part of every empirical study, we
will provide an overview of different families of models, encompassing
both the \emph{econometrics} and \emph{machine learning} fields. The
following presentation is a generalised vision of the existing discrete
modelling techniques, which can be used for classification tasks. As
general as it is, this part respect the setting of the discrete choice
behavioural modelling.

There exist several possibilities to divide ML algorithms into groups in
order to provide an exhaustive and complete taxonomy of this field and
the same reasoning may be applied to econometric models. However, the
existing taxonomies are rarely complete and focus mostly on one or
several grouping aspects. They define the general structure of a
particular taxonomy, but rarely take into account a sufficient number of
different descriptive features, which may vary across statistical
models. For example, we may take a look at Kotsiantis, Zaharakis, and
Pintelas (2006) work attempting to provide an overview of different
classification techniques on figure \ref{fig:kots}.

\begin{figure}
\centering
\caption{Taxonomy as proposed by Kotsiantis (2006)}
\label{fig:kots}
\begin{forest}
  for tree={
    align=center,
    edge+={ -{Stealth[]}},
    l sep'+=10pt,
    fork sep'=10pt,
  },
  forked edges,
  if level=0{
    inner xsep=0pt,
    tikz={\draw (.children first) -- (.children last);}
  }{},
  [Machine Learning 
    [Unsupervised]
    [Supervised
      [Logic-base\\techniques
        [Set of\\rules]
        [Inductive\\logic]
        [Decision\\trees]
      ]
      [Support Vector\\Machines]
      [Statistical\\techniques 
        [Neural\\Networks]
        [Instance-based\\learning]
        [Naive Bayesian\\networks]
      ]
    ]
  ]
\end{forest}
\end{figure}

This taxonomy is fairly simple and encompasses a large number of models'
families specifically designed for classification.\\
In the works of Hastie, Tibshirani, and Friedman (2009), Cascetta (2009)
and Ayodele (2010) we may see some more recent attempts to organise the
existing models into a single hierarchically related structure, although
neither of known to the author works offers sufficiently extended
reasoning over the relations between different classification techniques
(several of the resulting taxonomies could be seen in the Appendix A).
Moreover, not only the taxonomies may be based on the models'
themselves, but it can be constructed around their algorithmic
properties, as in Mullainathan and Spiess (2017). The resulting tree is
represented on the figure \ref{fig:mull}.

\begin{figure}
\centering
\caption{Taxonomy as proposed by Mullainathan (2017)}
\label{fig:mull}
\begin{forest}
  for tree={
    align=center,
    edge+={ -{Stealth[]}},
    l sep'+=10pt,
    fork sep'=10pt,
  },
  forked edges,
  if level=0{
    inner xsep=0pt,
    tikz={\draw (.children first) -- (.children last);}
  }{},
  [Machine Learning 
    [Combined\\predictors
      [Bagging]
      [Ensemble]
      [Boosting]
    ]
    [[Local/Nonparametric\\predictors
      [Kernel\\regression]
      [Decision\\trees]
      [Nearest\\neighbhors]
      [Random\\forest]
    ]]
    [Mixed\\predictors
      [Splines]
      [Neural\\Networks]
    ]
    [Global/Parametric\\predictors
      [Linear predictors\\(and generalizations)] 
    ]
  ]
\end{forest}
\end{figure}

In attempt to generalize the existing taxonomies and unite somehow the
different classification models and techniques, we may roughly divide
them in categories by different criteria. Usually there is no evident
hierarchical dependency between the different criteria, which immensely
complexifies the task of unified taxonomy construction.

First of all we may divide the models onto \emph{supervised} and
\emph{unsupervised} learning techniques (Hastie, Tibshirani, and
Friedman 2009), which is the most widely used model separation in ML
field. Sometimes this separation is complimented by various intermediate
combinations of these two. The supervised methods have the goal to
predict the value of an outcome measure based on a number of given input
measures, the outcome variable is available through the learning process
to guide the researcher and algorithm providing some baseline for
testing. In the statistical literature the inputs are often called the
predictors, the inputs, the features, or the independent variables. In
the econometrics the terms explicative or endogenous variables are more
popular. The outputs are denominated as responses, or, in econometrics,
the dependent or endogenous variables. The unsupervised learning is used
without any outcome measure available, with a main objective being to
describe the associations and patterns among a set of inputs. Such
formulation of a learning task is rather implemented to describe how the
data is organized or clustered, find the underlying patterns and
dependencies. As for the intermediate models' families, we may address
the article of Ayodele (2010), where authors present different mixed
types of learning tasks, although this particular classification is not
widely used. Among these models we find: \emph{semi-supervised}
learning, combining both labelled and unlabelled examples to generate an
appropriate function or classifier; \emph{reinforcement} learning, in
which algorithm learns to interact with the data generating source,
given an observation of the world, in this context every action of model
has some impact in the environment, and the environment provides
feedback that guides the learning algorithm; The \emph{transduction} is
nearly identical to supervised learning, although instead of an attempt
to construct a function it tries to predict new outputs based on
training inputs, training outputs, and new inputs; and finally
\emph{learning to learn}, when the algorithm learns its own inductive
bias based on previous experience, which is a more advanced
reinforcement learning problem.

Depending on the output variable structure we attempt to model we may
examine the taxonomy proposed by Agresti (2013). This taxonomy is based
on the output variable format: it may be either discrete or continuous.
The \emph{continuous} variables are the simplest case, where the output
is assumed to be continuous on a given interval and in the statistical
society is usually addressed as ``\emph{regression}'' task. It's
counterpart, the discrete dependent variable is sometimes addressed as
``\emph{classification}'' task and it is the focus of this particular
work. The \emph{categorical} variable has a measurement scale consisting
of a set of categories and these variables are of many types: binary
variables, nominal data, ordinal data or count variables. The
\emph{binary} data assumes that there exist only two categories, often
given the generic labels ``success'' and ``failure'' numerically
represented as 0 and 1. In the context of the undertaken study we may
imagine a binary variable representing the individual choice of ``Buy''
against ``No buy''. The \emph{nominal} variables represent categories
without a natural ordering and are measured on a nominal scale. The
perfect example for this data type is our choice set delimitation with
several unordered and independent options for individuals to consider:
buy rose A, buy rose B or do not buy anything. For nominal variables,
the order of listing the categories is irrelevant to the statistical
analysis, and the main importance is given by the choice of baseline
option, which is important for some of the statistical models.
\emph{Ordinal} data or ordered discrete data is an advanced
representation for nominal data, where many categorical variables do
have ordered categories, representing some given preferences order, for
example. For these variables, the distances between categories are
usually unknown and these intervals may be uneven between different
categories. An \emph{interval} variable is one that does have numerical
distances between any two values. For most variables of this type, it is
possible to compare two values by their ratio, in which case the
variable is also called a ratio variable. The final class if the
\emph{count} data, which is specific for special cases of
discrete-continuous data treatment.

By their structure the models may be separated into \emph{additive} and
\emph{non-additive} as described in Hastie, Tibshirani, and Friedman
(2009), both of which could be understood either as additive
(non-additive) in error term or having a full additive (non-additive)
structure. The first group encompasses different regression and
classification models where either the main function has additive
structure:

\begin{equation}
f(X) = E(Y \mid X)
\end{equation}

Or the error term is additive defining the following model:

\begin{equation}
Y = f(X) + \epsilon
\end{equation}

The \emph{non-additive} models, also denominated as
\emph{multiplicative} models, include all other eventual specifications
which could not be viewed or approximated by the additive relations.
This particular separation could be extended even further, as the models
could be viewed as \emph{linear} and \emph{non-linear} in their
parameters, or in their overall functional form. The former either
assume that the regression function \(E(Y \mid X)\) is linear, or that
the linear model is a reasonable approximation for the particular
situation. The non-linear models usually regroup the various extensions
and generalisations for the linear models integrating various non-linear
transformations.

One more possibility to separate different discrete choice models in
particular is by taking into account the probability structure they are
attempting to model as mentioned in Jebara (2004). The models are
separated into two major groups: generative and discriminative models,
to which sometimes a third ambiguous group of non-model techniques is
added. The \emph{generative} algorithms model the full structured joint
probability distribution over the examples and the labels given by
\(P(Y, X)\). The models in this context are typically cast in the
language of graphical models such as Bayesian networks. The joint
distribution modelling offers several attractive features such as the
ability to deal effectively with missing values, for example. On the
other hand, the \emph{discriminative} methods such as support vector
machines or boosting algorithms focus only on the conditional relation
of a label given the example, the probability being written as
\(P(Y \mid X)\). Their parametrized decision boundaries are optimized
directly according to the classification objective, encouraging a large
margin separation of the classes. They often lead to robust and highly
accurate classifiers.

The estimates structure differs across model families as well, as
described in Hastie, Tibshirani, and Friedman (2009). There are two
principal approaches to modelling given by \emph{parametric} estimators,
which are usually easy to read and interpret, and their
\emph{non-parametric} counterpart, offering the best results in terms of
precision in most cases. The multitude of non-parametric regression
techniques or learning methods can be separated into a number of classes
by the nature of the restrictions imposed, although we are not going to
provide an extensive description of all of them. What is more important,
that there exist different families of mixed models, profiting from both
the parametric and non-parametric feature. They are traditionally
regrouped into a single family of \emph{semi-parametric} models.

In this work we face a classification task which can be understood,
given the context, as consumer choice modelling. In order to correctly
model the consumer choice structure we will need to use the models
allowing to work with nominal discrete data, because the consumer
choices can not be positioned in some logical order defining a
continuous variable. The desire to obtain some explanatory results leads
us to restrict our choice to some additive and, moreover, linear models,
which would identify the parameters of a given relative utility
function. The latter argument implies that the models should be
parametric, producing some exact estimates for given set of parameters.

\hypertarget{description-of-models-to-be-compared}{%
\subsubsection{Description of models to be
compared}\label{description-of-models-to-be-compared}}

For our particular demonstrative task, which is restricted by the
context of the study of Michaud, Llerena, and Joly (2012), we have
already described the advantages and reasons behind the unrelenting
theoretical assumptions concerning the behaviour of individual, as well
as the dataset generation procedure. The two resulting datasets allow us
to explore the effects of the random effects of the alternatives'
attributes on the modelling. This possibility is particularly important,
as usually researchers ignore the possibility of random effects presence
in the population and use more simple and conventional multinomial
logistic models to model various discrete choice situations. However, we
are not going to test only one model over the obtained dataset, but
rather introduce several models with different specifications in order
to demonstrate a vast potential of our testing framework and its
advantages for research.

As we are exploring an over-simplified framework, we are going to study
first two different traditional models each perfectly adapted to model
one of the two generated datasets respectively. We are speaking about
the multinomial logistic regression, which should yield perfect fit
results on a fixed effects dataset and its counterpart - the mixed
multinomial logistic regression, which should be the most performant in
the presence of random effects in the utility functions. Many of the
existing applied econometrics papers use the most simple specification
of the Multinomial Logistic Regression (MNL), that may lead to erroneous
results and conclusions in the presence of random coefficients.
Eventually these models will allow us to verify, whether or not we are
able to obtain the same results as at the input.

What is more, as the main objective of this work is to demonstrate
proposed framework's flexibility, we are going to show how a completely
alien model to econometrics, such as neural networks model, may be
explored and compared with more traditional tools. More precisely, we
are going to use a neural networks imitating the procedure of the
multinomial logistic regression, while the other will be more
traditional multilayer neural network. It is because this model can be
viewed as an even wider generalisation of the generalised additive
models (GAM), that it is possible to simulate a model similar to MNL and
MMNL models. This choice was made because the seemingly identical model
by its structure may produce different results, depending on the
implemented estimation technique. The NN techniques offer us a great
number of different algorithms which are more advanced than the
algorithms traditionally implemented in econometrics, which make us
wonder, whether the changes in the estimation algorithm will allow us to
achieve better results.

In this part we will attempt as well to introduce some common notation
for the different models' families, issued from different disciplines.

\hypertarget{logistic-regressions}{%
\paragraph{Logistic regressions}\label{logistic-regressions}}

Multi-category logit models simultaneously use all pairs of categories
by specifying the odds of outcome in one category instead of another
(Agresti 2007). As described in Agresti (2013), many applications of
multinomial logit models relate to determining effects of explanatory
variables on a subject's choice from a discrete set of options.

\textbf{Multinomial Logit}

Even if in the original article of Michaud, Llerena, and Joly (2012) a
Mixed Logit model is used, here we start our study with an introduction
of the multinomial logistic regression (MNL) model, assuming the fixed
effects presence. This model will allow us to contrast the performances
in case of both fixed and random effect theoretical assumptions and
compare them with a more advanced version of mixed multinomial logistic
regression and NN model. This assumption is relaxed in the Mixed Logit
model (ML or MMNL), where coefficients (or some of them) vary by
individual (Agresti 2013). The logistic regression models are derived
from GLM specifications (Agresti 2007):

\begin{equation}
g(\mu_i) = \sum_r \beta_r x_{ir}
\end{equation}

Where \(g(.)\) is a link function, which is a logistic transformation
for binary logistic model. It is important to say that in this
theoretical introduction we ignore in some extent the previously
introduced terminology: \(i\) still denotes the individual observations,
laying in range of \(\{1, \dots, N \}\) in this case; the \(r\) index
here stands for different variables, because we do not use matrix
notation for the reasons of simplicity.

Here we propose the econometric specification of a \emph{multinomial
logit (MNL)} model as described by Cascetta (2009). The MNL model is one
of the simplest \emph{random utility model (RUM)} (McFadden 1974). This
class of models relies on the hypothesis, that an individual \(n\)
maximises his perceived utility over a set of alternatives \(\Omega\),
his utility determined by a fixed and a random parts, as described
earlier:

\begin{equation}
U_{ij} = V_{ij} + \eta_{ij} \text{ where } V_{ij} = \alpha_j + \beta_j X_i + \gamma Z_j
\end{equation}

Both \(\beta\), representing the alternative specific individual
coefficients, and \(\gamma\), standing for population-wide attributes
effects, are assumed to be fixed across population, meaning that all the
individuals have identical preferences and are subject to identical
effects. As precise in Agresti (2013) this approach enables
discrete-choice models to contain characteristics of the chooser and of
the choices. It offers the model an immense flexibility. The MNL is
based on the assumption that the residuals \(\eta_{ij}\) are identically
and independently distributed (iid.) as Gumbel random variables with
zero mean and scale parameter \(\theta\), which is usually equal to 1
(\(\theta = 1\)). This calibration is done due to computational reasons,
which will be explained later in this part.

One of the key concepts when it comes to modelling of the described
above process is the \emph{latent variable} notion. The latent variable
\(Y\) corresponds to its more meaningful counterpart \(V\) and is
sometimes understood as probability to choose a particular alternative.
Obviously, as in the experimental context we are unable to observe the
real choice probabilities, this variable takes values 0 or 1 depending
on whether or not a particular alternative was chosen:

\begin{equation}
Y_ij = I(V_{ij} > V_{il} | j \neq l, \forall l \in \Omega_i)
\end{equation}

Under the assumptions made here, the probability of choosing alternative
\(\omega_j\) from among those available
\(\{\omega_1, \dots, \omega_k\} \in \Omega\) by individual \(i\), can be
expressed in closed form as:

\begin{equation}
P_{ij} = \frac{
    e^{V_{ij} / \theta}
}{
    \sum_{l = 1}^{k} e^{V_{il} / \theta}
}
\end{equation}

The probability structure incorporates the theoretical assumptions of
the finite choice set, the uniqueness of the chosen alternative and the
idea of utility maximisation. In a more comprehensive form, we may say
that an individual chooses a particular alternative \(\omega_j\) or
simply \(j\) among all available for him alternatives \(\Omega_i\) only
if its utility is higher than any others' alternative utility:

\begin{equation}
P_{ij} = P(\eta_{il} - \eta_{ij} < V_{ij} - V_{il}) 
    \forall l : l \neq j, l \in \Omega_i
\end{equation}

Knowing the structure of \(V_{ij}\) and assuming the \(\theta\)
parameter for Gumble distribution of \(\eta\) is \(1\) we may rewrite
the probability as:

\begin{equation}
P_{ij} = \frac{
    e^{\alpha_j + \beta_j X_i + \gamma Z_j}
}{
    \sum_{l = 1}^{k} e^{\alpha_l + \beta_l X_i + \gamma Z_l}
}
\end{equation}

The alternative \(\omega_j\) in such case is denoted as reference
alternative or baseline alternative and is subject to several
restriction for the sake of identifiability. The most important one is
that we can not identify all the parameters in the probability function,
which require us to impose some restrictions over effects structure.
Traditionally (Agresti 2013) the reference level coefficients are
assumed to be 0, reducing this way the number of parameters to estimate.
This choice has some important consequences for the models'
interpretation, because the estimated effects for other alternatives in
this case should be treated as differences between the actual effects
for the baseline alternative and other alternative respectively. The
estimated parameters are in fact:

\begin{equation}
V_{ij} - V_{il} = (\alpha_j + \beta_j X_i + \gamma Z_j) - (\alpha_l + \beta_l X_i + \gamma Z_l)
\end{equation}

Where \(l \neq j\) and \(j, l \in \Omega_i\). Which could be transformed
into:

\begin{equation}
V_{ij} - V_{il} = (\alpha_j - \alpha_l) + (\beta_j - \beta_l) X_i + \gamma (Z_j - Z_l)
\end{equation}

At this stage an important remark should be made, which concerns the
understanding of individual characteristic effects and alternatives'
attributes effects. It is theoretically possible to estimate a common
individual effect for all the alternatives should we only wish to. The
main idea lies in the correct parametrisation of the initial framework.
To achieve identifiability for the individual characteristic specific
effects we should observe enough within choice set variance, as
otherwise the resulting singularity will incapacitate us to perform the
estimation. In other words, we can understand this procedure as manually
setting the individual effects to 0 for our baseline alternative and
estimating the resulting model. Speaking about the changes in the
dataset, the described above procedure is strictly equivalent to setting
the baseline alternative's individual characteristics vector to zeros
and estimating the resulting feature matrix as alternative specific
attributes.

The traditional vision of alternative specific individual
characteristics effects, assuming \(\beta_j = 0\), is:

\begin{equation}
(\beta_j - \beta_l) X_i = - \beta_l X_i \text{ if } \beta_j = 0
\end{equation}

The analogous vision for alternatives' attributes effects, when
reference attribute \(Z_j\) is set to 0 is:

\begin{equation}
\gamma (Z_j - Z_l) = - \gamma Z_l \text{ if } Z_j = 0
\end{equation}

As we can see \(\beta_l\) and \(\gamma\) parameters are roughly
equivalent in these two cases, assuming we are interested in means over
the set of individuals \(N\) and alternatives \(\Omega\).

\begin{equation}
E_{il} (- \beta_l X_i) = E_{il} (- \gamma Z_l) \forall i \in N, \forall l \in \Omega
\end{equation}

Which under transformation equals to:

\begin{equation}
- E_{l} (\beta_l) E_{i} (X_i) = - \gamma E_{l} (Z_l)
\end{equation}

Assuming \(X\) and \(Z\) here is the same variable, varying across
individuals and characteristics (\(Z_j = 0\)), we obtain that:

\begin{equation}
- E_{l} (\beta_l) X = - \gamma Z \Rightarrow E_{l} (\beta_l) = \gamma
\end{equation}

This could be empirically confirmed through estimation of two different
specifications and aggregation of obtained results.

However, were we in need to estimate an individual for all the
alternatives except the baseline one, we could benefit from this
transformation to do so. Such transformation allows us to take the
multiple choice context of the expiremental setup.

\textbf{Mixed Multinomial Logit}

Following Agresti (2007) presentation, generalized linear models (GLMs)
extend ordinary regression by allowing non-normal responses and a link
function of the mean. The generalized linear mixed model, denoted by
GLMM, is a further extension that permits random effects as well as
fixed effects in the linear predictor. We begin with the most common
case, in which is an intercept term in the model.

\begin{equation}
g(\mu_i) = \sum_r \beta_{ir} x_{ir}
\end{equation}

Where \(\beta_i\) is issued from some multivariate distribution.
Traditionally this distribution is assumed to be a multivariate normal
distribution (MNV) giving:

\begin{equation}
\beta_i \sim MNV(\beta, \Sigma)
\end{equation}

In more recent work of Agresti (2013) the more advanced models are
described. The multinomial logit and probability based discrete-choice
models can be further generalized by treating certain effects as random
rather than fixed.\\
A mixed logit model is the one in which choice probabilities are
obtained by integrating the logistic expression for choice probabilities
with respect to a distribution for certain model parameters. This allows
heterogeneity among subjects in the size of effects. It is useful as a
mechanism for inducing positive association among repeated responses
with panel data. Estimates of the parameters of the mixing distribution
provide information about the average effects and the extent of the
heterogeneity. Individual effects can also be predicted using this
technique.

The Mixed Logit is a further development and generalisation of a
traditional MNL and Conditional Logit models, because both of these
models may be constructed using Mixed Logit specification with a correct
parametrisation. The main difference from the more simple models is that
in this case it is assumed that effects vary across population and might
even be correlated. The utility specification in this case is
constructed identically to simple models, but the deterministic part
assumes that effects vary across population:

\begin{equation}
U_{ij} = V_{ij} + \eta_{ij} \text{ where } V_{ij} = \alpha_j + \beta_j X_i + \gamma_i Z_j
\end{equation}

Mathematically the random effects specification is achieved through the
parameter vector \(\gamma_i\), which is unobserved for each \(i\). The
\(\gamma\) in this case is assumed to vary in the population following
the continuous density \(f(\gamma_i \mid \theta)\), where \(\theta\) are
the parameters of this distribution. The simplest choice of the
distribution for the random effects is the normal distribution, which
was used by Michaud, Llerena, and Joly (2012), or more precisely a
multivariate normal distribution, because authors took into account the
correlation between coefficients:

\begin{equation}
\gamma_i \sim MVN(\gamma, \Sigma)
\end{equation}

In this case the vector of alternative specific effects can be
represented as:

\begin{equation}
\gamma_i = \gamma + L \sigma_i
\end{equation}

Where \(\sigma_i \sim N(0, I)\) , and \(L\) is the lower-triangular
Cholesky factor of \(\Sigma\) knowing which, the actual
variance-covariance matrix for random effects can be derived, as
presented in Croissant (2020):

\begin{equation}
LL^T = V(\gamma_i) = \Sigma
\end{equation}

Here we do not present the eventual possibility to incorporate the
individual specific characteristics covariates into the given framework,
because we will not use it, but such possibility is definitely worth
mentioning.

Where \(\beta\) are some fixed mean effects across population and
\(\psi\) stand for the random part with \(0\) mean and some imposed
variance-covariance structure, as it is technically possible to assume
that only some of the effects are random.

A more advanced description of MMNL models is available in the work of
McFadden and Train (2000), where some intuitions are given on the
estimation techniques necessary to evaluate such complex model. The
authors suggest, that numerical integration or approximation by
simulation is needed to evaluate MMNL probabilities. Maximum Simulated
Likelihood (MSLE) or Method of Simulated Moments (MSM) could be used to
estimate the MMNL model in practice, both of which are described in the
reference work (McFadden and Train 2000)

\hypertarget{neural-networks}{%
\paragraph{Neural Networks}\label{neural-networks}}

The second group of models focuses on more advanced and atypical
modelling techniques rarely implemented by the economists in their
studies, as usually this family is perceived as not offering enough
insight when it comes to the effects estimation. The ML techniques are
usually viewed by economists as some black boxes, which do not provide
any information about the underlying process. It is quite easy to comply
with their position, as even though the most advanced techniques perform
better in terms of predictive power, they rarely offer any insight into
the modelling process.

For this particular part we use the model's specifications described in
the handbook of Hastie, Tibshirani, and Friedman (2009) with some
additions and modifications, which aim at integration of this particular
specification in conformity with the specifications of the econometric
discrete model notation. \emph{Neural Networks (NN)} represent an
advanced class of models, being a further complexification of the
\emph{generalised additive models (GAM)}, which are a generalisation of
the \emph{generalised linear models (GLM)}, which was defined in
previous subsection. This GLM is generalised through assumption that
each explicative variable in \(X\) can undergo some transformation,
linear or not, resulting in a following GAM model:

\begin{equation}
g(\mu_i) = \sum_r s_r(x_{ir})
\end{equation}

Where \(s_r(.)\) is an unspecified smooth function of predictor
\(x_{ir}\). In order to obtain a NN model, this structure is further
developed as follows to obtain firstly a \emph{projection pursuit
regression (PPR)}:

\begin{equation}
f(X) = \sum_{r = m}^{M} g_m (\omega_{m}^{T} X)
\end{equation}

The \(X\) in this notation is a vector of inputs with \(p\) components,
and \(\omega_{m}\) with \(m \in \{1, 2, \dots, M \}\) are unit
\(p\)-vectors of unknown parameters. Before proceeding, we will
introduce some novelties to the notation used till this point by
introducing vectors \(X1\), \(X2\), \(\dots\), \(XS\), where \(X1\) is
the output of the first layer of neural network, each element of which
is some transformation (usually linear in parameters with some
``activation'' function) of the input vector \(X\). Then the simplest NN
for \(\Omega\) alternatives (classes) classification, with two layers,
may be represented as:

\begin{equation}
f_j (X) = g_j (X2) \text{ with } X2_j = \psi_{0j} + \psi_{k}^{T} X1
\end{equation}

Where \(f_j\) models the probability of a class \(j\), or in more
comprehensive language the probability that a given individual will
choose an alternative \(\omega_j\) from his choice set \(\Omega_i\):

\begin{equation}
X1_m = \sigma(\phi_{0m} + \phi_{m}^{T} X)
\end{equation}

While \(\sigma(.)\) is an activation function and \(g_k(.)\) a
probability transformation function, traditionally a \emph{softmax}
function. The latter is being used as well in \emph{multinomial logit
(MNL)} models:

\begin{equation}
g_j(T) = \frac{e^{T_j}}{\sum_{l = 1}^{\Omega} e^{T_l}} \text{ where } j,l \in \Omega
\end{equation}

This means, that single level NN with a softmax activation layer should
be identical to simple MNL model with all the coefficients varying by
alternatives. \(Z_m\) can be viewed as a basis expansion of the original
inputs \(X\) and the neural network is then a standard \emph{linear
multinomial logit (MNL)} model, using the transformations as inputs.

One of the supposed major problems for NN models in discrete choice
context is the inability to take into account all the influencing
factors across all the alternatives. Moreover, in this case study there
is major drawback in the ambiguity among choices A and B, as they are
interchangeable.

As we desire to obtain the effects assuming the alternatives A and B are
identical, this means that we should impose some additional restrictions
over the model. Traditional Multinomial Logistic regression (MNL) can be
potentially transcribed into a NN using convolution techniques. The
convolution layer operates iteratively on a given subset from the input
vector, calculating one single output per \(k\) inputs. In this case
\(k\) is denoted \emph{kernel size}. Another parameter, which defines a
convolutional layer is the \emph{stride} (\(s\)), which determines how
the ``window'' determined by kernel size should be moved over the input
layer. Consequently, the output layer consists of \(m\) values
determined as:

\begin{equation}
m = \frac{n - k}{s} + 1
\end{equation}

Where \(n\) is the length of the input vector to this layer. We may
attempt to define a convolution layer with linear activation function as
follows, assuming \(X = X_1, \dots, X_n\) is the input vector and
\(X1_1, \dots, X1_m\) is the output vector, while
\(\phi = \phi_1, \dots, \phi_k\) is the vector of weights:

\begin{align}
X1_1 = & \phi_1 X_1 + \phi_2 X_2 + \dots + \phi_k X_k \nonumber \\
& \vdots \\
X1_m = & \phi_1 X_{n-k} + \dots + \phi_k X_n \nonumber
\end{align}

The designed this way CNN consists of two transformation layers. The
first one is 1D convolutional layer with linear activation function,
which takes as input the dataset in ``wide'' format with 27 variables
overall (9 variables for each alternative), which produces a single
value as an output value for each individual for each choice set,
resulting in 3 output values in total. The second layer is a restricted
softmax transformation layer, which directly applies softmax
transformation over the inputs, without any supplementary permutations.

The vector of inputs issued from the dataset transformed into the
``wide'' format can be represented as:

\begin{multline}
X_i = Buy_{i,A}, Sex_{i,A}, Age_{i,A}, \dots, Habit_{i,C}, Price_{i,C}, Label_{i,C}, Carbon_{i,C}, LC_{i,C}
\end{multline}

Where all values with \(C\) index are set to zero in order to set the
baseline alternative. The first convolutional layer can be written as:

\begin{multline}
V_j = \alpha_{Buy} Buy_{ij} + \beta_{Sex} Sex_{ij} + \beta_{Age} Age_{ij} + \beta_{Income} Income_{ij} + \beta_{Habit} Habit_{ij} + \\
+ \gamma_{Price} Price_{ij} + \gamma_{Label} Label_{ij} + \gamma_{Carbon} Carbon_{ij} + \gamma_{Label \times Carbon} Label \times Carbon_{ij}
\end{multline}

Where \(j \in \{A, B, C\}\), with \(C\) denoting the ``No buy'' option.

We configure the convolution layer with linear activation function to
move across the input vector with strides 9, producing this way a vector
of length 3 as an output. This outputs of this layer may be interpreted
as utilities for each alternative respectively, identically to MNL
regression. The resulting design for a single convolution fold can be
schematically represented as in figure \ref{fig:convl}.

\begin{figure}[!htbp] \centering 
 \caption{Convolution layer} 
 \label{fig:convl} 
\begin{tikzpicture}[
    plain/.style={
        draw = none,
        fill = none,
    },
    net/.style={
        matrix of nodes,
        nodes={
            draw,
            circle,
            inner sep = 10pt
        },
        nodes in empty cells,
        column sep = 1.5cm,
        row sep = -9pt
    },
    >=latex
]

\matrix[net] (mat)
{
    |[plain]| \parbox{3cm}{\centering Convolution} & 
        |[plain]| \parbox{3cm}{\centering Deterministic Utility\\proxy} \\
    & |[plain]| \\
    |[plain]| & |[plain]| \\
    & |[plain]| \\
    |[plain]| \vdots & \\
    & |[plain]| \\
    |[plain]| & |[plain]| \\
    & |[plain]| \\
};


\draw[<-] (mat-2-1) -- node[above] {$Price_j$} +(-2cm,0);
\draw[<-] (mat-4-1) -- node[above] {$Label_j$} +(-2cm,0);
\draw[<-] (mat-6-1) -- node[above] {$Sex_j$} +(-2cm,0);
\draw[<-] (mat-8-1) -- node[above] {$Age_j$} +(-2cm,0);

\foreach \ai in {2,4,6,8}
    \draw[->] (mat-\ai-1) -- (mat-5-2);

\draw[->] (mat-5-2) -- node[above] {$V_j$} +(2cm,0);

\end{tikzpicture}
\end{figure}

The second transformation layer is a dense layer with a ``softmax''
activation function as described above, which has 3 coefficients for
each output, because it aggregates the inputs to an identical number of
outputs rescaling them in the process and producing choice
probabilities. Taking a set of \(V_A, V_B, V_C\) for inputs and
producing a vector of probabilities \(P(A), P(B), P(C)\) as outputs. The
second level may be synthetized as presented in figure \ref{fig:softl}.

\begin{figure}[!htbp] \centering 
 \caption{Softmax Layer} 
 \label{fig:softl} 
\begin{tikzpicture}[
    plain/.style={
        draw = none,
        fill = none,
    },
    net/.style={
        matrix of nodes,
        nodes={
            draw,
            circle,
            inner sep = 10pt
        },
        nodes in empty cells,
        column sep = 1.5cm,
        row sep = -9pt
    },
    >=latex
]

\matrix[net] (mat)
{
    |[plain]| \parbox{3cm}{\centering Deterministic Utility\\proxy} & 
        |[plain]| \parbox{3cm}{\centering Probability} \\
     & \\
    |[plain]| & |[plain]| \\
     & \\
    |[plain]| & |[plain]| \\
     & \\
};

\draw[<-] (mat-2-1) -- node[above] {$V_A$} +(-2cm,0);
\draw[<-] (mat-4-1) -- node[above] {$V_B$} +(-2cm,0);
\draw[<-] (mat-6-1) -- node[above] {$V_C$} +(-2cm,0);

\foreach \ai in {2,4,6}
    {\foreach \aii in {2,4,6}
        \draw[->] (mat-\ai-1) -- (mat-\aii-2);
    }

\draw[->] (mat-2-2) -- node[above] {$P_A$} +(2cm,0);
\draw[->] (mat-4-2) -- node[above] {$P_B$} +(2cm,0);
\draw[->] (mat-6-2) -- node[above] {$P_C$} +(2cm,0);

\end{tikzpicture}
\end{figure}

Finally, given the combination of these two layer we may construct the
whole CNN model. We may use the following graphical representation,
shown on figure \ref{fig:cnn} to visualise the resulting CNN
architecture:

\FloatBarrier

\begin{figure}[!htbp] \centering 
 \caption{Convolution Neural Network design} 
 \label{fig:cnn} 
\begin{tikzpicture}[
    plain/.style={
        draw = none,
        fill = none,
    },
    net/.style={
        matrix of nodes,
        nodes={
            draw,
            circle,
            inner sep = 10pt
        },
        nodes in empty cells,
        column sep = 1.5cm,
        row sep = -9pt
    },
    >=latex
]

\matrix[net] (mat)
{
    |[plain]| \parbox{1.3cm}{\centering Input\\layer} & 
        |[plain]| \parbox{1.3cm}{\centering Convolution\\layer} & 
        |[plain]| \parbox{1.3cm}{\centering Probability\\layer} & 
        |[plain]| \parbox{1.3cm}{\centering Output\\layer} \\
    & $V_A$ & $P(A)$ & |[plain]| \\
        |[plain]| & |[plain]| & |[plain]| \\
    & $V_B$ & $P(B)$ & \\
        |[plain]| & |[plain]| & |[plain]| \\
    & $V_C$ & $P(C)$ & |[plain]| \\
};


\draw[<-] (mat-2-1) -- node[above] {Rose A} +(-2.5cm,0);
\draw[<-] (mat-4-1) -- node[above] {Rose B} +(-2.5cm,0);
\draw[<-] (mat-6-1) -- node[above] {No buy (C)} +(-2.5cm,0);
\foreach \ai in {2,4,6}
    \draw[->] (mat-\ai-1) -- (mat-\ai-2);
\foreach \ai in {2,4,6}
    {\foreach \aii in {2,4,6}
        \draw[->] (mat-\ai-2) -- (mat-\aii-3);
    }
\foreach \ai in {2,4,6}
    \draw[->] (mat-\ai-3) -- (mat-4-4);
\draw[->] (mat-4-4) -- node[above] {Choice} +(2cm,0);

\end{tikzpicture}
\end{figure}

The figure \ref{fig:cnn} is no more than a simplified architecture
presentation for the chosen CNN design, imitating the MNL model in this
particular case. Each alternative input on this graph assumes entry of
the three attributes of a particular alternative, supported by five
individual characteristics each, the later being specific to a
particular alternative exactly as in the MNL model specification.

In this case the only difference between these two models is represented
by the algorithm used for estimation, which can yield absolutely
different results or even require some transformation of the input
dataset (ie. rescaling, which is used to prevent biases in weights
estimation). Consequently, the main interest of such implementation is
to observe, whether or not a ML algorithm will be able to bypass the MNL
model performances in the presence of heterogenous individual
preferences. Different convergence rates and different iterative
algorithms may result in absolutely distinct optimums for the parameters
vector. The particular algorithms implemented will be discussed later,
alongside the obtained results.

For NN modelling we use the advanced interface offered by Google's
\emph{Tensorflow} (Allaire and Tang 2020) with \emph{Keras} (Allaire and
Chollet 2020) back-end for \emph{R}-language. The flexibility offered by
this particular tool is astonishing compared to other neural networks
implementations in proposed in \emph{R}. This flexibility allows us to
simulate exactly the architecture of a MNL model and compare this way
how the different estimation techniques and algorithms perform in the
identical contexts.

\hypertarget{model-performance-evaluation-and-available-measures}{%
\subsection{Model performance evaluation and available
measures}\label{model-performance-evaluation-and-available-measures}}

In this subsection we are going to describe the different performance
measures, attempting at the same time to shun some light on the
complexity of this particular task and the multitude of different
questions that are usually aborded when a problem of performance
measures' choice arises.

The main problem in the case of classification context and particularly
in the multiple choice classification context relates to the fact that
rarely all of the models can use the same metrics for their comparison
(Baldi et al. 2000). The available metrics largely depend on the output
variable type, the models architecture and assumptions, the
specifications, the algorithms used and, finally and most importantly,
the context. As we have seen earlier, the work of Michaud, Llerena, and
Joly (2012) was focused on the identification of the willingness to pay
of consumers for particular environmental attributes of roses, rather
than general goodness of fit of particular model, which perfectly
illustrates the complexity of the posed question.

There exists a multitude of different target metrics to evaluate and
compare the performances of different models. For example, one may be
interested in exploration of a particular effects or the overall
goodness of fit, some predictive qualities or a possibility to derive
correct estimates for a particular socio-economic information. This
topic was already largely explored by some of the statisticians
(Japkowicz and Shah 2011) with some initial steps into producing an
integrated support containing all the necessary information for applied
studies. However, even given the amount of the work in reference, there
is still a strong need for contextualisation and constitution of
application specific methodological supports. The different possible
application scenarios require sometimes absolutely different metrics.
For example, econometricians rarely take into account the computational
efficiency of the models, while ML researchers are rarely considering
the possibility to derive the specific field specific metrics.

Nevertheless, this work aims at demonstrating the full potential of the
proposed experimental framework and we are bound to demonstrate at least
a fraction of its full potential, which inevitably addresses the
different performance metrics used to compare the models' performance in
terms of precision and predictive accuracy.

The measures available may roughly be divided into three parts following
the logic of Japkowicz and Shah (2011) (for an adaptation of the vision
of Japkowicz and Shah (2011) on the different measures' types see
Appendix B).

\begin{itemize}
\tightlist
\item
  The measures that take information solely from the \emph{confusion
  matrix}, which can be calculated using the estimated model over a know
  dataset (also denoted a test dataset). These measures are typically
  applied in the case of deterministic classification algorithms, but
  can be calculated for the probabilistic output algorithms as well.
\item
  The measures that not only use the confusion matrix, but integrate the
  information about the class distribution priors and classifier
  uncertainty. Logically, these metrics are useful for the
  \emph{scoring} classifiers" performance evaluation and could not be
  used with some more simple models.
\item
  Bayesian measures to account for probabilistic classifiers and
  measures for regression algorithms. Bayesian measures require a
  probabilistic structure of the models output.
\end{itemize}

The measures may be as well separated into two different groups by their
behaviour (Japkowicz and Shah 2011):

\begin{itemize}
\tightlist
\item
  A \emph{monotonic} performance measures \(pm(.)\), for which a strict
  increase (or decrease) in the value of \(pm(.)\) indicates a better
  (or worse) classifier throughout the range of the function \(pm(.)\)
  respectively.
\item
  \emph{Not strictly monotonic} can be thought of as the
  class-conditional probability estimate discussed in Kukar and
  Kononenko (2002) in the context of a multi-class problem.
\end{itemize}

As our framework can potentially treat multiple different aspects, we
will not only assess the general models' performances, but explore the
capacity to identify and estimate the target values of interest. Taking
into account the context of the target article we will be mostly
interested in exploring the willingness to pay (WTP) or the premium,
that the consumer is ready to add to the observed price for a particular
attribute.

\hypertarget{confusion-matrix}{%
\subsubsection{Confusion matrix}\label{confusion-matrix}}

Most of the performance measures for a classification task are derived
from the observed entries in the confusion matrix, denoted \(C\)
(Japkowicz and Shah 2011, @baldi2000ar). This matrix lies in the center
of most non-probabilistic performance measures for classification. A
confusion matrix \(C\) for a classifier defined by a function \(f(.)\)
over some dataset may be defined as:

\begin{equation}
C = {c_{ij}}, \text{  } i, j \in \{1, 2, \dots, k\}
\end{equation}

Where \(i\) is the row index and \(j\) is the column index, both
referring to some available alternatives for a given alternatives' set
\(\Omega\).

Generally, \(C\) is defined with respect to some fixed learning
algorithm. The confusion matrix can be extended to incorporate
information for the performance of more than one algorithm, resulting in
creation of a \emph{confusion tensor}, which can be imagined as a stack
of matrices. There exist specific metrics to be implemented on such
\emph{tensor}.

Given a training dataset and a test dataset, an algorithm learns on the
training set, outputting a fixed classifier \(f\). These datasets may be
identical, as it is frequently done in economics studies. The test-set
performance of \(f\) is then recorded in the confusion matrix. This
means that a confusion matrix, as well as its entries and the measures
derived from these are defined with respect to a fixed classifier \(f\)
over a given dataset. Consecutively, the matrix is sometimes denoted
with respect to \(f\) as \(C(f)\). It is a square \(k \times k\) matrix
for a dataset with \(k\) classes. Each element \(c_{ij}(f)\) of the
confusion matrix denotes the number of examples that actually have a
class \(i\) label and that the classifier \(f\) assigns to class \(j\).

In binary case these measures are simplified to four, that do not always
appear in matrix form for the sake of simplification. These measures, as
well as derived performance indicators are described in Baldi et al.
(2000). The binary classification case is the most common setting in
which the performance of the learning algorithm is measured. Also, this
setting serves well for illustration purposes with regard to the
strengths and limitations of the performance measures.

\hypertarget{general-performance-measures}{%
\subsubsection{General performance
measures}\label{general-performance-measures}}

The general measures (Baldi et al. 2000) describe the performance of a
given classifier \(f(.)\) (or shortly \(f\)) over a given set of
observation, taking into account all the possible classes, or choices in
the discrete choice context. In other words, these measures incorporate
all the information available for all the classes matches or mismatches,
which offers some good general overview of a given model performances,
but sometimes ignores some of the significant elements. For example,
given an unbalanced dataset, where one class dominates the other, the
general performance measures can have high positive values, signalling
the good overall performance, while all the observations will be
assigned to dominant class by the classifier.

The most known measures, which are usually implemented to assess the
general performance of the algorithms or even construct ``loss''
functions for some learning tasks include: the empirical risk, the
empirical error rate and the accuracy.

Accuracy and error rate effectively summarize the overall performance,
taking into account all data classes. This is the reason why these
measures are often implemented to assess general algorithms'
performances and are used in the learning tasks. Moreover, they offer an
insight into the generalization performance of the classifier by means
of studying their convergence behaviours, which may be important for
some algorithms.

Nevertheless, such general metrics have potential limitations (Japkowicz
and Shah 2011). Firstly, these measures suffer from the lack of
information on the varying degree of importance of different classes on
the performance. What is more, as we have already pointed out, the
metrics are incapacitated by the lack to produce any meaningful
information in the case of skewed class distribution. This results in
the situation, when as the distribution begins to skew in the direction
of a particular class, the more-prevalent class dominates the
measurement information in these metrics, making them biased.

\textbf{Empirical risk}

The \emph{empirical risk} \(R_{N} (f)\) of classifier \(f\) on test set
\(N\), defined as:

\begin{equation}
R_{N} (f) = \frac{1}{\mid N \mid} \sum_{i = 1}^{\mid N \mid} I (y_i \neq f (x_i))
\end{equation}

Where:

\begin{itemize}
\tightlist
\item
  \(I(a)\) is the indicator function if predicate \(a\) is true and zero
  otherwise;
\item
  \(f(x_i)\) is the label assigned to example \(x_i\) by classifier
  \(f\);
\item
  \(y_i\) is the true label of example \(x_i\), which indicates to ome
  of the alternatives \(\{\omega_1, \dots, \omega_k\} \in \Omega\);
\item
  \(\mid N \mid\) is the size of the test set.
\end{itemize}

This measure describes the average loss over the data points.

\textbf{Empirical error rate}

The \emph{empirical error rate} can be computed as follows:

\begin{equation}
R_N (f) = \frac
    {\sum_{i, j: i \neq j} c_{ij} (f)}
    {\sum_{i,j = 1}^{\Omega} c_{ij} (f)} =
  \frac
    {\sum_{i,j = 1}^{\Omega} c_{ij} (f) - \sum_{i = 1}^{\Omega} c_{ii} (f)}
    {\sum_{i,j = 1}^{\Omega} c_{ij} (f)}
\end{equation}

This rate measures the part of the instances from the given set that are
incorrectly classified by the learning algorithm \(f\).

\textbf{Accuracy}

The \emph{accuracy} describes the part of correctly classified instances
in a given set and is by its nature a complement to the empirical
error-rate measure. It can be computed as:

\begin{equation}
Acc_N (f) = \frac{1}{\mid N \mid} \sum_{i = 1}^{\mid N \mid} I (f (x_i) = y_i)
\end{equation}

Where \(y_i\) is the observed class for observation \(i\). Given a skew
ratio \(r_s\), it is possible to extend this measure and define the
\emph{skew-sensitive formulation of the accuracy}. Such modification
allows partially to solve the poor measures' utility problem on a skewed
class distribution dataset.

\hypertarget{single-class-performance-measures}{%
\subsubsection{Single-class performance
measures}\label{single-class-performance-measures}}

Apart from the general performance measures, there exist some more
specific performance measures, which instead of estimating the
performances of the overall classifier, target some specific aspects.
Usually in the modelling the consumer behaviour we may be interested in
his his choice ``Buy'' against ``No buy'' beforehand, and only
afterwards we are interested by his consumer habits and preferences.
Among these measure we may cite:

\begin{multicols}{2}
\begin{itemize}
\item True- and False-Positive/Negative Rates
\item Sensitivity
\item Specificity
\item Precision 
\item Recall 
\item Geometric means
\item Likelihood Ratio (LR) \footnote{This measure will be omitted in order to prevent the eventual confusion with Likelihood Ratio (LR) used in the MNL and MMNL models}
\item F-measured
\item Skew and Cost 
\end{itemize}
\end{multicols}

One of the important problems for discrete choice modelling and general
classification tasks resides in the form of the greater importance of
the algorithms' performance on a single class of interest. This
performance on a given class can be crucial with regard to the instances
of this class itself or with regard to the instances of other classes in
the training data. As it was mentioned earlier, in our particular study
case, we may be interested at how good the algorithm distinguishes the
``Buy'' and ``No buy'' choices.

A number of such measures can also allow us to measure the overall
performance of the classifier with an emphasis on the instances of each
individual class. Such precise metrics may be excessive, given a
particular case study, although they offer a good substitute for more
typical measures, such as the accuracy or error rate.

In this part we are going to introduce some new terminology, because
contrary to the precious parts, where we had to deal with classes, here
we are bound to simplify the problem to a binary case. This means that
one of the classes is considered as ``positive'', while the rest of the
alternatives is regrouped into a single ``negative'' class. Such
transformation allows us to define new variables, which will be used
later in the class-specific measures presentation. Among these values we
have:

\begin{itemize}
\tightlist
\item
  True Positive or \(TP\), which denotes the number of correctly
  classified observations which appertained to the ``positive'' class;
\item
  True Negative or \(TN\), where the number of correctly classified
  ``negative'' instances is regrouped;
\item
  False Positive or \(FP\) stands for the misclassified instances that
  in the dataset were encoded as ``positive'' class;
\item
  False Negative or \(FN\), which logically indicates the number of
  initially ``positive'' observations, which were identified as
  ``negative'' ones by the model.
\end{itemize}

All these values may be easily obtained from the confusion matrix \(C\).

\textbf{True- and False- positive/negative rates, specificity and
sensitivity}

The most natural metric aimed at measuring the performance of a learning
algorithm on instances of a single class is arguably its
\emph{true-positive rate}. The \emph{true-positive rate} of a classifier
is also referred to as the \emph{sensitivity} of the classifier. The
complement metric to this, in the case of the two-class scenario, would
focus on the proportion of negative instances is called the
\emph{specificity} of the learning algorithm. It is obtained as:

\begin{equation}
TPR_i (f) = \frac{c_{ii} (f)}{\sum_{j = 1}^{I} c_{ij} (f)} =
  \frac{c_{ii} (f)}{c_i (f)}
  \end{equation}

The \emph{false-positive rate} of a classifier:

\begin{equation}
FPR_i (f) = \frac{\sum_{j: j \neq i} c_{ji} (f)}
  {\sum_{j, k: k \neq i} c_{jk} (f)}
  \end{equation}

Some usefull derived formulas, which are easy to compute for a binary
case, are introduced hereafter. The True- and False- positive rates:

\begin{equation}
TPR (f) = \frac{TP}{TP + FN} = \text{Sensitivity} = 1 - FNR (f)
\end{equation}

\begin{equation}
FPR (f) = \frac{FP}{FP + TN}
\end{equation}

As well as their counterpart, the True- and False- negative rates, which
are focussed on the number of correctly classified instances from a
``negative class''.

\begin{equation}
TNR (f) = \frac{FN}{TN + FP} = \text{Specificity}
\end{equation}

\begin{equation}
FNR (f) = \frac{FN}{FN + TP}
\end{equation}

\textbf{Precision and recall}

The \emph{precision} or \emph{positive predictive value (PPV)} of a
classifier \(f\) on a given class of interest \(j\), denoted as well as
the ``positive'' class, in terms of the entries of \(C\), measures how
\emph{precise} the algorithm is when identifying the examples of a given
class and is defined as:

\begin{equation}
PPV_i (f) = Prec_i (f) \frac{c_{ii} (f)}{\sum_{j = 1}^{I} c_{ji} (f)} = \frac{c_{ii} (f)}{c_{.i} (f)}
\end{equation}

For binary case we can write the following simplified definition, which
should be more clear to the reader:

\begin{equation}
Prec (f) = PPV (f) = \frac{TP}{TP + FP}
\end{equation}

The PPV can be complimented with the sensitivity of the classifier over
this class. This measure is generally referred to as \emph{recall}:

\begin{equation}
Rec (f) = \frac{TP}{TP + FN}
\end{equation}

\textbf{Geometric means}

The \emph{geometric means} take into account the relative balance of
several performance measures for a given classifier. The most popular
option is to observe simultaneously the classifier's performance on both
the positive and the negative classes:

\begin{equation}
Gmean_1 (f) = \sqrt{TPR (f) \times TNR (f)}
\end{equation}

This implementation is of particular interest for our case study, as we
will be able to compare the performances of different models across
``Buy'' and ``No buy'' options. Another popular version of the measure,
which focusses on a single class of interest, can take the precision of
the classifier in combination with the classifiers performance on the
``positive'' class into account:

\begin{equation}
Gmean_2 (f) = \sqrt{TPR (f) \times Prec (f)}
\end{equation}

\textbf{F-measure}

The \emph{F-measure} as well attempts to address the issue of
convenience brought on by a single metric versus a pair of metrics. It
combines the information of precision and recall in a single value. More
precisely, the F-measure is a weighted harmonic mean of precision and
recall, with a weight \(\alpha\):

\begin{equation}
F_{\alpha} = \frac
  {(1 + \alpha)(Prec (f) \times Rec (f))}
  {\alpha Prec (f) + Rec (f)}
\end{equation}

For instance, the most comprehensive \emph{balanced F-measure} weights
the recall and precision of the classifier evenly:

\begin{equation}
F_{1} = \frac
  {2(Prec (f) \times Rec (f))}
  {Prec (f) + Rec (f)}
\end{equation}

In most practical cases, appropriate weights are generally not known,
which results in some complications in choice of the hyper-parameter
\(\alpha\) of such combinations of measures.

\textbf{Class ratio}

\emph{Class ratio} for a given class \(i\), which in the consumer choice
setting is usually denoted \(\omega_i\) refers to the number of
instances of class \(i\) as opposed to those of other classes in the
dataset:

\begin{equation}
ratio_i = r_i = \frac{\sum_j c_{ij}}
  {\sum_{j, j \neq i} c_{ji} + \sum_{j, j \neq i} c_{jj}}
\end{equation}

Or for a binary case:

\begin{equation}
ratio_{positive} = \frac{(TP + FN)}{(FP + TN)}
\end{equation}

Another issue worth considering when looking at misclassification is
that of classifier uncertainty. This lack of classifier uncertainty
information is also reflected in all the performance measures that rely
solely on the confusion matrix.

\hypertarget{information-theoretic-measures}{%
\subsubsection{Information-theoretic
measures}\label{information-theoretic-measures}}

These measures are probabilistic by their nature, as they explore the
performances of the classifier with respect to the (typically empirical)
prior distributions of the data. in contrast to the cost-sensitive
metrics that have been introduced earlier, the
\emph{information-theoretic measures}, because of accounting for the
data priors, are applicable only to probabilistic classifiers. What is
more, these metrics are independent of the cost considerations and can
be applied directly to the probabilistic output of a given model. These
measures are extensively implemented in Bayesian learning and take their
roots in physics. Among these metrics one may encounter:

\begin{itemize}
\tightlist
\item
  Kullback--Leibler Divergence, which estimates the difference between
  the entropies of the two distributions;
\item
  Kononenko and Bratko's Information Score, which explores the
  likelihood of correct classification.
\end{itemize}

In this work we will present only the first among these two.

\textbf{Kullback--Leibler Divergence}

Let the true probability distribution over the labels be denoted as
\(p(y)\). Let the posterior distribution generated by the learning
algorithm after seeing the data be denoted by \(P (y \mid f)\). Because
\(f\) is obtained after looking at the training samples \(x \in S\),
this empirically approximates \(P (y \mid x)\), the conditional
posterior distribution of the labels. Then the \emph{Kullback--Leibler
divergence} (KLD or KL) can be utilized to quantify the difference
between the estimated posterior distribution and the true underlying
distribution of the labels:

\begin{equation}
KLD [p(y) \mid \mid P (y \mid f) ] =
  \int p(y) ln p(y) dy - 
  \int p(y) ln P(y \mid f) dy
\end{equation}

\begin{equation}
KLD [p(y) \mid \mid P (y \mid f) ] =
  - \int p(y) ln \frac{P(y \mid f)}{p(y)} dy
\end{equation}

The KLD divergence basically just finds the difference between the
entropies of the two distributions \(P (y \mid f)\) and \(p(y)\). This
measure is also known as \emph{relative entropy} (see Baldi et al.
(2000)) for more information.

\begin{equation}
KLD [p(y) \mid \mid P (y \mid f) ] = 
  - \sum_{x \in S} p(y) ln \frac{P(y \mid f)}{p(y)} dy = \sum_{x \in S} p(y) ln \frac{p(y)}{P(y \mid f)} dy
\end{equation}

The KLD value is equal to zero if and only if the posterior distribution
is the same as the prior, when the perfect fit is achieved, meaning that
the classifier perfectly mimics the true underlying distribution of the
labels.

Even though the KLD measures the difference between the posterior
distribution obtained by the learner from the true distribution so there
is a significant drawback to it. The KLD needs the knowledge of the true
underlying prior distribution of the labels, which is rarely, if at all,
known in any practical application. In practice the estimated priors are
used, although in the experimental framework where a synthetic dataset
is used, we may theoretically impose some ``true'' structure over the
choice distribution.

\hypertarget{case-specific-metrics}{%
\subsubsection{Case specific metrics}\label{case-specific-metrics}}

The article of Michaud, Llerena, and Joly (2012) focuses on the WTP for
roses and derivation of the premiums for particular alternative
attributes of interest. This focus allows authors to explore the
consumer attitude towards the alternative specific environmental
attributes. Consequently, as we try to follow the logic introduced in
the article, we are going to attempt to derive the WTP and premiums for
attributes as well. However, before introducing the notion of the WTP
and premium, we should firstly describe the procedure of derivation of
the marginal effects, as the WTP and premiums are expressed using the
marginal effects.

In the conventional MNL models the coefficients \(\beta_{rj}\) can be
interpreted as the marginal effect of variable \(X_r\) on the log
odds-ratio of alternative \(j\) to the baseline alternative. The
marginal effect of \(X_r\) on the probability of choosing a specific
alternative \(j\) can be expressed as:

\begin{equation}
ME_{rj} = \frac{
    \Delta P(Y_i = \omega_j) 
}{
    \Delta X_r 
}
\end{equation}

Consequently, for the MNL model, the marginal effect of \(X_r\) on
alternative \(j\) not only takes into account the parameters specific to
\(j\) alternative, but the ones of all other alternatives as well. The
equation can be written in this case as:

\begin{equation}
\frac{
    \Delta P(Y_i = \omega_j) 
}{
    \Delta X_i 
} = P(Y_i = \omega_j) [
    \beta_{j1} - \sum_{l = 0}^k P(Y_i = \omega_l) \beta_{j1}
]
\end{equation}

The parameters such as WTP and premiums are more easy to interpret. They
can be estimated directly or can be obtained from the marginal utility
by dividing it by the effect estimate of a price, taken as a non random
parameter. The resulting ratio can afterwards be interpreted as a
monetary value. The WTP as it was described in the context presentation,
taking into account the case specific relative utility functions can be
represented as:

\begin{equation}
WTP = \frac{
  \frac{\Delta V}{\Delta BUY}
}{
  \frac{\Delta V}{\Delta Price}
} = \frac{
  - \alpha_{Buy}
}{
  \beta_{Price}
}
\end{equation}

The premiums for a given attribute \(X_r\) (\(Label\), \(Carbon\) or
their cross-product \(LC\)), can therefore be expressed as:

\begin{equation}
WTP = \frac{
  \frac{\Delta V}{\Delta X_r}
}{
   \frac{\Delta V}{\Delta Price}
}
\end{equation}

\hypertarget{selection-of-measures-to-implement}{%
\subsubsection{Selection of measures to
implement}\label{selection-of-measures-to-implement}}

In this work we are going to explore only a selection of the described
above most popular performance metrics, that are the most interesting
given the context of the study. Moreover, in our application we are
limited in the number of measures we can explore.

In the first place we are interested by the WTP for roses and the
premiums associated with particular alternative specific attributes.
These theoretical values could be easily derived for all the three
explored models and they will allow us to compare, how close are the
derived values from the theoretical input values, which were defined on
the dataset generation step.

Secondly, it is important to assess the overall goodness of fit over the
whole dataset for the selected models. For this particular task the most
suited measure is the \emph{accuracy}. This way we will be able to
observe the ratio of the overall correctly classified instances. We may
implement the KLD estimator for overall goodness of fit, based on the
probability distributions, because all the models predict the
probabilities for the available alternatives.

We may be interested as well in comparing the performances of the given
models in terms of distinguishing the ``Buy'' choice, irrelevant of the
alternative, and the ``No buy'' choice. This is a particularly
interesting question, because in the different choice settings and over
the datasets generated under different theoretical assumptions. For this
purpose the most interesting choice will be to select the F-measure or a
Geometric mean of the TPR and TNR.

Finally, we are going to observe the performance of these different
models in terms of computational efficiency in resources consumption.
For this task we will observe the computation times for given
models\footnote{This measure is one of the most complex, because it
  accounts at the same time for different models, different estimation
  algorithms, different numerical implementation in the statistical
  software and different PC configuration. It is valid in this
  particular case, because all models were estimated using the same
  hardware and software set-up.}. The obtained results will be discussed
at the end of this work.

\newpage

\hypertarget{model-comparison-in-practice-an-application}{%
\section{Model comparison in practice: an
application}\label{model-comparison-in-practice-an-application}}

This section is designed to present the results of the designed
theory-testing framework as well as to offer some more detailed view on
the adopted procedure. It will respect the following structure. First of
all we will start by a presentation of two generated datasets and their
comparison with the original dataset obtained through a controlled
experiment by Michaud, Llerena, and Joly (2012). Then we will discuss
the technical implementation of the models to test and the resulting
estimations over two of the simulated datasets. Finally the target
performance metrics will be constructed for all the models' performances
over both of the datasets and we will compare the obtained estimates
with the input values, assessing this way the biases suffered during
estimation.

\hypertarget{simulating-individual-choices}{%
\subsection{Simulating individual
choices}\label{simulating-individual-choices}}

Based on the article of Michaud, Llerena, and Joly (2012) we generate a
synthetic dataset assuming the utility function is as described in the
paper with some minor changes and adjustments. We have already delimited
the scope of study and delimited our area of interest to the exploration
of different models performance given the theoretical structure of
consumer preferences for the alternative specific attributes. For
simplicity we relax some of the assumptions made in the paper in order
to generate two different datasets. For the first dataset we assume that
estimations made in the paper and the derived utility functions are
correct and reflect the real world situation. For the second one, we
relax some of the advanced assumptions and regenerate a simplified
version, which will allow us to contrast the performances of different
models in different choice context assuming different nature of choice
functions.

In both situations the utility functions are determined as in paper: we
use the exact means for the coefficients estimates, assuming they are
correct. The relative utility's deterministic part for each individual
is defined by the following function, which was presented in a more
detailed way in previous section:

\begin{multline}
V_{ij} = \alpha_{i,Buy} + \beta_{Buy, Sex} Sex_i + \beta_{Buy, Age} Age_i + \beta_{Buy, Salary} Salary_i + \beta_{Buy, Habit} Habit_i + \\
+ \gamma_{Price} Price_{ij} + \gamma_{i, Label} Label_{ij} + \gamma_{i, Carbon} Carbon_{ij} + \gamma_{i, LC} LC_{ij}
\end{multline}

Where \(LC = Label \times Carbon\). The random component of the relative
utility \(U_{ij}\) is defined as identically and independently
distributed random variable \(\epsilon_{ij}\) issued from the Gumble
distribution parametrised with \((0, 1)\). The mean effects for the
components of the deterministic part are given as presented in the table
\ref{tab:params1}

\begin{table}[!htbp]\centering
    \caption{The assumed relative utility function parameters}
    \label{tab:params}
\begin{subfigure}[c]{.4\linewidth}
    \centering
    \caption{Mean effects}
    \label{tab:params1}
\begin{tabular}[b]{@{\extracolsep{5pt}}lc} 
\\[-1.8ex]\hline 
\hline \\[-1.8ex] 
& \multicolumn{1}{c}{\textit{Effects}} \\ 
\cline{2-2} 
\\[-1.8ex] & \multicolumn{1}{c}{\textit{Means}} \\
\hline \\[-1.8ex] 
\textbf{Individual characteristics ($\beta$)} & \\
 ~~~Sex & 1.420 \\ 
 ~~~Age & 0.009 \\ 
 ~~~Salary & 0.057 \\ 
 ~~~Habit & 1.027 \\ 
\textbf{Alternatives' attributes ($\gamma$)} & \\
 ~~~Price & $-$1.631 \\ 
 ~~~Buy & 2.285 \\ 
 ~~~Label & 2.824 \\ 
 ~~~Carbon & 6.665 \\ 
 ~~~LC & $-$2.785 \\
 ~~~ & \\
\hline \\[-1.8ex] 
\end{tabular} 
\end{subfigure}
\hspace{1.5cm}
\begin{subfigure}[c]{.4\linewidth}
    \centering
    \caption{Variance-covariance structure}
    \label{tab:params2}
\begin{tabular}{@{\extracolsep{5pt}}lcc} 
\\[-1.8ex]\hline 
\hline \\[-1.8ex] 
 & \multicolumn{2}{c}{\textit{Effects}} \\ 
\cline{2-3} 
\\[-1.8ex] & Fixed & Random \\ 
\hline \\[-1.8ex] 
\textbf{Variance} & & \\
 ~~~Buy & 0 & 3.202 \\  
 ~~~Label & 0 & 2.654 \\  
 ~~~Carbon & 0 & 3.535 \\  
 ~~~LC & 0 & 2.711 \\ 
\textbf{Covariance} & & \\ 
 ~~~Buy:Label & 0 & -0.54 \\  
 ~~~Buy:Carbon & 0 & -4.39 \\  
 ~~~Buy:LC & 0 & 6.17 \\  
 ~~~Label:Carbon & 0 & 8.77 \\  
 ~~~Label:LC & 0 & -2.33 \\  
 ~~~Carbon:LC & 0 & -4.82 \\ 
\hline \\[-1.8ex] 
\end{tabular} 
\end{subfigure}
\end{table}

The only difference between the two generated datasets is in the
specification of the randomness of these coefficients as they may vary
or not across population. It means, that the first dataset is generated
assuming the variance-covariance matrix for correlated random
coefficients is composed with 0's only and the resulting multivariate
normal distribution produces exact means for the coefficients. The
second dataset is generated using the exact estimates of the
variance-covariance matrix as provided in the article. The assumed
parameters for effects distributions are represented in the table
\ref{tab:params2}.

Additionally we impose some supplementary constraints to our data due to
the limitations of the simulation tool. Particularly, the individual
characteristics are supposed to be not correlated, which can be
explained by the fact that the context of a controlled experiment offers
a possibility to control this particular feature. Obviously, this is not
optimal decision, as naturally the age, sex, income and environmental
habits of individuals should be correlated. Unfortunately, the original
article does not provide information about the characteristics'
variance-covariance matrix.

The targeted features and requirements to the resulting dataset are
numerous and they make a contrast compared to the initial empirical
dataset.

The simulated dataset allows us to explore significant number of choice
sets for numerous artificial individuals, which ensures statistical
validity for obtained results and permits us to use advanced estimation
algorithms (such as neural networks, for example). It means that we
generate a large sample with exhaustive number of choice sets, in which
all the possible combinations of alternative attributes are represented.
Here by \emph{attributes} we understand the binary factors describing
rose's labelling and carbon footprint and ignore the price, the latter
being added afterwards using randomisation techniques. This choice is
similar to the experimental design described in the Michaud, Llerena,
and Joly (2012) work and is easily explained when we take a closer look
at the number of choice sets for different specifications. In simulated
datasets it is traditional to use Full-Factorial (FF) experimental
design as it uncovers completely the full potential of simulation tools:
it allows to observe all the possible combinations of factors affecting
some process and fully explore their implications. In our case, a simple
full factorial design for a binary choice context has 28 combinations of
factors (seven levels of prices, two levels for eco-label and two levels
for Carbon imprint), but a complete full factorial design for a choice
context with two alternatives implies 784 different combinations (as we
have two alternatives each having 28 possible variants), which is
unrealistic in a standard experimental study context and risks to be too
demanding in terms of calculation times.

The dataset should be equilibrated with relatively identical number of
choices for all three alternatives. In the field experiment the authors
managed to achieve satisfying result with 67.5\% of ``Buy'' choices and
32.5\% for ``Not to buy'' choices, although the ``A'' and ``B''
alternatives showed different properties. The resulting observed
descriptive statistics derived from the data proposed by Michaud,
Llerena, and Joly (2012) are presented in table \ref{tab:altdata}. The
table focusses on the choice ``Buy'' descriptive statistics, ignoring
the ``No buy'' option, for which all the attributes are considered to be
equal to 0. The \(p\)-values are the results of the two subsets (``A''
and ``B'') comparison\footnote{\(\chi^2\) test is used for discrete
  variables, while \emph{Anova} is implemented for continuous ones.}.

\begin{table}[!htbp] \centering 
  \caption{Alternatives' descriptive statistics by group, correlated random effects} 
  \label{tab:altdata} 
\begin{tabular}{@{\extracolsep{5pt}}lcccr}
\\[-1.8ex]\hline 
\hline \\[-1.8ex] 
 & A  & B  & Total  & p value\\
 & (N=1186) & (N=1186) & (N=2372) &  \\
\hline \\[-1.8ex] 
\textbf{Choice} &  &  &  & < 0.001\\
~~~Mean (SD) & 0.517 (0.500) & 0.159 (0.366) & 0.338 (0.473) & \\
~~~Range & 0.000 - 1.000 & 0.000 - 1.000 & 0.000 - 1.000 & \\
\textbf{Price} &  &  &  & 0.418\\
~~~Mean (SD) & 2.990 (0.881) & 3.020 (0.893) & 3.005 (0.887) & \\
~~~Range & 1.500 - 4.500 & 1.500 - 4.500 & 1.500 - 4.500 & \\
\textbf{Carbon} &  &  &  & < 0.001\\
~~~Mean (SD) & 0.167 (0.373) & 0.832 (0.374) & 0.500 (0.500) & \\
~~~Range & 0.000 - 1.000 & 0.000 - 1.000 & 0.000 - 1.000 & \\
\textbf{Label} &  &  &  & 0.837\\
~~~Mean (SD) & 0.502 (0.500) & 0.497 (0.500) & 0.500 (0.500) & \\
~~~Range & 0.000 - 1.000 & 0.000 - 1.000 & 0.000 - 1.000 & \\
\hline
\end{tabular}
\end{table}

Of particular interest in the table \ref{tab:altdata} to us is the
unbalanced structure of the resulting dataset. The \(Carbon\) imprint of
the different alternatives has not identical properties, which leads to
different \(Choice\) statistics, where the alternative with higher
carbon imprint is chosen less frequently. In the original study such
difference was not dangerous, because only the ``Buy'' option was
compared against ``No Buy'' one. However, in case of the NN modelling
such unbalanced dataset may lead to erroneous results, where the more
popular alternative will always have a higher choice probability. The
distribution inside the ``Buy'' group for different alternatives (``A''
and ``B'') should be quasi-identical, producing equally distributed
three groups of choices each nearing 33.3\%. Even if this property is
not as important for a traditional MNL model, we are interested to
observe the same choice structure in our artificial dataset, because it
may highly affect the performance of more advanced models, such as NN
for example.

\FloatBarrier

\hypertarget{generated-dataset-presentation}{%
\subsubsection{Generated dataset
presentation}\label{generated-dataset-presentation}}

\FloatBarrier

In this section we will discuss the resulting datasets simulated under
the listed above assumptions.

For our dataset we choose to generate 160000 observations, for 1000
individuals, each facing 16 different choice sets 10 times. The 16
choice sets include all the possible combinations of two roses (``A''
and ``B'') described by two environmental attributes, while prices are
randomly assigned within the choice sets. The prices are assumed to be
uniformly distributed over the choice sets, following a discrete uniform
distribution. The prices vary among the different replications. This
procedure resulted in sufficiently large dataset, which in the same time
was not difficult to treat without implementation of Big Data specific
techniques.

The original experimental design used to generate the choice sets
assumed no branding for the alternatives to avoid any undesired bias in
the results. Theoretically this design should have provided an
equilibrated dataset with no correlation between different attributes,
although the size of the final dataset might have affected the results.
In our case we assume that individuals have no additional information
about the roses in choice sets except the three observed attributes. As
in the original work we assign insignificant labels ``A'' and ``B'' to
the roses within choice sets, which is done mostly for convenience and
has no impact on the individuals' decisions.

It is interesting to explore the statistical properties of the resulting
datasets: the original one (Original), gathered by Michaud, Llerena, and
Joly (2012) and made available in anonymised format by Iragaël Joly; and
the two generated artificial datasets, assuming homogeneous (Generated
FE) and heterogeneous (Generated RE) preferences respectively of the
individuals for the environmental attributes. First of all, we may
observe the individuals descriptive statistics for three datasets in the
table \ref{tab:indivdata}.

\begin{table}[!htbp] \centering 
  \caption{Individuals' characteristics descriptive statistics by dataset} 
  \label{tab:indivdata} 
\begin{tabular}{@{\extracolsep{5pt}}lcccr}
\\[-1.8ex]\hline 
\hline \\[-1.8ex] 
 & Fixed Effects  & Random Effects  & Target  & p value\\
 & (N=1000) & (N=1000) & (N=102) &  \\
\hline \\[-1.8ex] 
\textbf{Sex} &  &  &  & 0.851\\
~~~Mean (SD) & 0.506 (0.500) & 0.515 (0.500) & 0.490 (0.502) & \\
~~~Range & 0.000 - 1.000 & 0.000 - 1.000 & 0.000 - 1.000 & \\
\textbf{Habit} &  &  &  & 0.182\\
~~~N-Miss & 0 & 0 & 1 & \\
~~~Mean (SD) & 0.683 (0.466) & 0.657 (0.475) & 0.604 (0.492) & \\
~~~Range & 0.000 - 1.000 & 0.000 - 1.000 & 0.000 - 1.000 & \\
\textbf{Salary} &  &  &  & < 0.001\\
~~~Mean (SD) & 2.750 (1.476) & 2.671 (1.438) & 2.147 (1.222) & \\
~~~Range & 1.000 - 6.000 & 1.000 - 6.000 & 1.000 - 6.000 & \\
\textbf{Age} &  &  &  & 0.255\\
~~~Mean (SD) & 41.862 (13.685) & 42.161 (13.820) & 39.755 (18.895) & \\
~~~Range & 18.000 - 84.000 & 18.000 - 84.000 & 18.000 - 85.000 & \\
\hline
\end{tabular}
\end{table}

Even though the \(p\)-values show no evident differences between the
simulated datasets and the original one, except for the \(Age\)
variable, we observe the differences in the means. This is explained by
the implemented dataset generation procedure. The variables in the
original dataset are integers, assuming continuous nature of the real
world variables. When synthesizing the dataset, we assume the quasi
continuous variables, such as \(Age\) and \(Salary\) (denoted as
\(Income\) in original work) to be issued from normal distribution with
parameters as figuring in the descriptive statistics for the original
dataset, and only afterwards we convert the resulting values to
integers. The binary variables \(Sex\) and \(Habit\) are generated with
random draws from Bernoully distribution and consequently produce more
realistic results. This procedure leads to potential biases in the
resulting datasets, which is true not only for the individual variables,
but for the alternatives' attributes as well.

\begin{table}[!htbp] \centering 
  \caption{Alternatives' descriptive statistics by dataset} 
  \label{tab:alt1} 
\begin{tabular}{@{\extracolsep{5pt}}lcccr}
\\[-1.8ex]\hline 
\hline \\[-1.8ex] 
 & Fixed Effects  & Random Effects  & Target  & p value\\
 & (N=320000) & (N=320000) & (N=2372) &  \\
\hline \\[-1.8ex] 
\textbf{Price} &  &  &  & 0.002\\
~~~Mean (SD) & 2.936 (0.958) & 2.936 (0.958) & 3.005 (0.887) & \\
~~~Range & 1.500 - 4.500 & 1.500 - 4.500 & 1.500 - 4.500 & \\
\textbf{Carbon} &  &  &  & 0.999\\
~~~Mean (SD) & 0.500 (0.500) & 0.500 (0.500) & 0.500 (0.500) & \\
~~~Range & 0.000 - 1.000 & 0.000 - 1.000 & 0.000 - 1.000 & \\
\textbf{Label} &  &  &  & 0.999\\
~~~Mean (SD) & 0.500 (0.500) & 0.500 (0.500) & 0.500 (0.500) & \\
~~~Range & 0.000 - 1.000 & 0.000 - 1.000 & 0.000 - 1.000 & \\
\hline
\end{tabular}
\end{table}

Secondly, we may as well observe the alternative specific descriptive
statistics. They are presented in table \ref{tab:alt1}. In this table we
present the cumulative statistics for the ``Buy'' option, including both
rose ``A'' and rose ``B'' properties, while 160000 entries (1186 entries
for the original dataset) describing the ``No buy'' alternative are
omitted, because their attributes are reduced to zeros in order to
achieve identifiability of the models (a complete presentation of
descriptive statistics par dataset and stratified by alternative may be
found in Appendix C). The distributions of \(Carbon\) footprint and
Eco-\(Label\) attributes follows perfectly the ones inside the original
dataset, although the prices differ. This particular divergence, may be
explained by the procedure implemented to assign prices to the
alternatives inside choice sets, because the random generator algorithms
different across statistical programs and potentially the procedures
implemented in \(R\) and \(SAS\) are not identical.

What is more interesting, is the difference in the \(Choice\)
statistics. We may be interested in comparing the statistics for
different classes in our sample to ensure that they are not biased in
favour of label ``A'' or label ``B'', as in this case it risks to bias
the estimates. For the artificial dataset the ratio of choices per
``Buy'' alternative is higher than 40\% and reaches 47.3\% for the fixed
effect utility (table \ref{tab:alt2}), while for the random effects
specification the numbers are lower, reaching only 42\% in mean for two
classes (table \ref{tab:alt3}). This particular observation is rather
interesting as it demonstrates how the heterogeneous effects for
alternatives' features the consumer decisions.

We will start with a close examination of the fixed effects dataset,
where we can see, that prices are not equally distributed among the
different choices.

\begin{table}[!htbp] \centering 
  \caption{Alternatives' descriptive statistics by group, fixed coefficients} 
  \label{tab:alt2} 
\begin{tabular}{@{\extracolsep{5pt}}lcccr}
\\[-1.8ex]\hline 
\hline \\[-1.8ex] 
 & A  & B  & Total  & p value\\
 & (N=160000) & (N=160000) & (N=320000) &  \\
\hline \\[-1.8ex] 
\textbf{Choice} &  &  &  & < 0.001\\
~~~Mean (SD) & 0.427 (0.495) & 0.518 (0.500) & 0.473 (0.499) & \\
~~~Range & 0.000 - 1.000 & 0.000 - 1.000 & 0.000 - 1.000 & \\
\textbf{Price} &  &  &  & < 0.001\\
~~~Mean (SD) & 3.069 (0.979) & 2.803 (0.917) & 2.936 (0.958) & \\
~~~Range & 1.500 - 4.500 & 1.500 - 4.500 & 1.500 - 4.500 & \\
\hline
\end{tabular}
\end{table}

The unbalanced prices potentially bias our dataset and we can see how
the option with inferior mean prices is chosen less frequently. Even
thought this differences do not affect the MNL and MMNL models, which
calculate average effects for all the alternatives, there may be an
impact over the performances of the NN models performances.

For the dataset with correlated random effects of the alternative
specific variables, we observe an identical situation in table
\ref{tab:alt3}. The class with lower average prices is chosen more
rarely by the consumers, while the overall choices are less frequent due
to the presence of stochastic individual preferences for particular
alternatives' attributes.

\begin{table}[!htbp] \centering 
  \caption{Alternatives' descriptive statistics by group, correlated random effects} 
  \label{tab:alt3} 
\begin{tabular}{@{\extracolsep{5pt}}lcccr}
\\[-1.8ex]\hline 
\hline \\[-1.8ex] 
 & A  & B  & Total  & p value\\
 & (N=160000) & (N=160000) & (N=320000) &  \\
\hline \\[-1.8ex] 
\textbf{Choice} &  &  &  & < 0.001\\
~~~Mean (SD) & 0.382 (0.486) & 0.462 (0.499) & 0.422 (0.494) & \\
~~~Range & 0.000 - 1.000 & 0.000 - 1.000 & 0.000 - 1.000 & \\
\textbf{Price} &  &  &  & < 0.001\\
~~~Mean (SD) & 3.069 (0.979) & 2.803 (0.917) & 2.936 (0.958) & \\
~~~Range & 1.500 - 4.500 & 1.500 - 4.500 & 1.500 - 4.500 & \\
\hline
\end{tabular}
\end{table}

We may conclude the preliminary datasets study and comparison with the
main impression that two artificial datasets may be assumed to be
quasi-identical. The slight differences in prices, captured by
statistical tests may be considered insignificant in comparison with the
biases present in the original dataset. What is more, even if the biases
were more significant, the models' specification, which assumes no
variable specific coefficients for choice A and B would have lead to the
correct estimates, exactly as it was done by Michaud, Llerena, and Joly
(2012). The heterogeneous preferences result in less probable decisions
to buy a rose in the population, which should definitely impact the
performances of our models. Now it rests to verify how well the number
of selected models will be able to derive the target values for the
relative utility function.

\FloatBarrier

\hypertarget{modelling-consumer-choices-under-different-assumptions}{%
\subsection{Modelling consumer choices under different
assumptions}\label{modelling-consumer-choices-under-different-assumptions}}

This part of the work aims at presenting the results of the estimation
for our selection of the econometric and ML models. We should
particularly underline the fact, that this section does not focus on the
performances of the models as they will be discussed more in detail
latter. There is still a double objective for this section, as before
presentation of the obtained results, we should discuss the methods and
techniques, which were implemented in order to estimate the models,
presented earlier.

The estimation procedure and choice of the estimation algorithms as well
as their numeric implementation in the statistical software are
important in the context of model performance comparison. The different
estimation procedures may lead to different results and different
conclusions.

We consecutively estimate the chosen models over the two datasets: with
and without the presence of heterogeneous preferences of the individuals
for the environmental attributes. Then we compare the estimates with the
target values we have used previously as inputs in defining the relative
utility functions.

\hypertarget{estimation-procedures}{%
\subsubsection{Estimation procedures}\label{estimation-procedures}}

In this section we will discuss the different techniques implemented in
order to estimate the different models, which were described in the
first theoretical part of this work. The different algorithms may result
in discrepancy in seminally identical mathematical models. This
particular difference will be demonstrated in comparison of the MNL
results and the estimates obtained through estimation of a CNN model
imitating MNL model. What is more, different models can provide
different insights into the real world state. For example, MMNL model
should account for heterogeneity in consumer preferences in the presence
of random alternative specific effects.

The econometric models focused on inference and understanding of the
underlying effects are usually estimated over the full dataset as there
is no question about the precision of the obtained results, but rather
the statistical power achieved in idenfication of the effects. We will
follow the same approach in order not to face the different question
related to external validity and verification of the estimated model, as
well as the questions related to verification and testing of the models'
performances over some external dataset.

In this part of the work we will firstly present the different
estimation techniques, starting with \emph{maximum likelihood} (Cosslett
1981) estimator for the MNL, as well as it's algorithmic implementation
within \emph{R}, and the \emph{Adam} algorithm (Kingma and Ba 2014)
traditionally used to estimate the NN models. Afterwards, we will
discuss the results of the estimations we obtain over the generated
datasets, presented in the previous part.

\hypertarget{maximum-likelihood-for-mnl-and-mmnl}{%
\paragraph{Maximum-Likelihood for MNL and
MMNL}\label{maximum-likelihood-for-mnl-and-mmnl}}

The MNL and MMNL models, both are estimated by the maximum likelihood
method. In this technique the estimator is used to derive the
parameters, which were the most likely to produce the observed results
(observed dataset).

Assuming we face probabilities defined by some function \(f(.)\)
parametrized \(\theta\), the joint probability density may be defined
as:

\begin{equation}
\mathcal{L}(\theta) = \prod_{j}^{\Omega} P_{i}(j \mid \theta) \text{  with  } \theta: max_{\theta} \mathcal{L}(\theta)
\end{equation}

This function is also known as likelihood function. The log-likelihood
is obtained through a \(log\) transformation of the likelihood function:

\begin{equation}
L(\theta) = \sum_{j}^{\Omega} log(P_{i}(j \mid \theta)) \text{  with  } \theta: min_{\theta} L(\theta)
\end{equation}

As we can see the obtained function is then minimised by adjusting
\(\theta\) in order to obtain the optimal parameters. The optimisation
problem is non-linear and requires an implementation of some iterative
technique to be solved. Under ``general conditions'' they are
consistent, asymptotically efficient and asymptotically normally
distributed (McFadden 2001).

Speaking about the algorithmic implementation within the statistical
software, the optimization is performed by iteratively updating the
vector of parameters by the amount given by \emph{step} \(\times\)
\emph{direction}. The \emph{step} in this case is a positive scalar and
the \emph{direction} is given by:

\begin{equation}
D = H^{-1} \times g
\end{equation}

Where \(g\) represents the gradient, while \(H^{-1}\) is an estimate of
the inverse of the Hessian matrix.

In this procedure the main question is the choice and estimation
procedure of \(H^{-1}\), which has several possible definitions. For
example, \emph{Broyden--Fletcher--Goldfarb--Shanno (BFGS)} (Broyden
1970) algorithm may be implemented, which is an iterative method for
solving unconstrained non-linear optimization problems. This algorithm
updates \(H^{-1}\) at each iteration using the variations of the vector
of parameters and the gradient. The initial value of the matrix in this
particular case is the inverse of the outer-product of the gradient. The
initial step equals to 1 and, if the new value of the function is
inferior to the previous value, it is divided by two, until a higher
value is obtained. This iterative procedure stops when the gradient is
sufficiently close to 0, which is achieved through comparison of the
\(g \times H^{-1} \times g\) product with the \emph{tolerance} argument.
An alternative stopping condition is achieved by introduction of the
maximum number of iterations for the algorithm, which ensures the
impossibility to fall into an eternal loop. We may summarise this
algorithm as follows, as described in \emph{mlogit} package
documentation by Croissant (2020):

\begin{enumerate}
\def\labelenumi{\arabic{enumi}.}
\tightlist
\item
  The likelihood for the baseline model is calculated (assuming all the
  parameters are 0);
\item
  The function is then evaluated, assuming a step equals to one;
\item
  If the value of likelihood function is lower than the baseline value,
  the step is divided by two until the likelihood increases;
\item
  The gradient \(g\) is then computed;
\end{enumerate}

The authors of \emph{mlogit} package insist that this method is more
efficient than other functions available in \emph{R} at this time. The
codes used to estimate MNL model are available in Appendix (Appendix
D.1).

For the MMNL model there exists an interesting modification for the
algorithm, because we need to estimate the random effects variances and
covariances in case of correlated random effects. The parameters are not
directly introduced inside the likelihood function, but rather the
elements of the Choleski decomposition of the covariance matrix are
used. The Choleski decomposition matrix \(L\) is defined in this case as
follows:

\begin{equation}
L=
    \begin{bmatrix} 
    chol_{11}  & 0         & 0     & 0 \\
    chol_{12}  & chol_{22}    & 0     & 0 \\
    chol_{13}  & chol_{23}    & chol_{33}& 0 \\
    chol_{14}  & chol_{24}    & chol_{34}& chol_{44} \\
    \end{bmatrix}
\quad
\end{equation}

Where indices correspond to the random effects variables, for example,
in our case study we have four parameters: \(Buy\) dummy variable,
eco-\(Label\), \(Carbon\) footprint and the \(LC\), which stands for the
\(Label\) and \(Carbon\) cross-product. Once the estimates of the matrix
elements are obtained a variance-covariance matrix can be obtained:

\begin{equation}
LL^T = 
    \begin{bmatrix} 
    \sigma_{1} ^2  & \sigma_{21}   & \sigma_{31}   & \sigma_{41} \\
    \sigma_{12}     & \sigma_{2}^2 & \sigma_{32}   & \sigma_{42} \\
    \sigma_{13}     & \sigma_{23}   & \sigma_{3}^2 & \sigma_{43} \\
    \sigma_{14}     & \sigma_{24}   & \sigma_{34}   & \sigma_{4}^2 \\
    \end{bmatrix}
= \Sigma
\end{equation}

Where \(\sigma_{i}^2\) stands for the variance of effect \(i\) and
\(\sigma_{ij}\) represents the covariance between two random parameters
\(i\) and \(j\). The codes used to estimate MMNL model may be found in
Appendix (Appendix D.2).

\hypertarget{backpropagation-algorithm-for-nn}{%
\paragraph{Backpropagation algorithm for
NN}\label{backpropagation-algorithm-for-nn}}

For the estimation of the NN model we benefit from the flexibility
offered by \emph{Keras} (Allaire and Chollet 2020), which is a
high-level NN API developed with a focus on the speed of computation,
offering at the same time an astonishing level of control over the
models. The port of \emph{Keras} inside \emph{R} offered by Allaire and
Chollet (2020) allows us to correctly specify our model, devised to
imitate the structure of the traditional MNL. This particular ML library
offers a choice of different model estimation algorithms, ranging from
the \emph{state-of-the-art} to the most recent and advanced techniques.
In this particular application it was decided to implement the
\emph{Adam} algorithm (Kingma and Ba 2014), which can be considered as
rather outdated estimation method by the standards of ML field, because
it was introduced only in 2014.

\emph{Adam} is an algorithm for first-order gradient-based optimization
of stochastic objective functions, based on adaptive estimates of
lower-order moments. This method was proved to be computationally
efficient, as well as to have low memory requirements. It is invariant
to diagonal rescaling of the gradients, and is well suited for problems
that are large in terms of data or parameters. This algorithm is also
considered appropriate for non-stationary objectives, as well as the
problems with very sparse gradients. What is more, one of the particular
advantage for us is that the hyper-parameters do not typically require
advanced tuning.

Historically, the \emph{Adam} algorithm is an extension to the
\emph{stochastic gradient descent} or \emph{SGD} (Kiefer, Wolfowitz, and
others 1952) method. The latter is an iterative method for optimizing a
differentiable or sub-differentiable objective functions. It can be
considered as a stochastic approximation of the \emph{gradient descent}
(GD) optimization, because it replaces the actual gradient, which is
typically calculated from the entire data set, by an estimate, which is
calculated from a randomly selected subset of the data. In
high-dimensional optimization problems this technique reduces the
computational complexity and hence the computation time, resulting in
faster iterations. SGD has a single learning rate, denoted \(\alpha\) by
convention, for all weight updates during training. The learning rate is
considered to fixed through an entire estimation procedure as well. The
two latter features, are sometimes regarded as disadvantage of the
particular estimation technique and may not be suitable in all the
contexts.

Once we have briefly presented its original predecessor, we may pass
directly to \emph{Adam} algorithm description as well as the procedures,
which influenced its creation. The chosen method combines the advantages
of two other extensions of SGD, which are:

\begin{itemize}
\tightlist
\item
  \emph{Adaptive Gradient Algorithm} (AdaGrad), where the per-parameter
  learning rate is maintained fixed, which is suitable for sparse data
  learning problems (Duchi, Hazan, and Singer 2011);
\item
  \emph{Root Mean Square Propagation} (RMSProp), for which the
  per-parameter learning rates are adapted based on the average of
  recent values of the gradients for the weight. Tthis is an unpublished
  method supported by the community, more information may be found in
  Bengio and CA (2015).
\end{itemize}

This properties make the algorithm especially well performing on a
non-stationary problems, including noisy data, as well as any other
problem types. Instead of adapting the parameter learning rates based on
the mean values (first moments) as in RMSProp, \emph{Adam} uses of the
average of the second moments of the gradients as well.

The \emph{Adam} is configured using a following set of hyper-parameters
(for more details on numerical implementation and working with
\emph{Keras} see Appendix D.3):

\begin{itemize}
\tightlist
\item
  \(alpha\), which stands for the learning rate or step size, designing
  the proportion at which the weights are updated. Traditionally, as it
  was proposed by authors Kingma and Ba (2014), the value of \(\alpha\)
  equals to \(1e-8\), but in our application we approach it to the
  values used in the DFGS algorithm, assuming that
  \(\alpha = 1e-1 = 0.1\). Large values results in faster initial
  learning rate, before it is updated, while inferior values slow
  learning significantly and require more runs;
\item
  \(beta_1\), describes the exponential decay rate for the first moment
  estimates. We assume this value to be fixed to the defaults of
  \emph{Keras}, which is \(\beta_1 = 0.9\);
\item
  \(beta_2\) is the exponential decay rate for the second moment
  estimates, which is by default \(\beta_2 = 0.999\);
\item
  \(\epsilon\) is the last hyper-parameter, which is a very small number
  to prevent any division by zero in the algorithm implementation. In
  \emph{Keras} this value is \(\epsilon = 1e-8\).
\end{itemize}

\hypertarget{estimation-results-presentation}{%
\subsubsection{Estimation results
presentation}\label{estimation-results-presentation}}

The comparison of the estimates obtained by the different models over
different datasets can be done in two steps. First of all, we are
interested in the observed mean effects over the datasets, because the
possibility to correctly identify the means for the coefficients is of
utmost importance for the analysis, regardless of the assumption on the
heterogeneity of these effects. Then we are going to explore the
additional dimension, provided by the MMNL estimates, which comprises
the estimates for the variance-covariance matrix of the correlated
random effects. The estimates obtained directly are the entries of the
Choleski decomposition matrix and need to be transformed in order to
observe the variances and covariances.

The results for the means estimates are regrouped in the table
\ref{tab:means} on page \pageref{tab:means}. Now we can pass to the
discussion of the obtained results and demonstrate the differences of
the performances observed for different algorithms. We are going to
start with the discussion of the estimates obtained with more
traditional to econometric field MNL and MMNL models. Effectively, the
MNL model allows us to obtain the exact estimates, due to the fast
convergence rate and the relative simplicity of the problem. What is
more, and what is of particular interest for us, it is how the MMNL
model performs on the MNL specific dataset with fixed effects. The
estimates obtained with the MMNL model for the fixed effects dataset
demonstrate quasi-identical estimates as traditional MNL model, which
nearly all the Choleski decomposition matrix element estimates
statistically insignificant to zeros. Observing the estimates obtained
from the two models we may rightfully conclude, that there is no evident
danger in implementing a MMNL model in place of a MNL model on the fixed
effects dataset, because the obtained estimates will point out the
absence of the heterogeneous preferences in such case. The only
disadvantage of the models misspecification in this case resides in the
significantly increased estimation time, which requires significantly
more iteration in order to estimate correctly the variance-covariance
matrix elements and, consequently, the estimation complexity.

\begin{table}[!htbp] \centering 
  \caption{Estimation results: mean effects} 
  \label{tab:means}  
  \small
\begin{tabular}{@{\extracolsep{0pt}}lD{.}{.}{3} D{.}{.}{3} D{.}{.}{3} D{.}{.}{3} D{.}{.}{3} D{.}{.}{3} D{.}{.}{3} } 
\\[-1.8ex]\hline 
\hline \\[-1.8ex] 
 & \multicolumn{3}{c}{\textit{Fixed effects}} & \multicolumn{3}{c}{\textit{Random effects}} & \multicolumn{1}{c}{\textit{Target}} \\ 
\cline{2-4}\cline{5-7} 
\\[-1.8ex] & \multicolumn{1}{c}{MNL} & \multicolumn{1}{c}{MMNL} & \multicolumn{1}{c}{CNN} & \multicolumn{1}{c}{MNL} & \multicolumn{1}{c}{MMNL} & \multicolumn{1}{c}{CNN} & \\ 
\hline \\[-1.8ex] 
\textbf{Characteristics} & & & & & & & \\ 
 ~~~Sex & 1.401^{***} & 1.400^{***} & 1.369 & 0.712^{***} & 1.297^{***} & 0.719 & 1.420 \\ 
  & (0.031) & (0.031) & & (0.016) & (0.024) & & \\ 
 ~~~Age & 0.009^{***} & 0.009^{***} & 0.010 & 0.007^{***} & 0.010^{***} & 0.005 & 0.009 \\ 
  & (0.001) & (0.001) & & (0.001) & (0.001) & & \\ 
 ~~~Salary & 0.048^{***} & 0.048^{***} & 0.060 & 0.066^{***} & 0.120^{***} & 0.062 & 0.057 \\ 
  & (0.010) & (0.010) & & (0.005) & (0.008) & & \\ 
 ~~~Habit & 1.070^{***} & 1.071^{***} & 1.056 & 0.361^{***} & 0.641^{***} & 0.343 & 1.027 \\ 
  & (0.030) & (0.030) & & (0.016) & (0.024) & & \\ 
\textbf{Attributes} & & & & & & & \\ 
 ~~~Price & -1.626^{***} & -1.628^{***} & -1.618 & -0.886^{***} & -1.586^{***} & -0.886 & -1.631 \\ 
  & (0.010) & (0.010) & & (0.006) & (0.010) & & \\ 
 ~~~Buy & 2.311^{***} & 2.313^{***} & 2.228 & 0.662^{***} & 2.180^{***} & 0.665 & 2.285 \\ 
  & (0.065) & (0.066) & & (0.036) & (0.054) & & \\ 
 ~~~Label & 2.815^{***} & 2.817^{***} & 2.810 & 1.279^{***} & 1.922^{***} & 1.277 & 2.824 \\ 
  & (0.022) & (0.022) & & (0.015) & (0.023) & & \\ 
 ~~~Carbon & 6.654^{***} & 6.662^{***} & 6.634 & 3.259^{***} & 5.430^{***} & 3.250 & 6.665 \\ 
  & (0.032) & (0.033) & & (0.016) & (0.030) & & \\ 
 ~~~LC & -2.781^{***} & -2.782^{***} & -2.765 & -1.546^{***} & -2.663^{***} & -1.558 & -2.785 \\ 
  & (0.028) & (0.028) & & (0.019) & (0.030) & & \\ 
\hline 
\hline \\[-1.8ex] 
\textit{Note:}  & \multicolumn{7}{r}{$^{*}$p$<$0.1; $^{**}$p$<$0.05; $^{***}$p$<$0.01} \\ 
\end{tabular} 
\end{table}

On the contrary, in the case of presence of the correlated random
effects in the preferences of the population the estimates are
significantly biased for the MNL model. Moreover, the estimates obtained
with the MMNL model are not identical to the input parameters, which
were used during the simulation step. In this situation the MNL model
tends to significantly underestimate the effects of all the
characteristics and attributes for the choice situation. This can
potentially lead to a notorious bias in case we were using incorrect
model specification during a field experiment data exploration.

The results for the Choleski matrix entries estimates are regrouped into
a single table \ref{tab:vars} on page \pageref{tab:vars} Based on these
estimates, we can comment as well the potential inefficiency of the
implemented algorithm, even if it is one of the best available to us.
Even though the estimates of the means obtained with MMNL in the
presence of the random effects are close to the theoretical ones, the
estimates of the variance-covariance matrix elements are rather close,
but not perfectly calculated. Which is important, as we had a rather
large dataset compared to the datasets typically collected during field
studies: 1000 individuals with 10 replications of 16 choice sets
situations for each totalling to 160000 choice situations. While in the
original field study 102 individuals with only 2 replications of 6
choice sets were present, mounting to 1224 observations.

This situation demonstrates the existing trade-off between the need to
correctly specify the model from the start and the potential computation
inconveniences in the case of implementation of a more complex model in
case of uncertainty. In other words, the scientists always face the
choice either to simply use more complex model, which requires more
data, calculation time and resources, or to perform an extensive
theoretical study beforehand in order to correctly specify and delimit
the model from the start.

\begin{table}[!htbp] \centering 
  \caption{Estimation results: standard deviations and covariances} 
  \label{tab:vars}
  \small
\begin{tabular}{@{\extracolsep{0pt}}lD{.}{.}{3} D{.}{.}{3} D{.}{.}{3} } 
\\[-1.8ex]\hline 
\hline \\[-1.8ex] 
 & \multicolumn{1}{c}{\textit{Fixed effects}} & \multicolumn{1}{c}{\textit{Random effects}} & \multicolumn{1}{c}{\textit{Target}} \\ 
\cline{2-2}\cline{3-3} 
\\[-1.8ex] & \multicolumn{1}{c}{MMNL} & \multicolumn{1}{c}{MMNL} & \\ 
\hline \\[-1.8ex] 
\textbf{Standard deviations} & & & \\ 
 ~~~Buy & 0.095 & 2.960^{***} & 3.202 \\ 
  & (0.061) & (0.028) & \\ 
 ~~~Label & 0.031 & 2.687^{***} & 2.654 \\ 
  & (0.077) & (0.023) & \\ 
 ~~~Carbon & 0.164^{*} & 3.734^{***} & 3.535 \\ 
  & (0.076) & (0.026) & \\ 
 ~~~LC & 0.145^{*} & 2.851^{***} & 2.711 \\ 
  & (0.071) & (0.031) & \\ 
\textbf{Covariances} & & & \\ 
 ~~~Buy:Label & -0.948 & -0.311^{***} & -0.54 \\ 
  & (5.116) & (0.026) & \\ 
 ~~~Buy:Carbon & -0.886 & -0.565^{***} & -4.39 \\ 
  & (1.954) & (0.026) & \\ 
 ~~~Label:Carbon & 0.891 & 0.959^{***} & 8.77 \\ 
  & (1.578) & (0.003) & \\ 
 ~~~Buy:LC & 0.669 & 0.789^{***} & 6.17 \\ 
  & (0.501) & (0.005) & \\ 
 ~~~Label:LC & -0.576 & -0.490^{***} & - 2.33 \\ 
  & (4.423) & (0.032) & \\ 
 ~~~Carbon:LC & -0.568 & -0.651^{***} & -4.82 \\ 
  & (1.604) & (0.030) & \\ 
\hline 
\hline \\[-1.8ex] 
\textit{Note:}  & \multicolumn{3}{r}{$^{*}$p$<$0.1; $^{**}$p$<$0.05; $^{***}$p$<$0.01} \\ 
\end{tabular} 
\end{table}

Now we can switch to the discussion of the estimates obtained with
\emph{Adam} estimated CNN model, identical in structure to the MNL
model. For reminder, we use convolution layers to calculate the relative
deterministic utilities for the population \(V_j\) for three
alternatives, which are then converted to probabilities using a softmax
dense layer with predefined unit weights for corresponding neurons.

As for the CNN estimates, the table \ref{tab:means} demonstrates, that
the obtained estimates are technically identical to the means, we could
see in the previous part for the MNL model estimates. These results
demonstrate the flexibility of the NN models and the hypothetical
possibility to implement them in place of traditional econometric models
with only inconvenience being the relative complexity to obtain the
variances for the weights estimates, as non known to us method allows
this. This only inconvenience renders impossible to analyse the
statistical significance for the obtained weight estimates, which can be
seen only over the marginal effects graphs for particular variable on
the probability, but this is other discussion's topic. For now, the most
important part is that the CNN imitation of the MNL models, estimated
with a high learning rate (\(\alpha = 1e-1\)) \emph{Adam} algorithm,
allows to obtain correct estimates for the means of the theoretical
utility function, assuming the variables were chosen correctly.

Because of the nature of the constructed CNN model latter performs
similarly to the traditional MNL model. This situation implies that the
proposed CNN algorithm is, identically to MNL model, unable to identify
correct parameters and consequently derive the true means for the
underlying coefficients of the relative utility function in the presence
of heterogeneous preferences among individuals. Nevertheless, given the
flexibility of the NN it is theoretically possible to device an
algorithm imitating the MMNL model's behaviour or even propose some
alternative modelling techniques which will be able to supply, not
directly through estimated weights but rather after a supplementary
study, the correct estimates for marginal effects of the attributes on
the choice probabilities.

To summarise this section, we can underline the successful
implementation of the chosen mathematical models over the artificially
created datasets simulating different choice situations. The effects
identified by all of the models are close to the target values, although
there exists clear evidence that the MMNL models perform significantly
better in mean effects identification in all the contexts. At the same
time the MNL model and its synthetically recreated NN counterpart
underestimate the coefficient of the given relative utility functions in
presence of the correlated random parameters in the individual
utilities.

\FloatBarrier

\hypertarget{performance-evaluation-and-comparison}{%
\subsection{Performance evaluation and
comparison}\label{performance-evaluation-and-comparison}}

This section comprises the results we managed to achieve in the
exploration of different performance metrics and provides insights on
the functioning of the discussed mathematical models in a given context.
As we have seen in the previous part, where the effects' estimates were
provided, all of the models are able to provide some estimates for the
retaliate utility function parameters in different discrete choice
set-ups. The most simple models performed well on the dataset defined by
the homogeneous preferences in the population for environmental
attributes, underestimating the effects in the presence of preference
heterogeneity. In the same time the more complex MMNL model performed
sufficiently well in both behavioural set-ups, although it demonstrated
some potential problems with the algorithmic implementation.

\hypertarget{overall-precision}{%
\subsubsection{Overall precision}\label{overall-precision}}

\FloatBarrier

First of all we focus our attention on the general performance metrics,
describing how well the estimated models fit the predicted outcomes over
an original dataset. As we have discussed earlier we use only some of
the available measures in an attempt not to make this work too
cumbersome. The retained performance metrics are: accuracy, describing
the overall goodness of fit over observed choices of the subjects; and
more complex KDL measure, which compares the distributions instead of
more simple metrics, which use only the information available in the
confusion matrix.

We can observe the values of these general performance measures,
describing overall performance of a given classifier in the table
\ref{tab:gpm}. The table regroups the metrics' values for all the
estimated models.

\begin{table}[!htbp] \centering 
  \caption{General performance measures} 
  \label{tab:gpm} 
\begin{tabular}{@{\extracolsep{5pt}} lcccccc} 
\\[-1.8ex]\hline 
\hline \\[-1.8ex] 
& \multicolumn{3}{c}{\textit{Fixed effects}} & \multicolumn{3}{c}{\textit{Random effects}} \\ 
\cline{2-4}\cline{5-7} 
\\[-1.8ex] & MNL & MMNL & CNN & MNL & MMNL & CNN \\ 
\hline \\[-1.8ex] 
\textbf{Overall measures} $ $ $ $ $ $ \\
~~~Accuracy & $0.863$ & $0.863$ & $0.723$ & $0.725$ & $0.863$ & $0.721$ \\ 
\textbf{Probabilistic measures} $ $ $ $ $ $ \\
~~~KLD & $0.623$ & $0.623$ & $0.328$ & $0.349$ & $0.625$ & $0.317$ \\ 
\hline \\[-1.8ex] 
\end{tabular} 
\end{table}

As we have underlined earlier we observe quite natural situation when
the best model in terms of overall performance is the model, which was
used in the data generation step. This situation perfectly demonstrates
the potential bias, which is explained by our choice of the artificial
data-generation algorithm. Nevertheless, it should be noted, that the
MNL and MMNL models perform equally well on the fixed effects dataset,
where the preferences for the environmental attributes are homogeneous.
This fact supports our initial hypothesis that an implementation of a
more complex model is preferred when the real effects are unknown to the
researcher.

Focusing our attention on the CNN model observe that the \emph{Adam}
algorithm did not outperform the \emph{BFGS} procedure. This observation
may be explained by the data-generation set-up, where the generative
algorithm favoured the MNL model, rather than \emph{Adam}. The latter
not supporting the fine tuning over the error distribution.

We can observe the results for the resources efficiency we managed to
obtain, which are regrouped in the table \ref{tab:time}. Even though we
present all the time values, we are mostly interested with the ``user''
and ``system'' time values. The first one indicates the CPU time charged
for the execution of user instructions of the calling process, while the
second one stand for the CPU time spent for execution by the system on
behalf of the calling process.

\begin{table}[!htbp] \centering 
  \caption{Ressources efficiency} 
  \label{tab:time} 
\begin{tabular}{@{\extracolsep{5pt}} lcccccc} 
\\[-1.8ex]\hline 
\hline \\[-1.8ex] 
& \multicolumn{3}{c}{\textit{Fixed effects}} & \multicolumn{3}{c}{\textit{Random effects}} \\ 
\cline{2-4}\cline{5-7} 
\\[-1.8ex] & MNL & MMNL & CNN & MNL & MMNL & CNN \\ 
\hline \\[-1.8ex] 
User & 20.910 & 452.414 & 17.433 & 18.722 & 2066.934 & 16.806 \\
System & 0.153 & 1.712 & 0.714 & 0.004 & 16.112 & 0.415 \\
Total & 21.068 & 454.192 & 8.412 & 18.726 & 2083.221 & 7.604 \\
\hline \\[-1.8ex] 
\end{tabular} 
\end{table}

The more advanced \emph{Adam} algorithm easily bypasses the algorithms
available in the \emph{mlogit} package, although this boost in
efficiency goes at the cost of lower overall performance and goodness of
fit. At the same time, the MMNL implementation is far less efficient and
takes 128 times more time, than CNN model. This situation clearly
illustrates us how the precision and flexibility come at higher costs.

\FloatBarrier

\hypertarget{alternative-specific-metrics}{%
\subsubsection{Alternative specific
metrics}\label{alternative-specific-metrics}}

We proceed with a look at some more specific measures. The table
\ref{tab:vspm} regroups response specific metrics, that describe the
precision of model in predicting only one target class of the dataset.
These metrics are mostly used when we are interested in some in-depth
insight into the model performance and allow to identify the models
which perform the best over a single class of interest. Given the
context of Michaud, Llerena, and Joly (2012) study we are interested in
identifying the algorithm which predicts the best ``buy'' (A and B
alternatives) against ``not buy'' (C) alternative, providing at the same
time some information about the alternative chosen. In order to evaluate
the performance at this dimension we use Geometric mean and the
F-measure performance estimators.

\begin{table}[!htbp] \centering 
  \caption{Variable specific performance measures, fixed effects data} 
  \label{tab:vspm} 
\begin{tabular}{@{\extracolsep{5pt}} lcccccc} 
\\[-1.8ex]\hline 
\hline \\[-1.8ex] 
& \multicolumn{3}{c}{\textit{Fixed effects}} & \multicolumn{3}{c}{\textit{Random effects}} \\
\cline{2-4}\cline{5-7} 
\\[-1.8ex] & C & A & B & C & A & B \\ 
\hline \\[-1.8ex] 
\textbf{Geometric mean} & & & & & & \\
  ~~~MNL & $0.454$ & $0.848$ & $0.868$ & $0.432$ & $0.696$ & $0.693$ \\
  ~~~MMNL & $0.454$ & $0.849$ & $0.867$ & $0.452$ & $0.848$ & $0.867$ \\ 
  ~~~CNN & $0.443$ & $0.697$ & $0.698$ & $0.447$ & $0.697$ & $0.700$ \\
\textbf{F-measure} & & & & & & \\
  ~~~MNL & $0.318$ & $0.834$ & $0.873$ & $0.282$ & $0.666$ & $0.704$ \\
  ~~~MMNL & $0.318$ & $0.834$ & $0.873$ & $0.316$ & $0.833$ & $0.873$ \\ 
  ~~~CNN & $0.291$ & $0.665$ & $0.706$ & $0.294$ & $0.665$ & $0.707$ \\
\hline \\[-1.8ex] 
\end{tabular} 
\end{table}

In the table \ref{tab:vspm} we are interested with the entries in the
columns corresponding to the ``No buy'' alternative (C). For the dataset
with fixed effects across the population, the MNL and MMNL models
perform identically according to both of the selected measures. The CNN
model falls behind the econometrics models on the fixed effects dataset,
although situation changes in the presence of heterogeneous effects. In
the more complex case scenario, when the individuals have varying across
population preferences towards one or another attribute, the CNN model
outperforms the simple MNL model in detecting ``No buy'' decisions for
given choice sets, which is rather interesting, because the overall
model's performance is still inferior to the MNL, as it was shown in
table \ref{tab:vspm}.

\FloatBarrier

\hypertarget{willingness-to-pay-and-premiums-1}{%
\subsubsection{Willingness to pay and
premiums}\label{willingness-to-pay-and-premiums-1}}

Here we should present the most important results comparing the
estimates for the WTP, as well as the premiums for particular attributes
derived for different models. The Premium to pay for a rose's particular
attribute as it was described previously can be represented as:

\begin{equation}
Premium = \frac{
  \frac{\delta V}{\delta X_k}
}{
   \frac{\delta V}{\delta Price}
}
\end{equation}

At the same time, the WTP for a rose may be seen as the ratio of two
corresponding coefficients of dummy variable and price. The table
\ref{tab:wtp} presents the estimated WTP and premiums for the models,
which output fixed coefficient estimates, without taking into account
the randomness of the individual effects. In other words, this table
regroups the results, which do not require bootstrapping for confidence
interval estimation.

\begin{table}[!htbp] \centering  
   \caption{WTP and Premiums obtained with MNL and CNN}  
   \label{tab:wtp}  
 \begin{tabular}{@{\extracolsep{5pt}} lccccc}  
 \\[-1.8ex]\hline  
 \hline \\[-1.8ex]  
& \multicolumn{2}{c}{\textit{Fixed effects}} & \multicolumn{2}{c}{\textit{Random effects}} & \multicolumn{1}{c}{\textit{Target}} \\
\cline{2-3}\cline{4-5} 
\\[-1.8ex] & MNL & CNN & MNL & CNN & \\  
 \hline \\[-1.8ex]  
 WTP & $1.421$ & $1.377$ & $0.747$ & $0.751$ & $1.401$ \\  
 Label & $1.731$ & $1.737$ & $1.445$ & $1.442$ & $1.731$ \\  
 Carbon & $4.091$ & $4.101$ & $3.679$ & $3.669$ & $4.086$ \\  
 LC & $4.112$ & $4.129$ & $3.378$ & $3.352$ & $4.110$ \\  
 \hline \\[-1.8ex]  
 \end{tabular}  
 \end{table}

For the estimation of the WTP and the premiums for more complex models
(the MMNL in our case) we use the same procedure, as was implemented by
Michaud, Llerena, and Joly (2012). Because the random parameters are
assumed to be correlated in the MMNL model's specification, the
estimated standard deviations and confidence intervals are obtained
using the Krinsky and Robb parametric bootstrapping method (Krinsky and
Robb 1986). This procedure consists of generating of multiple random
draws from a multivariate normal distribution and using the obtained
results to obtain the confidence interval estimates. Exactly as in the
original study we generate 1000 draws from a multivariate normal
distribution (\(MNV(\mu, \Sigma)\)), with the coefficient estimates as
means \(\mu\) and the estimated variance-covariance matrix of the random
parameters as \(\Sigma\).

The obtained results are then summarised as follows in the table
\ref{tab:wtpr}

\begin{table}[!htbp] \centering 
  \caption{WTP and Premiums obtained with MMNL} 
  \label{tab:wtpr} 
\begin{tabular}{@{\extracolsep{5pt}}lccccccc} 
\\[-1.8ex]\hline 
\hline \\[-1.8ex]  
& \multicolumn{6}{c}{\textit{Statistics}} \\
\cline{2-7} 
\\[-1.8ex] & \multicolumn{1}{c}{Mean} & \multicolumn{1}{c}{St. Dev.} & \multicolumn{1}{c}{Min} & \multicolumn{1}{c}{Pctl(25)} & \multicolumn{1}{c}{Pctl(75)} & \multicolumn{1}{c}{Max} \\ 
\hline \\[-1.8ex] 
\textbf{Fixed effects} & & & & & & \\
  ~~~WTP & 1.416 & 0.058 & 1.233 & 1.377 & 1.455 & 1.613 \\ 
  ~~~Label & 1.732 & 0.019 & 1.672 & 1.720 & 1.745 & 1.791 \\ 
  ~~~Carbon & 4.097 & 0.103 & 3.730 & 4.026 & 4.166 & 4.434 \\ 
  ~~~LC & 4.116 & 0.098 & 3.741 & 4.051 & 4.182 & 4.421 \\ 
\textbf{Random effects} & & & & & & \\ 
  ~~~WTP & 1.360 & 1.887 & $-$4.239 & 0.073 & 2.662 & 7.893 \\ 
  ~~~Label & 1.243 & 1.667 & $-$3.867 & 0.104 & 2.330 & 6.638 \\ 
  ~~~Carbon & 3.467 & 2.323 & $-$4.026 & 1.880 & 5.043 & 11.671 \\ 
  ~~~LC & 3.036 & 3.240 & $-$7.430 & 0.908 & 5.160 & 14.259 \\ 
\textbf{Target} & & & & & & \\
  ~~~WTP & 1.418 & 1.973 & $-$4.474 & 0.058 & 2.798 & 6.706 \\
  ~~~Label & 1.735 & 1.611 & $-$2.652 & 0.653 & 2.849 & 6.709 \\ 
  ~~~Carbon & 4.076 & 2.134 & $-$1.774 & 2.608 & 5.543 & 11.217 \\ 
  ~~~LC & 4.106 & 3.379 & $-$6.304 & 1.913 & 6.439 & 14.612 \\ 
\hline \\[-1.8ex] 
\textit{Note:}  & \multicolumn{6}{r}{The estimates are obtained with 1000 draws from MNV distribution} \\ 
\end{tabular} 
\end{table}

Comparing the estimates to the input values we observe that the variance
of the WTP and Premiums estimates, estimated over a fixed effects
dataset, do not potentially affect the conclusion one can derive from
the results. The values stay positive with the 75\% interval within 0.2€
of the mean estimate. Assuming the model is not re-estimated and
adjusted after the insignificant estimators are obtained for Choleski
matrix elements, the results remain valid.

We may conclude, that given sufficiently large dataset the
implementation of more complex model is preferable, because it will
allow to control for unknown parameters without adding a risk of
obtaining biased results. The more simple models, should be preferred in
a more restricted context. They allow to obtain the valid results only
in the case of correct theoretical assumptions, biasing the estimates in
other conditions. Consequently, in the presence of uncertainty about the
presence of heterogeneity in the customer choice modelling questions
there is a strong interest to implement a more complex model,
readjusting it afterwards if needed.

\FloatBarrier

\FloatBarrier

\newpage

\hypertarget{conclusion}{%
\section*{Conclusion}\label{conclusion}}
\addcontentsline{toc}{section}{Conclusion}

In this work we have introduced the reader to the problematic of the
different modelling paradigms in application to the consumer choice
studies. By means of an experimental theory-testing framework we
demonstrate the complexity of the model performance evaluation
problematic, showing the eventual bottlenecks and the questions to be
answered on all the levels of data exploration procedure. The correct
specification of the theoretical assumptions, the dataset generation,
the model choice as well as the performance measure choice were studied.
The main objective to propose a comprehensive methodology for
theory-testing framework creation was accomplished, illustrating the
devised frameworks' potential over an economic question issued from real
world.

Two different consumer choice situation were explored, issued from the
setting delimited by Michaud, Llerena, and Joly (2012). The discrete
choice context allowed us to compare how the presence of heterogeneous
preferences for environmental attributes affected the possibility to
identify correctly the underlying utility functions, as well as to
derive the WTP and premiums for the attributes. The implementation of
artificial dataset simulation techniques proved its potential in
creation of fully controlled data samples, providing two consistent
datasets constructed under RUM assumptions. Given the data, we could
observe, how taste heterogeneity affected the population's choice
distribution and the resulting datasets, as well as their impact on
models' performances.

A total of three models, issued from alien disciplines such as
econometrics (MNL and MMNL) and ML (CNN-MNL), were implemented over the
generated artificial datasets. We could demonstrate the differences and
similarities between the traditional econometrics models and such ML
techniques as NN. The econometric models allowed us to observe the
potential biases that researchers risk to induce using the simplest
models in unjustified context. The ML model made it possible to
demonstrate, how different approaches to optimisation and algorithmic
solutions influence the obtained results. Moreover, the framework
demonstrated, that ML models could be used instead of the traditional
econometrics techniques under correct specification, as technically NN
are able to approximate any other more simple linear or non-linear
model. All of the models demonstrated good overall performance given the
homogeneous individual preferences setting, while only the most complex
MMNL model achieved sufficient results in presence of taste
heterogeneity.

The multidimensionality of the explored situation allows us to tear
several solutions from this work in terms of model performances in
presence of heterogeneous preferences. The MMNL models demonstrated a
better adaptivity for the different datasets and consequently a better
adaptiveness in all the cases. This family of models showed a great
tolerance for the eventual misspecification in the assumptions of the
presence of random effects. On the contrary, the MNL models produced
biased estimates in the presence of the random effects in population,
which indicates a great danger and signal the importance of the correct
specifications a preliminary data studies to be performed before the
models estimation. The only observed difference was in the way the
resulting approximation was unable to directly estimate the variance for
the linear part coefficients, which is not initially the main focus of
the NN models. However, the marginal effects could still be derived for
the individual characteristics or the alternative specific attributes,
assuming a correct approximation was used, which does not inflate the
overall variance for the marginal effects.

Nevertheless, there exist potential biases that require particular
attention and caution in future research. The implemented
data-generation procedure risks to bias the results in favour of the
econometrics models, which were used to simulate the data. Speaking
about the models, we have observed that the adaptiveness and flexibility
of the MMNL model comets at some costs in resources efficiency. The
time, computation power and the data amount needed to achieve satisfying
results are significantly higher than for the other models.

This work demonstrates only a fraction of the full potential of the
theory-testing framework. Many extensions and generalisations should be
performed before it could be used at scale. For example, it is
particularly interesting to introduce an extension which will provide
the possibility to explore and compare how different behavioural
theories (RUM, RRM, QDM) affects the estimation results. Even more, with
this methodology it becomes possible to explore the effects of
non-additive utility presence or the behaviour of populations with mixed
behaviours presence. Another extension concerns the implemented
mathematical models and consists in incorporating the most recent
developments in the ML field into the framework, enabling users to
implement such models as decision trees or more advanced NN. Last, but
not the least, the framework could be complemented with a methodological
tool-set for hypothesis testing using the advantages of a controlled
experiment data collection.

To summarise, we conclude that the experimental framework has proven its
importance for the empirical and theoretical studies and has
demonstrated its potential. There clearly exist a strong need for a more
extensive study and development of this framework to provide the
research community with a hypothesis testing tool-set, which could be
used in the context of the consumer choice modelling. The exploration of
potential biases and theory-testing will allow us to establish a
comprehensive and consistent methodology to be implemented latter in
empirical work and controlled experiments in particular.

\newpage
\renewcommand\contentsname{}
\setcounter{tocdepth}{4}

\hypertarget{table-of-contents}{%
\section*{Table of contents}\label{table-of-contents}}
\addcontentsline{toc}{section}{Table of contents}

\vspace{-12mm}
\tableofcontents

\newpage

\hypertarget{list-of-figures}{%
\section*{List of figures}\label{list-of-figures}}
\addcontentsline{toc}{section}{List of figures}

\renewcommand\listfigurename{}
\vspace{-12mm}
\listoffigures

\newpage

\hypertarget{list-of-tables}{%
\section*{List of tables}\label{list-of-tables}}
\addcontentsline{toc}{section}{List of tables}

\renewcommand\listtablename{}
\vspace{-12mm}
\listoftables

\newpage

\hypertarget{bibliography}{%
\section*{Bibliography}\label{bibliography}}
\addcontentsline{toc}{section}{Bibliography}

\hypertarget{refs}{}
\leavevmode\hypertarget{ref-agrawal2019nber}{}%
Agrawal, Ajay, Joshua Gans, and Avi Goldfarb. 2019. \emph{The Economics
of Artificial Intelligence: An Agenda}. Book. National Bureau of
Economic Research; University of Chicago Press.
\url{https://doi.org/https://doi.org/10.7208/chicago/9780226613475.001.0001}.

\leavevmode\hypertarget{ref-agresti2007cd}{}%
Agresti, Alan. 2007. \emph{An Introduction to Categorical Data Analysis,
Second Edition}.

\leavevmode\hypertarget{ref-agresti2013cd}{}%
---------. 2013. \emph{Categorical Data Analysis, Third Edition}.

\leavevmode\hypertarget{ref-R-keras}{}%
Allaire, JJ, and François Chollet. 2020. \emph{Keras: R Interface to
'Keras'}. \url{https://CRAN.R-project.org/package=keras}.

\leavevmode\hypertarget{ref-R-tensorflow}{}%
Allaire, JJ, and Yuan Tang. 2020. \emph{Tensorflow: R Interface to
'Tensorflow'}. \url{https://CRAN.R-project.org/package=tensorflow}.

\leavevmode\hypertarget{ref-R-rmarkdown}{}%
Allaire, JJ, Yihui Xie, Jonathan McPherson, Javier Luraschi, Kevin
Ushey, Aron Atkins, Hadley Wickham, Joe Cheng, Winston Chang, and
Richard Iannone. 2018. \emph{Rmarkdown: Dynamic Documents for R}.
\url{https://CRAN.R-project.org/package=rmarkdown}.

\leavevmode\hypertarget{ref-anderson1992discrete}{}%
Anderson, Simon P, Andre De Palma, and Jacques-Francois Thisse. 1992.
\emph{Discrete Choice Theory of Product Differentiation}. MIT press.

\leavevmode\hypertarget{ref-athey2018iml}{}%
Athey, Susan. 2018. ``The Impact of Machine Learning on Economics.''
Book. In \emph{The Economics of Artificial Intelligence: An Agenda}, by
Ajay Agrawal, Joshua Gans, and Avi Goldfarb, 507--47. National Bureau of
Economic Research; University of Chicago Press.
\url{https://doi.org/https://doi.org/10.7208/chicago/9780226613475.001.0001}.

\leavevmode\hypertarget{ref-athey2019ml}{}%
Athey, Susan, and Guido W. Imbens. 2019. ``Machine Learning Methods That
Economists Should Know About.'' \emph{Annual Review of Economics} 11
(1): 685--725.
\url{https://doi.org/10.1146/annurev-economics-080217-053433}.

\leavevmode\hypertarget{ref-ayodele2010tml}{}%
Ayodele, Taiwo Oladipupo. 2010. ``Types of Machine Learning
Algorithms.'' \emph{New Advances in Machine Learning}. InTech, 19--48.

\leavevmode\hypertarget{ref-baayen2017gam}{}%
Baayen, Harald, Shravan Vasishth, Reinhold Kliegl, and Douglas Bates.
2017. ``The Cave of Shadows: Addressing the Human Factor with
Generalized Additive Mixed Models.'' \emph{Journal of Memory and
Language} 94: 206--34.
\url{https://doi.org/https://doi.org/10.1016/j.jml.2016.11.006}.

\leavevmode\hypertarget{ref-baldi2000ar}{}%
Baldi, Pierre, Søren Brunak, Yves Chauvin, Claus A. F. Andersen, and
Henrik Nielsen. 2000. ``Assessing the accuracy of prediction algorithms
for classification: an overview.'' \emph{Bioinformatics} 16 (5):
412--24. \url{https://doi.org/10.1093/bioinformatics/16.5.412}.

\leavevmode\hypertarget{ref-baltagi2008econometric}{}%
Baltagi, Badi. 2008. \emph{Econometric Analysis of Panel Data}. John
Wiley \& Sons.

\leavevmode\hypertarget{ref-bengio2015rmsprop}{}%
Bengio, Yoshua, and MONTREAL CA. 2015. ``Rmsprop and Equilibrated
Adaptive Learning Rates for Nonconvex Optimization.'' \emph{Corr
Abs/1502.04390}.

\leavevmode\hypertarget{ref-bernard2009organic}{}%
Bernard, John C, and Daria J Bernard. 2009. ``What Is It About Organic
Milk? An Experimental Analysis.'' \emph{American Journal of Agricultural
Economics} 91 (3). Oxford University Press: 826--36.

\leavevmode\hypertarget{ref-bhat1995evm}{}%
Bhat, Chandra R. 1995. ``A Heteroscedastic Extreme Value Model of
Intercity Travel Mode Choice.'' Suggested. \emph{Transportation Research
Part B: Methodological} 29 (6): 471--83.
\url{https://EconPapers.repec.org/RePEc:eee:transb:v:29:y:1995:i:6:p:471-483}.

\leavevmode\hypertarget{ref-joly2019qcm}{}%
Bouscasse, Hélène, Iragaël Joly, and Jean Peyhardi. 2019. ``A new family
of qualitative choice models: An application of reference models to
travel mode choice.'' \emph{Transportation Research Part B:
Methodological} 121 (C): 74--91.
\url{https://doi.org/10.1016/j.trb.2018.12.010}.

\leavevmode\hypertarget{ref-brathwaite2017machine}{}%
Brathwaite, Timothy, Akshay Vij, and Joan L Walker. 2017. ``Machine
Learning Meets Microeconomics: The Case of Decision Trees and Discrete
Choice.'' \emph{arXiv Preprint arXiv:1711.04826}.

\leavevmode\hypertarget{ref-breiman2001stat}{}%
Breiman, Leo, and others. 2001. ``Statistical Modeling: The Two Cultures
(with Comments and a Rejoinder by the Author).'' \emph{Statistical
Science} 16 (3). Institute of Mathematical Statistics: 199--231.

\leavevmode\hypertarget{ref-brock2003mcsi}{}%
Brock, William, and Steven Durlauf. 2003. ``Multinomial Choice with
Social Interactions.'' NBER Technical Working Papers 0288. National
Bureau of Economic Research, Inc.
\url{https://EconPapers.repec.org/RePEc:nbr:nberte:0288}.

\leavevmode\hypertarget{ref-broyden1970convergence}{}%
Broyden, Charles G. 1970. ``The Convergence of Single-Rank Quasi-Newton
Methods.'' \emph{Mathematics of Computation} 24 (110): 365--82.

\leavevmode\hypertarget{ref-cascetta2009tr}{}%
Cascetta, Ennio. 2009. \emph{Transportation Systems Analysis: Models and
Applications}. Vol. 29. Springer Science \& Business Media.

\leavevmode\hypertarget{ref-chen2013rac}{}%
Chen, Bryant, and Judea Pearl. 2013. ``Regression and Causation: A
Critical Examination of Six Econometrics Textbooks.'' \emph{Real-World
Economics Review, Issue}, no. 65: 2--20.

\leavevmode\hypertarget{ref-chorus2010rrm}{}%
Chorus, Caspar G. 2010. ``A New Model of Random Regret Minimization.''
\emph{European Journal of Transport and Infrastructure Research} 10 (2).

\leavevmode\hypertarget{ref-cosslett1981maximum}{}%
Cosslett, Stephen R. 1981. ``Maximum Likelihood Estimator for
Choice-Based Samples.'' \emph{Econometrica: Journal of the Econometric
Society}. JSTOR, 1289--1316.

\leavevmode\hypertarget{ref-coussement2010gam}{}%
Coussement, Kristof, Dries F. Benoit, and Dirk Van den Poel. 2010.
``Improved Marketing Decision Making in a Customer Churn Prediction
Context Using Generalized Additive Models.'' \emph{Expert Systems with
Applications} 37 (3): 2132--43.
\url{https://doi.org/https://doi.org/10.1016/j.eswa.2009.07.029}.

\leavevmode\hypertarget{ref-R-mlogit}{}%
Croissant, Yves. 2020. \emph{Mlogit: Multinomial Logit Models}.
\url{https://CRAN.R-project.org/package=mlogit}.

\leavevmode\hypertarget{ref-denuit2019as3}{}%
Denuit, Michel, and Donatien Hainaut. 2019. \emph{Effective Statistical
Learning Methods for Actuaries Iii: Neural Networks and Extentions}.
Springer.

\leavevmode\hypertarget{ref-denuit2019as1}{}%
Denuit, Michel, and Julien Trufin. 2019. \emph{Effective Statistical
Learning Methods for Actuaries I: GLMs and Extentions}. Springer.

\leavevmode\hypertarget{ref-depalma2011tr}{}%
De Palma, André, Robin Lindsey, Emile Quinet, and Roger Vickerman. 2011.
\emph{A Handbook of Transport Economics}. Edward Elgar Publishing.

\leavevmode\hypertarget{ref-donoho2017ds}{}%
Donoho, David. 2017. ``50 Years of Data Science.'' \emph{Journal of
Computational and Graphical Statistics} 26 (4). Taylor \& Francis:
745--66.

\leavevmode\hypertarget{ref-duchi2011adaptive}{}%
Duchi, John, Elad Hazan, and Yoram Singer. 2011. ``Adaptive Subgradient
Methods for Online Learning and Stochastic Optimization.'' \emph{Journal
of Machine Learning Research} 12 (7).

\leavevmode\hypertarget{ref-fiebig2010gmlm}{}%
Fiebig, Denzil, Michael Keane, Jordan Louviere, and Nada Wasi. 2010.
``The Generalized Multinomial Logit Model: Accounting for Scale and
Coefficient Heterogeneity.'' \emph{Marketing Science} 29 (3): 393--421.
\url{https://EconPapers.repec.org/RePEc:inm:ormksc:v:29:y:2010:i:3:p:393-421}.

\leavevmode\hypertarget{ref-furnkranz2011p}{}%
Fürnkranz, J., and E. Hüllermeier. 2010. \emph{Preference Learning}.
Springer Verlag, Berlin.

\leavevmode\hypertarget{ref-garrow2010gs}{}%
Garrow, Tudor D.; Lee, Laurie A.; Bodea. 2010. ``Generation of Synthetic
Datasets for Discrete Choice Analysis.'' \emph{Transportation} 37 (2):
183--202. \url{https://doi.org/10.1007/s11116-009-9228-6}.

\leavevmode\hypertarget{ref-greene2008econometric}{}%
Greene, William H. 2008. ``The Econometric Approach to Efficiency
Analysis.'' \emph{The Measurement of Productive Efficiency and
Productivity Growth} 1 (1): 92--250.

\leavevmode\hypertarget{ref-harrison2004experimental}{}%
Harrison, Glenn W, Ronald M Harstad, and E Elisabet Rutström. 2004.
``Experimental Methods and Elicitation of Values.'' \emph{Experimental
Economics} 7 (2). Springer: 123--40.

\leavevmode\hypertarget{ref-hastie2009sl}{}%
Hastie, Trevor, Robert Tibshirani, and Jerome Friedman. 2009. \emph{The
Elements of Statistical Learning: Data Mining, Inference, and
Prediction}. Springer Science \& Business Media.

\leavevmode\hypertarget{ref-R-arsenal}{}%
Heinzen, Ethan, Jason Sinnwell, Elizabeth Atkinson, Tina Gunderson, and
Gregory Dougherty. 2020. \emph{Arsenal: An Arsenal of 'R' Functions for
Large-Scale Statistical Summaries}.
\url{https://CRAN.R-project.org/package=arsenal}.

\leavevmode\hypertarget{ref-R-stargazer}{}%
Hlavac, Marek. 2018. \emph{Stargazer: Well-Formatted Regression and
Summary Statistics Tables}.
\url{https://CRAN.R-project.org/package=stargazer}.

\leavevmode\hypertarget{ref-japkowicz2011el}{}%
Japkowicz, Nathalie, and Mohak Shah. 2011. \emph{Evaluating Learning
Algorithms: A Classification Perspective}. Cambridge University Press.
\url{https://doi.org/10.1017/CBO9780511921803}.

\leavevmode\hypertarget{ref-jebara2004ml}{}%
Jebara, Tony. 2004. \emph{Machine Learning: Discriminative and
Generative}. Springer Science \& Business Media.

\leavevmode\hypertarget{ref-kiefer1952stochastic}{}%
Kiefer, Jack, Jacob Wolfowitz, and others. 1952. ``Stochastic Estimation
of the Maximum of a Regression Function.'' \emph{The Annals of
Mathematical Statistics} 23 (3). Institute of Mathematical Statistics:
462--66.

\leavevmode\hypertarget{ref-kingma2014adam}{}%
Kingma, Diederik P, and Jimmy Ba. 2014. ``Adam: A Method for Stochastic
Optimization.'' \emph{arXiv Preprint arXiv:1412.6980}.

\leavevmode\hypertarget{ref-kirk2012experimental}{}%
Kirk, Roger E. 2012. ``Experimental Design.'' \emph{Handbook of
Psychology, Second Edition} 2. Wiley Online Library.

\leavevmode\hypertarget{ref-kotsiantis2006tr}{}%
Kotsiantis, Sotiris, I. Zaharakis, and P. Pintelas. 2006. ``Machine
Learning: A Review of Classification and Combining Techniques.''
\emph{Artificial Intelligence Review} 26 (November): 159--90.
\url{https://doi.org/10.1007/s10462-007-9052-3}.

\leavevmode\hypertarget{ref-krinsky1986approximating}{}%
Krinsky, Itzhak, and A Leslie Robb. 1986. ``On Approximating the
Statistical Properties of Elasticities.'' \emph{The Review of Economics
and Statistics}. JSTOR, 715--19.

\leavevmode\hypertarget{ref-kuhfeld2003marketing}{}%
Kuhfeld, Warren F. 2003. \emph{Marketing Research Methods in Sas.}
Citeseer.

\leavevmode\hypertarget{ref-kukar2002re}{}%
Kukar, Matjaž, and Igor Kononenko. 2002. ``Reliable Classifications with
Machine Learning.'' In \emph{European Conference on Machine Learning},
219--31. Springer.

\leavevmode\hypertarget{ref-leong2015ram}{}%
Leong, Waiyan, and David A. Hensher. 2015. ``Contrasts of Relative
Advantage Maximisation with Random Utility Maximisation and Regret
Minimisation.'' \emph{Journal of Transport Economics and Policy (JTEP)}
49 (1): 167--86.
\url{https://www.ingentaconnect.com/content/lse/jtep/2015/00000049/00000001/art00010}.

\leavevmode\hypertarget{ref-louviere2000stated}{}%
Louviere, Jordan J, David A Hensher, and Joffre D Swait. 2000.
\emph{Stated Choice Methods: Analysis and Applications}. Cambridge
university press.

\leavevmode\hypertarget{ref-mccausland2013pd}{}%
McCausland, William J., and A.A.J. Marley. 2013. ``Prior Distributions
for Random Choice Structures.'' \emph{Journal of Mathematical
Psychology} 57 (3): 78--93.
\url{https://doi.org/https://doi.org/10.1016/j.jmp.2013.05.001}.

\leavevmode\hypertarget{ref-mcfadden1974utd}{}%
McFadden, Daniel. 1974. ``The Measurement of Urban Travel Demand.''
\emph{Journal of Public Economics} 3 (4): 303--28.
\url{https://doi.org/https://doi.org/10.1016/0047-2727(74)90003-6}.

\leavevmode\hypertarget{ref-mcfadden2001ec}{}%
---------. 2001. ``Economic Choices.'' \emph{The American Economic
Review} 91 (3). American Economic Association: 351--78.
\url{http://www.jstor.org/stable/2677869}.

\leavevmode\hypertarget{ref-mcfadden2000mmnl}{}%
McFadden, Daniel, and Kenneth Train. 2000. ``Mixed Mnl Models for
Discrete Response.'' \emph{Journal of Applied Econometrics} 15 (5).
Wiley Online Library: 447--70.

\leavevmode\hypertarget{ref-llerena2013rose}{}%
Michaud, Celine, Daniel Llerena, and Iragael Joly. 2012. ``Willingness
to pay for environmental attributes of non-food agricultural products: a
real choice experiment.'' \emph{European Review of Agricultural
Economics} 40 (2): 313--29. \url{https://doi.org/10.1093/erae/jbs025}.

\leavevmode\hypertarget{ref-michaud2013willingness}{}%
---------. 2013. ``Willingness to Pay for Environmental Attributes of
Non-Food Agricultural Products: A Real Choice Experiment.''
\emph{European Review of Agricultural Economics} 40 (2). Oxford
University Press: 313--29.

\leavevmode\hypertarget{ref-R-foreach}{}%
Microsoft, and Steve Weston. 2020. \emph{Foreach: Provides Foreach
Looping Construct}. \url{https://CRAN.R-project.org/package=foreach}.

\leavevmode\hypertarget{ref-molina2019soc}{}%
Molina, Mario, and Filiz Garip. 2019. ``Machine Learning for
Sociology.'' \emph{Annual Review of Sociology} 45 (1): 27--45.
\url{https://doi.org/10.1146/annurev-soc-073117-041106}.

\leavevmode\hypertarget{ref-mullainathan2017ml}{}%
Mullainathan, Sendhil, and Jann Spiess. 2017. ``Machine Learning: An
Applied Econometric Approach.'' \emph{Journal of Economic Perspectives}
31 (2): 87--106. \url{https://doi.org/10.1257/jep.31.2.87}.

\leavevmode\hypertarget{ref-munizaga2005mlyp}{}%
Munizaga, Marcela A., and Ricardo Alvarez-Daziano. 2005. ``Testing Mixed
Logit and Probit Models by Simulation.'' \emph{Transportation Research
Record} 1921 (1): 53--62.
\url{https://doi.org/10.1177/0361198105192100107}.

\leavevmode\hypertarget{ref-paredes2017machine}{}%
Paredes, Miguel, Erik Hemberg, Una-May O'Reilly, and Chris Zegras. 2017.
``Machine Learning or Discrete Choice Models for Car Ownership Demand
Estimation and Prediction?'' In \emph{2017 5th Ieee International
Conference on Models and Technologies for Intelligent Transportation
Systems (Mt-Its)}, 780--85. IEEE.

\leavevmode\hypertarget{ref-pigozzi2016pai}{}%
Pigozzi, Gabriella, Alexis Tsoukiàs, and Paolo Viappiani. 2016.
``Preferences in Artificial Intelligence.'' \emph{Annals of Mathematics
and Artificial Intelligence} 77 (3-4). Springer Verlag: 361--401.
\url{https://doi.org/10.1007/s10472-015-9475-5}.

\leavevmode\hypertarget{ref-R-foreign}{}%
R Core Team. 2018a. \emph{Foreign: Read Data Stored by 'Minitab', 'S',
'Sas', 'Spss', 'Stata', 'Systat', 'Weka', 'dBase', ...}
\url{https://CRAN.R-project.org/package=foreign}.

\leavevmode\hypertarget{ref-R-base}{}%
---------. 2018b. \emph{R: A Language and Environment for Statistical
Computing}. Vienna, Austria: R Foundation for Statistical Computing.
\url{https://www.R-project.org/}.

\leavevmode\hypertarget{ref-revelt1998mixed}{}%
Revelt, David, and Kenneth Train. 1998. ``Mixed Logit with Repeated
Choices: Households' Choices of Appliance Efficiency Level.''
\emph{Review of Economics and Statistics} 80 (4). MIT Press: 647--57.

\leavevmode\hypertarget{ref-rose2006constructing}{}%
Rose, John M, and Michiel CJ Bliemer. 2006. ``Constructing Efficient
Stated Choice Experimental Designs.'' \emph{Transport Reviews} 29 (5).
Taylor \& Francis: 587--617.

\leavevmode\hypertarget{ref-rose2008designing}{}%
Rose, John M, Michiel CJ Bliemer, David A Hensher, and Andrew T Collins.
2008. ``Designing Efficient Stated Choice Experiments in the Presence of
Reference Alternatives.'' \emph{Transportation Research Part B:
Methodological} 42 (4). Elsevier: 395--406.

\leavevmode\hypertarget{ref-R-sdcm}{}%
Talebijamalabad, Amirreza, Nikita Gusarov, and Iragael Joly. 2020.
\emph{Sdcm: Simulation of Discrete Choice Models}.

\leavevmode\hypertarget{ref-train2009dc}{}%
Train, Kenneth E. 2009. \emph{Discrete Choice Methods with Simulation}.
Cambridge university press.

\leavevmode\hypertarget{ref-tsoukias2013ph}{}%
Tsoukiàs, Alexis, and Paolo Viappiani. 2013. ``Tutorial on preference
handling.'' In \emph{ACM Conference on Recommender System (RecSys)},
497--98. Hong Kong, China.
\url{https://doi.org/10.1145/2507157.2508065}.

\leavevmode\hypertarget{ref-tsoumakas2007cm}{}%
Tsoumakas, Grigorios, and Ioannis Katakis. 2007. ``Multi-Label
Classification: An Overview.'' \emph{International Journal of Data
Warehousing and Mining (IJDWM)} 3 (3): 1--13.
\url{https://EconPapers.repec.org/RePEc:igg:jdwm00:v:3:y:2007:i:3:p:1-13}.

\leavevmode\hypertarget{ref-varian2014bd}{}%
Varian, Hal R. 2014. ``Big Data: New Tricks for Econometrics.''
\emph{Journal of Economic Perspectives} 28 (2): 3--28.
\url{https://doi.org/10.1257/jep.28.2.3}.

\leavevmode\hypertarget{ref-vitetta2016quantum}{}%
Vitetta, Antonino. 2016. ``A Quantum Utility Model for Route Choice in
Transport Systems.'' \emph{Travel Behaviour and Society} 3. Elsevier:
29--37.

\leavevmode\hypertarget{ref-R-tidyverse}{}%
Wickham, Hadley. 2019. \emph{Tidyverse: Easily Install and Load the
'Tidyverse'}. \url{https://CRAN.R-project.org/package=tidyverse}.

\leavevmode\hypertarget{ref-R-knitr}{}%
Xie, Yihui. 2020a. \emph{Knitr: A General-Purpose Package for Dynamic
Report Generation in R}. \url{https://CRAN.R-project.org/package=knitr}.

\leavevmode\hypertarget{ref-R-tinytex}{}%
---------. 2020b. \emph{Tinytex: Helper Functions to Install and
Maintain Tex Live, and Compile Latex Documents}.
\url{https://CRAN.R-project.org/package=tinytex}.

\leavevmode\hypertarget{ref-yukalov2017quantum}{}%
Yukalov, Vyacheslav I, and Didier Sornette. 2017. ``Quantum
Probabilities as Behavioral Probabilities.'' \emph{Entropy} 19 (3).
Multidisciplinary Digital Publishing Institute: 112.

\leavevmode\hypertarget{ref-zielesny2011cf}{}%
Zielesny, Achim. 2011. \emph{From Curve Fitting to Machine Learning}.
Vol. 18. Springer.

\newpage

\hypertarget{appendices}{%
\section*{Appendices}\label{appendices}}
\addcontentsline{toc}{section}{Appendices}

\hypertarget{a-taxonomies-of-statistical-models}{%
\subsection*{A Taxonomies of statistical
models}\label{a-taxonomies-of-statistical-models}}
\addcontentsline{toc}{subsection}{A Taxonomies of statistical models}

\FloatBarrier

\begin{figure}[!htbp]
\centering
\caption{Taxonomy as proposed by Hastie and Tibshirani (2009), reduced form}
\label{fig:ayod}
\begin{forest}
  for tree={
    align=center,
    edge+={ -{Stealth[]}},
    l sep'+=10pt,
    fork sep'=10pt,
  },
  forked edges,
  if level=0{
    inner xsep=0pt,
    tikz={\draw (.children first) -- (.children last);}
  }{},
  [Machine Learning 
    [Supervised\\learning
        [Non-linear\\methods
            [Additive\\models]
            [SVM]
            [Prototype\\methods]
            [Neural\\Networks]
        ]
        [Linear\\methods
            [Classification
                [Discriminant\\analysis]
                [Logistic\\regression\\and\\generalisations]
                [Separating\\hyperplanes]
                [Identity\\regression]
            ]
            [Regression]
        ]
    ]
    [Unsupervised\\learning]
  ]
\end{forest}
\end{figure}

\begin{figure}[!htbp]
\centering
\caption{Taxonomy as proposed by Ayodele (2010)}
\label{fig:hast}
\begin{forest}
  for tree={
    align=center,
    edge+={ -{Stealth[]}},
    l sep'+=10pt,
    fork sep'=10pt,
  },
  forked edges,
  if level=0{
    inner xsep=0pt,
    tikz={\draw (.children first) -- (.children last);}
  }{},
  [Machine Learning 
    [Semi-supervised\\learning]
    [Learning\\to\\learn]
    [Reinforcement\\learning]
    [Supervised\\Learning
        [Decision\\trees]
        [Neural\\networks]
        [Bayesian\\networks]
        [K-means\\clustering]
        [Linear\\classifier]
        [Quadratic\\classifier]
    ]
    [Unsupervised\\learning]
    [Transduction]
  ]
\end{forest}
\end{figure}

\begin{figure}[!htbp]
\centering
\caption{Taxonomy as proposed by Agresti (2013), based on data types}
\label{fig:agre}
\begin{forest}
  for tree={
    align=center,
    edge+={ -{Stealth[]}},
    l sep'+=10pt,
    fork sep'=10pt,
  },
  forked edges,
  if level=0{
    inner xsep=0pt,
    tikz={\draw (.children first) -- (.children last);}
  }{},
  [Machine Learning 
    [GLM
        [Multinomial\\responses
            [Ordinal\\responses
                [Cumulative\\logit]
            ]
            [Nominal\\responses
                [Baseline-\\Category\\logit]
            ]
        ]
        [Binary\\responses
            [Conditional\\logit]
            [Probit]
            [Logit]
            [Complementary\\Log-log\\models]
        ]
    ]
  ]
\end{forest}
\end{figure}

\FloatBarrier

\clearpage

\hypertarget{b-performance-measures-positioning}{%
\subsection*{B Performance measures
positioning}\label{b-performance-measures-positioning}}
\addcontentsline{toc}{subsection}{B Performance measures positioning}

\FloatBarrier

\begin{figure}[!htbp]
\centering
\caption{Performance measures as described by Japkowicz (2011)}
\label{fig:japk}
\begin{forest}
  for tree={
    align=center,
    edge+={ -{Stealth[]}},
    l sep'+=10pt,
    fork sep'=10pt,
  },
  forked edges,
  if level=0{
    inner xsep=0pt,
    tikz={\draw (.children first) -- (.children last);}
  }{},
  [All measures 
    [Confusion\\matrix\\based
        [Deterministic\\classifiers
            [Multiclass\\focused
                [Accuracy\\Error rate]
            ]
            [Single-class\\focused
                [TP-TN ratios\\F-measure\\Geometric means\\...]
            ]
        ]
    ]
    [Additional\\information
        [Scoring\\classifiers
            [Graphical\\measures
                [ROC\\PR\\DET\\Lift\\and Cost curves]
            ]
            [Summary\\statistics
                [AUC]
            ]
        ]
        [Continuous\\Probabilistic\\classifiers
            [Distance\\error\\measures
                [RMSE]
            ]
            [Information\\Theoretic\\measures
                [KLD\\KB IR\\BIR]
            ]
        ]
    ]
    [Alternative\\information
        [Multicriteria\\measures]
    ]
  ]
\end{forest}
\end{figure}

\FloatBarrier

\newpage

\FloatBarrier

\hypertarget{c-descriptive-statistics}{%
\subsection*{C Descriptive statistics}\label{c-descriptive-statistics}}
\addcontentsline{toc}{subsection}{C Descriptive statistics}

\hypertarget{c.1-comparing-datasets-over-a-alternative}{%
\subsubsection*{C.1 Comparing datasets over A
alternative}\label{c.1-comparing-datasets-over-a-alternative}}
\addcontentsline{toc}{subsubsection}{C.1 Comparing datasets over A
alternative}

\FloatBarrier

\FloatBarrier

\begin{table}[!htbp] \centering 
  \caption{Alternatives' descriptive statistics by dataset, stratified by alternative} 
  \label{tab:stratA} 
\begin{tabular}{@{\extracolsep{5pt}}llcccc}
\\[-1.8ex]\hline 
\hline \\[-1.8ex] 
Alternative &  & Fixed Effects  & Random Effects  & Target  & p value\\
 & & (N=320000) & (N=320000) & (N=2372) &  \\
\hline \\[-1.8ex] 
A & \textbf{Alternative} &  &  &  & \\
 & ~~~A & 160000 (100.0\%) & 160000 (100.0\%) & 1186 (100.0\%) & \\
 & ~~~B & 0 (0.0\%) & 0 (0.0\%) & 0 (0.0\%) & \\
 & \textbf{Choice} &  &  &  & < 0.001\\
 & ~~~Mean (SD) & 0.427 (0.495) & 0.382 (0.486) & 0.517 (0.500) & \\
 & ~~~Range & 0.000 - 1.000 & 0.000 - 1.000 & 0.000 - 1.000 & \\
 & \textbf{Price} &  &  &  & 0.022\\
 & ~~~Mean (SD) & 3.069 (0.979) & 3.069 (0.979) & 2.990 (0.881) & \\
 & ~~~Range & 1.500 - 4.500 & 1.500 - 4.500 & 1.500 - 4.500 & \\
 & \textbf{Carbon} &  &  &  & < 0.001\\
 & ~~~Mean (SD) & 0.500 (0.500) & 0.500 (0.500) & 0.167 (0.373) & \\
 & ~~~Range & 0.000 - 1.000 & 0.000 - 1.000 & 0.000 - 1.000 & \\
 & \textbf{Label} &  &  &  & 0.993\\
 & ~~~Mean (SD) & 0.500 (0.500) & 0.500 (0.500) & 0.502 (0.500) & \\
 & ~~~Range & 0.000 - 1.000 & 0.000 - 1.000 & 0.000 - 1.000 & \\
 & \textbf{Price by group} &  &  &  & < 0.001\\
 & ~~~1.5 & 16000 (10.0\%) & 16000 (10.0\%) & 82 (6.9\%) & \\
 & ~~~2 & 24000 (15.0\%) & 24000 (15.0\%) & 223 (18.8\%) & \\
 & ~~~2.5 & 27000 (16.9\%) & 27000 (16.9\%) & 214 (18.0\%) & \\
 & ~~~3 & 23000 (14.4\%) & 23000 (14.4\%) & 175 (14.8\%) & \\
 & ~~~3.5 & 22000 (13.8\%) & 22000 (13.8\%) & 187 (15.8\%) & \\
 & ~~~4 & 21000 (13.1\%) & 21000 (13.1\%) & 219 (18.5\%) & \\
 & ~~~4.5 & 27000 (16.9\%) & 27000 (16.9\%) & 86 (7.3\%) & \\
\hline
\end{tabular}
\end{table}

\FloatBarrier

\newpage

\hypertarget{c.2-comparing-datasets-over-b-alternative}{%
\subsubsection*{C.2 Comparing datasets over B
alternative}\label{c.2-comparing-datasets-over-b-alternative}}
\addcontentsline{toc}{subsubsection}{C.2 Comparing datasets over B
alternative}

\FloatBarrier

\begin{table}[!htbp] \centering 
  \caption{Alternatives' descriptive statistics by dataset, stratified by alternative} 
  \label{tab:stratB} 
\begin{tabular}{@{\extracolsep{5pt}}llcccc}
\\[-1.8ex]\hline 
\hline \\[-1.8ex] 
Alternative &  & Fixed Effects  & Random Effects  & Target  & p value\\
 & & (N=320000) & (N=320000) & (N=2372) &  \\
\hline \\[-1.8ex] 
B & \textbf{Alternative} &  &  &  & \\
 & ~~~A & 0 (0.0\%) & 0 (0.0\%) & 0 (0.0\%) & \\
 & ~~~B & 160000 (100.0\%) & 160000 (100.0\%) & 1186 (100.0\%) & \\
 & \textbf{Choice} &  &  &  & < 0.001\\
 & ~~~Mean (SD) & 0.518 (0.500) & 0.462 (0.499) & 0.159 (0.366) & \\
 & ~~~Range & 0.000 - 1.000 & 0.000 - 1.000 & 0.000 - 1.000 & \\
 & \textbf{Price} &  &  &  & < 0.001\\
 & ~~~Mean (SD) & 2.803 (0.917) & 2.803 (0.917) & 3.020 (0.893) & \\
 & ~~~Range & 1.500 - 4.500 & 1.500 - 4.500 & 1.500 - 4.500 & \\
 & \textbf{Carbon} &  &  &  & < 0.001\\
 & ~~~Mean (SD) & 0.500 (0.500) & 0.500 (0.500) & 0.832 (0.374) & \\
 & ~~~Range & 0.000 - 1.000 & 0.000 - 1.000 & 0.000 - 1.000 & \\
 & \textbf{Label} &  &  &  & 0.985\\
 & ~~~Mean (SD) & 0.500 (0.500) & 0.500 (0.500) & 0.497 (0.500) & \\
 & ~~~Range & 0.000 - 1.000 & 0.000 - 1.000 & 0.000 - 1.000 & \\
 & \textbf{Price by group} &  &  &  & < 0.001\\
 & ~~~1.5 & 25000 (15.6\%) & 25000 (15.6\%) & 108 (9.1\%) & \\
 & ~~~2 & 28000 (17.5\%) & 28000 (17.5\%) & 192 (16.2\%) & \\
 & ~~~2.5 & 26000 (16.2\%) & 26000 (16.2\%) & 158 (13.3\%) & \\
 & ~~~3 & 30000 (18.8\%) & 30000 (18.8\%) & 204 (17.2\%) & \\
 & ~~~3.5 & 18000 (11.2\%) & 18000 (11.2\%) & 232 (19.6\%) & \\
 & ~~~4 & 23000 (14.4\%) & 23000 (14.4\%) & 195 (16.4\%) & \\
 & ~~~4.5 & 10000 (6.2\%) & 10000 (6.2\%) & 97 (8.2\%) & \\
\hline
\end{tabular}
\end{table}

\FloatBarrier

\newpage

\hypertarget{d-r-code-for-implemented-models}{%
\subsection*{\texorpdfstring{D \emph{R} code for implemented
models}{D R code for implemented models}}\label{d-r-code-for-implemented-models}}
\addcontentsline{toc}{subsection}{D \emph{R} code for implemented
models}

\hypertarget{d.1-mnl-model}{%
\subsubsection*{D.1 MNL model}\label{d.1-mnl-model}}
\addcontentsline{toc}{subsubsection}{D.1 MNL model}

\begin{Shaded}
\begin{Highlighting}[]
\CommentTok{# Transform dataset to mlogit format}
\NormalTok{mnl_data =}\StringTok{ }\NormalTok{data }\OperatorTok
\StringTok{    }\KeywordTok{mlogit.data}\NormalTok{(}
        \DataTypeTok{choice =} \StringTok{"Choice"}\NormalTok{,}
        \DataTypeTok{alt.var =} \StringTok{"Alternative"}\NormalTok{, }
        \DataTypeTok{shape =} \StringTok{"long"}\NormalTok{, }\CommentTok{# Long format}
        \DataTypeTok{alt.levels =} \KeywordTok{c}\NormalTok{(}\StringTok{"C"}\NormalTok{, }\StringTok{"A"}\NormalTok{, }\StringTok{"B"}\NormalTok{) }\CommentTok{# Define order of alternatives}
\NormalTok{    )}

\CommentTok{# Function}
\NormalTok{utility =}\StringTok{ }\NormalTok{Choice }\OperatorTok{~}\StringTok{ }\NormalTok{Sex }\OperatorTok{+}\StringTok{ }\NormalTok{Age }\OperatorTok{+}\StringTok{ }\NormalTok{Salary }\OperatorTok{+}\StringTok{ }\NormalTok{Habit }\OperatorTok{+}\StringTok{ }\CommentTok{# Individual characteristics}
\StringTok{    }\NormalTok{Price }\OperatorTok{+}\StringTok{ }\NormalTok{Buy }\OperatorTok{+}\StringTok{ }\NormalTok{Label }\OperatorTok{+}\StringTok{ }\NormalTok{Carbon }\OperatorTok{+}\StringTok{ }\NormalTok{LC }\OperatorTok{+}\StringTok{ }\DecValTok{0} \OperatorTok{|}\StringTok{ }\DecValTok{0} \CommentTok{# Alternatives attributes}

\CommentTok{# Estimate MNL model}
\NormalTok{mnl_novar =}\StringTok{ }\KeywordTok{mlogit}\NormalTok{(}
\NormalTok{        utility,}
        \DataTypeTok{data =}\NormalTok{ mnl_data, }
        \DataTypeTok{reflevel =} \StringTok{"C"}\NormalTok{, }\CommentTok{# The No-buy option is the baseline}
        \DataTypeTok{print.level =} \DecValTok{3}\NormalTok{, }\CommentTok{# Print estimation details}
        \DataTypeTok{iterlim =} \DecValTok{1000}
\NormalTok{    )}
\end{Highlighting}
\end{Shaded}

\hypertarget{d.2-mmnl-model}{%
\subsubsection*{D.2 MMNL model}\label{d.2-mmnl-model}}
\addcontentsline{toc}{subsubsection}{D.2 MMNL model}

\begin{Shaded}
\begin{Highlighting}[]
\CommentTok{# Transform dataset to mlogit format}
\NormalTok{mmnl_data =}\StringTok{ }\NormalTok{data }\OperatorTok
\StringTok{    }\KeywordTok{mlogit.data}\NormalTok{(}
        \DataTypeTok{choice =} \StringTok{"Choice"}\NormalTok{,}
        \DataTypeTok{alt.var =} \StringTok{"Alternative"}\NormalTok{,}
        \DataTypeTok{id =} \StringTok{"ID"}\NormalTok{, }\CommentTok{# Set individuals' index}
        \DataTypeTok{chid =} \StringTok{"CHID"}\NormalTok{, }\CommentTok{# Set choice sets index}
        \DataTypeTok{shape =} \StringTok{"long"}\NormalTok{,}
        \DataTypeTok{alt.levels =} \KeywordTok{c}\NormalTok{(}\StringTok{"C"}\NormalTok{, }\StringTok{"A"}\NormalTok{, }\StringTok{"B"}\NormalTok{)}
\NormalTok{    )}

\CommentTok{# Function}
\NormalTok{utility =}\StringTok{ }\NormalTok{Choice }\OperatorTok{~}\StringTok{ }\NormalTok{Sex }\OperatorTok{+}\StringTok{ }\NormalTok{Age }\OperatorTok{+}\StringTok{ }\NormalTok{Salary }\OperatorTok{+}\StringTok{ }\NormalTok{Habit }\OperatorTok{+}\StringTok{ }\CommentTok{# Individual characteristics}
\StringTok{    }\NormalTok{Price }\OperatorTok{+}\StringTok{ }\NormalTok{Buy }\OperatorTok{+}\StringTok{ }\NormalTok{Label }\OperatorTok{+}\StringTok{ }\NormalTok{Carbon }\OperatorTok{+}\StringTok{ }\NormalTok{LC }\OperatorTok{+}\StringTok{ }\DecValTok{0} \OperatorTok{|}\StringTok{ }\DecValTok{0} \CommentTok{# Alternatives attributes}

\CommentTok{# Estimate MMNL model}
\NormalTok{mmnl =}\StringTok{ }\KeywordTok{mlogit}\NormalTok{(}
\NormalTok{        utility,}
        \DataTypeTok{data =}\NormalTok{ mmnl_data, }
        \DataTypeTok{reflevel =} \StringTok{"C"}\NormalTok{, }\CommentTok{# The No-buy option is the baseline}
        \DataTypeTok{correlation =} \OtherTok{TRUE}\NormalTok{, }\CommentTok{# Include covariance (and not variance only)}
        \DataTypeTok{rpar =}  \KeywordTok{c}\NormalTok{( }\CommentTok{# Normality assumption and four parameters}
            \StringTok{"Buy"}\NormalTok{ =}\StringTok{ "n"}\NormalTok{, }
            \StringTok{"Label"}\NormalTok{ =}\StringTok{ "n"}\NormalTok{, }
            \StringTok{"Carbon"}\NormalTok{ =}\StringTok{ "n"}\NormalTok{, }
            \StringTok{"LC"}\NormalTok{ =}\StringTok{ "n"}
\NormalTok{        ),}
        \DataTypeTok{panel =} \OtherTok{TRUE}\NormalTok{, }\CommentTok{# Estimate dataset as panel}
        \DataTypeTok{print.level =} \DecValTok{3}\NormalTok{, }\CommentTok{# Print estimation details}
        \DataTypeTok{iterlim =} \DecValTok{1000}
\NormalTok{    )}
\end{Highlighting}
\end{Shaded}

\hypertarget{d.3-cnn-model-with-adam-algorithm}{%
\subsubsection*{\texorpdfstring{D.3 CNN model with \emph{Adam}
algorithm}{D.3 CNN model with Adam algorithm}}\label{d.3-cnn-model-with-adam-algorithm}}
\addcontentsline{toc}{subsubsection}{D.3 CNN model with \emph{Adam}
algorithm}

\begin{Shaded}
\begin{Highlighting}[]
\CommentTok{# Used libraries }
\KeywordTok{library}\NormalTok{(tidyverse)}
\KeywordTok{library}\NormalTok{(tensorflow)}
\KeywordTok{library}\NormalTok{(keras)}

\CommentTok{# Define optimization algorithm to be used}
\NormalTok{adam_own =}\StringTok{ }\KeywordTok{optimizer_adam}\NormalTok{(}
    \DataTypeTok{lr =} \FloatTok{1e-1}\NormalTok{, }\CommentTok{# We adjust the learning rate, keeping the rest as defaults}
    \DataTypeTok{beta_1 =} \FloatTok{0.9}\NormalTok{, }
    \DataTypeTok{beta_2 =} \FloatTok{0.999}\NormalTok{,}
    \DataTypeTok{epsilon =} \OtherTok{NULL}\NormalTok{, }
    \DataTypeTok{decay =} \DecValTok{0}\NormalTok{, }
    \DataTypeTok{amsgrad =} \OtherTok{FALSE}\NormalTok{, }
    \DataTypeTok{clipnorm =} \DecValTok{6}\NormalTok{, }\CommentTok{# We limit as well the max value for weights}
    \DataTypeTok{clipvalue =} \OtherTok{NULL}
\NormalTok{)}

\CommentTok{# Set hyperparameters }
\CommentTok{## The number of epochs is a hyperparameter that defines the number times }
\CommentTok{## that the learning algorithm will work through the entire training }
\CommentTok{## dataset.}
\NormalTok{epoch =}\StringTok{ }\DecValTok{50}
\CommentTok{## The batch size is a hyperparameter that defines the number of samples }
\CommentTok{## to work through before updating the internal model parameters.}
\NormalTok{batch =}\StringTok{ }\DecValTok{16000}

\CommentTok{# Limit softmax weights }
\CommentTok{## (keras uses dense layer transformation inside softmax layer by default)}
\NormalTok{softmax_weights =}\StringTok{ }\KeywordTok{list}\NormalTok{(}
    \KeywordTok{matrix}\NormalTok{(}
        \KeywordTok{c}\NormalTok{(  }\DecValTok{1}\NormalTok{,  }\DecValTok{0}\NormalTok{,  }\DecValTok{0}\NormalTok{,}
            \DecValTok{0}\NormalTok{,  }\DecValTok{1}\NormalTok{,  }\DecValTok{0}\NormalTok{,}
            \DecValTok{0}\NormalTok{,  }\DecValTok{0}\NormalTok{,  }\DecValTok{1}\NormalTok{), }
        \DataTypeTok{nrow =} \DecValTok{3}
\NormalTok{    )}
\NormalTok{)}

\CommentTok{# Setup CNN model}
\NormalTok{model_cnn =}\StringTok{ }\KeywordTok{keras_model_sequential}\NormalTok{() }\OperatorTok
\StringTok{    }\CommentTok{# We reshape the dataset, as 1D convolution requires 3D tensor as input}
\StringTok{    }\KeywordTok{layer_reshape}\NormalTok{(}
        \DataTypeTok{target_shape =} \KeywordTok{c}\NormalTok{(}\DecValTok{27}\NormalTok{, }\DecValTok{1}\NormalTok{),}
        \DataTypeTok{input_shape =} \DecValTok{27}\NormalTok{,}
        \DataTypeTok{trainable =} \OtherTok{FALSE}
\NormalTok{    ) }\OperatorTok\StringTok{ }
\StringTok{    }\CommentTok{# 1D convolution layer}
\StringTok{    }\KeywordTok{layer_conv_1d}\NormalTok{(}
        \DataTypeTok{filters =}\NormalTok{ 1L, }\CommentTok{# Dimentions of the output space}
        \DataTypeTok{kernel_size =}\NormalTok{ 9L, }\CommentTok{# Number of parameters}
        \DataTypeTok{strides =}\NormalTok{ 9L, }\CommentTok{# Strides of convolution equal to parameters side}
        \CommentTok{# The starting value is 0 to ensure reproducibility}
        \DataTypeTok{kernel_initializer =} \StringTok{"zeros"}\NormalTok{, }
        \CommentTok{# The constant is not added, because we already have "Buy" dummy}
        \DataTypeTok{use_bias =} \OtherTok{FALSE}\NormalTok{, }
        \CommentTok{# We want a linear activation function }
        \DataTypeTok{activation =} \StringTok{"linear"}\NormalTok{, }
        \DataTypeTok{input_shape =} \KeywordTok{c}\NormalTok{(}\DecValTok{27}\NormalTok{, }\DecValTok{1}\NormalTok{)}
\NormalTok{    ) }\OperatorTok\StringTok{ }
\StringTok{    }\CommentTok{# An inverse transformation into a 2D tensor for softmax implementation}
\StringTok{    }\KeywordTok{layer_flatten}\NormalTok{(}
        \DataTypeTok{data_format =} \StringTok{"channels_first"}
\NormalTok{    ) }\OperatorTok
\StringTok{    }\CommentTok{# Softmax layer}
\StringTok{    }\KeywordTok{layer_dense}\NormalTok{(}
        \DataTypeTok{units =} \DecValTok{3}\NormalTok{, }\CommentTok{# Number of units equal to categories (3 utilities)}
        \DataTypeTok{use_bias =} \OtherTok{FALSE}\NormalTok{, }\CommentTok{# The bias constant is not estimated}
        \DataTypeTok{weights =}\NormalTok{ softmax_weights,}
        \DataTypeTok{trainable =} \OtherTok{FALSE}\NormalTok{, }\CommentTok{# This layer is fixed}
        \DataTypeTok{activation =} \StringTok{"softmax"} \CommentTok{# Softmax layer (to obtain probabilities)}
\NormalTok{    ) }\OperatorTok\StringTok{ }
\StringTok{    }\CommentTok{# Learning algorith definition}
\StringTok{    }\KeywordTok{compile}\NormalTok{(}
        \DataTypeTok{loss =} \StringTok{"categorical_crossentropy"}\NormalTok{, }\CommentTok{# Choice of loss function}
        \DataTypeTok{optimizer =}\NormalTok{ adam_own, }\CommentTok{# Parametrised Adam}
        \DataTypeTok{metrics =} \KeywordTok{c}\NormalTok{(}\StringTok{"accuracy"}\NormalTok{) }\CommentTok{# Target metrics}
\NormalTok{    ) }\OperatorTok\StringTok{ }
\StringTok{    }\CommentTok{# Training the model}
\StringTok{    }\KeywordTok{fit}\NormalTok{(}
\NormalTok{        X_train, Y_train, }\CommentTok{# To train the model we use 80% of our dataset}
        \DataTypeTok{epochs =}\NormalTok{ epoch,}
        \DataTypeTok{batch_size =}\NormalTok{ batch, }
        \DataTypeTok{validation_data =} \KeywordTok{list}\NormalTok{(X_test, Y_test) }\CommentTok{# 20% for validation}
\NormalTok{    )}
\end{Highlighting}
\end{Shaded}

\FloatBarrier

\newpage

\hypertarget{annexes}{%
\section*{Annexes}\label{annexes}}
\addcontentsline{toc}{section}{Annexes}

\hypertarget{i-simulation-tool-for-performance-comparison-of-discrete-choice-models}{%
\subsection*{I Simulation tool for performance comparison of discrete
choice
models}\label{i-simulation-tool-for-performance-comparison-of-discrete-choice-models}}
\addcontentsline{toc}{subsection}{I Simulation tool for performance
comparison of discrete choice models}

\textbf{Author:} Amirreza Talebijamalabad, M1 SIE (Grenoble INP)

\textbf{Under supervision of:} Iragaël Joly, HDR (GAEL, UGA, Grenoble
INP)

\textbf{Available at:} \url{https://github.com/Amirreza-96/sdcm}

\FloatBarrier

An experimental design is a plan which identifies the independent,
dependent, and nuisance variables and indicates the way in which the
randomization and statistical aspects of an experiment are to be carried
out. Speaking of experimental design, we need to bear randomization,
replication and blocking in our minds as three key elements of the
experimental design (Kirk 2012). Randomization as a rather new concept
in design of experiments, plays a pivotal role in distribution of
idiosyncratic characteristics and variables' levels so that they do not
selectively bias the outcome of the experiment. For example, in our
designs, we applied randomization to avoid dominant alternatives as much
as we can. Replication is the observation of two or more experimental
units under the same conditions. Replication enables us to validate the
proposed model and ensures the precise effects. Usually, in simulation,
we run a very long replication or we make relatively many replications
but small in dimensions, which we choose to replicate once but large
enough. Blocking, on the other hand, is an experimental procedure for
isolating variation attributable to a nuisance variable. Also, making
blocks, we can randomly assign respondents to the choice sets or control
the number of respondents in order to intimate the real scenarios;
However, blocking is not of great importance when we are talking in the
realm of simulation since we can control variations and variables.

Stated choice experiments present sampled respondents with a number of
different choice situations, each consisting of a universal but finite
set of alternatives defined on a number of attribute dimensions.
Respondents are then asked to specify their preferred alternatives given
a specific hypothetical choice context. In simulation, since the
respondents are artificial, it will not be wiseful to sample the
population, instead, replicationg the processes would be fruitful as the
population will be generated repeatedly which is more close to reality.
Moreover, to simulate the choice making process, based on decision rules
such as utility maximization, utilities are calculated to reveal the
choices of the individuals. SC data requires that the analyst designs
the experiment in advance by assigning attribute levels to the
attributes that define each of the alternatives which respondents are
asked to consider(Rose et al. 2008).

To generate experimental designs for SC studies we need to find out how
to allocate the attribute levels to the design matrix. Traditionally,
researchers have relied on the principle of orthogonality to populate
the choice situations shown to respondents. The orthogonality of an
experimental design relates to the correlation structure between the
attributes of the design. however, this class of designs may not be
statistically efficient, as they do not take the SC model specification
into account. These models are optimal for the linear models and assure
the researcher that multicollinearity does not exist in design.
Considering this, it is assumed that such designs can be used for the
non-linear models by linear arrangements(Kuhfeld 2003). It is important
to note however, that the orthogonality of a design suggests nothing
about whether two or more attributes are cognitively correlated in the
minds of the respondents (e.g.~price and quality attributes). As such,
orthogonality is purely a statistical property of the design and not a
behavioural property imposed upon the experiment(Rose and Bliemer 2006).
Moreover, by entreing non-design attributes such as socio-demographic
variables, any covariate within the dataset will unlikely be orthogonal,
not only amongst themselves, but also with the design attributes. For
example, if age, gender and income are added as variables in an
analysis, correlations are not only likely to exist for these variables,
but given that the variables described are constant over all choice
situations within individual respondents, correlations between these
variables and other attributes of the design are also likely to exist.
Simulation tool should allow us to enter or not such soci-demographic
variables to the simulation process so that at least we have some
control on correlations. Furthermore, more advanced data generation
methods should be applied to generate correlated data with specific
precision. In this research, we have made a very conventional and
widespread design so called full factorial design. It contains all of
the possible levels of factors, and allows us to estimate all of the
main effects and two-way interactions. Main effects are independent of
the levels of other attributes, however; interactions involve two or
more factors in which, effect of one factor depends on the level of
another(Kuhfeld 2003). Furthermore, there are fractional orthogonal
designs known as efficient designs providing ratherly small but
efficient designs, also, there are algorithms to determine the
correlations between columns as orthogonality is violated in these
designs. It would be excellent if various kinds of designs were
available in simulator.

Michaud, Llerena, and Joly (2013) conducted an empirical work to figure
out consumers' willingness to pay, and a price premium for two
environmental attributes of a non-food agricultural product(Roses). In
this research there are two unlabelled alternatives Rose A and B and one
no choice alternative. The two attributes, Label and Carbon, have two
levels which make four combinations, hence six pairs of alternatives can
be drawn from these combinations. Price ranges from 1.5 to 4.5 and is
randomly assigned to the combinations of two other attributes. Finally,
each respondent is faced with twelve choice sets(24 combinations of all
attributes or 12 questions), hence, considerring no choice mode, there
are three alternatives in each question. Trying to simulate the paper's
results, we made a design with the same attributes and attribute levels.
We sample put alternatives two by two in choice sets (16 choice
sittuation), hence, respondents are faced twice with six pairs of
alternatives, but the price is randomly assigned to each of the choice
sets. Moreover, as no-choice mode does not effect the design, we do not
add this mode to the design but finally, when it comes to utility
comparison and decision process, this alternative is taken into account.
Furthermore, we have not put interaction variable in the design since as
no-choice mode, it does not affect the combinations of the design. To
add, it makes the tool more flexible if we allow the user to decide
about these two options.

Michaud, Llerena, and Joly (2013) considered four socioeconomic
characteristics as well as sex, age, income and organic purchase habit.
Since no information were available in regard to these features'
correlation, we assumed that they are uncorrelated, and made each
feature independently. This is a limitation for data generation process.
Simulator must enable the user to specify whether data is correlated or
not. Moreover, it should allow the user to enter the inputs and
specifications as well as distributions and their parameters. In order
to generate sex data, we draw samples out of a uniform distribution with
parameters \(a=0,b=1\), then we assume that there is a \(0.49\) chance
that a respondent is female. Hence, if the random number is in range
\((0,0.49)\), hypothetical individual is female, otherwise, is male. The
same procedure applies to the habit feautre. If the random number is
among \((0,0.35)\), organic habit is assumed to be zero. In order to
generate age feature, the best distribution that we can draw samples
which exactly could resemble the real data is truncated normal
distribution. However, we just have the tnormal distribution's
parameters and we need to have the underlying normal distribution's
parameters(mean, and std.) to be able to draw samples. To tackle this,
we solve a system of non-linear equations utilising numerical
methods(Newton Raphson method) to find the underlying normal dist.
parameters. Another way to generate such data is to draw samples from a
log normal distribution. We still have a problem with this way since we
need to have data ranging from 18 to 85, nevertheless positive values
are generated. And finally, we simply take draws from normal
distribution with the same parameters. As future improvements,
simulation tool should be able to generate data based on theoretical
distributions or empirical ones. Hence, some curve fitting procedures to
find the distribution best fitting the real datashould be installed in
the tool.

So far, we have made the design and socioeconomic features for
artificial individuals. Now, we need to specify utilities per each
individual, and finally, due to the RUM model, we select the alternative
with highest utility per each individual per each choice set. As a
future improvement, we suggust that simulation tool should be able to
simulate decision making process based on different approaches for
example, regret minimization. In order to calculate utilities, we took
parameters from the paper (\emph{a priori}). All of the terms mentioned
in the paper including ASC are used. We take no-choice as reference
alternative. Firstly, a matrix of \(4\times 1000\) is constructed for
socioeconomic characteristics parameters. Each column indicates a person
, and all of the columns are similar since the parameters are constant
for all of the people. Then, this matrix is pointly multiplied with the
matrix of socioeconomic charachteristics matrix, and finally the sum of
each column is the socioeconomic utility of each person and a vector of
utility is achieved. Secondly, a matrix of \(1000\times 5\) is made to
contain the parameters corresponding to the alternatives(price, label,
carbon, label-carbon, constant). each row of this matrix is drawn from
multivariate normal distribution, \(\mu + L\times R\) where \(\mu\) is a
vector of means of parameters, L is derived from Cholesky
decomposition(\(L\times L^{\prime} = \sigma^2\)) and \(R\) is a vector
of \(K\) draws from a \(N~(0,1)\). Finally, this matrix is multiplied by
the inverse of design matrix which results in a matrix of
\(1000\times 36\) in which each element shows the utility of an
alternative for an individual. Consequently, we add up socioeconomic
utility to each of the columns of this matrix. This makes the observed
utility. In regard to unobserved utility, a matrix of \(1000\times 36\)
is containing the draws of \(Gumbel(0,1)\), then we add this matrix to
the previous one and this brings about the utility matrix. For the
choice selection process, columns of utility matrix are compared pair by
pair and also the max of these each of these paires is compared with an
element of \(Gumbel(0,1)\) to specify whether the individual buys or
not. Finally, we suggust that tool decode and clean the data. One
important issue is the difference between real data and simulated data
which arises from ommited variables. For example, when it comes to
reality, time is a very important factor affecting the choices made by
respondents, but when it comes to simulation, time is meaningless for
artificial individuals. These issues also need to be taken into account
specially when we are comparing estimation results of these two types of
data.

\newpage

\hypertarget{ii-reproducible-research}{%
\subsection*{II Reproducible research}\label{ii-reproducible-research}}
\addcontentsline{toc}{subsection}{II Reproducible research}

This work was accomplished with implementation of the most advanced
reproducible research techniques. First of all, a version control system
(\emph{git}) was used to track the changes and modification in the
working tree from the start of the internship. The collaboration with
other participants was organised through \emph{GitHub}, where a common
repository was maintained to store the data and document, as well as to
keep every element of the code or text available to everyone. The report
generation was automated with the use of a simplified markup language
with embedded executable \emph{R} code. For heavy tasks, such as data
generation, model estimation or big data exploration separate source
code files were used.

This short documents aims to introduce the reader to used research
methodology, that was used in this work and during the internship. The
used tool-set will be introduced.

\emph{Git} is one of the version control tools alongside SVN and
Mercurial-SCM, which allows to easily control changes and modifications
within text documents. Unfortunately the proposed functionality does not
function with more complex proprietary formats such as Word or image
based documents, such as PDF. Consequently, this tool is not practical
only for working with simple text documents: it remains absolutely
impractical for working with typical office tasks. Several text editors
for developers among which RStudio, VSCode, Atom and many other provide
possibilities to integrate \emph{git} functionality directly into the
editor and drastically optimise the workflow. This makes interacting
with \emph{git} much more comfortable than through the command line or a
standalone \emph{git} client.

\emph{GitHub} is an open source cooperative platform for developers
offered by Microsoft making it easier to work with the \emph{git}
version control service. The platform has an entire ecosystem of
extensions, expanding git functionality, as well as a set of project
management and communication tools. In total this platform offers:

\begin{itemize}
\tightlist
\item
  A cloud space to host the working files and publish the results;
\item
  A web interface to interact with \emph{git} from browser or through a
  standalone app;
\item
  A platform facilitating collaboration with other users, which
  gradually approaches in the functionality to a social network;
\item
  An integrated project management system.
\end{itemize}

To write the scientific report it was decided to use the \emph{LaTeX}
complete markup language. There are several distributions of LaTeX, one
of the verified versions to integrate well with \emph{R} being
\emph{tinytex} (which is available as \emph{tinytex} package in CRAN
repositories). However, even if \emph{LaTeX} produces well structured
documents that are easy to manage, there exists the problem of its
complexity in extending its functionalities. Consequently, it was
decided to use an intermediary simplified markup language, which is easy
to use and does not require advanced knowledge of \emph{LaTeX}:
Markdown. It allows to write documents with simple syntax, which could
be later transformed into PDF, HTML and Word documents using the
\emph{pandoc} converter.

Finally, to embed the \emph{R} code inside the document to automatically
generate the figures and tables, we used \emph{RMarkdown}, which is an
extension for \emph{Markdown} integrating \emph{R} language inside. Such
set-up ensured, that the documents will be easy to share and modify,
preserving at the same time all their functionality. This work offers
all the necessary elements to be fully reproducible.

\textbf{The resulting compedium is available at:}
\url{https://github.com/nikitagusarov/performance_exploration}

\newpage


\end{document}
