\documentclass[11pt,]{article}
\usepackage{lmodern}
\usepackage{amssymb,amsmath}
\usepackage{ifxetex,ifluatex}
\usepackage{fixltx2e} % provides \textsubscript
\ifnum 0\ifxetex 1\fi\ifluatex 1\fi=0 % if pdftex
  \usepackage[T1]{fontenc}
  \usepackage[utf8]{inputenc}
\else % if luatex or xelatex
  \ifxetex
    \usepackage{mathspec}
  \else
    \usepackage{fontspec}
  \fi
  \defaultfontfeatures{Ligatures=TeX,Scale=MatchLowercase}
\fi
% use upquote if available, for straight quotes in verbatim environments
\IfFileExists{upquote.sty}{\usepackage{upquote}}{}
% use microtype if available
\IfFileExists{microtype.sty}{%
\usepackage{microtype}
\UseMicrotypeSet[protrusion]{basicmath} % disable protrusion for tt fonts
}{}
\usepackage[margin=1in]{geometry}
\usepackage{hyperref}
\hypersetup{unicode=true,
            pdfborder={0 0 0},
            breaklinks=true}
\urlstyle{same}  % don't use monospace font for urls
\usepackage{graphicx,grffile}
\makeatletter
\def\maxwidth{\ifdim\Gin@nat@width>\linewidth\linewidth\else\Gin@nat@width\fi}
\def\maxheight{\ifdim\Gin@nat@height>\textheight\textheight\else\Gin@nat@height\fi}
\makeatother
% Scale images if necessary, so that they will not overflow the page
% margins by default, and it is still possible to overwrite the defaults
% using explicit options in \includegraphics[width, height, ...]{}
\setkeys{Gin}{width=\maxwidth,height=\maxheight,keepaspectratio}
\IfFileExists{parskip.sty}{%
\usepackage{parskip}
}{% else
\setlength{\parindent}{0pt}
\setlength{\parskip}{6pt plus 2pt minus 1pt}
}
\setlength{\emergencystretch}{3em}  % prevent overfull lines
\providecommand{\tightlist}{%
  \setlength{\itemsep}{0pt}\setlength{\parskip}{0pt}}
\setcounter{secnumdepth}{0}
% Redefines (sub)paragraphs to behave more like sections
\ifx\paragraph\undefined\else
\let\oldparagraph\paragraph
\renewcommand{\paragraph}[1]{\oldparagraph{#1}\mbox{}}
\fi
\ifx\subparagraph\undefined\else
\let\oldsubparagraph\subparagraph
\renewcommand{\subparagraph}[1]{\oldsubparagraph{#1}\mbox{}}
\fi

%%% Use protect on footnotes to avoid problems with footnotes in titles
\let\rmarkdownfootnote\footnote%
\def\footnote{\protect\rmarkdownfootnote}

%%% Change title format to be more compact
\usepackage{titling}

% Create subtitle command for use in maketitle
\newcommand{\subtitle}[1]{
  \posttitle{
    \begin{center}\large#1\end{center}
    }
}

\setlength{\droptitle}{-2em}

  \title{}
    \pretitle{\vspace{\droptitle}}
  \posttitle{}
    \author{}
    \preauthor{}\postauthor{}
    \date{}
    \predate{}\postdate{}
  
\usepackage{placeins}
\AtBeginDocument{\let\maketitle\relax}

\begin{document}

\FloatBarrier

\newpage

\hypertarget{conclusion}{%
\section*{Conclusion}\label{conclusion}}
\addcontentsline{toc}{section}{Conclusion}

In this work we have introduced the reader to the problematic of the
different modelling paradigms in application to the consumer choice
studies. By means of an experimental theory-testing framework we
demonstrate the complexity of the model performance evaluation
problematic, showing the eventual bottlenecks and the questions to be
answered on all the levels of data exploration procedure. The correct
specification of the theoretical assumptions, the dataset generation,
the model choice as well as the performance measure choice were studied.
The main objective to propose a comprehensive methodology for
theory-testing framework creation was accomplished, illustrating the
devised frameworks' potential over an economic question issued from real
world.

Two different consumer choice situation were explored, issued from the
setting delimited by Michaud, Llerena, and Joly (2012). The discrete
choice context allowed us to compare how the presence of heterogeneous
preferences for environmental attributes affected the possibility to
identify correctly the underlying utility functions, as well as to
derive the WTP and premiums for the attributes. The implementation of
artificial dataset simulation techniques proved its potential in
creation of fully controlled data samples, providing two consistent
datasets constructed under RUM assumptions. Given the data, we could
observe, how taste heterogeneity affected the population's choice
distribution and the resulting datasets, as well as their impact on
models' performances.

A total of three models, issued from alien disciplines such as
econometrics (MNL and MMNL) and ML (CNN-MNL), were implemented over the
generated artificial datasets. We could demonstrate the differences and
similarities between the traditional econometrics models and such ML
techniques as NN. The econometric models allowed us to observe the
potential biases that researchers risk to induce using the simplest
models in unjustified context. The ML model made it possible to
demonstrate, how different approaches to optimisation and algorithmic
solutions influence the obtained results. Moreover, the framework
demonstrated, that ML models could be used instead of the traditional
econometrics techniques under correct specification, as technically NN
are able to approximate any other more simple linear or non-linear
model. All of the models demonstrated good overall performance given the
homogeneous individual preferences setting, while only the most complex
MMNL model achieved sufficient results in presence of taste
heterogeneity.

The multidimensionality of the explored situation allows us to tear
several solutions from this work in terms of model performances in
presence of heterogeneous preferences. The MMNL models demonstrated a
better adaptivity for the different datasets and consequently a better
adaptiveness in all the cases. This family of models showed a great
tolerance for the eventual misspecification in the assumptions of the
presence of random effects. On the contrary, the MNL models produced
biased estimates in the presence of the random effects in population,
which indicates a great danger and signal the importance of the correct
specifications a preliminary data studies to be performed before the
models estimation. The only observed difference was in the way the
resulting approximation was unable to directly estimate the variance for
the linear part coefficients, which is not initially the main focus of
the NN models. However, the marginal effects could still be derived for
the individual characteristics or the alternative specific attributes,
assuming a correct approximation was used, which does not inflate the
overall variance for the marginal effects.

Nevertheless, there exist potential biases that require particular
attention and caution in future research. The implemented
data-generation procedure risks to bias the results in favour of the
econometrics models, which were used to simulate the data. Speaking
about the models, we have observed that the adaptiveness and flexibility
of the MMNL model comets at some costs in resources efficiency. The
time, computation power and the data amount needed to achieve satisfying
results are significantly higher than for the other models.

This work demonstrates only a fraction of the full potential of the
theory-testing framework. Many extensions and generalisations should be
performed before it could be used at scale. For example, it is
particularly interesting to introduce an extension which will provide
the possibility to explore and compare how different behavioural
theories (RUM, RRM, QDM) affects the estimation results. Even more, with
this methodology it becomes possible to explore the effects of
non-additive utility presence or the behaviour of populations with mixed
behaviours presence. Another extension concerns the implemented
mathematical models and consists in incorporating the most recent
developments in the ML field into the framework, enabling users to
implement such models as decision trees or more advanced NN. Last, but
not the least, the framework could be complemented with a methodological
tool-set for hypothesis testing using the advantages of a controlled
experiment data collection.

To summarise, we conclude that the experimental framework has proven its
importance for the empirical and theoretical studies and has
demonstrated its potential. There clearly exist a strong need for a more
extensive study and development of this framework to provide the
research community with a hypothesis testing tool-set, which could be
used in the context of the consumer choice modelling. The exploration of
potential biases and theory-testing will allow us to establish a
comprehensive and consistent methodology to be implemented latter in
empirical work and controlled experiments in particular.

\newpage
\renewcommand\contentsname{}
\setcounter{tocdepth}{4}

\hypertarget{table-of-contents}{%
\section*{Table of contents}\label{table-of-contents}}
\addcontentsline{toc}{section}{Table of contents}

\vspace{-12mm}
\tableofcontents

\newpage

\hypertarget{list-of-figures}{%
\section*{List of figures}\label{list-of-figures}}
\addcontentsline{toc}{section}{List of figures}

\renewcommand\listfigurename{}
\vspace{-12mm}
\listoffigures

\newpage

\hypertarget{list-of-tables}{%
\section*{List of tables}\label{list-of-tables}}
\addcontentsline{toc}{section}{List of tables}

\renewcommand\listtablename{}
\vspace{-12mm}
\listoftables

\newpage

\hypertarget{bibliography}{%
\section*{Bibliography}\label{bibliography}}
\addcontentsline{toc}{section}{Bibliography}

\hypertarget{refs}{}
\leavevmode\hypertarget{ref-llerena2013rose}{}%
Michaud, Celine, Daniel Llerena, and Iragael Joly. 2012. ``Willingness
to pay for environmental attributes of non-food agricultural products: a
real choice experiment.'' \emph{European Review of Agricultural
Economics} 40 (2): 313--29. \url{https://doi.org/10.1093/erae/jbs025}.


\end{document}
